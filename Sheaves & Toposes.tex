\documentclass[11pt]{article}


\usepackage[utf8]{inputenc}
\usepackage{amsfonts, amsmath, amssymb}
\usepackage[english]{babel}
% \usepackage{boisik}
\usepackage{amsthm}
\usepackage[margin=0.5in]{geometry}

%\usepackage{gfsartemisia}
%\usepackage[T1]{fontenc}
%\usepackage{mathpazo}
\usepackage[scr=euler,scrscaled=1.05]{mathalfa} %For different symbol for Categories

\usepackage{tikz}  
\newcommand*\circled[1]{\tikz[baseline=(char.base)]{ %For Circling
		\node[shape=circle,draw,inner sep=2pt] (char) {#1};}}

%\usepackage{CormorantGaramond}



\usepackage{sectsty}
\sectionfont{\fontsize{20}{20}\fontfamily{cmss} \selectfont}
\subsectionfont{\fontsize{15}{15}\fontfamily{cmss} \selectfont}
\subsubsectionfont{\fontsize{12}{12}\fontfamily{cmss} \selectfont}

\usepackage{quiver}
\usepackage{bm}

\usepackage{epigraph}
%\usepackage{tgbonum}
%\usepackage{cmbright}
%\usepackage{textcomp}
\usepackage[object=am]{pgfornament}
\usepackage{graphicx}
\usepackage{tikz-cd}

\usepackage{hyperref} %Uncomment for Hyperlinked Table of Contents.

\hypersetup{
	colorlinks,
	citecolor=blue,
	filecolor=black,
	linkcolor=blue,
	urlcolor=blue
}

%\usepackage[
%backend=biber,
%style=alphabetic,
%]{biblatex}
%\addbibresource{sample.bib}

\theoremstyle{definition}
\newtheorem{definition}{$\boxed{\star}$ Definition}
\newcommand{\tit}[1]{\textit{#1}}
\newtheorem{theorem}{$\boxed{\boxed{\circledast}}$ Theorem}


\theoremstyle{remark}
\newtheorem*{remark}{\textbf{Remark}}

\theoremstyle{remark}
\newtheorem{lemma}{\textbf{Lemma}}

\theoremstyle{remark}
\newtheorem*{example}{\textbf{Example}}

\theoremstyle{definition}
\newtheorem{corollary}{$ \to $ Corollary}

\theoremstyle{definition}
\newtheorem{proposition}{$\bigstar$ Proposition}

\theoremstyle{definition}
\newtheorem*{attempt}{Attempt}

\title{Category Theory\\%Toposes, Triples \& Theories\\
	\large Definitions, Propositions, Theorems \& Proofs}
\author{Animesh Renanse}
\date{\today}
\usepackage{amsthm}

\newcommand{\inv}[1]{#1^{-1}}
\newcommand{\gen}[1]{\left ( #1\right )}
\newcommand{\order}[1]{\left\vert #1 \right\vert}
\newcommand{\image}[0]{\text{Im }}
\newcommand{\kernel}[0]{\text{Ker }}
\newcommand{\nsg}[0]{\trianglelefteq}
\newcommand{\isomorph}{\cong}
\newcommand{\End}[1]{\text{\textbf{End}}\left(#1\right)}
\newcommand{\Auto}[1]{\text{\textbf{Aut}}\left(#1\right)}
\newcommand{\pset}{\mathbf{P}}
\newcommand{\proofref}[1]{\emph{Refer to proof in Appendix #1}}
%\makeatletter
%\newcommand*\bigcdot{\mathpalette\bigcdot@{.5}}

%For Categories
\newcommand{\cat}[1]{{\fontfamily{qhv}\selectfont
		\text{\textbf{#1}}
}}
%\newcommand{\cat}[1]{\mathscr{#1}}

\newcommand{\opcat}[1]{{\fontfamily{qhv}\selectfont
		\text{\textbf{#1}}^{\text{op}}
}}
%\newcommand{\opcat}[1]{\mathscr{#1}^{\text{op}}}

\newcommand{\obj}[1]{\text{\textit{Ob}}(#1)}
\newcommand{\arr}[1]{\text{\textit{Ar}}(#1)}

\newcommand{\Id}[1]{\text{id}_{#1}}
\newcommand{\homset}[3]{{\fontfamily{lmr}\selectfont 
		\text{Hom}_{#1}(#2,#3)
	}}

\renewcommand{\qedsymbol}{\ensuremath{\blacksquare}}
\newcommand{\point}[0]{$\blacktriangleright\;$}
\newcommand{\singobj}[1]{\bullet_{#1}}
\newcommand{\Func}[2]{\text{Func}\left (#1,#2\right )}
\newcommand{\Nat}[2]{{\fontfamily{lmr}\selectfont
		\text{Nat}\left (#1,#2\right )
		}}
\newcommand{\func}[2]{{#2}^{#1}}
\newcommand{\GL}[1]{\text{GL}_{#1}}
\newcommand{\Un}{\text{Un}}
\newcommand{\elem}[1]{\in ^{#1}}

\newcommand{\res}{$ \bigstar $ \textbf{Proposition.}\;}
\newcommand{\Cone}[2]{\text{Cone}\left (#1,#2\right )}
\newcommand{\Cocone}[2]{\text{Cocone}\left (#1,#2\right )}
\newcommand{\ulg}[1]{\left \vert #1\right\vert }
\newcommand{\Eq}[2]{\text{Eq}\left (#1,#2\right )}
\newcommand{\Coeq}[2]{\text{Coeq}\left (#1,#2\right )}
\newcommand{\src}[1]{\;\text{src}\left (#1\right )}
\newcommand{\tar}[1]{\;\text{tar}\left (#1\right )}
\newcommand{\colim}{{\fontfamily{lmr}\selectfont 
		\text{colim}
}}
\newcommand{\ang}[1]{\left \langle #1 \right \rangle}

%%For Topos Theory
\newcommand{\true}{{\fontfamily{qhv}\selectfont %Subobject Classifier
		\text{true}
}}
\newcommand{\Sub}[2]{{\fontfamily{lmr}\selectfont %Subobject functor applied at #1 in category #2
		\text{Sub}_{#2}\left (#1\right )
}}
\newcommand{\psheaf}[1]{\cat{Sets}^{\opcat{#1}}}
\newcommand{\ps}[1]{\widehat{\cat{#1}}}
\newcommand{\yembed}[1]{{\fontfamily{lmr}\selectfont
		\text{\textbf{Yon}} \left (#1\right )
	}}
\newcommand{\open}[1]{\mathcal{O}\left (#1\right )}
\newcommand{\bunion}{\bigcup}
\newcommand{\bintrs}{\bigcap}
\newcommand{\rest}[2]{\left . #1 \right \vert_{#2}}
\newcommand{\union}{\cup}
\newcommand{\intrs}{\cap}
\newcommand{\dirim}[2]{#2_{\star} #1}
\newcommand{\Sh}[1]{{\fontfamily{lmr}\selectfont
		\text{Sh}\left (#1\right )
}}
\newcommand{\basis}{\mathcal{B}}
\newcommand{\nbdsys}[1]{\mathfrak{U}\left (#1\right )}
\newcommand{\germ}[2]{{\fontfamily{lmr}\selectfont
		\text{germ}_{#2} #1
}}
\newcommand{\dom}[1]{{\fontfamily{lmr}\selectfont
		\text{dom} \left (#1\right )
}}
\newcommand{\cod}[1]{{\fontfamily{lmr}\selectfont
		\text{cod} \left (#1\right )
}}
\newcommand{\Match}[2]{{\fontfamily{lmr}\selectfont
		\text{Match} \left (#1,#2\right )
}}
\newcommand{\Ptr}[1]{\hat{#1}}
\newcommand{\chr}[1]{{\fontfamily{lmr}\selectfont
		\text{char }#1 
}}
\newcommand{\nme}[1]{\ulcorner #1\urcorner}
\newcommand{\diag}[1]{\Delta_{#1}}
\newcommand{\singarr}[1]{\{\cdot\}_{#1}}
\newcommand{\upair}[2]{\langle #1,#2 \rangle}
\newcommand{\tens}{\otimes}
\newcommand{\xn}{\overset{\bullet}{\longrightarrow}}
\newcommand{\power}[2]{\pitchfork\left (#2,#1\right )} %power of an object #1 by object #2.
\newcommand{\dimag}[1]{\exists_{#1}}
\newcommand{\join}{\vee}
\newcommand{\meet}{\wedge}
\newcommand{\Open}[1]{{\fontfamily{lmr}\selectfont
		\text{Open }\left (#1\right ) 
}}
\newcommand{\bmeet}{\bigwedge}
\newcommand{\bjoin}{\bigvee}
\newcommand{\Bund}[1]{{\fontfamily{lmr}\selectfont
		\text{\textbf{Bund}}\left (#1\right ) 
}}
\newcommand{\Etal}[1]{{\fontfamily{lmr}\selectfont
		\text{\textbf{\'Etale}}\left (#1\right ) 
}}
\newcommand{\sep}[2]{{\fontfamily{lmr}\selectfont
		\text{{Sep}}_{#1}#2
}}
\newcommand{\sh}[2]{{\fontfamily{lmr}\selectfont
		\text{{Sh}}_{#1}#2
}}
\newcommand{\nno}[1]{{\mathbb{N}}_{\cat{#1}}}
\newcommand{\N}{\mathbb{N}}
\newcommand{\Z}{\mathbb{Z}}
\newcommand{\ClSub}[2]{{\fontfamily{lmr}\selectfont
		\text{{ClSub}}_{#2}\left (#1\right )
}}


%For Categorical Logic
\newcommand{\sign}[0]{\Sigma}
\newcommand{\ssort}[0]{\Sigma-\text{Sort}}
\newcommand{\srel}[0]{\Sigma-\text{Rel}}
\newcommand{\sfun}[0]{\Sigma-\text{Fun}}
\newcommand{\theory}{\mathbb{T}}
\newcommand{\sstruc}[1]{\sign-\text{\textbf{Str}}\left (\cat{#1}\right )}
\newcommand{\uprs}[1]{\langle #1\rangle}
\newcommand{\tmod}[1]{\theory-\text{\textbf{Mod}}\left (\cat{#1}\right )}
\newcommand{\ev}[2]{{\fontfamily{lmr}\selectfont
		\text{{ev}}_{#2}\left (#1\right )
}}


\begin{document}
	\begin{titlepage}
		{\scshape\LARGE Indian Institute of Technology, Kanpur \par}
		\vspace{1cm}
		{\scshape\Large Summer Reading Project\par}
		\vspace{1.5cm}
		{\Huge\bfseries {{\fontfamily{lmr}\selectfont \underline{Sheaves \& Topos Theory}}\par}}
		\vspace{2cm}
		{\Large\itshape\textsc{Animesh Renanse}\par}
		\vfill
		\includegraphics[width=\textwidth]{CoverArt.pdf}
		\vfill
		
		% Bottom of the page
		{\large \today\par}
	\end{titlepage}
	\tableofcontents
	\newpage
	
\section{Towards the axioms of a Topos}
Consider the categories $ \cat{Sets} $, $ \cat{Sets}^{n} $, $ \cat{G-Sets} $, $ \cat{Sets}^{\cat{N}} $ for $ \cat{N} $ being the totally ordered $ \mathbb{N} $ regarded as a category and the slice category $ \cat{Sets}/S $ for some set $ S $ . These all, as we will see later, are important examples of topoi. But the more interesting fact arises from the realization that all of the above examples except the last one, can be shown to be a special case of another important example of a topos, the presheaf category:
\[\cat{Sets}^{\opcat{C}}.\]
For example, we can see that if we fix $ \cat{C} $ as the $ n $ object discrete category, then we get back the category $ \cat{Sets}^{n} $. The arrows in $ \cat{Sets}^{n} $ is just the arrow (natural transf.) in the functor category $ \cat{Sets}^{\opcat{C}} $. To see this, consider two presheaves $ F,G : \opcat{C} \longrightarrow\cat{Sets} $. A natural transformation $ \eta : F\longrightarrow G $ would take some object $ *_i $ in $ \cat{C} $ to the component $ \eta *_i : F{*_i} \to G{*_i} $ for $ 0\le i\le n $. Hence we have $ n $ arrows between two $ n $-tuple of sets, $ (F*_1,\dots,F*_n) $ and $ (G*_1,\dots,G*_n) $. This is exactly the arrow in $ \cat{Sets}^{n} $, so that we have established an isomorphism of categories. Similar constructions for the other examples realize the importance of the presheaf category in this aspect. However, the slice category $ \cat{Sets}/S $ is different in the sense that it is \emph{almost} a presheaf category.  \emph{Almost} because one can only construct the presheaf category $ \cat{Sets}^{S} $ of $ S $ indexed family of sets which will not be isomorphic to the slice, but equivalent only upto a natural isomorphism.
\\\\
Next thing we notice about the above examples is that they all have finite limits. Recall that any category have finite limits if and only if it has terminal object and all pullbacks. Now one may see the importance of having pullbacks by the following trivial example. Suppose we are working in the $ \cat{Sets} $ and we have $ f : X\to B $ and $ g : Y \hookrightarrow B $ for $ Y\subseteq B $, the latter of which is clearly a monic. Then the pullback of $ g $ along $ f $ would be the set $ P $ characterized as:
\[P=\{(x,y)\in X\times Y\;:\; f(x) = g(y) = y\}.\]
Clearly, we have that $ P \isomorph \inv{f}(Y) $. That is, we have a set $ P $ which is same as the inverse image of $ f $ for some subset $ Y $ of it's co-domain. We can hence quite easily turn this idea into any category in order to find a generalized notion of \emph{inverse} of an arrow. 
\\\\
The most immediate use, however, of existence of terminal object and pullbacks is in the generalization of the fact that in $ \cat{Sets} $, any subset $ S \subset X $ can be written either as a monic $ m : S \rightarrowtail X $ or as a characteristic function $ \chi_S : X \to \mathbf{2}$ where $ \mathbf{2} = \{0,1\} $ is a 2 object set which maps as following:
\[\chi_S(x) = \begin{cases}
	0&\text{ if }x\in S\\
	1&\text{ if }x\notin S
\end{cases}\]
where we regard $ \mathbf{2} = \{0,1\} $ as the \textbf{Truth} object, which contains the possible truth values of the category of sets, which clearly exists and has two elements. This latter convention for a subset of a set $ X $ can be generalized to an arbitrary category by the following definition:
\begin{definition}\label{D-1}
	(\textbf{Subobject Classifier})\emph{ Suppose }$ \cat{C} $ \emph{has all finite limits. A subobject classifier is a monic global element}
	\[\true : \mathbf{1} \rightarrowtail \Omega \]
	\emph{ where $ \Omega $ is the truth object, such that for any subobject $ m : S \rightarrowtail X$, there exists a unique arrow $ \phi : X \to \Omega $ such that the following forms a Pullback Square:}
	 \[\begin{tikzcd}
	 	S & {\mathbf{1}} \\
	 	X & \Omega
	 	\arrow["m"', tail, from=1-1, to=2-1]
	 	\arrow["\true", tail, from=1-2, to=2-2]
	 	\arrow[from=1-1, to=1-2]
	 	\arrow["\phi"', dashed, from=2-1, to=2-2]
	 	\arrow["\lrcorner"{anchor=center, pos=0.125}, draw=none, from=1-1, to=2-2]
	 \end{tikzcd}\]
\end{definition}

We can in-fact turn the whole construction of a Subobject to a Presheaf itself!
\begin{definition}\label{D-2}
	(\textbf{Subobject Functor}) \emph{In a well-powered}\footnote{A category which has small set of all subobjects for any object.} \emph{category} $ \cat{C} $\emph{, we have the following presheaf called the subobject functor}
	\[\Sub{-}{\cat{C}} : \opcat{C} \longrightarrow \cat{Sets}\]
\emph{	which takes:}
	\begin{enumerate}
		\item {\emph{\textbf{Object} $ X $ to the small set} $ \Sub{X}{\cat{C}} $,}
		\item {\emph{\textbf{Arrow} $ f : Y\to X $ to the arrow:}
	\begin{align*}
		\Sub{f}{\cat{C}} : \Sub{X}{\cat{C}} \to \Sub{Y}{\cat{C}}
	\end{align*}	
\emph{which takes any subobject $ m : S\rightarrowtail X $ in} $ \Sub{X}{\cat{C}} $\emph{ to the pullback of $ s $ along $ f $, i.e.:} 
\[\begin{tikzcd}
	S & {} & {} & P & S \\
	X &&& Y & X
	\arrow["{\circled{m}}"', tail, from=1-1, to=2-1]
	\arrow[maps to, from=1-2, to=1-3]
	\arrow["{\circled{m^\prime}}"', tail, from=1-4, to=2-4]
	\arrow[from=1-4, to=1-5]
	\arrow["m", tail, from=1-5, to=2-5]
	\arrow["f"', from=2-4, to=2-5]
	\arrow["\lrcorner"{anchor=center, pos=0.125}, draw=none, from=1-4, to=2-5]
\end{tikzcd}\]
}
	\end{enumerate}
\end{definition}
In-fact, the functor $ \Sub{-}{\cat{C}} $ is a Representable Functor represented exactly by the Truth object $ \Omega $!
\begin{proposition}\label{P-1}
	(\textbf{Representability of }$ \Sub{-}{\cat{C}} $) Suppose $ \cat{C} $ is a small category with finite limits. Then, $ \cat{C} $ has a subobject classifier if and only if $ \exists $ an object $ \Omega $ such that $ \exists \; \theta $
	\[\theta : \Sub{-}{\cat{C}} \Longrightarrow \homset{\cat{C}}{-}{\Omega}\]
	which is a natural isomorphism.
\end{proposition}
\begin{proof}
	(L $ \implies $ R) Suppose $ \cat{C} $ has a subobject classifier. Take any $ f : Y\to X $ in $ \cat{C} $. The $ \Sub{f}{\cat{C}} $ identifies an arrow $ \theta_X : \Sub{X}{\cat{C}} \longrightarrow \homset{\cat{C}}{X}{\Omega} $, given by $ \theta_X(m) = $ characteristic arrow of subobject $ m $. This $ \theta_X $ can be seen to be natural in $ X $ by direct verification. \\
	(R $ \implies $ L) Suppose $ \theta : \Sub{-}{\cat{C}}  \Rightarrow \homset{\cat{C}}{-}{\Omega}$ is a natural isomorphism. Take any monic $ m : A\rightarrowtail E $ in $ \cat{C} $. Consider the unique arrow $ \theta_E(m) : A \rightarrow \Omega$. We also have by the isomorphism a unique subobject $ t : \bm{1} \rightarrowtail \Omega$ obtained by such $ t $ with $ \theta_\Omega(t) = 1_\Omega $. The result then follows by observing the naturality condition on $ \theta_X(f) $ for a monic $ f : A \rightarrowtail E $.
\end{proof}
\subsection{Subobject Classifier in $ \cat{Sets}^{\opcat{C}} $}\label{SCiS}
%\textbf{TODO} : Construction of Subobject Classifier of $ \cat{Sets}^{\opcat{C}} $
Since we saw earlier that most of the examples of topoi are particular instances of presheaf categories, therefore the construction of a subobject classifier for $ \cat{Set}^{\opcat{C}} $ constructs a subobject classifier for them too. \\
To begin with, clearly, every monic in $ \cat{Set}^{\opcat{C}} $ is just a natural transformation whose each component is injective. That is, for a monic arrow $ \theta : Q \Longrightarrow P $ in the presheaf category, for any object $ C $ of $ \cat{C} $, the component $ \theta C : QC \rightarrowtail PC $ is monic, so that $ QC \subset PC $. Moreover, by naturality, for any arrow $ f : D\to C $ in $ \cat{C} $, we must have that $ Qf = \left . Pf\right \vert_{QC\subset PC} $. This exactly is the definition of $ Q $ being the subfunctor of $ P $. Therefore, subobjects in presheaf category are subfunctors. With this, we now construct the classifier object $ \Omega $.\\\\
Suppose $ \Omega $ in $ \psheaf{C} $ is the classifier object. Since it classifies all subobjects, therefore it must also classify the contravariant functor $ \homset{\cat{C}}{-}{C} $ for any object $ C $ of $ \cat{C} $. But, by Proposition \ref{P-1}, we have that:
\begin{align*}
	\Sub{\homset{\cat{C}}{-}{C}}{\ps{C}} &\isomorph \homset{\ps{C}}{\homset{\cat{C}}{-}{C}}{\Omega} \\
	&\isomorph \Omega C &&\text{By Yoneda Lemma}
\end{align*}
where $ \ps{C} = \psheaf{C}$. Hence, $ \Omega $ is that object/presheaf in $ \ps{C} $, which takes object $ C $ of $ \cat{C} $ to the set of all subobjects of the representable functor $ \homset{\cat{C}}{-}{C} $ in $ \ps{C} $. But this set would contain just the subfunctors of $ \homset{\cat{C}}{-}{C} $. Now, if we define the following:
\begin{definition}\label{D-3}
	(\textbf{Sieve over an Object}) \emph{In a category }$ \cat{C} $\emph{, a sieve on an object $ C $ is the following set : }
	\begin{align*}
		S_C = \{ f : A\to C\;\vert\; \text{ \emph{if} }h: B\to A\text{\emph{, then} }f\circ h \in S_C\text{\emph{ for any such} }A \}.
	\end{align*}
\end{definition}
Then we notice that a sieve on $ C $ is actually the same thing as a subfunctor of $ \homset{\cat{C}}{-}{C} $. Hence, $ \Omega C $ is just the set of all sieves over the object $ C $. But what about the action of $ \Omega $ on the arrows of $ \cat{C} $?\\
To answer this, we must remind ourselves first that pullback preserves the subobjects. This means that if $ g : B\to C $ in $ \cat{C}$, then pullback of the subobject identified by the monic $ Q \Longrightarrow \homset{\cat{C}}{-}{C} $ along the arrow $ \yembed{g} : \homset{\cat{C}}{-}{B} \Longrightarrow \homset{\cat{C}}{-}{C} $ is a subobject of $ \homset{\cat{C}}{-}{B} $. This translates to the fact that pullback of a sieve over $ C $, $ S_C $ is just the sieve over $ B $, $ S_B $ defined by:
\begin{align*}
	S_B = S_C \cdot g = \{h\;\vert\; g\circ h \in S_C\}.
\end{align*}
Hence, $ \Omega g $ for $ g : B\to C $ is the set function:
\begin{align*}
	 \Omega g : \Omega C &\longrightarrow \Omega B\\
	  S_C &\longmapsto S_B = S_C \cdot g.
\end{align*}
Hence the monic subobject classifier in $ \ps{C} $ is simply:
\begin{align*}
	\true : \mathbf{1} &\Longrightarrow \Omega\\
				\true_C : 1C = \{\star\} &\rightarrowtail \Omega C\\
				\star &\mapsto \text{Maximal Sieve over $ C $.}
 \end{align*}
The final piece left to settle is the unique characteristic function $\phi :  P \Longrightarrow \Omega $ for a given subobject $ Q\Longrightarrow P $. It can be seen that the following choice of $ \phi $ does make the corresponding classifier a pullback diagram:
\begin{align*}
	\phi: P &\Longrightarrow \Omega \\
	\phi_C : PC &\longrightarrow \Omega C\\
	p &\mapsto \{f : E\to C\;\vert\; (Pf)(p) \in Q(E)\text{ for any such }f : E\to C\}
\end{align*}
Note that component $ \phi_C $ maps each element to that sieve which contains those arrows of $ \cat{C} $ whose image under $ P $ takes that element to the subset mapped to by the $ Q $ under the domain of that arrow. Clearly, when $ p\in QC \subset PC $, then for any $f : E\to C  $, $ Pf(p) \in QE $ because $ Qf = \left .Pf \right \vert_{QC\subset PC} $, hence $  \phi_C(p) $ is the maximal sieve over $ C $.
\subsection{Colimits in $ \psheaf{C} $}
It turns out that the presheaf category has a peculiar property that each object/presheaf in it is the colimit of some particular diagram of representable presheaves, and that too in the most obvious way. Before stating it precisely, let us look at a very general result, whose corollary gives us the above result.
\begin{definition}\label{D-4}
	(\textbf{Category of Elements for a Presheaf})\emph{ Suppose} $ P : \opcat{C} \longrightarrow \cat{Sets}$\emph{. Then we can construct a category called the category of elements, }$ \int_{\cat{C}} P $\emph{, which has:}
	\begin{enumerate}
		\item {\emph{\textbf{Objects} as the pairs $ (C,p) $ where }$ C \in \obj{\cat{C}}$ \emph{and }$ p \in PC $,}
		\item{\emph{\textbf{Arrows} as }$ u : (C^{\prime},p^{\prime}) \to (C,p) $ \emph{where} $ u $ \emph{is just} $ u : C^{\prime} \to C$ \emph{in} $ \cat{C} $\emph{ but with property that}
	\[Pu(p) = p^{\prime}.\]	
	}
	\end{enumerate}
\emph{where composition is defined as in} $ \cat{C} $\emph{. Also, there clearly exists a functor:}
\begin{align*}
	\pi_P : \int_\cat{C} P &\longrightarrow \cat{C}\\
	(C,p) &\longmapsto C\\
	\left (u : (C^{\prime},p^{\prime}) \to (C,p)\right ) & \longmapsto \left (u: C^{\prime} \to C \right )
\end{align*}
\end{definition}
\begin{theorem}\label{T-1}
	Suppose $ \cat{C} $ is a small category and $ A : \cat{C} \longrightarrow \cat{E} $ is a functor to a co-complete category $ \cat{E} $. Then, the following functor:
	\begin{align*}
		R : \cat{E} &\longrightarrow \psheaf{C}\\
		E &\longmapsto \homset{\cat{E}}{A(-)}{E}
	\end{align*}
has a Left Adjoint given by:
\begin{align*}
	L : \psheaf{C} & \longrightarrow \cat{E}\\
	P &\longmapsto \colim \left (\int_\cat{C} P \overset{\pi_P}{\longrightarrow} \cat{C} \overset{A}{\longrightarrow } \cat{E}\right )
\end{align*}
\end{theorem}
\begin{proof}
	Suppose $ E $ is some object in $ \cat{E} $. Take any $ \lambda : P \Longrightarrow RE$ from $ \homset{\ps{C}}{P}{RE} $. For any object $ C $ of $ \cat{C} $, $ \lambda C : PC \to \homset{\cat{E}}{AC}{E}$ is therefore just the collection of $ (C,p) $ for $ p\in PC $. Hence, $ \lambda C $ is a subset of $ \obj{\int_\cat{C}P} $. The projection $ \pi_P $ of $ (C,p) $ for any $ p $ is simply the object $ C $ ov $ \cat{C} $. Further application of functor $ A $ on $ \pi_P (C,p) $ would be $ AC $. Similarly, for $ u : (C^{\prime},p^{\prime}) \longrightarrow (C,p) $, the arrow $Au :  A\pi_P (C^{\prime},p^{\prime}) \to A\pi_P(C,p)  $ would be such that the following commutes, because $ \lambda $ is a natural transformation:
	\[\begin{tikzcd}
		{A\pi_P(C,p)} \\
		& E \\
		{A\pi_P(C^\prime,p^\prime)}
		\arrow["u", from=3-1, to=1-1]
		\arrow["{\lambda C(p)}", from=1-1, to=2-2]
		\arrow["{\lambda C^\prime(p)}"', from=3-1, to=2-2]
	\end{tikzcd}\]
Clearly, this means that $ E $ forms a cocone over the diagram $ \int_\cat{C} P \overset{\pi_P}{\longrightarrow} \cat{C} \overset{A}{\longrightarrow } \cat{E} $. Therefore, there exists a unique arrow $ E \dashrightarrow LP $. This assignment establishes the corresponding natural isomorphism between $ \homset{\ps{C}}{P}{RE} $ and $ \homset{\cat{E}}{LP}{E} $.
\end{proof}
This Theorem now naturally leads to the result discussed in the beginning:
\begin{proposition}\label{P-2}
	In the presheaf category $ \ps{C} $, any presheaf $ P $ is the colimit of a particular diagram of representable functors, in a canonical way.
\end{proposition}
\begin{proof}
	In the above Theorem \ref{T-1}, if we set $ \cat{E} = \ps{C} $ and $ A = \yembed{-} $, we get the desired result, where the diagram is the following:
	\begin{align*}
		\int_\cat{C} P \overset{\pi_P}{\longrightarrow} \cat{C} \overset{\yembed{-}}{\longrightarrow } \ps{C}
	\end{align*}
of index as category of elements of $ P $.
\end{proof}
\subsection{Exponentials in $ \psheaf{C} $}
The next important general result about the presheaf category is that every object/presheaf in it is exponentiable. Now since $ \ps{C} $ is already complete, therefore existence of exponentials would imply that $ \ps{C} $ is Cartesian Closed! As we will note much later in a bit more detail, this is a general property of any topoi.
\begin{proposition}\label{P-3}
	Suppose $ \cat{C} $ is a small category, then the presheaf category $ \ps{C} $ is Cartesian Closed.
\end{proposition}
\begin{proof}
	It can be verified that for any object $ P, Q $ in $ \ps{C} $, the another object $ Q^{P} $ defined by:
	\begin{align*}
		 Q^{P} : \opcat{C} &\longrightarrow \cat{Sets}\\
		 C &\longmapsto \homset{\ps{C}}{\homset{\cat{C}}{-}{C}\times P}{Q} = \Nat{\homset{\cat{C}}{-}{C}\times P}{Q}
	\end{align*}
makes $ Q^{P} $ the exponential of $ P $ and $ Q $.
\end{proof}
With all the discussion above, we now see that presheaves is an important category and the examples seen in the beginning of topoi follows the following three rules: (1) it has finite (co)limits, (2) each object in it is exponentiable \& (3) it has a subobject classifier.\\
In fact, these three exactly constitute the definition of a topos, as we will see later!
%\subsection{Heyting Algebras}
\newpage
\section{Sheaves}
Sheaves over a space is a way of describing local property present on a given topological space. As we will see after the discussion on sheaves, this notion is related to topoi in the sense that sheaves over a space is a topos. Moreover, the further properties of topoi are motivated from sheaves.
\begin{definition}\label{D-5}
	(\textbf{Sheaf of Sets - Ordinary Defn.}) \emph{Suppose $ X $ is a topological space. Regard the collection of open sets of $ X $ as a posetal category ordered by inclusion,} $ \cat{O(X)} $\footnote{So that if for open sets $ U,V \in O(X) $\emph{,} $ V \subset U $\emph{, then} $ \exists   $ \emph{an arrow} $ V\to U $ in $ \cat{O(X)} $. }\emph{. Then a presheaf $ F $ over it},
	\begin{align*}
		F : \opcat{O(X)} \longrightarrow \cat{Sets}
	\end{align*}
\emph{is called a Sheaf if for any open set $ U $ and any covering of }$ U = \bunion_{i\in I} U_i$, \emph{we have} :
\begin{enumerate}
	\item {(\textbf{Locality})\emph{ For any} $ f,g \in FU $,
\[\rest{f}{U_i} =\rest{g}{U_i}  \;\forall\;i\in I\; \bm{\implies}\; f = g\]	
}
\item {(\textbf{Gluing}) }\emph{ For any} $ y_i \in FU_i \;\forall \;i\in I$,
\[\rest{y_i}{U_i\intrs U_j} = \rest{y_j}{U_i \intrs U_j}\forall \;i \in  I\;\bm{\implies}\; \exists\; y\in FU\text{ \emph{such that} } \rest{y}{U_i} = y_i\;\forall\;i\in I. \] 
\end{enumerate}
\emph{Here, the restriction map is given as the map, for $ V\subset U $ as}:
\begin{align*}
	\rest{(-)}{V} = F(V\subset U) = FU \longrightarrow FV
\end{align*}
\end{definition}
A more abstract formulation of sheaves in complete categorical language is also possible:
\begin{definition}\label{D-6}
	(\textbf{Sheaf of Sets - Categorical Defn.}) \emph{Suppose $ X $ is a topological space and }$ \cat{O(X)} $\emph{ is the posetal category of open sets of $ X $, ordered by inclusion. Then a presheaf}
	\[F : \opcat{O(X)} \longrightarrow \cat{Sets}\]
	\emph{is a Sheaf if for any open set $ U $ and any covering of $ U = \bunion_{i\in I}U_i $, we have that}
	\[\begin{tikzcd}
		FU \\
		& {\prod_{i\in I} FU_i} && {\prod_{i,j\in I}F(U_i \cap U_j)}
		\arrow["e", dashed, from=1-1, to=2-2]
		\arrow["q"', shift right=2, dashed, from=2-2, to=2-4]
		\arrow["p", shift left=2, dashed, from=2-2, to=2-4]
	\end{tikzcd}\]
	\emph{is an Equalizer diagram, where the unique maps $e, p\;\&\;q $ are given as:}
	\begin{itemize}
		\item {$ e $ \emph{: for a $ f \in FU$, $ e $ maps it as
				\[e(f) = \{\underbrace{F(U_i \subset U)}_{FU \to F(U_i)} (f)\}\in \prod_{i} F(U_i)\]	
				That is, $ e $ maps each element $ f $ of the $ FU $ via the set map under the functor $ F $ of the inclusion $ U_i \subset U$.}}
		\item {$ p $ \emph{: for a sequence $ \{f_i\} \in \prod_{i\in I}FU_i$, $ p $ maps it as
	\[p(\{f_i\}) = \{\underbrace{F(U_i \intrs U_j \subset U_i)}_{FU_i \to F(U_i\intrs U_j)} (f_i)\}\in \prod_{i,j\in I} F(U_i\intrs U_j)\]	
	That is, $ p $ maps each component $ y_i $ of the sequence $ \{y_i\} $ via the set map under the functor $ F $ of the inclusion $ U_i \intrs U_j \subset U_i$.}}
\item {$ q $ :\emph{ for a sequence $ \{f_i\} \in \prod_{i\in I}FU_i$, $ q $ maps it as
	\[q(\{f_i\}) = \{\underbrace{F(U_i \intrs U_j \subset U_j)}_{FU_j \to F(U_i\intrs U_j)} (f_j)\}\in \prod_{i,j\in I} F(U_i\intrs U_j)\]	
	That is, $ q $ maps each component $ y_i $ of the sequence $ \{y_i\} $ via the set map under the functor $ F $ of the inclusion $ U_i \intrs U_j \subset U_j$.\footnote{Refraining to write $ F(V\subset U) = FU \to FV $ to be equal to the restriction $ \rest{(-)}{V}  $ exaggerates the emphasis on the abstract nature of sheaf $ F $, that is, it helps to imagine that $ FU $ might not always be a set of specific maps over $ U $, even though in most examples of interest it is the case.}}}
	\end{itemize} 
\end{definition}
Both the Definitions \ref{D-5} \& \ref{D-6} are equivalent. In particular, the definition \ref{D-6} can be seen to respect the \emph{Locality} by reminding ourselves that an \emph{Equalizer is always a monic}. Finally, the \emph{Gluing} condition can be established by just using the \emph{Universality of the Equalizer}. The latter, in a more precise sense, is true because the gluing condition says that for any element of $ \prod_{i\in I}FU_i $, if it equalizes $ p $ and $ q $, then there exists an element of $ FU $ (which is equivalent to existence of a unique map to $ FU $) such that it's restriction (application of $ e $) is same as the original element of $ \prod_{i\in I} FU_i $. Clearly, $ FU $ is universal with this property, so it must be the equalizer.
\\
A high level definition of locality can hence be understood in the language of sheaves. Suppose we mean by an element of a property to be a realization of a given property on some subset of the underlying space. Then, a property on a space is said to be local when for any cover of the underlying space, if the element of the property for each member of the cover is such that it agrees on each member restricted to the intersection with any other member of the cover, then there must be an element of the property over whole space which is just formed by collation of these pieces over each member of the cover. This intuitive understanding can be seen to be followed by the property of being \emph{any function on space}, the property of being \emph{continuous function on a space}, the property of being \emph{differentiable function on a space} and so on and so-forth.
\subsection{Basic Properties of Sheaves}
We now quickly discuss some easy properties of sheaves. In the following, a \textbf{Subsheaf} of a sheaf $ F $ is defined as a subfunctor of $ F $ which also satisfies the sheaf property (is a sheaf itself).
\begin{proposition}
	A subfunctor $ S $ of a sheaf $ F $ is a subsheaf if and only if for any open set $ U $ and it's open covering $ \bunion_{i\in I} U_i $ together with an $ f \in FU $, we have $ f\in SU $ if and only if $ \rest{f}{U_i} \in SU_i\; \forall\;i\in I$.
\end{proposition}
\begin{proof}
	(\textbf{L $ \implies $ R}) Suppose $ S $ is a subsheaf, then clearly for any $ f\in SU \subset FU $, we must have $ \rest{f}{U_i} \in SU_i $ for all $ i\in I $ and for any such collection of $ \rest{f}{U_i} $, by the sheaf property of $ S $, $ f\in SU $.\\
	(\textbf{R $ \implies $ L}) Since $ S $ is a subfunctor of $ F $, therefore $ SV \subset FV $ for any open $ V $. With this, because $ F $ is a sheaf, we have the following diagram:
	\[\begin{tikzcd}
		SU & {\prod_iSU_i} & {\prod_{i,j} S(U_i\cap U_j)} \\
		FU & {\prod_iFU_i} & {\prod_{i,j}F(U_i\cap U_j)}
		\arrow[tail, from=1-1, to=2-1]
		\arrow[tail, from=1-2, to=2-2]
		\arrow[tail, from=1-3, to=2-3]
		\arrow[shift left=1, from=1-2, to=1-3]
		\arrow[shift right=1, from=1-2, to=1-3]
		\arrow[from=1-1, to=1-2]
		\arrow[from=2-1, to=2-2]
		\arrow[shift left=1, from=2-2, to=2-3]
		\arrow[shift right=1, from=2-2, to=2-3]
	\end{tikzcd}\]
where the bottom row is the equalizer. The condition on the right says that for $ f\in FU $, $ f\in SU \iff \{\rest{f}{U_i}\} \in \prod_{i} SU_i$, which means that the left square is a pullback. Now because $ SU $ is universal due to it being a pullback, and since the top row infact commutes, therefore $ SU $ is universal with top row commuting, hence, it is an equalizer. 
\end{proof}
\subsubsection{Sheaf itself is local}
%\begin{definition}
%	(\textbf{Direct Image Sheaf}) Suppose $ f : X\to Y $ is a continuous map and $ F $ is a sheaf over $ X $. We can construct a sheaf over $ Y $ with $ f $ and $ F $, called Direct Image of $ F $ under $ f $ as follows:
%	\begin{align*}
%		\dirim{F}{f} : \opcat{O(Y)} &\longrightarrow \cat{Sets}\\
%		V&\longmapsto F(\inv{f}(V))\\
%		(U\subset V) &\longmapsto F\left (\inv{f}(U) \subset \inv{f}(V)\right )
%	\end{align*}
%If $ f $ is a homeomorphism, then $ \dirim{F}{f} $ is an isomorphism of categories.
%\end{definition}
Define \textbf{restriction of a sheaf} $ F $ on $ X $ restricted to open $ U\subset X $ to be the $ \rest{F}{U} (V) = F(V) $ where $ V\subset U $, and $\rest{F}{U}(U) = F(\phi) = \{*\}$ if $ V\not \subset U $.
\begin{theorem}
	Suppose $ X $ is a space with a given open covering $ X = \bunion_{k\in I} W_k$. If there are sheaves for each $ k $,
	\begin{align*}
		F_k : \opcat{O(Wk)} \longrightarrow \cat{Sets}
	\end{align*}
	\footnote{where $ \opcat{O(Wk)} $ is the opposite category of all open subsets of open set $ W_k $ and inclusion.}such that 
	\begin{align*}
		\rest{F_k}{W_k \intrs W_l} = \rest{F_l}{W_k\intrs W_l} 
	\end{align*}\footnote{This condition implies that for any open subsets $ V_k \subset W_k $ and $ V_l \subset W_l $, $ F(V_k\intrs W_k\intrs W_l ) = F(V_l \intrs W_k \intrs W_l) $ and for arrows $ X_1 \subset X_2 $ in $ \cat{O(Wk)} $ \& $ Y_1 \subset Y_2 $ in $ \cat{O(Wl)} $, $ F(X_1 \intrs W_k\intrs W_l \subset X_2 \intrs W_k\intrs W_l ) = F(Y_1 \intrs W_k\intrs W_l \subset Y_2 \intrs W_k\intrs W_l) $. }
then, $ \exists $ a sheaf $ F $ on $ X $,
\begin{align*}
	F : \opcat{O(X)}\longrightarrow \cat{Sets}
\end{align*}
unique upto isomorphism such that 
\[\rest{F}{W_k} \isomorph F_k.\]
\end{theorem}
This theorem hence shows that the restriction functor $ U \mapsto \Sh{U}$ and $ V\subset U \mapsto (\Sh{U}\to \Sh{V}, \rest{F}{U} \mapsto \rest{F}{V})$ on \cat{O(X)} is local enough to be \emph{almost} a sheaf. If only for any sheaf $ F, G $ on $ X $, we had that $ \rest{F}{W_k} = \rest{G}{W_k} \forall \;k$ would imply that $ F= G $, which is not the case in general however, then we would have said that this restriction functor is also a sheaf.
\newpage
\subsubsection{Sheaf over a Basis of $ X $}
A \textbf{basis} of a space $ X $ is a subset of topology $ \basis \subset \mathcal{O}(X)$ such that for any open $ U \in \mathcal{O}(X) $, $ \exists \{B_i\} \subseteq \basis $ such that $ U = \bunion_{i} B_i $. \\
It turns out that the restriction functor $ r : \Sh{X} \longrightarrow \Sh{X_\basis} $ which restricts each sheaf over $ X $ to that of open sets of basis $ \basis $ establishes an equivalence of categories!
\begin{theorem}\footnote{Exercise $ 4 $ of the text.}
	Suppose $ X $ is a topological space and $ \basis $ is a basis for $ X $. Then, the restriction functor 
	\begin{align*}
		r : \Sh{X}&\longrightarrow \Sh{X_\basis}\\
		F &\longmapsto \rest{F}{\basis}\\
		\eta : F\Longrightarrow G &\longmapsto \rest{\eta}{\basis} : \rest{F}{\basis} \Longrightarrow \rest{G}{\basis}
	\end{align*}
establishes an equivalence of categories between $ \Sh{X} $ and $ \Sh{X_\basis} $.
\end{theorem}
\begin{proof}
	For any sheaves $ F,G$ in $ \Sh{X} $, we want to show that $ \homset{\Sh{X}}{F}{G} \isomorph \homset{\Sh{X_\basis}}{rF}{rG}$, that is, $ r $ is fully faithful. One can see that there $ r $ is an injection between the above hom-sets as for any $ \epsilon, \eta : F\Rightarrow G$, if $ rF = \rest{F}{\basis} = \rest{G}{\basis} = rG $, then due to the commutation of the two squares below because of naturality, (take $ U  = \bunion_i B_i$ to be any open set and it's trivial open covering from basic open sets)
	\[\begin{tikzcd}
		FU && {\prod_iFB_i} \\
		GU && {\prod_iGB_i}
		\arrow["{\epsilon U}"', shift right=2, from=1-1, to=2-1]
		\arrow["{\eta U}", shift left=2, from=1-1, to=2-1]
		\arrow["{e_F}", from=1-1, to=1-3]
		\arrow["{e_G}", from=2-1, to=2-3]
		\arrow["{\prod_i \epsilon B_i}"', shift right=2, from=1-3, to=2-3]
		\arrow["{\prod_i \eta B_i}", from=1-3, to=2-3]
	\end{tikzcd}\]
one can infer $ \epsilon U = \eta U $ ($ e_F $ and $ e_G $ are equalizers, so are monic).\\
With the information $ \kappa : rF \Rightarrow rG $, one can construct a natural transformation $ \gamma : F\Rightarrow G $ by defining $ FU $ and $ GU $, for any open $ U $ with it's basic cover $ U = \bunion_i B_i $ where $ B_i \in \basis $, as the equalizer of the parallel arrows $ \prod_i \rest{F}{\basis} B_i \rightrightarrows \prod_{i,j} \rest{F}{\basis} B_i \intrs B_j $ and $ \prod_i \rest{G}{\basis} B_i \rightrightarrows \prod_{i,j} \rest{G}{\basis} B_i \intrs B_j $, respectively. Then, one defines $ \gamma U : FU \to GU $ by noticing that the former forms a cone over the latter, due to arrows $ \prod_i \kappa B_i : \prod_i \rest{F}{\basis} B_i \to \prod_i \rest{G}{\basis} B_i $ and $\prod_{i,j} \kappa(B_i \intrs B_j) : \prod_{i,j} \rest{F}{\basis} B_i \intrs B_j \Rightarrow \prod_{i,j} \rest{G}{\basis} B_i \intrs B_j $, so that there exists a unique arrow $ FU \to GU $, which we just define as $ \gamma U $. \\
With this, we see that $ r $ is fully faithful. Finally, with the above definitions, $ rF \isomorph \rest{F}{\basis} $ where $ F \in \Sh{X} $ is the sheaf obtained by the above process from $ \rest{F}{\basis}\in \Sh{X_\basis} $ because both of them are equalizers of the same diagram for any open set $ U = \bunion_i B_i $ and it's basic covering (note that any covering of $ U $ can be decomposed into basic covering).
\end{proof}
\subsection{Sieves as General Covers}
Our discussion now turn back to sieves. As we saw in Definition \ref{D-3}, a subfunctor of $ \yembed{C} = \homset{}{-}{C} $ is a sieve, therefore this notion would allow us to generalize the notion of \emph{covering} of a space, as we will see later. But for now, the \emph{shadow} of that more general notion can still be felt in the usual category $ \cat{O(X)} $ of open sets of $ X $.
\begin{definition}\label{D-7}
	(\textbf{Principal Sieve})\emph{ Suppose $ X $ is a topological space and $ U $ is open. Then the sieve $ S $, \emph{generated} from $ U $, that is,
	\begin{align*}
		S = \{V \;:\; \text{ open }V\subset U\} 
	\end{align*}
    is said to be a principal sieve, denoted $ S= \gen{U} $, generated by a single open set.}
\end{definition}
With Definition \ref{D-7}, we can now define a new notion of \emph{covering} of an open set, purely in terms of arrows onto it! 
\begin{definition}\label{8}
	(\textbf{Covering Sieve})\emph{ Suppose $ X $ is a topological space and $ U $ is open in it. A sieve $ S $ on $ U $ is said to cover $ U $ if 
	\begin{align*}
		U = \bunion_{W\in S} W.
	\end{align*}
That is, when $ U $ is union of all open sets in the sieve $ S $.}
\end{definition}
\begin{remark}
	It can be seen quite easily that a subfunctor $ S $ of $ \yembed{U} $ is a principal sieve over $ U $ if and only if $ S $ is a subsheaf. L $\implies $ R by Proposition \ref{P-1} and R $ \implies $ L by noting that the union of all sets in $ S $ would generate it. Remember that you can take covers of only those open sets which are members of $ S $ because $ S $ is a subsheaf.
\end{remark}
The above definition in effect can be replaced with in the definition of sheaves!
\begin{proposition}\label{P-5}
	A presheaf $ P : \opcat{O(X)}  \longrightarrow \cat{Sets}$ on a topological space $ X $ is a sheaf if and only if for any open $ U $ and a covering sieve $ S $ over $ U $, we have that the inclusion nat. trans. $ i_S : S\Longrightarrow \yembed{U} $ induces an isomorphism:
	\[\homset{\ps{O(X)}}{S}{P} \isomorph \homset{\ps{O(X)}}{\yembed{U}}{P}.\]
\end{proposition}
\begin{proof}
	We can re-derive the sheaf condition in terms of the covering sieve as follows. For an open $ U = \bunion_{i} U_i $, if$ \{f_i\}\in \prod_i PU_i $ is such that $ \rest{f_i}{U_i \intrs U_j} = \rest{f_j}{U_i \intrs U_j} $, then because $ S $ is a covering sieve of $ U $, therefore this condition is equivalent to a sequence $ \{f_V\} \in PV $ for all $ V\in S $ such that $ \rest{f_V}{V^{\prime}} = \rest{f}{V^{\prime}}$ whenever $ V^{\prime}\subset V $. It can also be seen that every natural transformation $ \eta $ between $ S $ and $ P $ can be mapped to an element of $ \prod_{V\in S} PV$ by forming the collection $ \{\eta_V (*)\} $. Similarly, for any $ \{f_V\} \in \prod_{V\in S} PV $ we can construct a nat. trans. $ \{f_V : SV = \{*\} \to PV\} $. Now, with this, we can obtain the result by a basic diagram chase around the left square of the following
	 \[\begin{tikzcd}
	 	{\homset{\ps{O(X)}}{S}{P}} & {\prod_i PU_i} & {\prod_{i,j}P(U_i\intrs U_j)} \\
	 	{\homset{\ps{O(X)}}{\yembed{U}}{P}} & PU
	 	\arrow["d", from=1-1, to=1-2]
	 	\arrow[shift left=1, from=1-2, to=1-3]
	 	\arrow[shift right=1, from=1-2, to=1-3]
	 	\arrow[Rightarrow, no head, from=2-1, to=2-2]
	 	\arrow["{\homset{\ps{O(X)}}{i_S}{P}}", from=2-1, to=1-1]
	 	\arrow["e"', from=2-2, to=1-2]
	 \end{tikzcd}\]
 where $ d $ is the equalizer of the parallel arrows on the right (the fact that this set is the equalizer is established in the prev. paragraph)
\end{proof}
\subsection{$ \Sh{X} $ has all small limits}
We now see that $ \Sh{X} $ has all small limits and the inclusion of $ \Sh{X} $ to $ \ps{O(X)} $ preserves these limits.
\begin{proposition}\label{P-6}
	For any topological space $ X $, the category $ \Sh{X} $ has all small limits and the inclusion functor 
	\begin{align*}
		 i : \Sh{X} &\rightarrowtail \ps{O(X)}
	\end{align*}
preserves all those limits.
\end{proposition}
\begin{proof}
	To show that $ \Sh{X} $ has all small limits, we can first notice that the singleton functor is a sheaf, which is the terminal object in $ \Sh{X} $. Now, to see equalizers, take any parallel arrows in $ \Sh{X} $ as $ F \rightrightarrows G $. Since $ \ps{O(X)} $ has all small limits, therefore, we can take the equalizer of this in it, in turn of taking equalizer in $ \Sh{X} $. With this, there exists $ E $, the equalizer of $ F\rightrightarrows G $ in $ \ps{O(X)} $. Now because covariant hom-functors preserves limits, therefore for any open $ U $, the $ \homset{\ps{O(X)}}{\yembed{U}}{E} $ and $ \homset{\ps{O(X)}}{S}{E} $ acts as equalizers in the diagram below:
	\[\begin{tikzcd}
		{\homset{\ps{O(X)}}{\yembed{U}}{E}} & {\homset{\ps{O(X)}}{\yembed{U}}{F}} & {\homset{\ps{O(X)}}{\yembed{U}}{G}} \\
		{\homset{\ps{O(X)}}{S}{E}} & {\homset{\ps{O(X)}}{S}{F}} & {\homset{\ps{O(X)}}{S}{G}}
		\arrow["{e\circ -}", from=1-1, to=1-2]
		\arrow["{e\circ -}"', from=2-1, to=2-2]
		\arrow["{-\circ i_s}"', from=1-1, to=2-1]
		\arrow["{-\circ i_s}"', from=1-2, to=2-2]
		\arrow["{g\circ -}"', shift right=1, from=1-2, to=1-3]
		\arrow["{f\circ -}", shift left=1, from=1-2, to=1-3]
		\arrow["{g\circ -}"', shift right=1, from=2-2, to=2-3]
		\arrow["{f\circ -}", shift left=1, from=2-2, to=2-3]
		\arrow["{-\circ i_s}"', from=1-3, to=2-3]
		\arrow["{f_E}", from=1-1, to=2-1]
		\arrow["{f_F}", from=1-2, to=2-2]
		\arrow["{f_G}", from=1-3, to=2-3]
	\end{tikzcd}\]
Using Proposition \ref{P-5}, $ f_F $ and $ f_G $ are isomorphisms. A simple diagram chase on the left square then shows $ f_E $ is also an isomorphism. Binary products exists by the same process.
\end{proof}
The above proposition hence allows us to infer what it means to be a subobject of a sheaf in $ \Sh{X} $.
\begin{corollary}
	For any topological space $ X $, any subobject of a sheaf $ F $ in $ \Sh{X} $ is isomorphic to a subsheaf of $ F $.
\end{corollary}
\begin{proof}
	Suppose $ H \Rightarrow F $ is a monic, so a subobject of $ F $. Since $ \Sh{X} $ has all limits (Proposition \ref{P-6}), so the kernel pair of this arrow would exist in $ \Sh{X} $ and it's inclusion in $ \ps{O(X)} $ would preserve it.	By point-wise construction of presheaves in $ \ps{O(X)} $, we can see that $ H $ would be isomorphic to some some subfunctor of $ F $, which would be a sheaf too because it is isomorphic to $ H $, a sheaf.
\end{proof}
\subsubsection{Topology of X $ \isomorph $ Subobjects of $ \yembed{X} $ in $ \Sh{X} $}
Finally, we observe that the topology of $ X $ is actually isomorphic to subobjects of $ \yembed{X} $\footnote{Remember that $ \yembed{X} $ is the terminal object in $ \Sh{X} $.} in $ \Sh{X} $!
\begin{proposition}\label{P-7}
	For any topological space $ X $, there exists an isomorphism of the following posets
	\begin{align*}
		\mathcal{O}(X) \isomorph \Sub{\yembed{X}}{\Sh{X}}
	\end{align*}
which is moreover order preserving.\footnote{Remember Proposition \ref{P-1}. Therefore this isomorphism could be extended as:
\begin{align*}
	\mathcal{O}(X) \isomorph \Sub{\yembed{X}}{\Sh{X}} \isomorph \homset{\Sh{X}}{\yembed{X}}{\Omega}
\end{align*}
when $ \Omega $ exists. This is the first sign of how sheaves might be related to topoi.
}

\end{proposition}
%\begin{proof}
%	
%\end{proof}
\subsection{Bundles}
A \textbf{bundle} is just a continuous map between two topological spaces. As we will discuss in this section, every sheaf can be seen to be a sheaf of continuous right inverses (cross-sections) of a suitable bundle. The statement of this theorem would require a sizable amount of new terminology, which we introduce now.
\subsubsection{The Bundle Terminology}\label{TBT}
As mentioned earlier, a map $ p : Y\to X $ is called a bundle if $ p $ is continuous. One then defines the continuous right inverse of $ p $, denoted $ s : X\to Y $, as the \textbf{cross-section} of $ p $ if $ p\circ s = \Id{X}$. The name \emph{bundle} arises from the fact that a continuous function $ p : Y\to X $ is just a \emph{collection} of mappings of elements of $ Y $ to that of $ X $. Hence, all the elements of $ Y $ which map to some $ x\in X $ are called \textbf{fiber of $ Y $ over $ x $}. This is equivalently just $ \inv{p}x $.
\\
Now suppose $ p : Y\to X $ is a bundle and $ U\subseteq X $ is open. Then the restriction of $ p $ to $ \inv{p}(U) $ is again a bundle but on $ U $, usually called the \textbf{restriction bundle}
\begin{align*}
	p_U : \inv{p}(U)\longrightarrow U.
\end{align*}
Since the restriction $ p_U : \inv{p}(U) \longrightarrow U$ is a bundle, therefore one can construct a cross-section of it. A \textbf{cross-section of a restriction bundle} $ p_U $ is a continuous map $ s : U\longrightarrow \inv{p}(U) $ such that $ p_U \circ s = \Id{U} $, or equivalently, $ p\circ s  = \iota : U \hookrightarrow X$. 
\subsubsection{Sheaf of Cross-Sections of a Bundle}\label{SCSB}
Consider a bundle $ p : Y\to X $. Note that for any open set $ U $ of $ X $, we can have cross-sections $ s : U \to \inv{p}(U) $. Moreover, for $ V\subset U $ for any open subsets $ U,V $ of $ X $, we can take a cross-section $ s : U \to \inv{p}(U) $ of $ p_U $ and restrict it to $ V $ to get $ \rest{s}{V} : V \to \inv{p}(V) \subset \inv{p}(U) $ to get a cross section of $ p_V $. \\
As the above discussion shows, we can construct the following presheaf $ \Gamma_p $ for a bundle $ p : Y\to X $:
\begin{align*}
	\Gamma_p : \opcat{O(X)} &\longrightarrow \cat{Sets}\\
	U &\longmapsto \Gamma_pU : = \{s : U \to Y \;\vert\; p\circ s = \iota : U \hookrightarrow X\}\\
	(V\subset U) &\longmapsto (\Gamma_pU \to \Gamma_pV)\\
	& \;\;\;\;\;\;\;\;\;\;\;\;\;\;\;s\mapsto \rest{s}{V}
\end{align*}
\textbf{Question} : \emph{Is $ \Gamma_p $ a sheaf? }\\
\textbf{Yes}. Locality can be seen trivially while the gluing condition can be established by the following: Suppose $ U = \bunion_i U_i$ is an open set with a given open covering and let $ \{s_i\} \in\prod_i \Gamma U_i $ be such that $ \rest{s_i}{U_i \intrs U_j} = \rest{s_j}{U_i \intrs U_j} $. Define the map $ s : U \to Y $ by $ s(x) = s_i(x) $ where $ x \in U_i \subset U $, the existence of such an $ i $ for any $ x \in U$ is always guaranteed. We now see that this map $ s $ is continuous, because for an open subset $ T \subseteq Y $, we get $ \inv{s}(T) = \{x\in U\;\vert\; s(x) \in T\} = \bunion_i \{x\in U_i \;\vert\; s_i(x)\in T\} $ where the last equality wouldn't been true if we had $ \rest{s_i}{U_i \intrs U_j} \neq \rest{s_j}{U_i \intrs U_j} $. Therefore $ \inv{s}(T) $ is open, proving $ s $ is continuous. Clearly, $ p\circ s(x) = p(s(x)) = p(s_i(x)) = x $ for $ i $ such that $ x\in U_i \subset U$. Hence $ s\in \Gamma_pU $ with $ \rest{s}{U_i} = s_i \;\forall\;i$, making $ \Gamma_p $ a sheaf.\\
 This sheaf $ \Gamma_p $ is called the \textbf{Sheaf of Cross Sections of the Bundle $ p $.}
\\\\
With this, we have now introduced a very important sheaf, the sheaf of cross-sections of a bundle. To move towards stating the theorem mentioned in the beginning, we now need to introduce the additional terminology of germs and stalks, which we do now.
\subsubsection{Germs \& Stalks}\label{G&S}
Consider the following presheaf:
\begin{align*}
	P : \opcat{O(X)} \longrightarrow \cat{Sets}
\end{align*}
Take a point $ x\in X $ and any two open sets $ U,V \in \nbdsys{x}$ where $ \nbdsys{x} $ is a neighborhood system of $ x $\footnote{Collection of all open sets of $ X $ containing $ x $.}. Then two points $ s \in PU $ and $ t\in PV $ are said to have \textbf{same germ at $ x $}, or $ s\sim_x t $, if
\begin{align*}
	\exists \;\text{open}\;W \subseteq U\intrs V \text{ where $ W\in \nbdsys{x} $ such that } \rest{s}{W} = \rest{t}{W}\in PW. 
\end{align*} 
The relation $ \sim_x $ is an equivalence relation on $ \bunion_{U \in \nbdsys{x}} PU $. An equivalence class represented by $ s\in PU $ where $ U \in \nbdsys{x} $ is denoted as $ \germ{s}{x} $, called \textbf{germ of $ s \in PU $ at $ x \in U$}.
\\
Now, for any given $ x\in X $, we can collect all possible germs at $ x $, which we denote as the following:
\begin{align*}
	P_x = \{\germ{s}{x}\;\vert\; s\in PU \;\& \; U \in \nbdsys{x}\}.
\end{align*}
This set $ P_x $ of all possible germs at $ x $ (equivalence classes) is called the \textbf{stalk at $ x \in X$.}\\\\
Consider now the restriction of presheaf $ P $ to neighborhood system $ \nbdsys{x} $ of $ x$. Denote this restriction by $ \rest{P}{\nbdsys{x}} $. It can be seen quite easily that the stalk at $ x $, $ P_x $, is the colimit of the restriction diagram: 
\begin{align*}
	\rest{P}{\nbdsys{x}} : \bm{\nbdsys{x}}^{\text{op}}&\longrightarrow \cat{Sets}\\
	U &\longmapsto PU\\
	(V\subset U) &\longmapsto (PU \to PV)\\
		& \;\;\;\;\;\;\;\;\;\;\;\;\;\;s\mapsto \rest{s}{V}
\end{align*}
That is,
\begin{align*}
	P_x = \colim \rest{P}{\nbdsys{x}}
\end{align*}
This can be seen via the fact that if we take any other cocone of $ \rest{P}{\nbdsys{x}} $ with vertex, say $ L $, then any component of this cocone $ \alpha_U : PU \to L $ for $ U \in \nbdsys{x} $ can be trivially factored through the unique map $ P_x \to L $ which sends the $ \germ{t}{x} $ to $ \alpha_U (t) $. This just means that $ P_x $ is the colimit of the said diagram.
\\\\
In the whole discussion above, we fixed the presheaf $ P $ and \emph{calculated stalks at various $ x\in X $} by restricting $ P $ to the corresponding neighborhood systems $ \nbdsys{x} $. We can in-fact do the opposite. That is, for a fixed $ x\in X $, we can calculate stalks of various presheaves at $ x $. The former process was formalized by the presheaf $ \rest{P}{\nbdsys{x}} $. Similarly, the latter process would then be formalized by the following functor, which \emph{takes the stalk at $ x\in X $}:
\begin{align*}
	(-)_x : \psheaf{O(X)} &\longrightarrow \cat{Sets}\\
	P &\longmapsto P_x\\
	(h:P\Rightarrow Q) &\longmapsto (h_x : P_x \to Q_x)\\
		& \;\;\;\;\;\;\;\;\;\germ{s}{x}\mapsto \germ{h_U(s)}{x}\;\;\text{ where }s\in PU,\;U \in \nbdsys{x}
\end{align*}
With the above terminology, we finally turn to the \emph{bundle of interest to our discussion}.\\\\
Collect all the stalks for each $ x\in X $, for a fixed presheaf $ P $, into a disjoint union as shown below:
\begin{align*}
	\Lambda_P = \coprod_{x\in X} P_x = \{\germ{t}{x} \;\vert\;\text{ any }t\in PU \text{ any } U \in \nbdsys{x} \text{ where any }x\in X \}.
\end{align*}
That is, $ \Lambda_P $ is the \textbf{set of all germs for $ P $}. \\\\
\textbf{The Bundle of Interest}\\\\
Consider the following map which \emph{projects germs} back to their evaluation point:
\begin{align*}
	p : \Lambda_P &\longrightarrow X\\
	\germ{s}{x} &\longmapsto x.
\end{align*}
Now take any $ s\in PU $ where $ U\in \nbdsys{x} $, then we have a corresponding function to $ s $, given as follows:
\begin{align*}
	\germ{s}{(-)} : U &\longrightarrow \Lambda_P\\
	x&\longmapsto \germ{s}{x}
\end{align*}
This maps \emph{evaluates germs} of $ s $ at various points of $ U $.\\\\
This \emph{projection-evaluation} seems like a bundle \& cross-section pair, but we cannot really say that right away because we don't have a topology on $ \Lambda_p $. Hence, we now topologize $ \Lambda_P $!\\
Endow $ \Lambda_P $ with the following base of a topology:
\begin{align*}
	\Lambda_{P_{\basis}} = \left \{\germ{s}{(U)} = \{\germ{s}{x}\;\vert\;x\in U\} \;\vert\; \text{ any }s\in PU\text{ ,  any open }U\subseteq X\right \}
\end{align*}
so that any open subset $T\subseteq \Lambda_P $ would be written as:
\begin{align*}
	T = \bunion_{s} \germ{s}{(U_s)} \;\;\;\text{ where }s\in PU_s  \text{ and $ U_s \subseteq X$ is open.}
\end{align*}
With this topology on $ \Lambda_P $, we now observe the following:
\begin{enumerate}
	\item {\textbf{$ p : \Lambda_P \longrightarrow X $ is a continuous map} : Take any open set $ U\subseteq X $. Since 
\begin{align*}
	\inv{p}(U) &= \{\germ{t}{x}\;\vert\;\text{ any }t\in PU \text{ where any }x\in U \}\\
	&= \bunion_{t} \germ{t}{(U)}
\end{align*}
so $ \inv{p}(U) $ is open.
\\
Moreover, this map is a \textbf{local homeomorphism} in the sense that for any point $ \germ{s}{x} \in \Lambda_p$ where $x\in U,\; s\in PU $, we have an open neighborhood around it given by $ \germ{s}{(U)} $, such that $ \rest{p}{\germ{s}{(U)}} : \germ{s}{(U)} \longrightarrow U \subseteq X$ is a homeomorphism where it's inverse is $ \germ{s}{(-)} $.
 }
\item {$ \germ{s}{(-)} : U \longrightarrow \Lambda_P$ \textbf{is a continuous map} : Take any open $ T = \bunion_{t} \germ{t}{(U_t)}\subseteq \Lambda_P $. Since
\begin{align*}
	\inv{(\germ{s}{(-)})}(T) &= \left \{ x\in U \;\vert\; \germ{s}{x} \in T\right \}
\end{align*}
Now for $ s\in PU $ and $ t\in PV $, if we have $ \germ{s}{x} = \germ{t}{x} $ for $ x\in U\intrs V $, then $ s $ and $ t $ have same germ. This means that $ \exists W \subseteq U\intrs V$ open containing $ x $ such that $ \rest{s}{W} = \rest{t}{W} \implies \germ{\rest{s}{W}}{(-)} = \germ{\rest{t}{W}}{(-)}$. The above inverse image is just union of these open $ W $, one for each $ \germ{s}{x} \in T$ , therefore the inverse image is open, making $ \germ{s}{(-)} $ a continuous map. 
}
\end{enumerate}
We finally have that \emph{\textbf{$ p : \Lambda_P \longrightarrow X $ is a bundle}} and $ \germ{s}{(-)} : U \longrightarrow \Lambda_P$\textbf{\emph{ is a cross-section of $ p $}} and this cross section is available for each $ s\in PU $ for any open $ U \subseteq X $.
\\\\
With the above, we now turn to the actual transformation between a presheaf $ P : \opcat{O(X)} \longrightarrow \cat{Sets}$ to the sheaf of cross-sections of the bundle $ p : \Lambda_P \longrightarrow X$.
\subsubsection{From a Presheaf to the Sheaf of Cross-Sections}
Suppose that 
\begin{align*}
	P : \opcat{O(X)} \longrightarrow \cat{Sets}
\end{align*}
 is a presheaf over a given topological space $ X $. As discussed in subsection \ref{G&S}, we have a bundle:
\begin{align*}
	p : \Lambda_P &\longrightarrow X\\
	\germ{t}{x} &\longmapsto x.
\end{align*}
By the discussion in subsection \ref{SCSB}, we also have a corresponding sheaf of cross-sections:
\begin{equation}\label{E1}
	\begin{aligned}
			\Gamma \Lambda_P :\opcat{O(X)} &\longrightarrow \cat{Sets}\\
		U &\longmapsto \Gamma\Lambda_PU := \{c : U \to \Lambda_P \;\vert\; p\circ c : U \hookrightarrow X\;,\;\;\text{ $ c $ is continuous}\}\\
		(V\subset U) &\longmapsto (\Gamma \Lambda_P U \to \Gamma\Lambda_P V)\\
		& \;\;\;\;\;\;\;\;\;\;\;\;\;\;\;\;\;\;c \mapsto \rest{c}{V}
	\end{aligned}
\end{equation}
Now, consider the following map between the functors $ P $ and $ \Gamma \Lambda_P $ as:
\begin{align*}
	\eta : P \Longrightarrow \Gamma \Lambda_P 
\end{align*}
such that for any object $ U  $ of $ \opcat{O(X)} $, we have 
\begin{align*}
	\eta_U: PU &\longrightarrow \Gamma \Lambda_P U\\
	s&\longmapsto \germ{s}{(-)}
\end{align*}
\textbf{Question}. \emph{Is $ \eta $ a Natural Transformation?}\\
\textbf{Yes}. This can be seen by noticing that $ \germ{s}{x} = \germ{\rest{s}{V}}{x} $ where $ s\in PU $, $ x\in U  $ \& $ x\in V\subseteq U $. This observation makes the natural square commute.
\\\\
We have hence established a way to convert any presheaf into a sheaf of cross-sections ($ \germ{s}{(-)} $) of a bundle ($ p $). Finally, we can state the result for which the subsections \ref{TBT}, \ref{SCSB} \& \ref{G&S} were devoted to.
\begin{theorem}\label{T-4}
	Suppose $ X $ is a topological space and $ P : \opcat{O(X)} \longrightarrow \cat{Sets} $ is a presheaf on it. Then,
	\begin{align*}
		P\text{ is a sheaf }\bm{\implies} \; \eta: P \Longrightarrow \Gamma \Lambda_P \text{ is a natural isomorphism}
	\end{align*}
where $ \Gamma\Lambda_P $ is as given in \eqref{E1}.
\end{theorem}
\begin{proof}
	Suppose $ P $ is a sheaf. To show that $ \eta  $ is a natural isomorphism is same as showing that any $ \eta_U : PU \to \Gamma \Lambda_P U, t \mapsto  \germ{t}{(-)}$ is a bijection. This map $ \eta_U $ is trivially an injection because if $ \germ{t}{(-)} = \germ{s}{(-)}$, then $ \germ{t}{x} = \germ{s}{x} \forall x\in U$, then $ \exists \text{ open }W_x \subseteq U\intrs U $ such that $ \rest{s}{W_x} = \rest{t}{W_x} $. Clearly, $ \bunion_{x\in U} W_x = U $. Therefore by locality of sheaf $ P $, $ s= t $.\\
	For surjectivity, take any cross-section $ h : U \to \Lambda_P $. Since $ p\circ h = \iota : U \hookrightarrow X $, therefore $ h(x) = \germ{t_x}{x} $ where $ x\in U_x $ \& $ t_x \in PU_x $. Correspondingly, due to continuity of $ h $ we have an open neighborhood of $ h(x) $, $ \germ{t_x}{(U_x)} $, and therefore, an open neighborhood of $ x $, namely $V_x : =  \inv{h}\left (\germ{t_x}{(U_x)}\right ) $. Clearly, $ x\in V_x $, $ V_x \subseteq U_x $ and $ \rest{h}{V_x} = \germ{t_x}{(-)} $. Now, because $ U = \bunion_{x\in U} V_x $ and $ \rest{h}{V_x \intrs V_y} = \germ{t_x}{(-)} = \germ{t_y}{(-)} $ on $ V_x\intrs V_y $, therefore by injectivity above, we get $ t_x = t_y $ on $ V_x \intrs V_y $. Therefore if we denote $ \bar{t}_x = \rest{t_x}{V_x} $, then $ \rest{\bar{t}_x}{V_x \intrs V_y} = \rest{\bar{t}_y}{V_x \intrs V_y}$. By gluing condition, $ \exists \;s\in PU $ such that $ \rest{s}{V_x} = \bar{t}_x $. Therefore $ h(x) = \germ{t_x}{x} = \germ{\rest{t_x}{V_x}}{x} $, because of property of same germ. Moreover, as shown just now $ \germ{\rest{t_x}{V_x}}{x} = \germ{\bar{t}_x}{x} = \germ{\rest{s}{V_x}}{x} = \germ{s}{x} $. Hence, $ \eta_U : PU \to \Gamma\Lambda_P $ is also a surjection.\\
	Combining above two gives us that $ \eta: P \Longleftarrow \Gamma\Lambda_P $ is a Natural Isomorphism.
\end{proof}
The morphism $ \eta  $ in $ \ps{O(X)} $ is also universal for sheaves, as explained below.
\begin{theorem}\label{T-5}
	For any presheaf $ P $ over $ \cat{O(X)} $ where $ X $ is a topological space and for any sheaf $ F $ over $ \cat{O(X)} $ with any given morphism 
	\begin{align*}
		\theta: P \Longrightarrow F
	\end{align*}
in $ \ps{O(X)} $, $ \exists  $ a unique morphism $ \sigma: \Gamma\Lambda_P \Longrightarrow F $, such that the following commutes:
\[\begin{tikzcd}
	P & {\Gamma\Lambda_P} \\
	& F
	\arrow["\theta"', Rightarrow, from=1-1, to=2-2]
	\arrow["\sigma", Rightarrow, dashed, from=1-2, to=2-2]
	\arrow["\eta", Rightarrow, from=1-1, to=1-2]
\end{tikzcd}\]
 \end{theorem}
\begin{proof}
	The map $ \sigma $ defined by $ \sigma_U = \inv{\left (\eta^{F}_U\right )} \circ \alpha_U $ does the job, where the other arrows are defined in the following commutative diagram and below, for $ V\subseteq U $:
	\[\begin{tikzcd}
		PU && {} && {\Gamma\Lambda_PU} \\
		& FU && {\Gamma\Lambda_FU} \\
		& FV && {\Gamma\Lambda_FV} & {} \\
		PV &&&& {\Gamma\Lambda_PV}
		\arrow["{\eta^F_U}", Rightarrow, no head, from=2-2, to=2-4]
		\arrow["{\eta^F_V}", Rightarrow, no head, from=3-2, to=3-4]
		\arrow["{\theta_U}", from=1-1, to=2-2]
		\arrow["{\theta_V}", from=4-1, to=3-2]
		\arrow["{\alpha_U}"', from=1-5, to=2-4]
		\arrow["{\alpha_V}"', from=4-5, to=3-4]
		\arrow["{P(V\subset U)}"', from=1-1, to=4-1]
		\arrow["{\Gamma\Lambda_P(V\subset U)}", from=1-5, to=4-5]
		\arrow["{\eta_U}", from=1-1, to=1-5]
		\arrow["{\eta_V}"', from=4-1, to=4-5]
		\arrow["{F(V\subset U)}"{description}, from=2-2, to=3-2]
		\arrow["{\Gamma\Lambda_F(V\subset U)}"{description}, from=2-4, to=3-4]
	\end{tikzcd}\]
	where $ \alpha_U $ is defined as:
	\begin{align*}
		\alpha_U: \Gamma\Lambda_P U &\longrightarrow \Gamma \Lambda_F U\\
		c: U \to \Lambda_P, c(x) = \germ{t_x}{x} &\longmapsto \theta_Uc : U \to \Lambda_F, \theta_Uc(x) = \germ{\theta_U(t_x)}{x}
	\end{align*}
and $ \eta^{F}_U $ is isomorphism from Theorem \ref{T-4}.
\end{proof}
\subsubsection{The Sheafification Functor for $ \cat{O(X)} $} \label{TSF}
Theorem \ref{T-5} is very interesting in the sense that there is a left adjoint hiding in it! Moreover, this is actually the left adjoint of the inclusion functor for $ \Sh{X} $ into $ \ps{O(X)} $, therefore making $ \Sh{X} $ a reflective subcategory of $ \ps{O(X)} $. More precisely:
\begin{theorem}\label{T-6}
	For any topological space $X $, category $ \Sh{X} $ is a reflective subcategory of $ \ps{O(X)} $. That is, $ \exists \;\;L : \ps{O(X)} \longrightarrow \Sh{X}$ for which the following are adjoints:
\[\begin{tikzcd}
	{\ps{O(X)}} && {\Sh{X}}
	\arrow[""{name=0, anchor=center, inner sep=0}, "L", curve={height=-12pt}, from=1-1, to=1-3]
	\arrow[""{name=1, anchor=center, inner sep=0}, "i", curve={height=-12pt},hook', from=1-3, to=1-1]
	\arrow["\dashv"{anchor=center, rotate=-90}, draw=none, from=0, to=1]
\end{tikzcd}\]
where $ i $ is the inclusion of $ \Sh{X} $ into presheaf category $ \ps{O(X)} $.
\end{theorem}
\begin{proof}
	The Theorem \ref{T-5} shows that for any presheaf $ P $ in $ \ps{O(X)} $ and for a sheaf $ F $ in $ \Sh{X} $ and for an arrow $ \theta: P \Longrightarrow F $ in $ \ps{O(X)} $, we have the following commutative diagram:
	\[\begin{tikzcd}
		P \\
		{i\Gamma\Lambda_P} & iF
		\arrow["\sigma"', Rightarrow, dashed, from=2-1, to=2-2]
		\arrow["\theta", Rightarrow, from=1-1, to=2-2]
		\arrow["\eta"', Rightarrow, from=1-1, to=2-1]
	\end{tikzcd}\]
which implies that $ \Gamma\Lambda_{(-)} : \ps{O(X)} \longrightarrow \Sh{X}$ is the left adjoint of $ i $.
\end{proof}
Hence, the above functor turns a presheaf to the "best approximation" of it to a sheaf\footnote{The "best approximation" because it is the adjoint of inclusion, therefore if the best way to get a presheaf from sheaf is inclusion, then it's adjoint ought to be the best way to get a sheaf from presheaf. Nonetheless, the functor $ \Gamma\Lambda_{(-)} $ is clearly not as simple as inclusion is.}:
\begin{definition}
	(\textbf{Sheafification Functor}) \emph{Suppose $ X $ is any topological space. The functor }
	\begin{align*}
		\Gamma\Lambda_{(-)} : \ps{O(X)} &\longrightarrow \Sh{X}\\
		P &\longmapsto \Gamma \Lambda_P \\
		P\Rightarrow Q & \longmapsto \Gamma \Lambda_P \Rightarrow \Gamma \Lambda_Q
	\end{align*}
	\emph{which assigns to each presheaf $ P $ the sheaf of cross-sections for the bundle $ p : \Lambda_P \to X $ is called the sheafification functor or the associated sheaf functor.}
\end{definition}
%\subsection{(\emph{\textbf{In Progress}}) \'Etale Spaces and Sheaves}
From the Section \ref{G&S}, we saw that for each topological space $ X $ and a presheaf $ P $ on it, there exists a bundle $ p : \Lambda_P \longrightarrow X$ where $ \Lambda_P $ is a topological space consisting of all germs at all points of $ X $ endowed with a topology which makes the bundle $ p $ a local homeomorphism. We then saw that the corresponding presheaf of cross-sections attached to this bundle $ p $ is exactly the sheaf needed to make $ \Sh{X} $ reflective into $ \ps{O(X)} $. We will now see that one can generalize this idea of a space (here $ \Lambda_P $) being locally homeomorphic to $ X $ to any bundle on $ X $.
\begin{definition}
	(\textbf{\'Etale Bundle}) \emph{Suppose $ X $ is a topological space and $ p :E\longrightarrow X $ is any bundle. The bundle $ p $ is called to be \'etale or \'etale over $ X $ if
	\begin{align*}
		\forall\;e\in E\;,\;\;\exists \;\text{ open } V \ni e \text{ such that }pV\text{ is open in $ X $}\;\&\; \rest{p}{V} : V\longrightarrow pV \text{ is a homeomorphism.}
	\end{align*}
Or, equivalently, that $ \rest{p}{V} $ is an open embedding into $ X $.}
\end{definition}
\begin{remark}
	Clearly, any covering space $p: \tilde{X} \longrightarrow X $ is \'etale over $ X $ as by definition, for any $ e\in \tilde{X} $, since it exists in some $ \inv{p}(U) $ of some open $ U\subseteq  X $ such that $ e\in \inv{p}(U) $ and clearly $ \rest{p}{\inv{p}(U)} : \inv{p}(U) \longrightarrow p\left (\inv{p}(U)\right ) = U $ is a homeomorphism. But not all \'etale spaces are covering spaces. \\
	Also note that a \textbf{section of an \'etale bundle} has the same meaning as discussed earlier. That is, $ s : U\longrightarrow E $ is a section of $ p : E\longrightarrow  X $ which is \'etale where $ U\subseteq X $ if $ p\circ s = \iota: U\hookrightarrow X $.
\end{remark}
\begin{remark}
	One can collect all the bundles over $ X $ as the slice category over $ X $ to define the category $ \Bund{X} = \cat{Top}/X $. Moreover, we can then define $ \Etal{X} $ to be the full subcategory of $ \Bund{X} $. 
\end{remark}
\newpage
\section{Grothendieck Topologies \& Sheaves}
Suppose we wish to generalize sheaves on to arbitrary categories. The first problem that one might face to this goal would be that defining a sheaf requires a notion of \emph{cover} of an open subset of the given topological space. Therefore sheaves on arbitrary categories would require a notion of \emph{cover} of each object in that category. This is precisely the problem handled (among many others) by Grothendieck topologies and the notion of a site. Reconciling such topologies on any categories would lead us to a vast generalization of the notion of a \emph{cover} of a given space, but now arbitrary objects, which has very deep connotations within algebraic geometry, which we would discuss after defining them.
\subsection{Grothendieck Topologies}
\begin{definition}
	(\textbf{Grothendieck Topology \& Site}) \emph{Suppose }$ \cat{C} $\emph{ is a small category. A \textbf{Grothendieck topology} on }$ \cat{C} $ \emph{is a functor:}
	\begin{align*}
		J(-) : \cat{\obj{C}} &\longrightarrow \cat{Sets}\\
		C &\longmapsto JC : = \{S \;\vert\; S\text{ is a sieve over }C\}
	\end{align*}
\emph{where} $ \cat{\obj{C}} $ \emph{is the discrete category of objects of} $ \cat{C} $\emph{, such that for any object $ C $, the collection $ JC $ of sieves over $ C $ in category $ \cat{C} $ must satisfy:}
\begin{enumerate}
	\item [\textbf{GT.1}]{\emph{(\textbf{Maximal Cover}) The maximal sieve over $ C $, $ S^{max}_C $, must be in $ JC $.}}
	\item[\textbf{GT.2}]{\emph{\textbf{(Stability of Covers)} For any $ S\in JC $ and $ f : D\to C $ in $ \cat{C} $, we must have that:}
\begin{align*}
	f^{*}(S) := \underbrace{\left \{g \;\vert\; f\circ g \in S\right \}}_{\text{\emph{a sieve over $ D $}}} \in JD
\end{align*}	
} 
\item [\textbf{GT.3}]{\emph{(\textbf{Transitivity of Covers}) If $ S\in JC $ and $ R $ is any sieve over $ C $ such that }
\[\forall\; f \in S\;,\;\;f^{*}(R)\in J(\text{dom}(f))\]
\emph{then we must have that}
\begin{align*}
	R\in JC.
\end{align*}
}
\end{enumerate}
\emph{A \textbf{site} $ (\cat{C},J) $ is just a small category $ \cat{C} $ coupled with a Grothendieck topology $ J $ on $ \cat{C} $}.
\end{definition}
\begin{lemma}\label{L-1}
	Suppose $ (\cat{C},J) $ is a site. Then for any object $ C $,
	\begin{align*}
		R,S \in JC \;\bm{\implies}\; R\intrs S \in JC 
	\end{align*}
\end{lemma}
\begin{proof}
	Take any $ f: D\to C $ and note that $ f^{*}(R\intrs S) = f^{*}(S) $. Then use GT.3 on $ R\intrs S $, in which GT.2 would be used.
\end{proof}
\begin{example}
	Suppose $ X $ is a topological space. Then a functor $ J : \cat{\obj{O(X)}} \longrightarrow \cat{Sets}$ where $ \cat{\obj{O(X)}} $ is the discrete category of open sets of $ X $ defined by:
	\begin{align*}
		S\in JU \;\bm{\iff} \; U \subseteq  \bunion_{V\in S} V
	\end{align*}
 forms a Grothendieck topology over $ \cat{O(X)} $.
\end{example}
\begin{proof}
	GT.1 : The set of all arrows over $ U $ in $ \cat{O(X)} $, when we take all their union, would equal $ U $. So maximal sieve over $ U $ is in $ JU $.\\
	GT.2 : Let $ S \in JU $ and $ V\subset U $. Then $ V^{*}(S) = \{W \in S\;\vert\; W \subset V \subset U\} $ is such that $ \bunion_{W\in V^{*}(S)} W \supseteq V $.\\
	GT.3 : Let $ S\in JU $ and $ R $ be any sieve over $ U $ such that for any $ V\subset U$ with $ V\in S $, the $ V^{*}(R) \in JV $. Since $ \bunion_{W\in S} W \supseteq U $ and $ V^{*}(R) = \{T \;\vert\; T\subset V\subset U\;,\;\;T\in R\} \in JV \implies \bunion_{T\in V^{*}(R)} T \supseteq V $ and $ V^{*}(R) \subset R\;\forall\;V\in S$ as sieve, therefore $ \bunion_{Q \in R} Q \supseteq \bunion_{W\in S} W\supseteq U $. Hence $ R\in JU $.
\end{proof}
\newpage 
\subsection{Basis for a Grothendieck Topology}
Note that in $ \cat{O(X)} $, the usual notion of an open cover of $ U $ as $ U = \bunion_{i\in I} U_i $ does not makes $ \{U_i\}_{i\in I} $ a sieve over $ U $. But one could generate a sieve from this open cover, by collection all $ V\subseteq U $ with $ V\subseteq U_i $ for some $ i\in I $. Therefore the collection $ \{U_i\}_{i\in I} $ forms a \emph{base cover} for $ U $ from which we can generate a usual cover over $ U $. We now extend this to any category with pullbacks\footnote{because the pullback in $ \cat{O(X)} $ of $ U \hookrightarrow X $ and $ V\hookrightarrow X $ is just the intersection of $ U $ and $ V $.}.
\begin{definition}\label{D-11}
	(\textbf{Basis for a Grothendieck Topology}) \emph{Suppose} $ \cat{C}$ \emph{is a small category. A basis for Grothendieck topology is a functor}
	\begin{align*}
		K : \cat{\obj{C}} \longrightarrow \cat{Sets}
 	\end{align*}
 \emph{where} $ \cat{\obj{C}} $ \emph{is the discrete category of objects of} $ \cat{C} $\emph{, such that for any object $ C $ of} $ \cat{C} $\emph{, the collection $ KC $ of sets of arrows over $ C $ must satisfy:
 \begin{enumerate}
 	\item [\textbf{BGT.1}] {(\textbf{Isomorphic Objects are Bases}) If $ f : C^{\prime} \to C $ is an isomorphism, then 
 		\begin{align*}
 			\{f\} \in KC 
 		\end{align*}
 		}
 	\item [\textbf{BGT.2}] {(\textbf{Pullback Stability of Bases}) If $ \{f_i : C_i \to C\;\vert\; i\in I\} \in KC$ and $ g : D\to C $ is any arrow, then
 		\begin{align*}
 			\{\pi_i : C_i \times_C D \to D\; \vert\; i\in I\} \in KD
 		\end{align*}
 }
\item [\textbf{BGT.3}]{\textbf{(Transitivity of Bases)} If $ \{f_i : C_i \to C \;\vert\; i\in I\} \in KC$ and $ \{f_i^j : C_i^{j} \to C_i \;\vert\; j\in I_i\} \in KC_i$, then
\begin{align*}
	\{f_i \circ f_i^{j} : C_i^{j} \to C \;\vert\; j\in I_i\;,\;\;i\in I\} \in KC
\end{align*}
}
 \end{enumerate}}
\end{definition}
One uses the same terminology of sites, even when we just have a base. That is, we will call $ (\cat{C},K) $ a site, even when $ K $ is a base, not a Grothendieck topology. One also generates a Grothendieck topology $ J $ from a base $ K $ as follows:
\begin{lemma}\label{L-2}
	Suppose $ \cat{C} $ is small and $ K $ is a base defined over it. Consider the topology $ J $ generated by the base $ K $ as follows:
	\begin{align*}
		S\in JC \;\bm{\iff}\; \exists \;R\in KC \text{ such that } R\subseteq S.
	\end{align*}
Then $ J $ is indeed a Grothendieck Topology.
\end{lemma}
\begin{proof}
	GT.1 : Any $ R\in KC $ is subset of the maximal sieve over $ C $. So maximal sieve is in $ JC $.\\
	GT.2 : Suppose $ S = \{f_i : C_i\to C\;\vert\;i\in I\}\in JC $ and $ g :D\to C $. We want to show $ g^{*}(S) \in JD $. By BGT.2, we have $ \{\pi_i : C_i \times_C D \to D \;\vert\; i\in I\}\in KD $. Remember that $ g\circ \pi_i = f_i \circ  \sigma_i $ where $ \sigma_i : C_i \times_C D \to C_i $ because of pullback square. But $ f_i \in S $, therefore $ f_i\circ \sigma_i = g\circ \pi_i\in S $. Hence, $ \pi_i \in g^{*}(S) $, which shows that $ \{\pi_i : C_i \times_C D \to D \;\vert\; i\in I\} \subseteq S $ so that $ g^{*}(S) \in JD $.\\
	GT.3 : The canonical idea works.
\end{proof}
\begin{remark}
	Conversely to Lemma \ref{L-2}, given a site $ (\cat{C},J) $, we can construct a basis $ K $ of $ J $ as follows:
	\begin{align*}
		R\in KC \;\bm{\iff}\; (R) \in JC
	\end{align*}
where $ (R) $ is the sieve generated by the basic cover $ R $. More precisely:
\begin{align*}
	(R) = \{f\circ g\;\vert\; f\in R\;,\;\;g\text{ is any composable arrow}\}.
\end{align*}
\end{remark}
Similar to Lemma \ref{L-1}, we have the following for basic covers:
\begin{lemma}
	Suppose we have a site $ (\cat{C},K) $. For any two basic covers $ R,P \in KC$ of $ C $, there exists a common \emph{refinement} of $ R $ and $ P $ in $ KC $.
\end{lemma}
\begin{proof}
	A collection $ \{f_i : C_i \to C\;\vert\;i\in I\} $ is said to be a \emph{refinement} of $ \{g_j : D_j \to C\;\vert\; j\in J\} $ if every $ f_i $ factors through some $ g_j $. Suppose $ J  $ is the Grothendieck topology generated from $ K $. Therefore $ (R),(P) \in JC $. By Lemma \ref{L-1}, $ (R)\intrs (P) \in JC $, so that $ \exists \;Q\in KC $ such that $ Q \subseteq (R) \intrs (P) $. Moreover, any arrow $ f\in Q $ is such that $ f\in (R) $ and $ f\in (P) $. This means that $ f $ factors through some arrow in $ R $ and $ P $, so that $ Q $ is a common refinement of $ R $ and $ P $.
\end{proof}
%\subsection{\textbf{TODO} : Some examples of Grothendieck Topologies}
%\emph{\textbf{TODO} : Write the open cover topology on $ \cat{Top} $ and complete Heyting algebra $ \cat{cHA} $.}
\subsection{Sheaves on a Site}
With the notion of Grothendieck topologies over a small category in place, we have now a notion of what it means to \emph{cover} an object in a small category. Therefore, the next step would be now to generalize the notion of sheaves over sites.
\subsubsection{Matching Families and Amalgamations}
In ordinary definition of sheaves over topological spaces (Definition \ref{D-5}), the gluing condition requires the elements of the members of the cover of some open set which match over all restrictions of their corresponding members to their intersections, to be collatable to form an element of the whole open set under the sheaf functor. The analogue of the elements of members of the cover which satisfy the above property in sheaves over a site is known as a \emph{matching family}:
\begin{definition}
	(\textbf{Matching Family of a Cover for a Presheaf}) \emph{Suppose} $ (\cat{C}, J) $ \emph{is a site. Let $ S\in JC $ be an arbitrary cover of object $ C $ and $ P $ be a presheaf over} $ \cat{C} $. \emph{Then a matching family of $ S $ for the presheaf $ P $ is a set $ M^{P}_S $ given by:}
	\emph{\begin{align*}
		M^{P}_S = \left \{ x_f \in P(\dom{f}) \;\vert\; f\in S \text{ and }\forall \;g\text{ post-composable with }f\;,\;\;P(g)(x_f) = x_{f\circ g} \right \}
	\end{align*}}
\emph{Note that $ M^{P}_S $ may not be the only matching family for the cover $ S $.}
\end{definition}
The notion of a global collation is then captured by the following:
\begin{definition}\label{D-13}
	(\textbf{Amalgamation of a Matching Family}) \emph{Suppose} $ (\cat{C},J) $ \emph{is a site, $ S\in JC $ is a cover of $ C $, $ P $ is a presheaf over} $ \cat{C} $\emph{ and $ M^{P}_S $ is a matching family of $ S $. An amalgamation of $ M^{P}_S $ is then the following element of $ PC $:
	\begin{align*}
		x\in PC\text{ such that } Pf(x) = x_f\;\forall\;f\in S\;,\;\;\text{ where }x_f\in M^{P}_S. 
	\end{align*}}
\end{definition}
Finally, a sheaf over a site is then defined as the following:
\begin{definition}\label{D-14}
	(\textbf{Sheaves over a Site - Ordinary Defn.}) \emph{Suppose} $ (\cat{C},J) $ \emph{is a site. A presheaf} $ P : \opcat{C} \longrightarrow \cat{Sets}$ \emph{is a sheaf if and only if:}
	\begin{align*}
		\text{\emph{any matching family $ M^{P}_S $ for any cover $ S\in JC $ for any object} $ C\in \obj{\cat{C}} $ \emph{has a \textbf{unique amalgamation}.}}
	\end{align*}
\end{definition}
In-fact, the above definition can be written purely in categorical language as follows:
\begin{definition}\label{D-15}
	(\textbf{Sheaves over a Site - Categorical Defn.}) \emph{Suppose} $ (\cat{C},J) $ \emph{is a site. A presheaf} $ P : \opcat{C} \longrightarrow \cat{Sets}$ \emph{is a sheaf if and only if }$ \forall C\in \obj{\cat{C}} $ \emph{and} $ \forall S\in JC $\emph{, the following is an equalizer diagram }
	\begin{equation*}
		\begin{tikzcd}
			PC \\
			& {\prod_{f\in S} P(\dom{f})} && {\prod_{f\in S,g,\;\dom{f} = \cod{g}} P(\dom{g})}
			\arrow["e", from=1-1, to=2-2]
			\arrow["p", shift left=3, from=2-2, to=2-4]
			\arrow["a"', shift right=3, from=2-2, to=2-4]
		\end{tikzcd}
	\end{equation*}
\emph{where $ e $, $ p $ and $ a $ are:}
\begin{align*}
	e : PC &\longrightarrow \prod_{f\in S} P(\dom{f})\\
	x&\longmapsto \{Pf(x)\}_{f\in S}
\end{align*}
\begin{align*}
	a :\prod_{f\in S} P(\dom{f}) &\longrightarrow \prod_{f\in S,g,\;\dom{f} = \cod{g}} P(\dom{g})\\
	\{x_f\}_{f\in S}&\longmapsto \{Pg(x_f)\}_{\dom{f} = \cod{g}}
\end{align*}
\begin{align*}
	p :\prod_{f\in S} P(\dom{f}) &\longrightarrow \prod_{f\in S,g,\;\dom{f} = \cod{g}} P(\dom{g})\\
	\{x_f\}_{f\in S}&\longmapsto \{x_{f\circ g}\}_{\dom{f} = \cod{g}}
\end{align*}
\emph{Note that $ a $ is just mapping the $ x_f $ to the corresponding member $ x_{f\circ g} \in \{x_f\}_{f\in S}$ because $ f\circ g \in S$.}
\end{definition}
\subsubsection{Sheaves in terms of Basis}
A sheaf over a site $ (\cat{C}, K) $ where $ K $ is a basis (Definition \ref{D-11}) can also be realized, albeit we require different notion of matching families for a basic cover:
\begin{definition}
	(\textbf{Matching Family of a Basic Cover for a Presheaf})\emph{ Suppose }$ (\cat{C},K) $\emph{ is a site where $ K $ is a basis, $ R\in KC $ is a basic cover of $ C $ and $ P $ is a presheaf over} $ \cat{C} $. \emph{A matching family $ M^{P}_R $ of the basic cover $ R $ for the presheaf $ P $ is then defined as the following set:}
	\begin{align*}
		M^{P}_R &= \left \{ x_{f_i}\in PC_i \;\vert\; f_i : C_i \to C\in R\text{ such that }P(\pi_{ij}^{1})(x_{f_i}) = P(\pi_{ij}^{2})(x_{f_j})\right \}
	\end{align*}
\emph{where $ \pi_{ij}^{1} $ and $ \pi_{ij}^{2} $ are as follows:}
\[\begin{tikzcd}
	{C_i\times_CC_j} & {C_j} \\
	{C_i} & C
	\arrow["{f_i}"', from=2-1, to=2-2]
	\arrow["{f_j}", from=1-2, to=2-2]
	\arrow["{\pi^{1}_{ij}}"', from=1-1, to=2-1]
	\arrow["{\pi^2_{ij}}", from=1-1, to=1-2]
	\arrow["\lrcorner"{anchor=center, pos=0.125}, draw=none, from=1-1, to=2-2]
\end{tikzcd}\] 
\end{definition}
\begin{remark}
	An \textbf{amalgamation} of a matching family of a basic cover for a presheaf is then defined as done previously (Definition \ref{D-13}), i.e. $ x \in PC$ is an amalgamation of $ M^{P}_R $ if $ Pf(x) = x_f \in M^{P}_R \;\forall\; f\in R$.
\end{remark}
A sheaf is over a basic site is then obtained as follows (note the similarity of the result with that of Definition \ref{D-14}) : 
\begin{theorem}
	Suppose $ (\cat{C},K) $ is a site and $ K $ is a basis and $ P $ is a presheaf over $ \cat{C} $. Then, $ P $ is a sheaf if and only if any basic cover $ \{f_i :C_i\to C \;\vert\; i \in I\} \in KC $ for any object $ C $ has a unique amalgamation.
 \end{theorem}
\begin{proof} \emph{Proof is a bit long so we only provide a very brief sketch.} \\
	(L $ \implies $ R) Take the topology $ J $ generated from $ K$. $ P $ is a sheaf so $ (R) \in JC$ has an unique amalgamation. Canonical observations lead to the realization that this unique amalgamation of $ (R) $ is an amalgamation for $ R $ too.\\
	(R $ \implies  $ L) Again take the $ J $ as above. Take any matching family of a cover $ S $ from $ JC $. Since $ \exists\; R\subseteq S $ where $ R\in KC $ and premise of the question says that this $ R $ has a unique amalgamation, therefore argue that this unique amalgamation of $ R $ is a unique amalgamation for $ S $ too.
\end{proof}
An example of sheaves on a site $ (\cat{Top},K) $ where $ K $ is the open cover topology is the usual contravariant hom-functor:
\begin{align*}
	\yembed{Y} = \homset{\cat{C}}{-}{Y}
\end{align*}
This is quite trivial to see by a simple un-ravelling of definitions.
\subsection{The Grothendieck Topos}
We are now at a good footing to understand one of the central themes of this text, the Grothendieck topos. Denote the category of sheaves over a site $ (\cat{C},J) $ and natural transformations between them as the following:
\begin{align*}
	\Sh{\cat{C},J}
\end{align*}
Note that this category of sheaves is a full subcategory of the presheaf category $ \ps{C} $. Hence we have the following inclusion functor:
\begin{align*}
	\Sh{\cat{C},J} \rightarrowtail \ps{C}.
\end{align*}
Hence, we have the following definition:
\begin{definition}
	(\textbf{Grothendieck Topos}) \emph{A Grothendieck Topos is a category} $ \cat{T} $ \emph{which is functorially equivalent to a category} $ \Sh{\cat{C},J} $\emph{ of sheaves over some site} $ (\cat{C},J) $.
\end{definition}
\begin{remark}
	Clearly, the category $ \Sh{\cat{C},J} $ is a trivial example of a Grothendieck Topos.
\end{remark}
We will later see some of the basic properties of the category of sheaves.
\newpage
\subsection{The Sheafification Functor}
Suppose $ (\cat{C},J) $ is a site and we have the sheaf category $ \Sh{\cat{C},J} $. As mentioned in Section \ref{TSF} for $ \ps{O(X)} $, since the inclusion of $ \Sh{\cat{C},J} $ into $ \ps{\cat{C}} $ is the \emph{simplest} way to get a presheaf from a sheaf, then it's adjoint has to be the \emph{simplest} way to get a sheaf from a presheaf. Of-course, the aspect of the construction which transforms a presheaf to a sheaf is interesting, but as we will see (as we had already seen in \ref{TSF}) this construction is a bit non-trivial.\\
The construction which we would now study is known as the \textbf{$ (-)^{+} $-Construction}:
\subsubsection{The $ (-)^{+} $-Construction}
Suppose $ (\cat{C},J) $ is a site and $ P :\opcat{C} \longrightarrow \cat{Sets} $ be a presheaf. We define a new presheaf $ P^{+} $ as follows:
\begin{itemize}
	\item {Define
\begin{align*}
	\Match{R}{P}_C := \text{Set of all matching families of $ R\in JC $}
\end{align*}	
}
\item {Define the following functor:
\begin{align*}
	\Match{-}{P}_C : \opcat{(JC)} &\longrightarrow \cat{Sets}\\
	R &\longmapsto \Match{R}{P}_C\\
	S\subset R&\longmapsto \Match{R}{P}_C \to \Match{S}{P}_C
\end{align*}
where $ \Match{R}{P}_C \to \Match{S}{P}_C $ takes a matching family of $ R $ to that of $ S $ by restricting the elements of the family to that of $ S \subset R $. 
}
\item {Define presheaf $ P^{+}$ as the following:
\begin{align*}
	 P^{+} : \opcat{C} &\longrightarrow \cat{Sets}\\
	 C&\longmapsto \colim\;\Match{-}{P}_C
\end{align*}
}
\end{itemize}
A more illuminating equivalent definition of $ P^{+} $ by reminding ourselves the definition of colimits in $ \cat{Sets} $, however, would be the following:
\begin{definition}
	(\textbf{$ (-)^{+} $-Construction of a Presheaf}) \emph{Suppose} $ (\cat{C},J) $\emph{ is a site and $ P $ is a presheaf over }$ \cat{C} $. \emph{The presheaf $ P^{+} $ is given as follows:}
	\begin{itemize}
		\item {\emph{Define an equivalence relation on the set $ \bunion_{S\in JC} \bunion_{M^{P}_S \in \Match{S}{P}_C} M^{P}_S  $	where two matching families are related as follows:}
			\begin{align*}
				M^{P}_{S} \sim M^{P}_{R} \;\bm{\iff}\; &\exists \text{ a refinement } T\subseteq S\intrs R\;,\;\;T\in JC \text{ such that }\\
				&x_f = y_f \;\forall\;f\in T\;,\;\;x_f \in M^{P}_S \;\&\;y_f \in M^{P}_R
			\end{align*}
	}
\item {\emph{Define action of $ P^{+} $ on objects as:}
\begin{align*}
	 P^{+} : \opcat{C} &\longrightarrow \cat{Sets}\\
	 C&\longmapsto  \text{Set of equivalence classes }\left (\bunion_{S\in JC} \bunion_{M^{P}_S \in \Match{S}{P}_C} M^{P}_S\right ) / \sim 
\end{align*}}
\item {\emph{Define action of $ P^{+} $ on arrows as:}
\begin{align*}
	P^{+} : \opcat{C} &\longrightarrow \cat{Sets}\\
	f : D\to C &\longmapsto P^{+}f : P^{+}C \to P^{+}D
\end{align*}
\emph{where $ P^{+}f $ is as follows:}
\begin{align*}
	P^{+}f : P^{+}C &\longrightarrow P^{+}D\\
	[M^{P}_S]&\longmapsto [f\circ M^{P}_{S}]
\end{align*}
\emph{and $ f\circ M^{P}_S $ is defined as }(\emph{let} $ M^{P}_S = \{x_g \in P(\dom{g})\;\vert\;g\in S\} $):
\begin{align*}
	f\circ M^{P}_S&:= \{x_{f\circ h}\in M^{P}_S\;\vert\;h\in f^{*}(S)\}.
\end{align*}
}
	\end{itemize} 
\end{definition}
\begin{remark}(\textbf{$ P^{+} $ is well defined})
	One may wonder whether the $ P^{+}f $ as mentioned above is well-defined or not. That is, does $ M^{P}_S \sim M^{P}_R \;\implies\; f\circ M^{P}_S \sim f\circ M^{P}_R$? Well it can be seen quite easily that this is true because if we take the refinement $ T \subset R\intrs S$ and it's pullback $ f^{*}(T) $, one can see that it would be a refinement too of $ f^{*}(R)\intrs f^{*}(S) $. Clearly, elements of $ f\circ M^{P}_S $ and $ f\circ M^{P}_R $ are same when restricted to $ f^{*}(T) $. So $ P^{+} $ is indeed well-defined.
\end{remark}
There is a canonical natural transformation which takes any presheaf to it's $ (-)^{+} $ presheaf. This would be important in the following constructions. 
\begin{lemma}\label{L-4}
	Suppose $ (\cat{C},J) $ is a site and $ P $ is a presheaf over $ \cat{C} $. Then the following is an important natural transformation:
	\begin{align*}
		\eta : P \Longrightarrow P^{+}
	\end{align*}
	defined by 
	\begin{align*}
		\eta_C : PC&\longrightarrow P^{+}C\\
		x&\longmapsto \left [M^{P}_{S^{\text{max}}_C}\right ]
	\end{align*}
	where 
	\begin{align*}
		M^{P}_{S^{\text{max}}_C} &:= \left \{Pf(x)\in P(\dom{f})\;\vert\;f\in S^{\text{max}}_C\right \}
	\end{align*}
\end{lemma}
\subsubsection{A Presheaf to the Sheaf via $ (-)^{++} $}
We will now see how we would transform a presheaf to a sheaf via the above construction. More specifically, we would take a presheaf $ P $ and then do the following cascade of transformations:
\[\begin{tikzcd}
	P & {P^+} & {(P^+)^+}
	\arrow["{\eta^P}", maps to, from=1-1, to=1-2]
	\arrow["{\eta^{P^+}}", maps to, from=1-2, to=1-3]
\end{tikzcd}\]
where $ \eta^{P} $ is the natural transformation $ \eta^{P} : P \Longrightarrow P^{+} $, as given in Lemma \ref{L-4}. As we will see now, the $ (P^{+})^{+} $ is guaranteed to be a sheaf over $ (\cat{C},J) $ for any presheaf $ P $.\\
We now quickly state some canonical lemmas without proof (in the following, we assume $ (\cat{C},J) $ is a given site):
\begin{lemma}
	A presheaf $ P $ is separated $ \bm{\iff} $ $ \eta: P \Longrightarrow P^{+} $ is a monic. \\
	Similarly, a presheaf $ P $ is a sheaf $ \bm{\iff} $ $ \eta: P\Longrightarrow P^{+} $ is an isomorphism.
\end{lemma}
\begin{lemma}\label{L-6}
	(\textbf{Universality of $ P^{+} $}) If $ F $ is a sheaf and $ P $ is a presheaf over $ \cat{C} $ and we are given a map $ \theta : P \Longrightarrow F $ in $ \ps{C} $, then $ \exists $ a unique map $ P^{+} \Longrightarrow F $ such that the following commutes:
	\[\begin{tikzcd}
		P \\
		{P^+} & F
		\arrow["\theta", Rightarrow, from=1-1, to=2-2]
		\arrow["\eta"', Rightarrow, from=1-1, to=2-1]
		\arrow[Rightarrow, dashed, from=2-1, to=2-2]
	\end{tikzcd}\]
\end{lemma}
\begin{lemma}\label{L-7}
	For any presheaf $ P $, the presheaf $ P^{+} $ is separated.
\end{lemma}
\begin{lemma}\label{L-8}
	For any separated presheaf, the presheaf $ P^{+} $ is a sheaf.
\end{lemma}
With the above lemmas we can state the following theorem, whose proof is now quite easy with their help:
\begin{theorem}\label{T-8}
	Suppose $ (\cat{C}, J) $ is a site. Then sheaves over this site $ \Sh{\cat{C},J} $ forms a reflective subcategory of $ \ps{C} $, where the corresponding adjunction is given by the following:
	\[\begin{tikzcd}
		{\ps{C}} && {\Sh{\cat{C},J}}
		\arrow[""{name=0, anchor=center, inner sep=0}, "i", curve={height=-18pt}, hook', from=1-3, to=1-1]
		\arrow[""{name=1, anchor=center, inner sep=0}, "a", curve={height=-18pt}, from=1-1, to=1-3]
		\arrow["\dashv"{anchor=center, rotate=-90}, draw=none, from=1, to=0]
	\end{tikzcd}\]
where 
\begin{align*}
	a : \ps{C} &\longrightarrow \Sh{\cat{C},J}\\
	P&\longmapsto a(P) := \eta^{P^{+}} \circ \eta^{P} = (P^{+})^{+}.
\end{align*} 
Moreover, we have:
\begin{align*}
	a\circ i \isomorph_{\text{Nat}}  \Id{\Sh{\cat{C},J}}.
\end{align*}
\end{theorem}
\begin{proof}
	 Lemma \ref{L-7} means $ P^{+} $ is separated and Lemma \ref{L-8} means that $ (P^{+})^{+} $ is a sheaf. That $ a $ as defined above is the left adjoint of inclusion can be seen via the Lemma \ref{L-6}.
\end{proof}
\begin{remark}
	The \textbf{sheafification functor $ a(-) $ preserves finite limits}. This is due to the fact that the functor $ \Match{R}{-} $ preserves finite limits as it is isomorphic to $ \homset{\ps{C}}{R}{-} $ (Proposition \ref{P-5} generalized to sites), which preserves limits, for a fixed cover $ R $. But the $ P^{+} $ is defined to be a filtered colimit. But filtered colimit commutes with finite limits. Therefore $ P\mapsto P^{+} $ preserves limits and hence $ a(-) $ preserves limits. This shows \textbf{that $ i \vdash a $ is a geometric morphism between topoi}.
\end{remark}
\subsection{Properties of $ \Sh{\cat{C},J} $}
We now look at some of the basic properties of categories of sheaves over a site. In particular, we will see that $ \Sh{\cat{C},J} $ satisfies all properties of an elementary topos. In the following, we assume that a site $ (\cat{C},J) $ is given to us.
\begin{proposition}\label{P-8}
	$ \Sh{\cat{C},J} $ has all small limits.
\end{proposition}

\begin{proof}
	Take any diagram in $ \Sh{\cat{C},J} $. Compute their limit in the $ \ps{C} $. But limits in $ \ps{C} $ are computed point-wise. Then use the Definition \ref{D-15} on each member of the diagram in $ \ps{C} $ evaluated at some object $ C $ to get an equalizer diagram. Take the limit of each member of the equalizer diagram. This new diagram would also be an equalizer. Hence limit of the original diagram also follows the equalizer definition of sheaves over a site and hence it is also in $ \Sh{\cat{C},J} $.
\end{proof}
Remember the following functor from our previous discussions:
\begin{align*}
	\Delta: \cat{Sets} &\longrightarrow \ps{C}\\
	S&\longmapsto \Delta S := \text{Constant presheaf to $ S $}
\end{align*}
The following is an useful adjunction, called the \emph{global sections adjunction}:
\begin{definition}\label{D-19}
	(\textbf{Global Sections Adjunction}) \emph{Suppose} $ (\cat{C},J) $ \emph{is a site and left adjoint of inclusion} $ a : \ps{C} \longrightarrow \Sh{\cat{C},J} $ \emph{as given in Theorem \ref{T-8}. We then have the following adjunction:}
	 	\[\begin{tikzcd}
	 		{\cat{Sets}} && {\Sh{\cat{C},J}}
	 		\arrow[""{name=0, anchor=center, inner sep=0}, "a\circ\Delta", curve={height=-24pt}, from=1-1, to=1-3]
	 		\arrow[""{name=1, anchor=center, inner sep=0}, "\Gamma", curve={height=-24pt}, from=1-3, to=1-1]
	 		\arrow["\dashv"{anchor=center, rotate=-90}, draw=none, from=0, to=1]
	 	\end{tikzcd}\]
 	\emph{where }$ \Gamma: \Sh{\cat{C},J} \longrightarrow \cat{Sets} $ \emph{takes a sheaf $ F $ to }$ \Nat{\bm{1}}{F} $, \emph{i.e. $ \Gamma $ is the global sections functor.}
\end{definition}
We now see that sheaf category has small colimits.
\begin{proposition}\label{P-9}
	$ \Sh{\cat{C},J} $ has all small colimits.
\end{proposition}
\begin{proof}
 From Theorem \ref{T-8}, we have that $ a $ preserves colimits as it is the left adjoint. Therefore, to find colimit of a diagram in $ \Sh{\cat{C},J} $, first take it's limit in $ \ps{C} $ and then apply $ a $ to it, since it would preserves colimits, we then have colimit of the original diagram in $ \Sh{\cat{C},J} $.
\end{proof}
Finally, we show without proof that the exponentials in $ \Sh{\cat{C},J} $ exists and how they are given by:
\begin{proposition}\label{P-10}
	$ \Sh{\cat{C},J} $ has all exponentials and for two sheaves $ F $ \& $ G $, the exponential $ F^{G} $ is given by:
	\begin{align*}
		F^{G} : \opcat{C}&\longrightarrow \cat{Sets}\\
		C &\longmapsto F^{G}(C) := \Nat{\homset{\cat{C}}{-}{C}\times G}{F}
	\end{align*}
\end{proposition}
\begin{remark}
	Since $ \Sh{\cat{C},J} $ is a full subcategory of $ \ps{C} $, therefore it doesn't matter whether we take all natural transformations in $ \ps{C} $ or $ \Sh{\cat{C},J} $ in the above. Also note that the above construction of exponential was already proved in Proposition \ref{P-3}.
\end{remark}
Hence we are just one step away from proving that $ \Sh{\cat{C},J} $ is an elementary topos; we just need to show that $ \Sh{\cat{C},J} $ has a subobject classifier (Definition \ref{D-1}) which would classify one subobject from the other. Constructing that requires a bit more insight, which we gain now.
\subsubsection{Subobject Classifier in $ \Sh{\cat{C},J} $}
The truth object and the subobject classifier in $ \Sh{\cat{C},J} $ are to be constructed now. We first define a closed sieve:
\begin{definition}
	(\textbf{Closed Sieve over an object}) \emph{Let} $ (\cat{C},J) $\emph{ be a category. A closed sieve $ S $ over an object $ C $ in the given site is such a sieve which follows:}\emph{
	\begin{align*}
		\text{ If for any }f: D\to C\text{, }f^{*}(S)\in J(C) \text{, then }f\in S.
	\end{align*}}
\end{definition}
\begin{remark}
	It's quite trivial to see that closed sieves are stable under pullback under any compatible arrow. Also, all maximal sieves are closed sieves.
\end{remark}
Then, the truth object is given by the following :
\begin{proposition}
	Suppose $ (\cat{C},J) $ is a site and $ \Sh{\cat{C},J} $ is the sheaf category over it. Then, the truth object $ \Omega : \opcat{C}\longrightarrow \cat{Sets} $ is given by:
	\begin{align*}
		\Omega: \opcat{C} &\longrightarrow \cat{Sets}\\
		C&\longmapsto \Omega C:= \text{Set of all $ J $-closed sieves over $ C $}\\
		f:B\to C&\longmapsto \Omega f : \Omega C\to \Omega B\\
		&\;\;\;\;\;\;\;\;\;\;\;\;\;\;\;\;\;\;\;\;S \mapsto f^{*}(S)
	\end{align*}
and this truth object $ \Omega $ is indeed a sheaf.
\end{proposition}
\begin{proof}(\emph{Sketch}) First prove that $ \Omega $ is a separated presheaf. For this, take any matching family $ R $ of any object $ C $ and take two amalgamations $ S^{1} $ and $ S^{2} $ in $ \Omega C $. Note now that $ S^{1}\intrs R = S^{2}\intrs R $. With this take $ g\in S^{1} $ and show that $ g\in S^{2} $. Similarly the converse to assert that $ S^{1} = S^{2} $.\\
	After this, now show that $ \Omega $ has amalgamations for any matching family. For this, take any matching family of $ R\in JC$ and form a different sieve over $ C $ by collecting all closed sieves' pre-composed with the corresponding member of $ R $ to get arrows over $ C $. Argue that the closure\footnote{The closure of a sieve $ S $ over $ C $ in a site $ (\cat{C},J) $ is defined as $ \bar{S} := \{g\;\vert\;\cod{g}= C\;,\;\;g^{*}(S) \in J(\dom{g})\} $.} of this sieve is the amalgamation of that matching family.
\end{proof}
Now, the \textbf{"truth" monic} and the unique \textbf{characteristic map} is then given as (the proof requires a small result but is fairly trivial afterwards, so is not presented below):
\begin{proposition}\label{P-12}
	Suppose $ (\cat{C},J) $ is a site and $ \Sh{\cat{C},J} $ is the sheaf category. Then the monic 
	\begin{align*}
		\true : \bm{1} &\Longrightarrow \Omega\\
		&\text{in $ \Sh{\cat{C},J} $, given as}\\
		\true_C : \{\star\} &\longrightarrow\Omega C\\
		\star&\longmapsto S_C^{max}
	\end{align*}
is the subobject classifier for $ \Sh{\cat{C},J} $. The unique characteristic map $ \chi : G \Longrightarrow \Omega $ corresponding to the monic $ m : F \Longrightarrow G $ in $ \Sh{\cat{C},J} $ is given as:
\begin{align*}
	\chi : G&\Longrightarrow\Omega\\
	\chi_C : GC&\longrightarrow \Omega_C\\
	x&\longmapsto \{f\;\vert\; \cod{f} = C\;,\;\;Gf(x) \in F(\dom{f})\}.
\end{align*}
\end{proposition}
Hence by Propositions \ref{P-8},\ref{P-9},\ref{P-10} \& \ref{P-12}, $ \Sh{\cat{C},J} $ is an elementary topos, as was required to be shown.
\newpage
\section{Basic Properties and Results in Topoi}
We now study the first properties observed from the definition of a topos. We will see that there are a lot of striking similarity between $ \cat{Sets} $ and a topos $ \cat{E} $. For example, each subobject in a topos will have a clearly defined way of identifying "elements" as whether they are indeed in the given subobject or not, similarly, whether two "elements" are same or not, image of a subobject along other arrow and so on, purely in categorical terms, and all of which is generalized from their usual notions in $ \cat{Sets} $. \\
Let's begin with the definition of a topos, with the underlying help of sets:
\begin{definition}\label{D-21}
	(\textbf{Elementary Topos - I}) \emph{A category} $ \cat{E} $ \emph{is a topos if:}
	\begin{itemize}
		\item [\textbf{ETI.1}] {$ \cat{E} $ \emph{has all finite limits.}}
		\item [\textbf{ETI.2}] { \emph{The subobject functor (Definition \ref{D-2})} $ \Sub{-}{\cat{E}} : \opcat{E} \longrightarrow \cat{Sets} $ \emph{is representable and the object which represents it, denoted $ \Omega $, is called the subobject classifier. That is,}
			\begin{align*}
				\Sub{A}{\cat{E}} \isomorph_{\text{Nat}} \homset{\cat{E}}{A}{\Omega}.
			\end{align*}
		}
	\item[\textbf{ETI.3}] {\emph{The functor }$ \homset{\cat{E}}{B\times -}{\Omega} $ \emph{is representable for all objects $ B $ and the representing object is denoted as $ PB $, called the power object of object $ B $. That is,}
\begin{align*}
	\homset{\cat{E}}{B\times A}{\Omega} \isomorph_{\text{Nat}} \homset{\cat{E}}{A}{PB}.
\end{align*}	
}
	\end{itemize} 
\end{definition}
\begin{remark}
	One can combine ETI.2 and ETI.3 to get 
	\begin{align*}
		\Sub{B\times A}{\cat{E}} \isomorph_{\text{Nat}} \homset{\cat{E}}{A}{PB}
	\end{align*}
\end{remark}
As pointed earlier, we can actually state a definition of topos in complete categorical language without any use of underlying sets. Hence, the following is the first order theory of an elementary topos: 
\begin{definition}\label{D-22}
	(\textbf{Elementary Topos -II}) \emph{A category} $ \cat{E} $ \emph{is a topos if:}
	\begin{itemize}
		\item [\textbf{ETII.1}]{\emph{All pullbacks exists.}}
		\item [\textbf{ETII.2}]{\emph{A terminal object $ \bm{1} $ exists}.}
		\item [\textbf{ETII.3}]{\emph{There exists an object $ \Omega $ and a monomorphism $ \true : \bm{1} \rightarrowtail \Omega $ in }$ \cat{E} $ \emph{such that for any monomorphism $ m : E \rightarrowtail B $, there exists a unique map $ \phi : B\rightarrow \Omega $ such that the following is a pullback square:}
	\[\begin{tikzcd}
		E & {\bm{1}} \\
		B & \Omega
		\arrow["m"', tail, from=1-1, to=2-1]
		\arrow["\phi"', dashed, from=2-1, to=2-2]
		\arrow[from=1-1, to=1-2]
		\arrow["\true", tail, from=1-2, to=2-2]
		\arrow["\lrcorner"{anchor=center, pos=0.125}, draw=none, from=1-1, to=2-2]
	\end{tikzcd}\]	
	}
\item [\textbf{ETII.4}]{\emph{For any object $ B $, there exists an object $ PB $ in} $ \cat{E} $ \emph{and an arrow:}
	\begin{align*}
		\epsilon_B : B\times PB \longrightarrow \Omega
	\end{align*}
\emph{such that for any object $ A $ and any arrow $ f : B\times A \to  \Omega $, we have a unique arrow $ \Ptr{f} : A \longrightarrow PB $ such that the following commutes:}
\[\begin{tikzcd}
	A && {B\times A} \\
	PB && {B\times PB} & \Omega
	\arrow["f", from=1-3, to=2-4]
	\arrow["{1\times \Ptr{f}}"', from=1-3, to=2-3]
	\arrow["{\epsilon_B}"', from=2-3, to=2-4]
	\arrow["{\Ptr{f}}"', dashed, from=1-1, to=2-1]
\end{tikzcd}.\]
\emph{The $ \Ptr{f} $ is usually called the P-transpose of $ f $.}}
	\end{itemize}
\end{definition}
\newpage
The construction of power object can be defined as the following functor:
\begin{definition}
	(\textbf{Power Object Functor}) :\emph{ Suppose} $ \cat{E} $ \emph{is a topos. The power object for each object $ B $ of} $ \cat{E} $ \emph{is a functor }
	\begin{align*}
		P : \opcat{E} &\longrightarrow \cat{E}\\
		B &\longmapsto PB \;\;\text{(the object defined in Definition \ref{D-22}, ETII.4)}\\
		f : B\to C &\longmapsto Pf : PC\to PB
	\end{align*}
\emph{where the $ Pf $ is the unique map for which the corresponding diamond commutes:}
\[\begin{tikzcd}
	& {B\times PB} \\
	{B\times PC} && \Omega \\
	& {C\times PC}
	\arrow["{\epsilon_B}", from=1-2, to=2-3]
	\arrow["{\epsilon_C}"', from=3-2, to=2-3]
	\arrow["{1\times Pf}", from=2-1, to=1-2]
	\arrow["{f\times 1}"', from=2-1, to=3-2]
\end{tikzcd}.\]
\emph{That is,}
\begin{align*}
	\epsilon_B\circ\left (1\times Pf\right ) = \epsilon_C \circ \left (f\times 1\right ).
\end{align*}
\end{definition}
\begin{remark}
	This means that the arrow $ \epsilon_B: B\times PB \longrightarrow \Omega $ is \emph{dinatural} in $ B $.
\end{remark}
\subsection{Extension, Characteristic \& Name of a subobject}
Remember that $ A \isomorph A\times 1 $ in any category. Therefore by the Definition \ref{D-21}, we have the following isomorphisms in a topos $ \cat{E} $: 
\begin{align*}
	\Sub{A}{\cat{E}} \isomorph \homset{\cat{E}}{A}{\Omega} \isomorph \homset{\cat{E}}{1}{PA}
\end{align*}
Hence there are three equivalent ways to talk about a subobject. These are denoted as follows:
\begin{definition}
	(\textbf{Extension, Characteristic \& Name}) \emph{Suppose} $ \cat{E} $ \emph{is a topos. Let $ m : S \rightarrowtail A $ be a subobject of $ A $. Then we denote the total three corresponding arrows (from above isomorphisms) as follows}:
\[\begin{tikzcd}
	S & A && A & \Omega && 1 & PA \\
	{} & {} && {} & {} && {} & {} \\
	{} &&& {} &&& {}
	\arrow["m", tail, from=1-1, to=1-2]
	\arrow["\phi", from=1-4, to=1-5]
	\arrow["s", from=1-7, to=1-8]
	\arrow["{S:=\{m\vert\phi\}}"{description}, draw=none, from=3-1, to=2-2]
	\arrow["{\text{Extension of } \phi}"{description}, from=2-1, to=2-2]
	\arrow["{\text{Characteristic of }S}"{description}, from=2-4, to=2-5]
	\arrow["{\phi := \chr{S}}"{description}, draw=none, from=3-4, to=2-5]
	\arrow["{\text{Name of } \phi}"{description}, from=2-7, to=2-8]
	\arrow["{s:=\nme{\phi}}"{description}, draw=none, from=3-7, to=2-8]
\end{tikzcd}.\]

\end{definition}
This convention becomes useful when we realize that an arrow $ b : X\to A $ can be treated as a "generalized element" of $ A $. Hence, we can talk whether an "element" $ b $ of $ A $ is in a subobject of $ A $. The role of characteristic map $\phi = \chr{S}$ is illuminating. It tells us about the property/predicate that is followed by the subobject that it characterizes. For example, the statement of the subobject classifier (Definition \ref{D-22}, ETII.3) can be reformulated as the condition that an "element" of $ A $, $ a : X\to A $, is in the subobject $ m : S\rightarrowtail A $ if and only if $ (\chr{S})\circ a = \true\circ(X\to \bm{1}) $. Let us denote $ \true_X :=  \true \circ (X\to \bm{1})$. Note that $ \true \circ (X\to \bm{1}) $ is simply the arrow $ X\to \Omega $ corresponding to all truth. Therefore, the fact that $ (\chr{S})\circ a = \true_X $ means that the element $ X $ forms a cone over the pullback square and hence we have a unique arrow $ X \to S $, meaning that $ X $ is also a "generalized element" of $ S $.\\\\
We now develop more interesting predicates, like the one which tells us whether an element is in a given subobject named by a name $ \nme{\phi} $.
\subsubsection{Membership Predicate}
Suppose we are given an element $ b : X\to B $ of $ B $ and a subobject named $ s:= \nme{\phi} $ of $ B $. We then have the following commutative diagram:
\begin{equation}\label{MP}
	\begin{tikzcd}
		X & B & \Omega \\
		{X\times 1} & {B\times 1} & {B\times PB}
		\arrow["b", from=1-1, to=1-2]
		\arrow["\phi", from=1-2, to=1-3]
		\arrow[Rightarrow, no head, from=1-1, to=2-1]
		\arrow["{b\times 1}"', from=2-1, to=2-2]
		\arrow["{1\times s}"', from=2-2, to=2-3]
		\arrow["{\epsilon_B}"', from=2-3, to=1-3]
		\arrow[Rightarrow, no head, from=1-2, to=2-2]
	\end{tikzcd}
\end{equation}
This means that:
\begin{align*}
	\boxed{\epsilon_B\circ (b\times s)  = \true_{X\times 1} \;\text{\textbf{if and only if}}\;\phi\circ b = \true_{X}.}
\end{align*}
That is, $ \epsilon_B\circ(b\times s) $ is true exactly when $ \phi\circ b $ is true. But the latter means that $ b $ factors through the subobject (is in) named by $ s $ because of universality of pullback. Hence $ \epsilon_B\circ(b\times s)$ is true only when $ b $ is in the subobject named by $ s $.\\\\
This is exactly the reason why $ \epsilon_B $ is called the \textbf{membership predicate} for object $ B $.
\subsubsection{Equality Predicate}
Suppose now that we have two elements $ b,b^{\prime}  : X\to B$ of $ B $. How can we know that these are same elements, that is, what is the condition for $ b = b^{\prime} $? Clearly, this is a property of the elements of $ B $, hence there must be a predicate for $B $ to answer this question. This predicate for an object which tells us when two elements of it are same is exactly what we construct now.\\\\
Remember first the diagonal arrow of $ B $, $ \diag{B} : B\longrightarrow B\times B $, which is such that $ p_1 \circ \diag{B} = p_2\circ \diag{B} = 1 $, where $ p_1, p_2 $ are projections of $ B\times B $. It can be seen without much effort that $ \diag{B} $ is actually a monic. We can hence talk about the subobject $ \diag{B} : B\rightarrowtail B\times B $ of $ B\times B $:
\begin{equation}\label{E-3}
	\begin{tikzcd}
		B & {\bm{1}} \\
		{B\times B} & \Omega
		\arrow["{\diag{B}}"', tail, from=1-1, to=2-1]
		\arrow["{\chr{\diag{B}}}"', from=2-1, to=2-2]
		\arrow[from=1-1, to=1-2]
		\arrow["\true", tail, from=1-2, to=2-2]
		\arrow["\lrcorner"{anchor=center, pos=0.125}, draw=none, from=1-1, to=2-2]
	\end{tikzcd}
\end{equation}
where $ \chr{\diag{B}} $ is the unique characteristic map of subobject $ \diag{B} $. But now, by power object adjunction (Definition \ref{D-22}, ETII.4), we have the following unique arrow:
\[\begin{tikzcd}
	B && {B\times B} & \Omega \\
	PB && {B\times PB}
	\arrow["{\chr{\diag{B}}}", from=1-3, to=1-4]
	\arrow["{1\times \Ptr{\chr{\diag{B}}}}"', from=1-3, to=2-3]
	\arrow["{\epsilon_B}"', from=2-3, to=1-4]
	\arrow["{\Ptr{\chr{\diag{B}}}}"', dashed, from=1-1, to=2-1]
\end{tikzcd}\]
where $ \Ptr{\chr{\diag{B}}} $ is the $ P $-transpose of $ \chr{\diag{B}} $. Let us denote this transpose as
\begin{align*}
	\singarr{B} := \Ptr{\chr{\diag{B}}}.
\end{align*}
Now, take any two elements $ b,b^{\prime} : X\to B $ of $ B $. We then have a unique arrow $ \upair{b}{b^{\prime}} : X \to B\times B$ due to the universality of the product. Hence, by universality of the pullback in \eqref{E-3}, we get the following condition:
\begin{align*}
	\boxed{(\chr{\diag{B}}) \circ \upair{b}{b^{\prime}} = \true_{X}\;\text{\textbf{if and only if}}\;b = b^{\prime}.}
\end{align*}
Due to this exact reason, the characteristic map of subobject $ \diag{B} $, $ \chr{\diag{B}} $ is called the \textbf{equality predicate} for object $ B $\footnote{At this point one should notice how the above two predicates have been constructed. We are using the subobject classifier to \emph{distinguish} between elements based on the object's \emph{properties}. Hence it is the subobject classifier (Definition \ref{D-22}, ETII.3) which provides us with the opportunity to talk about \emph{properties} of an object in a topos, which can not be done otherwise in any arbitrary category.}.
\\\\
The arrow $ \{\cdot\}_B : B \longrightarrow PB $ is always a monic:
\begin{proposition}
	Suppose $ \cat{E} $ is a topos. For any object $ B $ of $ \cat{E} $, the arrow
	\begin{align*}
		\{\cdot\}_B := \Ptr{\chr{\diag{B}}} : B \longrightarrow PB
	\end{align*}
is always a monic.
\end{proposition}
\begin{proof}
	Let $ b,b^{\prime} : X\to B $ be two arrows such that
\begin{align*}
	\{\cdot\}_B\circ b &= \{\cdot\}_B \circ b^{\prime}\\
	(1\times \{\cdot\}_B) \circ (1\times b) & = (1\times \{\cdot\}_B) \circ (1\times b^{\prime})\\
	\epsilon_B\circ (1\times \{\cdot\}_B) \circ (1\times b) & = \epsilon_B \circ(1\times \{\cdot\}_B) \circ (1\times b^{\prime})\\
	(\chr{\diag{B}}) \circ (1\times b) & = (\chr{\diag{B}})(1\times b^{\prime})
\end{align*}
	Now note the following diagram:
	\[\begin{tikzcd}
		X & B & {\bm{1}} \\
		{B\times X} & {B\times B} & \Omega
		\arrow["b", shift left=2, from=1-1, to=1-2]
		\arrow["{b^{\prime}}"', shift right=2, from=1-1, to=1-2]
		\arrow["{\Delta_B}"', from=1-2, to=2-2]
		\arrow["{\chr{\diag{B}}}"', from=2-2, to=2-3]
		\arrow[from=1-2, to=1-3]
		\arrow["\true", from=1-3, to=2-3]
		\arrow["\lrcorner"{anchor=center, pos=0.125}, draw=none, from=1-2, to=2-3]
		\arrow["{\upair{b^{\prime}}{1}}", shift left=2, from=1-1, to=2-1]
		\arrow["{\upair{b}{1}}"', shift right=2, from=1-1, to=2-1]
		\arrow["{1\times b}"', shift right=2, from=2-1, to=2-2]
		\arrow["{1\times b^{\prime}}", shift left=2, from=2-1, to=2-2]
	\end{tikzcd}\]
The right square is a pullback, whereas the two left squares are also pullback by an easy observation. Hence the whole two bigger rectangles in the above diagram are pullbacks. But this means that the $ \chr{\upair{b}{1}} = \chr{\diag{B}} \circ (1\times b) = \chr{\diag{B}} \circ (1\times b^{\prime}) = \chr{\upair{b^{\prime}}{1}} $. Hence $ \upair{b}{1} $ and $ \upair{b^{\prime}}{1} $ are same pullbacks. Therefore $ \exists f : X\to X $ isomorphism such that $ \upair{b}{1} = \upair{b^{\prime}}{1} \circ f $. But this means that $ b = b^{\prime} \circ f $ and $ 1 = f $ because $ \upair{a}{b}\circ f = \upair{a\circ f}{b\circ f} $. Hence $ b=b^{\prime}\circ 1 = b^{\prime} $.
\end{proof}
\subsubsection{Logical Morphism}
The fact that a topos has it's own defining properties like a subobject classifier and power objects means that any functor may or may not preserve these properties. To distinguish such a functor, we define what we call a logical morphism. Before formally defining them, let's look at one of the more defining properties of a topos; that a topos is \emph{balanced}: 
\begin{proposition}
	Suppose $ \cat{E} $ is a topos. Any monomorphism $ f : A\rightarrowtail B $ is an equalizer of some parallel pair.
\end{proposition}
\begin{proof}
	The arrow $ f : A\rightarrowtail B $ equalizes the following:
	\[\begin{tikzcd}
		B & \Omega
		\arrow["{\chr{f}}"', shift right=2, from=1-1, to=1-2]
		\arrow["{\true \circ !_B}", shift left=2, from=1-1, to=1-2]
	\end{tikzcd}\]
where $ !_B : B \longrightarrow \bm{1} $, because of the following:
\begin{align*}
	\chr{f}\circ f &= \true\circ !_{A}\\
	&= \true \circ !_B\circ f
\end{align*}
as $ !_A = !_B\circ f $. Now because $ A $ is universal with this property (note we are just stating the subobject classifier's definition) so we have that $ f : A\rightarrowtail B $ equalizes the above pair.
\end{proof}
\begin{corollary}
	In a topos $ \cat{E} $, every arrow $ f : A\to B $ which is both monic and epic is an isomorphism.
\end{corollary}
\begin{proof}
	If $ f : A\to B $ is monic, so it equalizes two parallel pairs $ x,y : B\to C $. But then $ x\circ f = y\circ f \implies x = y $ because $ f $ is epic. Therefore $ f $ is an equalizer of the same pair of arrows, which is always the identity at the domain. Hence $ f $ is an isomorphism.
\end{proof}
\begin{remark}
	A category in which every monic + epic arrow is an isomorphism is called a \textbf{balanced} category. Hence every topos is balanced.
\end{remark}
\begin{definition}\label{D-25}
	(\textbf{Logical Morphism})\emph{ Suppose }$ \cat{E} $\emph{ and }$ \cat{E}^{\prime} $ \emph{are two topoi. A functor }
	\begin{align*}
		T : \cat{E}\longrightarrow \cat{E}^{\prime}
	\end{align*}
\emph{is called a logical morphism if it preserves all finite limits, subobject classifier and all exponentials, upto isomorphism.}
\end{definition}
\subsubsection{Direct Image}
Suppose in $ \cat{Sets} $, we have a monic $ m : S^{\prime} \longrightarrow B^{\prime} $ and some other function $ f : B^{\prime}  \longrightarrow B$. With this, we have another function $ PB^{\prime} \longrightarrow PB $, which takes a subset $ S^{\prime} $ in $ PB^{\prime} $ to the set $ f(S^{\prime}) $. We can generalize it to an arbitrary topos.
\begin{definition}\label{D-26}
	(\textbf{Direct Image Arrow}) \emph{Suppose} $ \cat{E} $ \emph{is a topos and let $ m : S^{\prime} \rightarrowtail B^{\prime}$ be a monic and $ k : B^{\prime} \to B $ be any arrow. Then there is an arrow }
	\begin{align*}
		\dimag{k} : PB^{\prime} \longrightarrow PB
	\end{align*}
\emph{called the direct image arrow of $ k $ is defined by:}
\[\begin{tikzcd}
	U & {\bm{1}} && {\bm{1}} \\
	{B^{\prime}\times PB^{\prime}} & \Omega \\
	{B\times PB^{\prime}} &&& \Omega \\
	{PB^{\prime}} &&& PB
	\arrow[""{name=0, anchor=center, inner sep=0}, "{e_k := \chr{(k\times 1)\circ u_{B^{\prime}}}}", dashed, from=3-1, to=3-4]
	\arrow["\true", dashed, tail, from=1-4, to=3-4]
	\arrow["{u_{B^{\prime}}}"', dashed, tail, from=1-1, to=2-1]
	\arrow["{\epsilon_B}", from=2-1, to=2-2]
	\arrow["\true", tail, from=1-2, to=2-2]
	\arrow["{k\times 1}"', from=2-1, to=3-1]
	\arrow[from=1-1, to=1-2]
	\arrow[Rightarrow, no head, from=1-2, to=1-4]
	\arrow[""{name=1, anchor=center, inner sep=0}, "{\dimag{k}}"', dashed, from=4-1, to=4-4]
	\arrow["\lrcorner"{anchor=center, pos=0.125}, draw=none, from=1-1, to=2-2]
	\arrow[shorten <=4pt, shorten >=4pt, squiggly, from=0, to=1]
	\arrow[shorten <=4pt, shorten >=4pt, no head, from=0, to=1]
\end{tikzcd}\]
\end{definition}
Direct image arrow preserves names of the subobject, as was expected from the beginning discussion of the same arrow in $ \cat{Sets} $:
\begin{proposition}
	Suppose $ \cat{E} $ is a topos. Then, for monics
\[\begin{tikzcd}
	S & {B^{\prime}} & B
	\arrow["m", tail, from=1-1, to=1-2]
	\arrow["k", tail, from=1-2, to=1-3]
\end{tikzcd}\]
we have that the following commutes:
\[\begin{tikzcd}
	{\bm{1}} \\
	{PB^{\prime}} & PB
	\arrow["{\nme{\chr{m}}}"', from=1-1, to=2-1]
	\arrow["{\dimag{k}}"', from=2-1, to=2-2]
	\arrow["{\nme{\chr{k\circ m}}}", from=1-1, to=2-2]
\end{tikzcd}\]
\end{proposition}

\subsection{Factorization in a Topos}
A distinguishing feature of a topos is that each arrow in it can be factored into a product of a monomorphism and an epimorphism:
\begin{proposition}\label{P-16}
	Suppose $ \cat{E} $ is a topos and let $ f $ be any arrow in it. Then there is a monic $ m $ and an epic $ e $ composable such that 
	\begin{align*}
		f = m\circ e.
	\end{align*}
\end{proposition}
\begin{proof}(\textit{Sketch})
	The $ m $ is constructed by taking the equalizer of the cokernel pair\footnote{The pushout of $ f : A\to B $ with itself.} of $ f $ and $ e $ as the universal arrow from $ A $ to the object representing this equalizer. In more concrete setting, for the given $ f $, denote the $ x $ and $ y $ as the following pushout components:
	\[\begin{tikzcd}
		C & B \\
		B & A
		\arrow["f"', from=2-2, to=1-2]
		\arrow["f", from=2-2, to=2-1]
		\arrow["x", from=2-1, to=1-1]
		\arrow["y"', from=1-2, to=1-1]
		\arrow["\ulcorner"{anchor=center, pos=0.125}, draw=none, from=1-1, to=2-2]
	\end{tikzcd}\]
	Then the $ m $ and $ e $ are the following arrows (with their defining properties mentioned above):
	\[\begin{tikzcd}
		A & M & B & C
		\arrow["f", shift left=2, from=1-1, to=1-3]
		\arrow["x", shift left=1, from=1-3, to=1-4]
		\arrow["y"', shift right=1, from=1-3, to=1-4]
		\arrow["e"', shift right=2, dashed, from=1-1, to=1-2]
		\arrow["m"', shift right=2, dashed, from=1-2, to=1-3]
	\end{tikzcd}.\]
It then follows that $ m $ is monic and $ e $ is epic.
\end{proof}
\newpage
We now show that the collection of all subobjects for an object in a topos forms a lattice. Later we will show that it actually forms a \emph{Heyting Algebra}.
\begin{definition}
	(\textbf{Lattice}) \emph{Suppose} $ \cat{C} $ \emph{is a posetal category.} $ \cat{C} $ \emph{is a lattice when it has all binary products and coproducts. The product is alternatively called "meet" and coproduct the "join" and denoted $ \meet $ \& $ \join $ respectively.}
\end{definition}
Now, we get the following important theorem:
\begin{theorem}\label{T-4.1}
	Suppose $ \cat{E} $ is a topos. Let $ k : A\to B $ be any arrow in $ \cat{E} $. Then,
	\begin{enumerate}
		\item {$ \Sub{D}{\cat{E}} $ is a lattice for any object $ D $ in $ \cat{E} $.}
		\item {The functor $ \dimag{k} : \Sub{A}{\cat{E}} \longrightarrow \Sub{B}{\cat{E}} $\footnote{Note that $ \dimag{k} $ here is the "external" direct image functor, in contrast to Definition \ref{D-26}, which was internal.} which takes each subobject to the object which mono-epi factorizes\footnote{That is, for $ u : S\rightarrowtail A $, $ \dimag{k}(u) $ would be the monic $ m_{k\circ u} : \dimag{k}S \rightarrowtail B $ in the figure given below:
			\[\begin{tikzcd}
				S \\
				{\dimag{k}S} \\
				B
				\arrow["{e_{k\circ u}}"', two heads, from=1-1, to=2-1]
				\arrow["{m_{k\circ u}}", tail, from=2-1, to=3-1]
			\end{tikzcd}\]
		 composes to $ k\circ u $.} it, and the pullback arrow forms the following adjunct pair between lattices (regarded as categories):
		\[\begin{tikzcd}
			{\Sub{A}{\cat{E}}} && {\Sub{B}{\cat{E}}}
			\arrow[""{name=0, anchor=center, inner sep=0}, "{\dimag{k}}", curve={height=-12pt}, from=1-1, to=1-3]
			\arrow[""{name=1, anchor=center, inner sep=0}, "{\inv{k}}", curve={height=-12pt}, from=1-3, to=1-1]
			\arrow["\dashv"{anchor=center, rotate=-90}, draw=none, from=0, to=1]
		\end{tikzcd}\]
where 
\begin{align*}
	\inv{k} : \Sub{B}{\cat{E}} &\longrightarrow \Sub{A}{\cat{E}}\\
	m : S \rightarrowtail B & \longmapsto \pi_1 : A\times_B S \rightarrowtail A.
\end{align*}	
}
	\end{enumerate}
\end{theorem}
\begin{proof}
	To show that $ \Sub{D}{\cat{E}} $ is a lattice, we first note that it is partially ordered by inclusion, as if $ S,T,U \in \Sub{D}{\cat{E}} $ then $ S\subseteq S $, $ S\subseteq T $ \& $ T\subseteq S $ implies $ S = T $ and finally if $ S\subseteq T $ and $ T\subseteq U $ then $ S\subseteq U $. Moreover, for any two subobjects $ S,T$, we can form two more subobjects as follows:
	\[\begin{tikzcd}
		{S\meet T: = S\times_DT} & T \\
		S & D
		\arrow[tail, from=2-1, to=2-2]
		\arrow[tail, from=1-2, to=2-2]
		\arrow[tail, from=1-1, to=1-2]
		\arrow[tail, from=1-1, to=2-1]
		\arrow["\lrcorner"{anchor=center, pos=0.125}, draw=none, from=1-1, to=2-2]
	\end{tikzcd}\]
and
\[\begin{tikzcd}
	&& D \\
	&& {M} \\
	S && {S\amalg T} && T
	\arrow["{i_1}"', from=3-1, to=3-3]
	\arrow["{i_2}", from=3-5, to=3-3]
	\arrow[tail, from=3-1, to=1-3]
	\arrow[tail, from=3-5, to=1-3]
	\arrow["f"', curve={height=12pt}, dashed, from=3-3, to=1-3]
	\arrow["{e_f}", two heads, from=3-3, to=2-3]
	\arrow["{m_f}", tail, from=2-3, to=1-3]
\end{tikzcd}\]
where $ S\join T := M $ and $ f $ is the unique universal arrow from the coproduct of $ S $ and $ T $ to $ D $ and $ m_f, e_f $ are it's mono-epic factors (Proposition \ref{P-16}). Therefore $ \left (\Sub{D}{\cat{E}}, \join, \meet\right ) $ is a lattice.\\\\
Now, to show that $ \dimag{k} $ and $ \inv{k} $ are adjoints as required in the theorem, we first take any subobject $ m_A : S_A \rightarrowtail A $ of $ A $ and $ m_B : S_B \rightarrowtail B $ of $ B $. We want to establish the following natural isomorphism:
\begin{align*}
	\homset{\Sub{B}{\cat{E}}}{\dimag{k}m_A}{m_B} \isomorph \homset{\Sub{A}{\cat{E}}}{m_A}{\inv{k}m_B}.
\end{align*}
Hence, take any arrow $ g : \dimag{k} S_A \longrightarrow S_B $, so that $ g $ is just an arrow $ \dimag{k}m_A \longrightarrow m_B $. Then consider the following diagram:
\[\begin{tikzcd}
	{S_A} & {k^{-1}S_B} && {S_B} & {\dimag{k}S_A} \\
	A & A && B & B
	\arrow["{k^{-1}m_B}", tail, from=1-2, to=2-2]
	\arrow["{m_B}"', tail, from=1-4, to=2-4]
	\arrow["g"', from=1-5, to=1-4]
	\arrow["k"', from=2-2, to=2-4]
	\arrow[from=1-2, to=1-4]
	\arrow["\lrcorner"{anchor=center, pos=0.125}, draw=none, from=1-2, to=2-4]
	\arrow[Rightarrow, no head, from=2-4, to=2-5]
	\arrow["{m_{k\circ m_A}}", tail, from=1-5, to=2-5]
	\arrow[Rightarrow, no head, from=2-1, to=2-2]
	\arrow["{m_A}"', tail, from=1-1, to=2-1]
	\arrow["{e_{k\circ m_A}}"', curve={height=-30pt}, two heads, from=1-1, to=1-5]
	\arrow["f", dashed, from=1-1, to=1-2]
\end{tikzcd}\]
where, because the pair $ m_A $ and $ g\circ e_{k\circ m_A} $ forms a cone over the middle pullback as $ m_B \circ g\circ e_{k\circ m_A} = m_{k\circ m_{A}} \circ e_{k\circ m_A} = k\circ m_A$, $ \exists \;!\;f : S_A \longrightarrow \inv{k}S_B $ which is the required arrow to establish the adjunction.
\end{proof}
\begin{remark}
	In continuation of "external"/"internal" debate mentioned in footnote 15, we note that $ \Sub{A}{\cat{E}} $ is the external subobject lattice. We will later see that the internal analogue of $ \Sub{A}{\cat{E}} $ is the power object $ PA $ and it too forms an appropriate notion of an \emph{internal lattice}. In-fact, both $ \Sub{A}{\cat{E}} $ and $ PA $ forms a \emph{Heyting algebra} and an \emph{internal Heyting algebra}, respectively.
\end{remark}
\subsubsection{$ \Sub{1}{\cat{E}} \equiv \Open{\cat{E}}$}
There's more to the story than just the fact that all subobjects of an object forms a lattice. As we will see now, the subobject lattice of terminal object is a bit special in the sense that it is equivalent to a particular category of all objects which have a monic to the terminal. Let's retrospect this in the category of $ \cat{Sets} $, in which the terminal object is the singleton $ \{\star\} $. Clearly, any subset of the singleton is either the singleton $ \{\star\} $ itself or null-set $ \phi $. But note that $ \{\star\} $ and $ \phi $ are also the only sets with a monic arrow to $ \{\star\} $. Therefore the result mentioned in the beginning clearly holds in the prototypical elementary topos $ \cat{Sets} $. Let's now prove this in any arbitrary elementary topos $ \cat{E} $:
\begin{definition}\label{D-29}
	(\textbf{Open Objects}) \emph{Suppose }$ \cat{E} $\emph{ is a topos. Then an object $ B $ is an open object if the unique arrow to the terminal object is a monic, $ B \rightarrowtail \bm{1} $ .}
\end{definition}
We then have the following:
\begin{lemma}\label{L-9}
	Suppose $ \cat{E} $ is a topos. An object $ U $ is open if and only if $ \homset{\cat{E}}{X}{U} = \{\star\} \;\forall\;X\in \obj{\cat{E}}$.	
\end{lemma}
\begin{proof}
	$ U $ is open $ \iff $ $ m : U\rightarrowtail \bm{1} $ $ \iff $ any parallel pair $ x,y : X\to U $ would be such that $ m\circ x = m\circ y $, but because $ m $ is monic, we get $ x= y $.
\end{proof}
\begin{proposition}
	Suppose $ \cat{E} $ is a topos. The lattice $ \Sub{\bm{1}}{\cat{E}} $ is equivalent to the full subcategory $ \Open{\cat{E}} $ of open objects of $ \cat{E} $, i.e.
	\begin{align*}
		\Sub{\bm{1}}{\cat{E}} \equiv \Open{\cat{E}}.
	\end{align*}
\end{proposition}
\begin{proof}
	Since $ \homset{\Sub{\bm{1}}{}}{S}{T} = \{\star\} = \homset{\Open{\cat{E}}}{X}{Y} $ where last equation follows from Lemma \ref{L-9}, therefore $ \Sub{\bm{1}}{\cat{E}} \equiv \Open{\cat{E}}$.
\end{proof}
\subsubsection{$ \Sub{B}{\cat{E}} $ is reflective into $ \cat{E} /B$}
We now show that the subobject category (lattice) is a reflective subcategory of the slice over that object.
\begin{proposition}
	Suppose $ \cat{E} $ is a topos. For any object $ B $ in $ \cat{E} $, the inclusion $ \Sub{B}{\cat{E}} \hookrightarrow \cat{E}/B$ has a left adjoint $ \sigma $, i.e.
	\[\begin{tikzcd}
		{\cat{E}/B} && {\Sub{B}{\cat{E}}}
		\arrow[""{name=0, anchor=center, inner sep=0}, "\sigma", curve={height=-18pt}, from=1-1, to=1-3]
		\arrow[""{name=1, anchor=center, inner sep=0}, "i", curve={height=-18pt}, hook', from=1-3, to=1-1]
		\arrow["\dashv"{anchor=center, rotate=-90}, draw=none, from=0, to=1]
	\end{tikzcd}\]
	where 
	\begin{align*}
		\sigma: \cat{E}/B &\longrightarrow \Sub{B}{\cat{E}}\\
		(f : A\to B) &\longmapsto m_f : S \rightarrowtail B\text{, where }
	\end{align*}
	\[\begin{tikzcd}
		& S \\
		A && B
		\arrow["f"', from=2-1, to=2-3]
		\arrow["{e_f}", two heads, from=2-1, to=1-2]
		\arrow["{m_f}", tail, from=1-2, to=2-3]
	\end{tikzcd}.\]
\end{proposition}
\begin{proof}
	Take any arrow $ \sigma f \subset g $ in $ \Sub{B}{\cat{E}} $ where $ f : A\to B $ is an object in $ \cat{E}/B $ and $ g : S\rightarrowtail B $ is an object in $ \Sub{B}{\cat{E}} $. Now note the following diagram:
	\[\begin{tikzcd}
		S & A & T \\
		B & B
		\arrow["f", from=1-2, to=2-2]
		\arrow["g"', tail, from=1-1, to=2-1]
		\arrow["{e_f}", two heads, from=1-2, to=1-3]
		\arrow["{\sigma f:=m_f}", tail, from=1-3, to=2-2]
		\arrow[Rightarrow, no head, from=2-1, to=2-2]
		\arrow["h"', curve={height=18pt}, from=1-3, to=1-1]
		\arrow["{h\circ e_f}", from=1-2, to=1-1]
	\end{tikzcd}\]
	where $ h : T \to S $ is the arrow corresponding to the subobject inclusion $ \sigma f \subset g $. We claim that the arrow $ h\circ e_f : A\to S $ is the unique arrow $ f \to ig $ in $ \cat{E}/B $, because, firstly $ (g\circ h)\circ e_f = (m_f) \circ e_f = f$ and, secondly, for any other arrow $ k : A\to S $ with $ g\circ k = f $, since $ g\circ h\circ e_f = f $ too, therefore by monic nature of $ g $, we would have $ k = h\circ e_f $.
\end{proof}
\newpage
\subsection{Internal Structures}\label{IS}
Any mathematical structure which admits definition in set theory can be translated into \emph{internal object} in a suitable category with enough structure. One prime example of such internalization is the group object in a category:
\begin{definition}\label{D-4.9}
	(\textbf{Internal Group Object}) \emph{Suppose} $ \cat{C} $\emph{ is a category with binary products and a terminal object. An object $ G $ in} $ \cat{C} $\emph{ is said to be a group object if there are:}
	\begin{enumerate}
		\item {\emph{An arrow} $ m : G\times G \longrightarrow G$,}
		\item {\emph{An arrow }$ i : G\longrightarrow G $,}
		\item {\emph{An arrow }$ 0 : \bm{1} \longrightarrow G $}
	\end{enumerate}
	\emph{	which satisfy the following three commutative diagrams:}
	\[\begin{tikzcd}
		{G\times G\times G} & {G\times G} & {} & {G\times G} & G \\
		{G\times G} & G && G & {G\times G} \\
		{} & {} && {} & {} \\
		& {G\times G} && G & {G\times\bm{1}} \\
		{G\times \bm{1} } & { G} && {G\times G} \\
		& {} && {}
		\arrow["m\times1", from=1-1, to=1-2]
		\arrow["m", from=1-2, to=2-2]
		\arrow["{1\times m}"', from=1-1, to=2-1]
		\arrow["m"', from=2-1, to=2-2]
		\arrow["{\text{Associativity}}", draw=none, from=3-1, to=3-2]
		\arrow["m", from=1-4, to=1-5]
		\arrow["{\upair{1}{i}}", from=2-4, to=1-4]
		\arrow["{\upair{i}{1}}"', from=2-4, to=2-5]
		\arrow["m"', from=2-5, to=1-5]
		\arrow["{\text{Unital}}", draw=none, from=3-4, to=3-5]
		\arrow["{0\circ !_G}", from=2-4, to=1-5]
		\arrow["1", from=5-2, to=4-4]
		\arrow["{\upair{1}{0}}", from=5-2, to=4-2]
		\arrow["m", from=4-2, to=4-4]
		\arrow["{\upair{0}{1}}"', from=5-2, to=5-4]
		\arrow["m"', from=5-4, to=4-4]
		\arrow["{\text{Identity}}", draw=none, from=6-2, to=6-4]
		\arrow["{g+0 = g = 0+g}"', draw=none, from=6-2, to=6-4]
		\arrow["{g+(-g) = 0 = -g+g}"', draw=none, from=3-4, to=3-5]
		\arrow["{(g+h)+k = g+(h+k)}"', draw=none, from=3-1, to=3-2]
		\arrow["\isomorph"', no head, from=5-1, to=5-2]
		\arrow["\isomorph", no head, from=4-4, to=4-5]
	\end{tikzcd}\]
	\emph{where $ !_G : G\longrightarrow \bm{1} $ is the unique terminal arrow.}
\end{definition}
In a similar fashion, we define an internal meet semilattice in the truth object $ \Omega $ in any topos, which would prove to be very beneficial in the following discussion on generalization of sheaves on an arbitrary topos.\\\\
However, the definition of this internal meet semilattice in $ \Omega $ is not as direct as Definition \ref{D-4.9}. 
\subsubsection{Internal Meet in $ \Omega $}
\begin{definition}\label{D-4.10}
	(\textbf{Internal Meet Semilattice\footnote{a poset which has a meet for any nonempty finite subset.} in $ \Omega $}) Suppose $ \cat{E} $ is a topos and $ \Omega $ is it's truth object. There is an arrow $ \bmeet : \Omega \times \Omega \longrightarrow \Omega$ which is given by the following construction:
	\begin{enumerate}
		\item {By Theorem \ref{T-4.1}, 2, for any $ k : A\to B $ in $ \cat{E} $, $ \inv{k}  $ preserves finite limits, hence we get:
			\begin{align*}
				\inv{k}(S\intrs T) \isomorph \inv{k} (S) \intrs \inv{k}(T)
			\end{align*}	
			\footnote{The "$ \intrs $" is the meet of two subobjects in the lattice (Theorem \ref{T-4.1},1).}for any two subobjects $ S,T $ of $ B $.
		}
		\item {This determines the following functor:
			\begin{align*}
				-\intrs - :\Sub{B}{\cat{E}} \times \Sub{B}{\cat{E}} \longrightarrow \Sub{B}{\cat{E}}
			\end{align*}
			which is natural in $ B $, due to the 1.
		}
		\item {We then have the following diagram:
			\[\begin{tikzcd}
				{\Sub{B}{\cat{E}}\times\Sub{B}{\cat{E}}} & {\Sub{B}{\cat{E}}} \\
				{\homset{\cat{E}}{B}{\Omega}\times \homset{\cat{E}}{B}{\Omega}} & {\homset{\cat{E}}{B}{\Omega}} \\
				{\homset{\cat{E}}{B}{\Omega\times \Omega}} & {\homset{\cat{E}}{B}{\Omega}}
				\arrow["\isomorph", from=1-1, to=2-1]
				\arrow["\isomorph", from=2-1, to=3-1]
				\arrow["{-\intrs-}", from=1-1, to=1-2]
				\arrow["\isomorph", from=1-2, to=2-2]
				\arrow[Rightarrow, no head, from=2-2, to=3-2]
				\arrow["{\bmeet_B}"', from=3-1, to=3-2]
			\end{tikzcd}\]
			which gives rise to the above $ \bmeet_B : \homset{\cat{E}}{B}{\Omega\times \Omega} \longrightarrow \homset{\cat{E}}{B}{\Omega} $, which is also natural in $ B $ because $ -\intrs - $ was too.
		}
		\item {Now, because $ \bmeet_B $ is natural in $ B $, this would translate that $ \bmeet_{(-)} $ is the following natural transformation:
			\begin{align*}
				\bmeet_{(-)} : \homset{\cat{E}}{-}{\Omega\times \Omega} \Longrightarrow \homset{\cat{E}}{-}{\Omega}
			\end{align*}
			But by Yoneda Lemma, we would have that
			\begin{align*}
				\Nat{\homset{\cat{E}}{-}{\Omega \times \Omega}}{\homset{\cat{E}}{-}{\Omega}} \isomorph \homset{\cat{E}}{\Omega \times \Omega}{\Omega}
			\end{align*}
			Therefore by the above, the natural transformation $ \bmeet_{(-)} $ determines a unique arrow $ \bmeet $ as follows:
			\begin{align*}
				\bmeet_{\Omega\times \Omega}(1_{\Omega\times \Omega}) = \bmeet : \Omega \times\Omega \longrightarrow \Omega
			\end{align*}
		}
	\end{enumerate}
	The collection $ \left (\Omega, \bmeet, \true : \bm{1} \longrightarrow \Omega\right ) $ forms what we call an internal meet semilattice in $ \Omega $.
\end{definition}
\begin{remark} ($ \bmeet $ \textbf{gives the characteristic of intersection})
	The internal meet operation $ \bmeet $ in $ \Omega $ gives the characteristic map of the intersection of two subobjects $ S,T $ of some arbitrary object $ B $. That is, if $ s,t: B\longrightarrow \Omega $ are characteristic maps for $ S $ and $ T $ respectively then, the characteristic map for $ S\intrs T $ would be given by:
	\[\begin{tikzcd}
		B & \Omega\times\Omega & \Omega
		\arrow["{\upair{s}{t}}", from=1-1, to=1-2]
		\arrow["\bmeet", from=1-2, to=1-3]
	\end{tikzcd}.\]
	This follows from the construction in Definition \ref{D-4.10} and the trivial natural isomorphism $ \Sub{\cat{E}}{B} \isomorph \homset{\cat{E}}{B}{\Omega} $ (Definition \ref{D-21}, ETI.2).
\end{remark}
\begin{remark}(\textbf{Internal meet in $ PA $})
	Moreover, one can consider an internal meet not just in $ \Omega $ but in any power object $ PA $ in a topos $ \cat{E} $. The construction is roughly similar and is done in Proposition \ref{P-23}.
\end{remark}
\newpage
\subsection{Slice Topos}
One important result in the theory of topoi is that a slice of a topos is itself a topos. Moreover, the change of base functor between two slices provides a way of giving \emph{sets-like} properties of topoi, which we will discuss later.
\begin{theorem}\label{T-10}
	Suppose $ \cat{E} $ is a topos. The slice category $ \cat{E} /B$ is a topos for any object $ B $ in $ \cat{E} $.
\end{theorem}
\begin{proof}
	Let us not derive power object here as it's construction is a bit involved and would defeat the purpose of the notes. It is done in detail in Theorem 1, pp 190 of \cite{MacMoer}. Let's show that $ \cat{E}/B $ has finite limits. For this, the terminal object in $ \cat{E}/B $ is the identity $ 1 : B\to B $. The equalizer of two parallel arrows $ a,c : f\to g $ where $ f:A\to B,g: C\to B $ are two objects in $ \cat{E}/B $ is just the equalizer of $ a,c : A\to C $ in $ \cat{E} $ itself. Binary product of $ a : A\to B $ and $ c: C\to B $ in $ \cat{E}/B $ is given by their pullback in $ \cat{E} $. Hence, $ \cat{E}/B $ has all finite limits.\\
	Next, the subobject classifier of $ \cat{E}/B $ is given as follows: for a subobject $ m : s\rightarrowtail a $ in $ \cat{E}/B $ where $ s : S\to B $ and $ a : A \to B $ is given by the following diagram:
\[\begin{tikzcd}
	S && B & {\bm{1}\times B} \\
	& B \\
	A && {\Omega\times B} & {\Omega\times B}
	\arrow["{!_S}", from=1-1, to=1-3]
	\arrow["m"', tail, from=1-1, to=3-1]
	\arrow["{\chr{m}}"', from=3-1, to=3-3]
	\arrow["{\true_{\cat{E}/B}}", tail, from=1-3, to=3-3]
	\arrow["s"{description}, from=1-1, to=2-2]
	\arrow["a"{description}, from=3-1, to=2-2]
	\arrow["\isomorph", no head, from=1-3, to=1-4]
	\arrow["1"{description}, from=1-3, to=2-2]
	\arrow["{\upair{\true}{1}}", tail, from=1-4, to=3-4]
	\arrow[Rightarrow, no head, from=3-3, to=3-4]
	\arrow["{\pi_2}"{description}, from=3-3, to=2-2]
\end{tikzcd}\]
	Hence, the subobject classifier in $ \cat{E}/B $ is $ \upair{\true}{1} : \bm{1} \times B \rightarrowtail \Omega \times B$.
\end{proof}
\subsubsection{Change of base functor}
The following theorem would be used to derive \emph{sets-like} properties of a topos:
\begin{theorem}\label{T-11}
	Suppose $ \cat{E} $ is a topos and $ k : B\longrightarrow A $ is any arrow in it. The change of base functor defined as:
	\begin{align*}
		k^{*} : \cat{E}/A &\longrightarrow \cat{E}/B
	\end{align*}
which takes an object of $ \cat{E}/A $ to it's pullback along $ k $, that is:
\[\begin{tikzcd}
	X && {k^*(X)} & X \\
	A && B & A
	\arrow[""{name=0, anchor=center, inner sep=0}, "{\circled{x}}"', from=1-1, to=2-1]
	\arrow[""{name=1, anchor=center, inner sep=0}, "{\circled{\pi_1}}", from=1-3, to=2-3]
	\arrow["k"', from=2-3, to=2-4]
	\arrow["x", from=1-4, to=2-4]
	\arrow["{\pi_2}", from=1-3, to=1-4]
	\arrow["\lrcorner"{anchor=center, pos=0.125}, draw=none, from=1-3, to=2-4]
	\arrow[shorten <=13pt, shorten >=13pt, maps to, from=0, to=1]
\end{tikzcd}\]
has both left and right adjoints where left adjoint is given by pre-composition with $ k $:
\[\begin{tikzcd}
	{\cat{E}/B} && {\cat{E}/A} && {\cat{E}/A} && {\cat{E}/B}
	\arrow[""{name=0, anchor=center, inner sep=0}, "{\sum_k := k\circ-}", curve={height=-18pt}, from=1-1, to=1-3]
	\arrow[""{name=1, anchor=center, inner sep=0}, "{k^*}", curve={height=-18pt}, from=1-3, to=1-1]
	\arrow[""{name=2, anchor=center, inner sep=0}, "{k^*}", curve={height=-18pt}, from=1-5, to=1-7]
	\arrow[""{name=3, anchor=center, inner sep=0}, "{\prod_k}", curve={height=-18pt}, from=1-7, to=1-5]
	\arrow["\dashv"{anchor=center, rotate=-90}, draw=none, from=0, to=1]
	\arrow["\dashv"{anchor=center, rotate=-90}, draw=none, from=2, to=3]
\end{tikzcd}.\]
Moreover, $ k^{*} $ is a logical morphism of topoi (Definition \ref{D-25}).
\end{theorem}
\begin{proof}
	Since a slice category in a topos is also a topos (Theorem \ref{T-10}), therefore each slice is cartesian closed, so we have both the adjoints by Theorem 4, Chapter I, pp 59 of \cite{MacMoer}. It remains to be seen that $ k^{*} $ is a logical morphism. \\
	Since $ k^{*} $ is a right adjoint, therefore it preserves all finite limits. To see that $ k^{*} $ preserves the subobject classifier, first take the pullback of each member of the subobject classifier diagram in $ \cat{E}/A $ so that we obtain a diagram in $ \cat{E}/B $. Since pullbacks preserves monic, so we just need to show that $ B\times_{A}(\Omega \times A) \isomorph \Omega \times B $, where $ \Omega $ is the truth object of $ \cat{E} $. Now since it can be verified quite easily by universality of pullbacks that $ B\times_{A} (\Omega \times A) \isomorph \Omega \times (B\times_{A} A) \isomorph \Omega \times B$, we hence have that $ k^{*} $ preserves the subobject classifier.
	Lastly, we need to show that $ k^{*} $ preserves exponentials, that is, there is a following natural isomorphism:
	\begin{align*}
		k^{*}(-) \circ (-)^{x} \isomorph (-)^{k^{*}x} \circ k^{*}(-)
	\end{align*}
where $ y : Y\to A $ and $ x : X\to A $ are objects in slice $ \cat{E}/A $. If we could show that the corresponding left adjoints of the above equation is also isomorphic, then the result would follow. Therefore the left transpose equivalent to be shown is:
\begin{align*}
	(X\times_{A} -) \circ \Sigma_k(-) \isomorph \Sigma_k(-)\circ  (k^{*}(X) \times_B -). 
\end{align*}
For some object $ z : Z\to B $ in $ \cat{E}/B $, we have by definition of $ \Sigma_k $: 
\begin{align*}
	(X\times_A -) (\Sigma_kz) = (X\times_A -)(k\circ z) = \overline{k\circ z}_x : X\times_A Z \to X
\end{align*}
where $ \overline{k\circ z}_x $ is the pullback of $ k\circ z : Z\to A $ along $ x : X \to A $. Similarly, we have:
\begin{align*}
	\Sigma_k (-)\circ (k^{*}(X) \times_B z) = \Sigma_{k} \left (k^{*}(X) \times_B z\right ) = k\circ \left (\overline{z}_{\overline{x}_k}\right )
\end{align*}
where $ \overline{z}_{\overline{x}_k} $ is the pullback of $ z : Z\to B $ along the pullback of $ x $ along $ k $. Clearly, both the objects are isomorphic. Hence the left adjoint commutes, then so does the right adjoint.
\end{proof}
\subsubsection{"Sets-like" properties of topoi}
From Theorem \ref{T-11}, we can derive various \emph{sets-like} properties of topoi, the proofs of all which depends on the translation of the problem from the topos $ \cat{E} $ to some appropriate slice topos and using the fact that pullback would preserve both finite limits and colimits and would be a logical morphism. We derive some such results below.\\\\
The first result draws motivation from the fact that in $ \cat{Sets} $, for a surjective function $ e : A\to B$ and any function $ f : C\to B $, the set $ C\times_B A = \{(c,a)\in C\times A\;\vert\; e(a) = f(c)\}$ has an obvious surjection to $ C $ given by $ (c,a) \mapsto c $ as for all $ c\in C $, $ \exists a\in A $ with $ f(c) = e(a) $ because $ \image f \subseteq B = \image e $. It's generalization in an arbitrary topos is the following:
\begin{proposition}\label{P-19}
	In a topos $ \cat{E} $, the pullback of an epimorphism is an epimorphism.
\end{proposition}
\begin{proof}
	Suppose $ e : A\to B $ is an epimorphism in a topos $ \cat{E} $. The fact that $ e $ is an epimorphism can be equivalently stated as the pushout condition on the left below:
	\[\begin{tikzcd}
		B & B &&& {1_B} & {1_B} \\
		B & A &&& {1_B} & e
		\arrow[""{name=0, anchor=center, inner sep=0}, "e"', two heads, from=2-2, to=1-2]
		\arrow["e", two heads, from=2-2, to=2-1]
		\arrow["\ulcorner"{anchor=center, pos=0.125}, draw=none, from=1-1, to=2-2]
		\arrow["1", from=2-1, to=1-1]
		\arrow["1"', from=1-2, to=1-1]
		\arrow["e"', two heads, from=2-6, to=1-6]
		\arrow["e", two heads, from=2-6, to=2-5]
		\arrow[""{name=1, anchor=center, inner sep=0}, "1", from=2-5, to=1-5]
		\arrow["1"', from=1-6, to=1-5]
		\arrow["\ulcorner"{anchor=center, pos=0.125}, draw=none, from=1-5, to=2-6]
		\arrow["{\text{In Slice } \cat{E}/B}", shorten <=19pt, shorten >=19pt, from=0, to=1]
	\end{tikzcd}\]
	Now treat $ e $ as an object in the slice $ \cat{E}/B $, where this pushout condition would translate to the pushout condition on the right. For any arrow $ f : C\to B $, the corresponding change of base $ f^{*} : \cat{E}/B \longrightarrow \cat{E}/C$ preserves colimits (Theorem \ref{T-11}), therefore we would have the same pushout condition in $ \cat{E}/C $ by the pullback along $ f $. We hence have that epics are pullback stable in a topos.
\end{proof}
Next, consider the initial object in $ \cat{Sets} $, which is the null set $ \varnothing $. By definition, any map $ f : A\to \varnothing $ is an isomorphism by default. In general:
\begin{proposition}\label{P-20}
	In a topos $ \cat{E} $, any arrow $ k : A\to 0 $ is an isomorphism.
\end{proposition}
\begin{proof}
	Denote $ !_A : 0 \to A $ as the unique initial arrow. Since $ k\circ !_A = 1_0$ by uniqueness, we hence need to show $ !_A \circ k = 1_A $. Focus on the objects $ k $ and $ 1_0 $ in the topos $ \cat{E}/0 $. Clearly, $ 1_0 $ is the initial object in $ \cat{E}/0 $. But it is also terminal in this slice. Now, since the pullback $ !_A : 0 \to A $ of $ 1_0 : 0 \to 0$ along $ k : A\to 0 $ would also be initial and final in $ \cat{E}/A $, then, because $ 1_A : A\to A  $ and $ k : A\to 0 $ forms a cone over this pullback, therefore we get $ !_A\circ k = 1_A $.
\end{proof}
Another consequence in $ \cat{Sets} $ of null-set $ \varnothing $ is that the unique arrow $ f : \varnothing \to A $ is always injective. In general:
\begin{corollary}\label{C-3}
	In a topos $ \cat{E} $, the unique arrow $ !_A :0\longrightarrow A $ for any object $ A $ is a monomorphism.
\end{corollary}
\begin{proof}
	Let $ x,y : B\to 0 $ be two arrows such that $ !_A \circ x = !_A\circ y $. But since $ x $ and $ y $ are isomorphisms by Proposition \ref{P-20}, therefore, $ \inv{x},\inv{y} : 0 \to B $ are two arrows from initial $ 0 $, hence $ \inv{x} = \inv{y} \implies x = y$, so $ !_A $ is a monic.  
\end{proof}
In $ \cat{Sets} $, the product of two surjective functions is also surjective. This holds in general:
\begin{proposition}\label{P-21}
	In a topos $ \cat{E} $, if $ f : X\rightarrow Y $ and $ g : W\rightarrow Z $ are epimorphisms, then $ f\times g : X\times W \rightarrow Y\times Z $ is also an epimorphism.
\end{proposition}
\begin{proof}
	Note $ f\times g = (f\times 1_Z) \circ (1_X\times g) $. So if we could show that $ f\times 1_Z $ and $ 1_X\times g $ are epics then we would be done. To this end, note that $ (X\times Z) \times_Z W \isomorph X\times W$ follows from universality of pullbacks and products, and then we have the following diagram:
	\[\begin{tikzcd}
		{X\times W} & {(X\times Z)\times_ZW} & W \\
		{X\times Z} & {X\times Z} & Z
		\arrow["g", two heads, from=1-3, to=2-3]
		\arrow["{p_2}"', from=2-2, to=2-3]
		\arrow["{\pi_1}"', two heads, from=1-2, to=2-2]
		\arrow["{\pi_2}", from=1-2, to=1-3]
		\arrow["\lrcorner"{anchor=center, pos=0.125}, draw=none, from=1-2, to=2-3]
		\arrow["\isomorph"', no head, from=1-2, to=1-1]
		\arrow["{1_X\times g}"', from=1-1, to=2-1]
		\arrow[Rightarrow, no head, from=2-1, to=2-2]
	\end{tikzcd}\]
where $ \pi_1 \isomorph 1_X\times g $. But by Proposition \ref{P-19}, $ \pi_1 $ is an epimorphism, then so is $ 1_X\times g $. Similarly, $ f\times 1_Z $ is also an epimorphism, so we have our result.
\end{proof}
We finally have the following important result:
\begin{theorem}\label{T-12}
	In a topos $ \cat{E} $, every epimorphism is the coequalizer of it's kernel pair.
\end{theorem}
\begin{proof}
	Suppose $ f : C\to B $ is an epimorphism. Let $ \pi_1,\pi_2 : C\times_B C \to C$ be the kernel pair of $ f $. The coequalizer of this kernel pair would be denoted:
	\[\begin{tikzcd}
		{C\times_BC} & C & Q
		\arrow["{\pi_2}"', shift right=2, from=1-1, to=1-2]
		\arrow["{\pi_1}", shift left=2, from=1-1, to=1-2]
		\arrow["c", from=1-2, to=1-3]
	\end{tikzcd}.\]
	Let's translate this coequalizer to a diagram in slice topos $ \cat{E}/B $. For this, the object would be the epic $ f : C\to B$. Also notice that $ f $ being epi means that $ !_f : f\to 1 $ is an epic in $ \cat{E}/B $. Take the product $ f\times f $ in $ \cat{E}/B $. By Theorem \ref{T-10}, the product is exactly the pullback of $ f $ along itself. The projections $ f\times f \to f $ is therefore exactly the $ \pi_1 $ and $ \pi_2 $ as above. Hence, to find the coequalizer in $ \cat{E} $ of pullback projections $ \pi_1,\pi_2 : C\times_B C\to C $ is same as finding coequalizer of product projections $\pi_1,\pi_2 : f\times f \to f$ in $ \cat{E}/B $.\\
	Now, for any topos $ \cat{F} $, let $ X $ be an object and $ p_1,p_2 : X\times X \to X $ be the projection of the product. Let the coequalizer of $ p_1,p_2 $ be denoted as $ L $ as shown:
	\[\begin{tikzcd}
		{X\times X} & X & L
		\arrow["{p_2}"', shift right=2, from=1-1, to=1-2]
		\arrow["{p_1}", shift left=2, from=1-1, to=1-2]
		\arrow["v", from=1-2, to=1-3]
	\end{tikzcd}\]
	We wish to show that when $ !_X :X \to 1 $ is an epic, then $ L \isomorph 1$. Our motivation for this stems from the fact that $ !_f : f\to 1 $ is an epic in $ \cat{E} /B$ and $ L\isomorph 1 $ would mean directly that $ f : C\to B $ would be the coequalizer of $ \pi_1, \pi_2 : C\times_BC\to C $. To this end, we use the fact that in a topos, any arrow which is both monic and epic is an isomorphism. $!_L: L\to 1 $ is monic, because if we let two arrows $ x $ and $ y $ in $ \cat{F} $ with domain some arbitrary $ K $ be such that $ !_L\circ x = !_L\circ y $, which just means that $ !_K = !_K $, then we have the following:
	\[\begin{tikzcd}
		& K \\
		{X\times X} & {L\times L} \\
		X & L
		\arrow["{\upair{x}{y}}", from=1-2, to=2-2]
		\arrow["{q_i}", from=2-2, to=3-2]
		\arrow["{v\times v}", two heads, from=2-1, to=2-2]
		\arrow["{p_i}"', from=2-1, to=3-1]
		\arrow["v"', two heads, from=3-1, to=3-2]
	\end{tikzcd}\;\text{$ i=1,2 $}\]
 	Remember the coequalizer (here, $ q $) is always an epimorphism and so $ q\times q $ is an epimorphism by Proposition \ref{P-21}. Since both the squares commute, $ q_1\circ(v\times v) = v\circ p_1 = v\circ p_2 = q_2\circ(v\times v) \implies q_1 = q_2$. Therefore $ q_1 \circ (\upair{x}{y}) = q_2\circ (\upair{x}{y}) \implies x = y$, so that $ !_L :L \to 1 $ is a monic. Now to show $ !_L $ is an epic, if $ a\circ !_L = b\circ !_L  $, then $ a\circ !_L \circ v = b\circ !_L \circ v \implies a\circ !_X = b\circ !_X \implies a = b$ as $ !_X : X\to 1 $ is a given epic. Hence $ L\isomorph 1 $. 
\end{proof}
\newpage
\subsection{Internal Lattices, Heyting Algebras and Subobject Lattice}
We saw in Section \ref{IS} that one can define internal group objects and internal meet "in" some object. In particular, we saw internal meet in the truth object $ \Omega $, which determines the characteristic of intersection or meet of two subobjects in a given subobject lattice $ \Sub{\cat{E}}{B} $. This would be important to define the closure of a subobject when generalizing sheaves over arbitrary topoi.\\
 We now continue this line of \emph{internalization} of algebraic structures and, in the same vein as internal group object, introduce lattice and Heyting algebra objects and then study the "external" subobject lattice $ \Sub{A}{\cat{E}} $ in more detail.
\subsubsection{Internal Lattices} 
\begin{definition}
	(\textbf{Internal Lattice}) \emph{Suppose} $ \cat{C} $\emph{ is a category with finite limits. An internal lattice or a lattice object $ \left (L,\meet ,\join\right ) $ is an object $ L $ in} $ \cat{C} $ \emph{with two arrows
	\begin{align*}
		\bmeet: L\times L \longrightarrow L \;\;\;\;\&\;\;\;\; \bjoin :L\times L \longrightarrow L
	\end{align*}
which satisfy the following commutative diagrams:}
% https://q.uiver.app/?q=WzAsNDIsWzAsMSwiTFxcdGltZXMgTCBcXHRpbWVzICBMIl0sWzIsMSwiTFxcdGltZXMgTCJdLFswLDIsIkxcXHRpbWVzIEwiXSxbMiwyLCJMIl0sWzQsMSwiTFxcdGltZXMgTCBcXHRpbWVzICBMIl0sWzYsMSwiTFxcdGltZXMgTCJdLFs0LDIsIkxcXHRpbWVzIEwiXSxbNiwyLCJMIl0sWzAsMF0sWzIsMF0sWzYsMF0sWzQsMF0sWzIsM10sWzQsM10sWzAsNCwiTFxcdGltZXMgTCJdLFsyLDQsIkwiXSxbMCw1LCJMXFx0aW1lcyBMIl0sWzQsNCwiTFxcdGltZXMgTCJdLFs0LDUsIkxcXHRpbWVzIEwiXSxbNiw0LCJMIl0sWzAsM10sWzYsM10sWzMsNl0sWzIsNl0sWzQsNl0sWzAsNl0sWzYsNl0sWzAsOCwiTFxcdGltZXMgTCJdLFswLDcsIkxcXHRpbWVzIEwiXSxbMiw3LCJMIl0sWzQsNywiTFxcdGltZXMgTCJdLFs0LDgsIkxcXHRpbWVzIEwiXSxbNiw3LCJMIl0sWzIsOV0sWzQsOV0sWzEsMTAsIkwiXSxbNSwxMCwiTFxcdGltZXMgTCJdLFsxLDEyLCJMIl0sWzUsMTIsIkxcXHRpbWVzIEwiXSxbNSwxMSwiTFxcdGltZXMgTFxcdGltZXMgTCJdLFsxLDExLCJMXFx0aW1lcyBMIl0sWzMsMTEsIkxcXHRpbWVzIExcXHRpbWVzIEwiXSxbMCwxLCIxXFx0aW1lcyBcXG1lZXQiXSxbMCwyLCJcXG1lZXQgXFx0aW1lcyAxIiwyXSxbMiwzLCJcXG1lZXQiLDJdLFsxLDMsIlxcbWVldCJdLFs0LDUsIjFcXHRpbWVzIFxcam9pbiJdLFs2LDcsIlxcam9pbiIsMl0sWzQsNiwiXFxqb2luIFxcdGltZXMgMSIsMl0sWzUsNywiXFxqb2luICJdLFs4LDksInhcXG1lZXQoeVxcbWVldCB6KT0oeFxcbWVldCB5KVxcbWVldCB6IiwyLHsic3R5bGUiOnsiYm9keSI6eyJuYW1lIjoibm9uZSJ9LCJoZWFkIjp7Im5hbWUiOiJub25lIn19fV0sWzExLDEwLCJ4XFxqb2luKHlcXGpvaW4geik9KHhcXGpvaW4geSlcXGpvaW4geiIsMix7InN0eWxlIjp7ImJvZHkiOnsibmFtZSI6Im5vbmUifSwiaGVhZCI6eyJuYW1lIjoibm9uZSJ9fX1dLFs5LDExLCJcXHRleHR7QXNzb2NpYXRpdml0eX0iLDAseyJzdHlsZSI6eyJib2R5Ijp7Im5hbWUiOiJub25lIn0sImhlYWQiOnsibmFtZSI6Im5vbmUifX19XSxbMTIsMTMsIlxcdGV4dHtDb21tdXRhdGl2aXR5fSIsMCx7InN0eWxlIjp7ImJvZHkiOnsibmFtZSI6Im5vbmUifSwiaGVhZCI6eyJuYW1lIjoibm9uZSJ9fX1dLFsxNCwxNSwiXFxtZWV0Il0sWzE2LDE1LCJcXG1lZXQiLDJdLFsxNCwxNiwiXFx0YXUiLDJdLFsxNywxOSwiXFxqb2luIl0sWzE4LDE5LCJcXGpvaW4iLDJdLFsxNywxOCwiXFx0YXUiLDJdLFsyMCwxMiwieFxcbWVldCB5ID0geVxcbWVldCB4IiwyLHsic3R5bGUiOnsiYm9keSI6eyJuYW1lIjoibm9uZSJ9LCJoZWFkIjp7Im5hbWUiOiJub25lIn19fV0sWzEzLDIxLCJ4XFxqb2luIHkgPSB5XFxqb2luIHgiLDIseyJzdHlsZSI6eyJib2R5Ijp7Im5hbWUiOiJub25lIn0sImhlYWQiOnsibmFtZSI6Im5vbmUifX19XSxbMjMsMjQsIlxcdGV4dHtJZGVtcG90ZW5jeX0iLDAseyJzdHlsZSI6eyJib2R5Ijp7Im5hbWUiOiJub25lIn0sImhlYWQiOnsibmFtZSI6Im5vbmUifX19XSxbMjUsMjMsInhcXG1lZXQgeCA9IHgiLDIseyJzdHlsZSI6eyJib2R5Ijp7Im5hbWUiOiJub25lIn0sImhlYWQiOnsibmFtZSI6Im5vbmUifX19XSxbMjQsMjYsInhcXGpvaW4geCA9IHgiLDIseyJzdHlsZSI6eyJib2R5Ijp7Im5hbWUiOiJub25lIn0sImhlYWQiOnsibmFtZSI6Im5vbmUifX19XSxbMjgsMjksIlxcbWVldCJdLFsyOCwyNywiMVxcdGltZXMgMSIsMl0sWzI3LDI5LCJcXG1lZXQiLDJdLFszMCwzMiwiXFxqb2luIl0sWzMxLDMyLCJcXGpvaW4iLDJdLFszMCwzMSwiMVxcdGltZXMgMSIsMl0sWzMzLDM0LCJcXHRleHR7QWJzb3JwdGlvbn0iLDAseyJzdHlsZSI6eyJib2R5Ijp7Im5hbWUiOiJub25lIn0sImhlYWQiOnsibmFtZSI6Im5vbmUifX19XSxbMzMsMzQsInhcXG1lZXQgKHlcXGpvaW4geCkgPSB4ID0gKHhcXG1lZXQgeSlcXGpvaW4geCIsMix7InN0eWxlIjp7ImJvZHkiOnsibmFtZSI6Im5vbmUifSwiaGVhZCI6eyJuYW1lIjoibm9uZSJ9fX1dLFs0MCw0MSwiXFxEZWx0YV9MXFx0aW1lcyAxIiwyXSxbNDEsMzksIjFcXHRpbWVzIFxcdGF1IiwyXSxbNDAsMzUsInBfMSJdLFs0MCwzNywicF8xIiwyXSxbMzksMzYsIjFcXHRpbWVzIFxcbWVldCAiLDJdLFszNiwzNSwiXFxqb2luIiwyXSxbMzgsMzcsIlxcbWVldCJdLFszOSwzOCwiXFxqb2luIFxcdGltZXMgMSJdXQ==
\[\begin{tikzcd}
	{} && {} && {} && {} \\
	{L\times L \times  L} && {L\times L} && {L\times L \times  L} && {L\times L} \\
	{L\times L} && L && {L\times L} && L \\
	{} && {} && {} && {} \\
	{L\times L} && L && {L\times L} && L \\
	{L\times L} &&&& {L\times L} \\
	{} && {} & {} & {} && {} \\
	{L\times L} && L && {L\times L} && L \\
	{L\times L} &&&& {L\times L} \\
	&& {} && {} \\
	& L &&&& {L\times L} \\
	& {L\times L} && {L\times L\times L} && {L\times L\times L} \\
	& L &&&& {L\times L}
	\arrow["{1\times \meet}", from=2-1, to=2-3]
	\arrow["{\meet \times 1}"', from=2-1, to=3-1]
	\arrow["\meet"', from=3-1, to=3-3]
	\arrow["\meet", from=2-3, to=3-3]
	\arrow["{1\times \join}", from=2-5, to=2-7]
	\arrow["\join"', from=3-5, to=3-7]
	\arrow["{\join \times 1}"', from=2-5, to=3-5]
	\arrow["{\join }", from=2-7, to=3-7]
	\arrow["{x\meet(y\meet z)=(x\meet y)\meet z}"', draw=none, from=1-1, to=1-3]
	\arrow["{x\join(y\join z)=(x\join y)\join z}"', draw=none, from=1-5, to=1-7]
	\arrow["{\text{Associativity}}", draw=none, from=1-3, to=1-5]
	\arrow["{\text{Commutativity}}", draw=none, from=4-3, to=4-5]
	\arrow["\meet", from=5-1, to=5-3]
	\arrow["\meet"', from=6-1, to=5-3]
	\arrow["\tau"', from=5-1, to=6-1]
	\arrow["\join", from=5-5, to=5-7]
	\arrow["\join"', from=6-5, to=5-7]
	\arrow["\tau"', from=5-5, to=6-5]
	\arrow["{x\meet y = y\meet x}"', draw=none, from=4-1, to=4-3]
	\arrow["{x\join y = y\join x}"', draw=none, from=4-5, to=4-7]
	\arrow["{\text{Idempotency}}", draw=none, from=7-3, to=7-5]
	\arrow["{x\meet x = x}"', draw=none, from=7-1, to=7-3]
	\arrow["{x\join x = x}"', draw=none, from=7-5, to=7-7]
	\arrow["\meet", from=8-1, to=8-3]
	\arrow["{1\times 1}"', from=8-1, to=9-1]
	\arrow["\meet"', from=9-1, to=8-3]
	\arrow["\join", from=8-5, to=8-7]
	\arrow["\join"', from=9-5, to=8-7]
	\arrow["{1\times 1}"', from=8-5, to=9-5]
	\arrow["{\text{Absorption}}", draw=none, from=10-3, to=10-5]
	\arrow["{x\meet (y\join x) = x = (x\meet y)\join x}"', draw=none, from=10-3, to=10-5]
	\arrow["{\Delta_L\times 1}"', from=12-2, to=12-4]
	\arrow["{1\times \tau}"', from=12-4, to=12-6]
	\arrow["{p_1}", from=12-2, to=11-2]
	\arrow["{p_1}"', from=12-2, to=13-2]
	\arrow["{1\times \meet }"', from=12-6, to=11-6]
	\arrow["\join"', from=11-6, to=11-2]
	\arrow["\meet", from=13-6, to=13-2]
	\arrow["{\join \times 1}", from=12-6, to=13-6]
\end{tikzcd}\]
\emph{where $ \Delta_L := \upair{1}{1} : L \longrightarrow L\times L $ is the diagonal map and $ \tau : L\times L \longrightarrow L\times L $ is the twist arrow, given as $ \tau := \upair{p_2}{p_1} $ where $ p_1,p_2 $ are the projection arrows of $ L\times L $. }
\end{definition}
\newpage
An internal lattice can also have \emph{top} and \emph{bottom} elements:
\begin{definition}
	(\textbf{Internal Lattice with $ \top $ and $ \bot $})\emph{ Suppose category} $ \cat{C} $ \emph{has finite limits and $ L $ is an internal lattice in} $ \cat{C} $.\emph{ $ L $ is said to be an internal lattice with top $ (\top) $ and bottom $ (\bot) $ elements if there are additionally two arrows $ \top: 1\longrightarrow L $ and $ \bot : 1\longrightarrow L $ which satisfy the following commutative diagrams:}
	\[\begin{tikzcd}
		& {L\times L} && L &&& {L\times L} && L \\
		L & {1\times L} &&&& L & {1\times L} \\
		& {} && {} &&& {} && {}
		\arrow["\meet", from=1-2, to=1-4]
		\arrow["{\top \times 1}", from=2-2, to=1-2]
		\arrow["1"', from=2-2, to=1-4]
		\arrow["\isomorph"', no head, from=2-1, to=2-2]
		\arrow["{\top \meet x = x}", draw=none, from=3-2, to=3-4]
		\arrow["{\bot \times 1}", from=2-7, to=1-7]
		\arrow["\isomorph"', no head, from=2-6, to=2-7]
		\arrow["\join", from=1-7, to=1-9]
		\arrow["1"', from=2-7, to=1-9]
		\arrow["{\bot \join x = x}", draw=none, from=3-7, to=3-9]
	\end{tikzcd}.\]
\end{definition}
We now define an internal Heyting algebra object:
\subsubsection{Internal Heyting Algebras}
\begin{definition}
	(\textbf{Internal Heyting Algebra - I}) \emph{Suppose} $ \cat{C} $\emph{ is a category with finite limits and $ H $ is an internal lattice with $ \top $ and $ \bot $ in} $ \cat{C} $.\emph{ $ H$ is then said to be an internal Heyting algebra if there is an additional arrow $ (\Rightarrow): H\times H \longrightarrow H$ such that the following commutes}
	% https://q.uiver.app/?q=WzAsMjcsWzEsMCwiSFxcdGltZXMgSCJdLFsyLDAsIkgiXSxbMSwxLCJIIl0sWzEsMl0sWzIsMl0sWzQsMSwiSFxcdGltZXMgSCJdLFs2LDEsIkgiXSxbNCwwLCJIXFx0aW1lcyBIXFx0aW1lcyBIIl0sWzYsMCwiSFxcdGltZXMgSCJdLFs0LDJdLFs2LDJdLFswLDMsIkhcXHRpbWVzIEhcXHRpbWVzIEgiXSxbMSwzLCJIXFx0aW1lcyBIXFx0aW1lcyBIIl0sWzIsMywiSFxcdGltZXMgSCJdLFswLDUsIkhcXHRpbWVzIEgiXSxbMiw1LCJIIl0sWzAsNl0sWzIsNl0sWzQsNSwiSFxcdGltZXMgSFxcdGltZXMgSCJdLFs1LDUsIkhcXHRpbWVzIEgiXSxbNiw1LCJIIl0sWzQsNCwiSFxcdGltZXMgSFxcdGltZXMgSFxcdGltZXMgSCJdLFs0LDMsIkhcXHRpbWVzIEhcXHRpbWVzIEhcXHRpbWVzIEgiXSxbNSwzLCJIXFx0aW1lcyBIXFx0aW1lcyBIIl0sWzYsMywiSFxcdGltZXMgSCJdLFs0LDZdLFs2LDZdLFswLDEsIihcXFJpZ2h0YXJyb3cpIl0sWzIsMCwiXFxEZWx0YV9IIl0sWzIsMSwiXFx0b3BcXGNpcmMgIV9IIiwyXSxbMyw0LCIoeFxcUmlnaHRhcnJvdyB4KSA9IFxcdG9wICIsMCx7InN0eWxlIjp7ImJvZHkiOnsibmFtZSI6Im5vbmUifSwiaGVhZCI6eyJuYW1lIjoibm9uZSJ9fX1dLFs1LDYsIlxcbWVldCIsMl0sWzUsNywiXFxEZWx0YV9IXFx0aW1lcyAxIl0sWzcsOCwiMVxcdGltZXMoXFxSaWdodGFycm93KSJdLFs4LDYsIlxcbWVldCJdLFs5LDEwLCJ4XFxtZWV0ICh4XFxSaWdodGFycm93IHkpID0geFxcbWVldCB5IiwwLHsic3R5bGUiOnsiYm9keSI6eyJuYW1lIjoibm9uZSJ9LCJoZWFkIjp7Im5hbWUiOiJub25lIn19fV0sWzExLDEyLCIxXFx0aW1lcyBcXHRhdSJdLFsxMiwxMywiMVxcdGltZXMgKFxcUmlnaHRhcnJvdykiXSxbMTQsMTEsIlxcRGVsdGFfSFxcdGltZXMgMSJdLFsxMywxNSwiXFxtZWV0Il0sWzE0LDE1LCJwXzEiLDJdLFsxNiwxNywieFxcbWVldCh5XFxSaWdodGFycm93IHgpID0geCIsMCx7InN0eWxlIjp7ImJvZHkiOnsibmFtZSI6Im5vbmUifSwiaGVhZCI6eyJuYW1lIjoibm9uZSJ9fX1dLFsxOCwxOSwiMVxcdGltZXMgXFxtZWV0IiwyXSxbMTksMjAsIihcXFJpZ2h0YXJyb3cpIiwyXSxbMTgsMjEsIlxcRGVsdGFfSFxcdGltZXMgMVxcdGltZXMgMSJdLFsyMSwyMiwiMVxcdGltZXMgXFx0YXVcXHRpbWVzIDEiXSxbMjIsMjMsIihcXFJpZ2h0YXJyb3cpXFx0aW1lcyAxXFx0aW1lcyAxIl0sWzIzLDI0LCIxXFx0aW1lcyAoXFxSaWdodGFycm93KSJdLFsyNCwyMCwicF8yIl0sWzI1LDI2LCJ4XFxSaWdodGFycm93KHlcXG1lZXQgeikgPSAoeFxcUmlnaHRhcnJvdyB5KSBcXG1lZXQgKHhcXFJpZ2h0YXJyb3cgeikiLDAseyJzdHlsZSI6eyJib2R5Ijp7Im5hbWUiOiJub25lIn0sImhlYWQiOnsibmFtZSI6Im5vbmUifX19XV0=
	\[\begin{tikzcd}
		& {H\times H} & H && {H\times H\times H} && {H\times H} \\
		& H &&& {H\times H} && H \\
		& {} & {} && {} && {} \\
		{H\times H\times H} & {H\times H\times H} & {H\times H} && {H\times H\times H\times H} & {H\times H\times H} & {H\times H} \\
		&&&& {H\times H\times H\times H} \\
		{H\times H} && H && {H\times H\times H} & {H\times H} & H \\
		{} && {} && {} && {}
		\arrow["{(\Rightarrow)}", from=1-2, to=1-3]
		\arrow["{\Delta_H}", from=2-2, to=1-2]
		\arrow["{\top\circ !_H}"', from=2-2, to=1-3]
		\arrow["{(x\Rightarrow x) = \top }", draw=none, from=3-2, to=3-3]
		\arrow["\meet"', from=2-5, to=2-7]
		\arrow["{\Delta_H\times 1}", from=2-5, to=1-5]
		\arrow["{1\times(\Rightarrow)}", from=1-5, to=1-7]
		\arrow["\meet", from=1-7, to=2-7]
		\arrow["{x\meet (x\Rightarrow y) = x\meet y}", draw=none, from=3-5, to=3-7]
		\arrow["{1\times \tau}", from=4-1, to=4-2]
		\arrow["{1\times (\Rightarrow)}", from=4-2, to=4-3]
		\arrow["{\Delta_H\times 1}", from=6-1, to=4-1]
		\arrow["\meet", from=4-3, to=6-3]
		\arrow["{p_1}"', from=6-1, to=6-3]
		\arrow["{x\meet(y\Rightarrow x) = x}", draw=none, from=7-1, to=7-3]
		\arrow["{1\times \meet}"', from=6-5, to=6-6]
		\arrow["{(\Rightarrow)}"', from=6-6, to=6-7]
		\arrow["{\Delta_H\times 1\times 1}", from=6-5, to=5-5]
		\arrow["{1\times \tau\times 1}", from=5-5, to=4-5]
		\arrow["{(\Rightarrow)\times 1\times 1}", from=4-5, to=4-6]
		\arrow["{1\times (\Rightarrow)}", from=4-6, to=4-7]
		\arrow["{p_2}", from=4-7, to=6-7]
		\arrow["{x\Rightarrow(y\meet z) = (x\Rightarrow y) \meet (x\Rightarrow z)}", draw=none, from=7-5, to=7-7]
	\end{tikzcd}\]
    \emph{where $ \Delta_H: H\longrightarrow H\times H $ is the diagonal map and $ \tau : H\times H\longrightarrow H\times H $ is the twist map.}
\end{definition}
For any lattice $ (S,\meet,\join) $, we have an partial order induced by the meet, given by $ x\le y \iff x = x\meet y $. We can do the same in an internal lattice:
\begin{definition}\label{D-35}
	(\textbf{Internal Partial Order in an Internal Lattice})\emph{ Suppose $ (L,\meet,\join) $ is an internal lattice in a category} $ \cat{C} $ \emph{with finite limits. Then $ \left (L,\le_L\right ) $ is called an internal partial order in $ L $ or an internal poset where the internal order in $ L $,$ \le_L $, is given by the following equalizer diagram\footnote{The diagram is the \emph{internal way} of saying that for $ x,y\in L $, $ x\le y \iff x = x\meet y$.}:}
	\[\begin{tikzcd}
		{\le_L} & {L\times L} & L
		\arrow["e", tail, from=1-1, to=1-2]
		\arrow["\meet", shift left=2, from=1-2, to=1-3]
		\arrow["{p_1}"', shift right=2, from=1-2, to=1-3]
	\end{tikzcd}.\]
\end{definition}
\begin{remark}
	Note that internal partial order $ \le_L $ is hence a subobject of $ L\times L $.
\end{remark}
One may remember from the usual set-theoretic definition of a Heyting algebra $ (H,\meet,\join,\top,\bot,\Rightarrow) $ that it was just a lattice with top and bottom where each object additionally was exponentiable, meaning that for the partial order induced from the meet of the lattice, $ \forall x,y\in H $, $ \exists (x\Rightarrow y)\in H $ such that for any $ z\in H $:
\begin{align*}
	z\le (x\Rightarrow y) \text{ if and only if } z\meet x \le y.
\end{align*}
Since by Definition \ref{D-35}, we now have a way to induce the internal partial order in an internal lattice, we can hence redefine internal Heyting algebra as follows:
\begin{definition}
	(\textbf{Internal Heyting Algebra - II}) \emph{Suppose} $ \cat{C} $ \emph{is a category with finite limits. Let $ (H,\meet,\join,\top,\bot) $ be an internal lattice with $ \top $ and $ \bot $ in $ \cat{C} $. $ H $ is said to be an internal Heyting algebra if there exists an additional arrow $ (\Rightarrow) : H\times H \longrightarrow H $ such that the subobjects $ P $ and $ Q $ are equivalent in the following:}
	\[\begin{tikzcd}
		P & {\le_H} & Q \\
		{H\times H\times H} & {H\times H} & {H\times H\times H} \\
		{} & {} & {}
		\arrow["{\meet\times 1}"', from=2-1, to=2-2]
		\arrow["e", tail, from=1-2, to=2-2]
		\arrow[from=1-1, to=1-2]
		\arrow[dashed, tail, from=1-1, to=2-1]
		\arrow["\lrcorner"{anchor=center, pos=0.125}, draw=none, from=1-1, to=2-2]
		\arrow[dashed, tail, from=1-3, to=2-3]
		\arrow[from=1-3, to=1-2]
		\arrow["{1\times (\Rightarrow)}", from=2-3, to=2-2]
		\arrow["\lrcorner"{anchor=center, pos=0.125, rotate=-90}, draw=none, from=1-3, to=2-2]
		\arrow["{x\meet y \le z}", draw=none, from=3-1, to=3-2]
		\arrow["{x\le y\Rightarrow z}", draw=none, from=3-2, to=3-3]
	\end{tikzcd}.\]
\end{definition}
A lattice homomorphism $ f : (A,\meet,\join, \top,\bot) \longrightarrow (B,\meet,\join, \top,\bot) $ is defined to be the one which respects all the structure ($ \meet $ and $ \join $) and preserves $ \top $ and $ \bot $. We can similarly define internal lattice homomorphism as the following:
\begin{definition}\label{D-37}
	(\textbf{Internal Lattice homomorphism}) \emph{Suppose $ \cat{C} $ is a category with finite limits and $ L,L^{\prime} $ are two internal lattices with $ \top $ and $ \bot $. An arrow $ f : L\longrightarrow L^{\prime} $ is said to be an internal lattice homomorphism if the following diagrams commute}\footnote{The diagrams are the \emph{internal way} of saying that $ f $ preserves all structure.}:
\[\begin{tikzcd}
	L & {L\times L} && L & {L\times L} \\
	L & {L\times L} && L & {L\times L} \\
	{} & {} && {} & {} \\
	1 & L && 1 & L \\
	L &&& L \\
	{} & {} && {} & {}
	\arrow["f"', from=1-1, to=2-1]
	\arrow["{\meet }"', from=1-2, to=1-1]
	\arrow["{f\times f}", from=1-2, to=2-2]
	\arrow["\meet", from=2-2, to=2-1]
	\arrow["f"', from=1-4, to=2-4]
	\arrow["\join"', from=1-5, to=1-4]
	\arrow["\join", from=2-5, to=2-4]
	\arrow["{f\times f}", from=1-5, to=2-5]
	\arrow["\top", from=4-1, to=4-2]
	\arrow["\top"', from=4-1, to=5-1]
	\arrow["f"', from=5-1, to=4-2]
	\arrow["\bot", from=4-4, to=4-5]
	\arrow["\bot"', from=4-4, to=5-4]
	\arrow["f"', from=5-4, to=4-5]
	\arrow["{f(x\meet y) = f(x)\meet f(y)}", draw=none, from=3-1, to=3-2]
	\arrow["{f(x\join y) = f(x)\join f(y)}", draw=none, from=3-4, to=3-5]
	\arrow["{f(\top) = \top}", draw=none, from=6-1, to=6-2]
	\arrow["{f(\bot) =\bot}", draw=none, from=6-4, to=6-5]
\end{tikzcd}.\]
\end{definition}
\subsubsection{Subobject Lattices, Internal \& External}
We now discuss the two subobject lattices, the external, $ \Sub{A}{\cat{E}} $, and the internal, $ PA $. We first begin by showing that $ \Sub{A}{\cat{E}} $ is in-fact a Heyting Algebra, which extends the result obtained in Theorem \ref{T-4.1}, 1.
\begin{proposition}\label{P-22}
	Suppose $ \cat{E} $ is a topos. The lattice $ \Sub{A}{\cat{E}} $ of subobjects of object $ A $ is a Heyting algebra where the top and bottom elements are $ \bot : 0 \rightarrowtail  A $\footnote{See Corollary \ref{C-3}.} and $ \top := 1 : A\rightarrowtail A $. Moreover, for any $ k : A\to B $ in $ \cat{E} $, the functor $ \inv{k} : \Sub{B}{\cat{E}} \rightarrow \Sub{A}{\cat{E}} $ is a Heyting algebra homomorphism.
\end{proposition}
\begin{proof}
	First note that $ \Sub{A}{\cat{E}} \isomorph \Sub{1}{\cat{E}/A}$ therefore it is enough to talk about the subobjects of identity in slice $ \cat{E} /A$. To construct the $ (\Rightarrow) $ in $ \Sub{1}{\cat{E}/A} $, take any two open objects\footnote{See Definition \ref{D-29}.} $ U \rightarrowtail 1 $ and $ V \rightarrowtail 1 $ in the $ \Sub{1}{\cat{E}/A} $. Clearly, $ U^{V} $ is also open, so $ U^{V} \rightarrowtail 1  $ in $ \Sub{1}{\cat{E}/A} $. The corresponding $ U^{V} \rightarrowtail A$ in $ \cat{E}$ is the required exponential.\\
	For next result, take an arrow $ k : A \to B $ in $ \cat{E} $. Since change of base functor $ k^{*} : \cat{E}/B \longrightarrow \cat{E}/A$ is a logical morphism and preserves limits and colimits (Theorem \ref{T-11}), therefore the corresponding arrow $ \inv{k} : \Sub{B}{\cat{E}} \longrightarrow \Sub{A}{\cat{E}} $ is also structure preserving as it preserves the meet (limit), join (image), top \& bottom (by functoriality) and exponents (logical morphism).
\end{proof}
As to what we eluded earlier in remark of Theorem \ref{T-4.1}, we have proved the external part of it in the Proposition \ref{P-22}. The remaining thing to do is to show that the internal subobject lattice $ PA $ is also an internal Heyting algebra. This is exactly what we do now.
\begin{proposition}\label{P-23}
	Suppose $ \cat{E} $ is a topos. The power object $ PA $ for any object $ A $ is an internal Heyting algebra. Moreover, for any arrow $ k : A\to B $ in $ \cat{E} $, the map $ Pk : PB \to PA $ is an internal Heyting algebra homomorphism.
\end{proposition}
\begin{proof}
	The proof follows the canonical construction of Definition \ref{D-4.10}. Construct internal meet in $ PA $ as the arrow $ \bmeet : PA\times PA \longrightarrow PA $ given by the Yoneda lemma on the natural transformation $ \bmeet_{(-)} : \homset{\cat{E}}{-}{PA \times PA} \Rightarrow \homset{\cat{E}}{-}{PA} $. Similarly for $ \bjoin $. Now for internal $ (\Rightarrow) : PA \times PA \longrightarrow PA$, we again follow the same construction on the $ (\Rightarrow)^{\prime} $ of the Heyting algebra $ \Sub{A\times X}{\cat{E}} $, which gives the following implication arrow in external subobject lattice: 
	\begin{align*}
		(\Rightarrow)^{\prime} : \Sub{A\times X}{\cat{E}} \times \Sub{A\times X}{\cat{E}} \longrightarrow \Sub{A\times X}{\cat{E}}
 	\end{align*}
 	which can be seen to give rise to a natural transform $ (\Rightarrow)_{(-)}^{\prime} : \homset{\cat{E}}{-}{PA\times PA} \longrightarrow \homset{\cat{E}}{-}{PA} $ which then by Yoneda lemma gives an arrow $ (\Rightarrow) : PA\times PA \longrightarrow PA $ which is the required implication for internal Heyting algebra $ PA $. Since top element of $ \Sub{A}{\cat{E}} $ is the $ 1 : A\rightarrowtail A $, which gives the corresponding arrow $ 1 \to PA $ by the natural isomorphism $ \Sub{A\times 1}{\cat{E}} \isomorph \homset{\cat{E}}{1}{PA} $. Similarly for $ \bot $. To see the $ Pk $ is an internal Heyting algebra homomorphism, we use generalized elements and argue the $ \homset{\cat{E}}{X}{Pk} $ is induces an external Heyting algebra homomorphism between $ \Sub{B\times X}{\cat{E}} $ and $ \Sub{A\times X}{\cat{E}} $. This can be seen easily since $ \Sub{B\times X}{\cat{E}} \isomorph \homset{\cat{E}}{X}{PB}$ and similarly for $ PA $. Since the square thus formed commutes, therefore we have an external Heyting algebra homomorphism by externalization of $ Pk $, which means $ Pk $ is an internal Heyting algebra homomorphism.
\end{proof}
\begin{remark}
	(\textbf{Internal logic of a topos is Intuitionistic}) What we have just proved in Propositions \ref{P-22} \& \ref{P-23} is a very striking fact that in a topos, all the subobjects of an object forms a Heyting algebra instead of a Boolean algebra. This is striking because law of excluded middle (\emph{either there is something or nothing}, more concretely, $ x\join \neg x = \top $ or $ \neg \neg x = x $) does not hold in a Heyting algebra. This means that for a subobject $ S$ of $ A $ in a topos, $ S \join  \neg S \neq \top $, which when unraveled means $ S \join (S\Rightarrow 0) \neq A $ since $ \neg S := (S\Rightarrow \bot) $ and $ \top :=1 : A\rightarrowtail A $ and $ \bot : 0 \rightarrowtail A $.  \\
	For example, the topos $ \cat{Sets} $ is such that the subobject lattice in $ \cat{Sets} $ forms a Boolean algebra since $ S\union S^{\complement} = A$ for $ S\subset A $. But the fact that this doesn't hold in an arbitrary topos suggests that a topos is a \textbf{generalized universe to do sets-like mathematics}.
\end{remark}
This concludes the basic properties of topoi. We now study how one can generalize the concept of a topology, and therefore a sheaf, to an arbitrary topos.
\newpage
\section{Sheaves in an arbitrary Topos}
We have studied two notions of sheaves, one on a topological space $ X $, whose sheaf category is denoted $ \Sh{X} $ and the other one on a site $ (\cat{C},J) $, whose sheaf category is denoted $ \Sh{\cat{C},J} $. Both times we saw that the sheaf category is a reflective subcategory of $ \ps{O(X)} $ (for $ \Sh{X} $, Theorem \ref{T-6}) and $ \ps{C}$ (for $ \Sh{\cat{C},J} $, Theorem \ref{T-8}). We now generalize the notion of a sheaf to an arbitrary topos. We will see that the same relations of the sheaf category in a topos and the underlying topos holds as one eluded to earlier. We would in-fact see that the notion of a sheaf in a topos is indeed a generalization of sheaves over a site, and hence over a topological space. But the interesting observation would here be that we do not access the "space" itself in the following generalization of sheaves. That is, we assume that our arbitrary topoi acts as if it is a presheaf topos of some notion of "generalized space" and this "space" is completely inaccessible to us\footnote{We would later see how to "access" it, via what we would call \emph{points of a topos}.}.
\subsection{The Lawvere-Tierney Topology on a Topos}
Of-course, to define a sheaf, we would first need to define a notion of a cover. But since we are working with a given topos as if it were a presheaf topoi, so we try to look at what properties are sufficient to give a "topology" on that "generalized space" over whom we consider our topos is actually it's presheaf topos. \\\\
To see what are the sufficient conditions to identify a topology on the "generalized space" while only having access to it's presheaf topos, we first look at the example of sheaves over a site $ (\cat{C},J) $. That is, we first study the notion of a Grothendieck topology $ J $ but not looking at the "space" $ \cat{C} $, but looking at the effect of $ J $ on the presheaf category $ \ps{C} $. \\
First note that the presheaf topos has the subobject classifier $ \Omega $ which maps each object of $ \cat{C} $ to the corresponding collection of sieves on it. It is also quite easy to see that a sieve $ S $ over $ C $ is a $ J $-covering sieve if and only if $ \bar{S} $ is maximal. Therefore $ J $ determines a natural transformation $ j : \Omega \Rightarrow \Omega $ where $ j_C(S) := \bar{S} $ where $ \bar{S} := \{f : \dom{f} \to C \;\vert\; f^{*}(S) \in J\dom{f}\}$ (See footnote 12). Let's analyze this natural transformation $ j : \Omega \Rightarrow \Omega $. First, if $ \true : \bm{1} \Rightarrow \Omega $ is the subobject classifier of $ \ps{C} $, then $ j\circ \true =  \true$ because $ \true_C(\star) := S^{\text{max}}_C $ (Section \ref{SCiS}). Next, we note that $ j\circ j = j $ and this is obvious. Finally, note that $ j_C(S\intrs T) = \overline{S\intrs T} $ is such that for any $ f\in \overline{S\intrs T} $, $ f^{*}(S\intrs T) \in J\dom{f} $ which implies that $ f^{*}(S)\intrs f^{*}(T) \in J\dom{f}  $ which further means that $ f^{*}(S), f^{*}(T) \in J\dom{f} $ so that $ f\in  \bar{S}\intrs \bar{T}$. Similarly, for $ f\in \bar{S} \intrs \bar{T}$, we get $ f^{*}(S), f^{*}(T) \in J\dom{f} $ therefore $ f^{*}(S\intrs T) = f^{*}(S)\intrs f^{*}(T) \in J\dom{f} $. Therefore $ j_C(S\intrs T) = j_C(S)\intrs j_C(T)$. \\
Therefore, we are motivated to define the following axioms for a "topology" on a topos $ \cat{E} $. The topology, is in-fact on the underlying space, but if see $ j $ as above on the presheaf topos, then we can safely say that it corresponds to a topology in that underlying generalized space.
\begin{definition}\label{D-38}
	(\textbf{Lawvere-Tierney Topology}) \emph{Suppose }$ \cat{E} $ \emph{is a topos and $ \Omega$ is it's truth object. An arrow $ j : \Omega \longrightarrow \Omega$ is called a Lawvere-Tierney topology on} $ \cat{E} $ \emph{if $ j $ satisfies the following commutative diagrams:}
	\[\begin{tikzcd}
		{} & {} && {} & {} && {} & {} \\
		{\bm{1}} &&& \Omega &&& {\Omega\times \Omega} & \Omega \\
		\Omega & \Omega && \Omega & \Omega && \Omega\times\Omega & \Omega \\
		{} & {} && {} & {} && {} & {}
		\arrow["\true"', tail, from=2-1, to=3-1]
		\arrow["j"', from=3-1, to=3-2]
		\arrow["\true", tail, from=2-1, to=3-2]
		\arrow["{j\circ \true = \true}", draw=none, from=4-1, to=4-2]
		\arrow["j"', from=3-4, to=3-5]
		\arrow["j"', from=2-4, to=3-4]
		\arrow["j", from=2-4, to=3-5]
		\arrow["{j\circ j = j}", draw=none, from=4-4, to=4-5]
		\arrow["\meet", from=2-7, to=2-8]
		\arrow["\meet"', from=3-7, to=3-8]
		\arrow["{j\times j}"', from=2-7, to=3-7]
		\arrow["j", from=2-8, to=3-8]
		\arrow["{j\circ \meet = \meet \circ (j\times j)}", draw=none, from=4-7, to=4-8]
		\arrow["{\textbf{LTT.1}}"', draw=none, from=1-1, to=1-2]
		\arrow["{\textbf{LTT.2}}"', draw=none, from=1-4, to=1-5]
		\arrow["{\textbf{LTT.3}}"', draw=none, from=1-7, to=1-8]
	\end{tikzcd}\]
\emph{where $ \meet : \Omega\times \Omega \longrightarrow \Omega $ is the internal meet in $ \Omega $ as in Definition \ref{D-4.10}.}
\end{definition}
\begin{remark}
	As mentioned in Definition \ref{D-4.10}, the truth object in a topos defines an internal meet-semilattice object $ (\Omega, \meet, \true : \bm{1}\longrightarrow \true)$ where $ \true $ is the top element. Now, the axiom LTT.1 says that $ j $ preserves 0-ary meet/top element in $ \Omega $. LTT.3 on the other hand tells us that $ j $ preserves the meet in $ \Omega $ and LTT.2 says that $ j $ is idempotent. Hence, in accordance with Definition \ref{D-37}, we can say that a Lawvere-Tierney topology on a topos $ \cat{E} $ is equivalently \textbf{an idempotent internal meet-semilattice endomorphism on $ \Omega $}, where $ \Omega $ is the truth object of $ \cat{E} $.
\end{remark}
\begin{example}
	An easy to see example of this is the arrow $ j : \Omega \Rightarrow \Omega $ in $ \cat{Sets}^{\opcat{O(X)}} $ which is the characteristic arrow of the subobject $ J : \opcat{O(X)}\longrightarrow \cat{Sets} $ which takes an open set $ U \subseteq X $ to the collection of all those sieves over $ U $ which forms an open cover of $ U $. In particular, this natural transformation $ j $ is given by components $ j_U : \Omega U \to \Omega U $ which takes a sieve $ S $ over $ U $ to it's \emph{closure} $ \bar{S} $, i.e., $ \bar{S} $ is that sieve which contains all those open subsets $ V $ of $ X $ for which $ V\intrs S $ forms an open cover of $ V $.
\end{example}
\subsubsection{The Closure Operator}
Any arrow $ j $ on a topos $ \cat{E} $ gives equivalently an operator which sends each subobject to it's \emph{$ j $-closure}. 
\begin{definition}
	(\textbf{$ j $-Closure Operator}) \emph{Suppose} $ \cat{E} $ \emph{is a topos and $ j : \Omega \longrightarrow \Omega $ is any such arrow}\emph{. Then $ j $ determines a map} $ \bar{(-)} : \Sub{A}{\cat{E}} \longrightarrow \Sub{A}{\cat{E}}$ \emph{for any object $ A $ given as follows:}
	\[\begin{tikzcd}
		{\Sub{A}{\cat{E}}} && {\homset{\cat{E}}{A}{\Omega}} \\
		\\
		{\Sub{A}{\cat{E}}} && {\homset{\cat{E}}{A}{\Omega}}
		\arrow["{\homset{\cat{E}}{A}{j}}", from=1-3, to=3-3]
		\arrow["\isomorph"{description}, no head, from=1-1, to=1-3]
		\arrow["\isomorph"{description}, no head, from=3-1, to=3-3]
		\arrow["{\bar{(-)}}"', from=1-1, to=3-1]
	\end{tikzcd}\]
\emph{which, as the above diagram says, maps each subobject of $ m: S \rightarrowtail A $ of $ A $ to another subobject of $ A $, $\bar{m}: \bar{S} \rightarrowtail A $ which is called $ j $-closure of $ S $ and this subobject $ \bar{S} $ is given by the subobject characterized by the arrow $ j\circ \chr{m} $, that is,}
\begin{align*}
	\chr{\bar{m}} = j\circ \chr{m}.
\end{align*}
\end{definition}
One can see that the closure operator $ \bar{(-)} : \Sub{A}{\cat{E}} \longrightarrow\Sub{A}{\cat{E}}$ is natural in $ A $:
\begin{lemma}
	Suppose $ \cat{E} $ is a topos and $ j : \Omega \longrightarrow \Omega $ is any such arrow in $ \cat{E} $. For any arrow $ f : A\longrightarrow B $ in $ \cat{E} $ and it's corresponding subobject pullback functor $ \inv{f} : \Sub{B}{\cat{E}} \longrightarrow \Sub{A}{\cat{E}}$, we have that 
	\begin{align*}
		\inv{f}(\bar{S}) = \overline{\inv{f}(S)}.
	\end{align*}
for any subobject $ S\rightarrow B $.
\end{lemma}
\begin{proof}
	The following diagram directly shows the result:
\[\begin{tikzcd}
	{\inv{f}(S)} & S & {\bm{1}} & {\bm{1}} \\
	A & B & \Omega & \Omega
	\arrow["f"', from=2-1, to=2-2]
	\arrow["m", tail, from=1-2, to=2-2]
	\arrow[tail, from=1-1, to=2-1]
	\arrow[from=1-1, to=1-2]
	\arrow["\lrcorner"{anchor=center, pos=0.125}, draw=none, from=1-1, to=2-2]
	\arrow["{\chr{m}}"', from=2-2, to=2-3]
	\arrow["\true", from=1-3, to=2-3]
	\arrow[from=1-2, to=1-3]
	\arrow["\lrcorner"{anchor=center, pos=0.125}, draw=none, from=1-2, to=2-3]
	\arrow["j"', from=2-3, to=2-4]
	\arrow[Rightarrow, no head, from=1-3, to=1-4]
	\arrow["\true", from=1-4, to=2-4]
	\arrow["\lrcorner"{anchor=center, pos=0.125}, draw=none, from=1-3, to=2-4]
\end{tikzcd}\]
as characteristic arrow for both $ \inv{f}(\bar{S}) $ and $ \overline{\inv{f}(S)}$ are same.
\end{proof}
One can give the description of a Lawvere-Tierney topology from the closure operator.
\begin{proposition}
	Suppose $ \cat{E} $ is a topos. An arrow $ j : \Omega \longrightarrow \Omega $ is a Lawvere-Tierney topology on $ \cat{E} $ if and only if the corresponding $ j $-closure operator $ \bar{(-)} $ satisfies for any subobjects $ S,T \rightarrowtail A $ for any object $ A $ the following
	\begin{align*}
		S\subseteq \bar{S}\;\;\;\;\bar{S} = \bar{\bar{S}}\;\;\;\;\overline{S\intrs T} = \bar{S}\intrs \bar{T}
	\end{align*}
\end{proposition}
\begin{proof}
	(L $ \implies $ R) Let $ j : \Omega \longrightarrow \Omega $ to be a Lawvere-Tierney topology on $ \cat{E} $. To show that $ S\subseteq \bar{S} $, we need a monomorphism $ S\rightarrowtail \bar{S} $, which can be seen to exist by the fact that $ S $ forms a cone $ (m, !_A) $ over the subobject pullback of $ \bar{S} $ and the fact that this does forms a cone depends on LTT.1. To show that $ S= \bar{\bar{S}} $, we can simply note that from LTT.2, we have $ \chr{\bar{m}} = j\circ \chr{m} = (j\circ j)\circ \chr{m} = \chr{\bar{\bar{m}}}$. To show $ \overline{S\intrs T} = \bar{S}\intrs \bar{T} $, we can note the following by the help of LTT.3 ($ n : A\to \Omega $ is the characteristic of $ T $):
	\begin{align*}
		\chr{\overline{m\intrs n}} &= j\circ \chr{m\intrs n}\\
		&= j\circ \meet \upair{\chr{m}}{\chr{n}}\\
		&= \meet \circ (j\times j) \circ \upair{\chr{m}}{\chr{n}}\\
		&= \meet \circ \upair{j\circ \chr{m}}{j\circ \chr{n}}\\
		&= \meet \circ \upair{\chr{\bar{m}}}{\chr{\bar{n}}}\\
		&= \chr{\bar{m}}\meet\chr{\bar{n}}
	\end{align*}
which is indeed the required result.\\
(R $ \implies  $ L) LTT.2 and LTT.3 essentially follows from following the arguments above in reverse. Whereas for LTT.1, since we have $ S\rightarrowtail \bar{S} $ for any subobject $ S $, then for $ S = \bm{1} $ and $ A = \Omega $, that is, if we take the subobject classifier $ \true : \bm{1}\rightarrowtail \Omega$ as our subobject, then because characteristic arrow of $ \true  $ is $ 1_\Omega $, therefore $ \bar{\bm{1}} = \bm{1} $ and so $ j\circ 1_\Omega \circ \true = 1\circ \true $, i.e. $ j\circ \true = \true $.
\end{proof}
\subsubsection{Closed \& Dense Subobjects}
Based on what the $ j $-closure of a subobject $ m : S\rightarrowtail A $ looks like in relation to $ S $ and $ A $, each $ j $ determines two classes of subobjects:
\begin{definition}
	(\textbf{Closed \& Dense Subobjects}) \emph{Suppose }$ \cat{E} $ \emph{is a topos and $ j : \Omega \longrightarrow \Omega $ is a Lawvere-Tierney topology on} $ \cat{E} $.\emph{ Let $ m : S\rightarrowtail A $ be any subobject. We then define
	\begin{itemize}
		\item {The subobject $ m $ is \textbf{closed} if $ \bar{S} = S $.}
		\item {The subobject $ m $ is \textbf{dense} if $ \bar{S} = A $.}
	\end{itemize}}
\end{definition}
As expected, any Lawvere-Tierney topology $ j : \Omega \Rightarrow \Omega$ on $ \psheaf{C} $ determines a Grothendieck topology on $ \cat{C} $, therefore generalizing it.
\begin{proposition}\label{P-25}
	Every Grothendieck topology $ J $ on a small category $ \cat{C} $ determines a Lawvere-Tierney topology $ j : \Omega \Rightarrow \Omega $ in the presheaf topos $ \psheaf{C} $.
\end{proposition}
\begin{proof}
	Let $ (\cat{C}, J) $ be a site. Define $ j : \Omega \Rightarrow \Omega $ to be a natural transformation with components $ j_C : \Omega C \to \Omega C $ which takes a sieve to it's $ J $-closure. The fact that $ j $ is a Lawvere-Tierney topology in $ \psheaf{C} $ can be seen from the beginning discussion above Definition \ref{D-38}.
\end{proof}
\subsection{$ j $-Sheaves in a topos}
A Grothendieck topology $ J $ on a small category $ \cat{C} $ leads to a notion of sheaves over a site as given by Definition \ref{D-13}. Remember that a $ J $-covering sieve $ S_C $  is a subpresheaf of $ \yembed{C} $ such that the closure of sieve $ S_C $ is $ S_C^{\text{max}} $. As in the proof of Proposition \ref{P-25}, the Lawvere-Tierney topology $ j $ in $ \psheaf{C} $ corresponding to $ J $ is such that the $ j $-closure of a sieve is maximal if and only if that sieve is a $ J $-cover, that is, a $ J $-cover $ S_C $ is a $ j $-dense subobject of $ \yembed{C} $. A matching family of a $ J $-cover $ S_C $ is simply a natural transformation $ S_C \Rightarrow P $ where $ S_C $ is viewed as a subpresheaf of $ \yembed{C} $. An amalgamation of a matching family $ S_C \Rightarrow P $ is hence a natural transformation from $ \yembed{C} \Rightarrow P$. The condition when $ P $ is a sheaf says \emph{every matching family has a unique amalgamation}, which from above discussion surmounts to the fact that the following commutes:
\[\begin{tikzcd}
	{S_C} & P \\
	{\yembed{C}}
	\arrow["{\text{(dense)}}"', Rightarrow, 2tail, from=1-1, to=2-1]
	\arrow["\forall", Rightarrow, from=1-1, to=1-2]
	\arrow["{\exists !}"', Rightarrow, from=2-1, to=1-2]
\end{tikzcd}.\]
This motivates the following definition of a sheaf in any arbitrary topos:
\begin{definition}
	(\textbf{Sheaf Object in a Topos}) \emph{Suppose} $ \cat{E} $ \emph{is a topos and $ j : \Omega \longrightarrow \Omega $ is a Lawvere-Tierney topology on it. An object $ F $ is called a sheaf i}n $ \cat{E} $ \emph{if for all dense subobjects $m :  A\rightarrowtail E $, any arrow $ A\to F $ can be factored uniquely via $ m $, that is the following commutes:}
	\[\begin{tikzcd}
		A & F \\
		E
		\arrow["{\text{(dense)}}"', tail, from=1-1, to=2-1]
		\arrow["{\exists !}"', from=2-1, to=1-2]
		\arrow["{\forall }", from=1-1, to=1-2]
		\arrow["m", tail, from=1-1, to=2-1]
	\end{tikzcd}.\]
\emph{In other words, the following is an isomorphism for all dense subobjects $ m $:}
\[\begin{tikzcd}
	{\homset{\cat{E}}{E}{F}} && {\homset{\cat{E}}{A}{F}}
	\arrow["{-\circ m}", from=1-1, to=1-3]
	\arrow["\isomorph"', from=1-1, to=1-3]
\end{tikzcd}.\]
\end{definition}
\begin{remark}
	The full subcategory of sheaf objects is denoted 
	\begin{align*}
		\sh{j}{\cat{E}}.
	\end{align*}
\end{remark}
A weaker condition of the above definition would gives us the following definition which generalizes separated presheaves\footnote{To remind, a presheaf is separated if every matching family has an amalgamation (not necessarily unique).}:
\begin{definition}
	(\textbf{Separated Object in a Topos})\emph{ Suppose }$ \cat{E} $ \emph{is a topos and $ j:\Omega \longrightarrow \Omega $ is a Lawvere-Tierney topology on it. An object $ G $ in} $ \cat{E} $ \emph{is called separated if for all dense subobjects $ m : A\rightarrowtail E$, the following is a monomorphism:}
	\[\begin{tikzcd}
		{\homset{\cat{E}}{E}{G}} && {\homset{\cat{E}}{A}{G}}
		\arrow["{-\circ m}", tail, from=1-1, to=1-3]
	\end{tikzcd}.\]
\end{definition}
\begin{remark}
	The full subcategory of separated objects is denoted 
	\begin{align*}
		\sep{j}{\cat{E}}.
	\end{align*}
\end{remark}
\subsection{$ \sh{j}{\cat{E}} $ is a Topos}
As in the case of sheaves over a site, $ \sh{j}{\cat{E}} $ is a topos. We begin with proving that $ \sh{j}{\cat{E}} $ has finite limits and exponentials.
\begin{lemma}\label{L-11}
	Let $ \cat{E} $ be a topos. The full subcategory $ \sh{j}{\cat{E}} $ has all finite limits and exponentials with any object in $ \cat{E} $.
\end{lemma}
\begin{proof}
	To show finite limits, we just have to show that it has terminal object, equalizers and binary products. The terminal object $ \bm{1} $ of $ \cat{E} $ is clearly a sheaf. To show equalizer of any two parallel arrows of sheaves is a sheaf, take any parallel pair and it's equalizer in $ \cat{E} $
	\[\begin{tikzcd}
		E & A & B
		\arrow["f", shift left=2, from=1-2, to=1-3]
		\arrow["g"', shift right=2, from=1-2, to=1-3]
		\arrow["e", tail, from=1-1, to=1-2]
	\end{tikzcd}\]
   take any dense subobject $ m : S\rightarrowtail C  $ and any arrow $ k_A : S\to A $ and $ k_B : S\to B $. Take any arrow $ k_E : S\to E $. We then have the following diagram due to $ A $ and $ B $ being sheaf objects and $ m $ being dense:
   \[\begin{tikzcd}
   	& S && A \\
   	{(\text{dense})} \\
   	& C && B
   	\arrow["m"', tail, from=1-2, to=3-2]
   	\arrow["{e\circ k_e}", from=1-2, to=1-4]
   	\arrow["{u_{e\circ k_e}}"', dashed, from=3-2, to=1-4]
   	\arrow["g", shift left=2, from=1-4, to=3-4]
   	\arrow["f"', shift right=2, from=1-4, to=3-4]
   	\arrow["{u_{g\circ e\circ k_e}}"{description}, shift right=2, dashed, from=3-2, to=3-4]
   	\arrow["{u_{f\circ e\circ k_e}}"{description}, shift left=2, dashed, from=3-2, to=3-4]
   \end{tikzcd}\]
	It can be seen now that the arrows $ f\circ u_{e\circ k_E} = g\circ u_{e\circ k_E}$. Hence, $ \exists ! v : C \longrightarrow E $ by universality of equalizer such that $ e\circ v = u_{e\circ k_E} $. Therefore $ e\circ v\circ m = e\circ k_E $ and since $ e $ is monic, hence $ v\circ m = k_E $.\\
	Binary products can also be seen as above. Therefore finite limits exists.\\
	To see about exponentials, we can see that for any sheaf object $ F $ and any object $ B $, $ F^{B} $ is a sheaf by the following; Suppose $ m : A\rightarrowtail E $ is a dense subobject, therefore we have:
	\begin{align*}
		\homset{\cat{E}}{E}{F} \isomorph \homset{\cat{E}}{A}{F}
	\end{align*}
	Now, because $ m $ is dense, then $ m\times 1_B : A\times B \rightarrowtail E\times B$ is dense because $ \overline{A\times B} \isomorph \overline{\inv{\pi}(A)} \isomorph \inv{\pi}(\overline{A}) =\inv{\pi}(E) = E\times B $  where $ \pi : E\times B \longrightarrow E $ is the first projection. Using the density of $ m\times 1_B $ and the fact that $ F$ is a sheaf, we can now see the following:
	\[\begin{tikzcd}
		{\homset{\cat{E}}{E\times B}{F}} && {\homset{\cat{E}}{E}{F^B}} \\
		\\
		{\homset{\cat{E}}{A\times B}{F}} && {\homset{\cat{E}}{A}{F^B}}
		\arrow["{\homset{\cat{E}}{m\times 1}{F}}"', from=1-1, to=3-1]
		\arrow["{\homset{\cat{E}}{m}{F^B}}", from=1-3, to=3-3]
		\arrow["\isomorph", no head, from=1-1, to=1-3]
		\arrow["\isomorph"', no head, from=3-1, to=3-3]
	\end{tikzcd}\]
 where the $ \homset{\cat{E}}{m\times 1}{F} $ is an isomorphism and since the square commutes, so $ \homset{\cat{E}}{m}{F^B} $ is an isomorphism, proving that $ F^{B} $ is a sheaf. This shows exponentials exists in $ \sh{j}{\cat{E}} $.
\end{proof}
Next, the subobject classifier of $ \sh{j}{\cat{E}} $ is given by the mono-epi factor (image) of $ j : \Omega \longrightarrow \Omega $:
\begin{lemma}\label{L-12Con}
	Suppose $ \cat{E} $ is a topos, $ j : \Omega \longrightarrow \Omega $ is an LT topology in $ \cat{E} $ and $ \Omega $ is it's subobject classifier. Then the equalizer of $ j $ and $ 1_\Omega $, denoted $ \Omega_j $ classifies the $ j $-closed subobjects, that is, for any object $ E $ in $ \cat{E} $, there is an isomorphism
	\begin{align*}
		\homset{\cat{E}}{E}{\Omega_j} \isomorph \ClSub{E}{\cat{E}}
	\end{align*}
	where $ \ClSub{E}{\cat{E}} $ is the sub-lattice of $ j $-closed subobjects of $ E $.
\end{lemma}
\begin{proof}
	The object $ \Omega_j $ is constructed, as stated above, by the equalizer of $ j $ and $ 1_\Omega $ can equivalently be realized by the mono-epic factor of $ j $, i.e. the following commutes:
	\[\begin{tikzcd}
		{\Omega_j} & \Omega \\
		& {\Omega_j} & \Omega
		\arrow["r"', two heads, from=1-2, to=2-2]
		\arrow["m"', tail, from=2-2, to=2-3]
		\arrow["j", from=1-2, to=2-3]
		\arrow["{1_{\Omega_j}}"', no head, from=1-1, to=2-2]
		\arrow["m", tail, from=1-1, to=1-2]
	\end{tikzcd}.\]
This follows from the fact that $ j\circ j = j $ and therefore $ \Omega $ itself forms a cone over the equalizer diagram of $ 1_\Omega $ and $ j $. Now, the main result follows as from the fact that for any closed subobject $ s : A\rightarrowtail E $ of $ E $, we have $ \bar{A} = A $, that is 
\begin{align*}
	j\circ \chr{s} &= \chr{s}\\
	m\circ r \circ \chr{s} &= \chr{s}
\end{align*}
and hence $ \chr{s}  $ is factored via the unique arrow $ r\circ \chr{s} $.
\end{proof}
Now if we ought to show that $  \Omega_j$ is the subobject classifier of $ \sh{j}{\cat{E}} $, then $ \Omega_j $ must be a sheaf first of all:
\begin{lemma}\label{L-13}
	Suppose $ \cat{E} $ is a topos, $ \j : \Omega \longrightarrow \Omega $ is an LT topology in $ \cat{E} $ and $ \Omega_j $ is the equalizer of $ j $ and $ 1_\Omega $. Then for any dense subobject $ m : A\rightarrowtail E $, the pullback functor along $ m $:
	\[\begin{tikzcd}
		{\inv{m} : \ClSub{E}{\cat{E}}} && {\ClSub{A}{\cat{E}}} \\
		{(k:C\rightarrowtail E)} && {\inv{m}(k)}
		\arrow[from=1-1, to=1-3]
	\end{tikzcd}\]
	is an isomorphism.
\end{lemma}
\begin{proof}
	We only wish to find a map $ \tau : \ClSub{A}{\cat{E}}\longrightarrow \ClSub{E}{\cat{E}}$ such that $ \tau \circ \inv{m} = 1 $ and $ \inv{m}\circ \tau = 1 $. Consider the following candidate for $ \tau $:
	\[\begin{tikzcd}
		{\tau : \ClSub{A}{\cat{E}}} && {\ClSub{E}{\cat{E}}} \\
		{h:B\rightarrowtail A} && B \\
		&& {\text{Im}(B)} & E \\
		&& {\overline{\text{Im}(B)}} & E
		\arrow[from=1-1, to=1-3]
		\arrow["{{u_{m\circ h}}}"', tail, from=3-3, to=3-4]
		\arrow["{e_{m\circ h}}"', two heads, from=2-3, to=3-3]
		\arrow["{m\circ h}", tail, from=2-3, to=3-4]
		\arrow[shorten <=7pt, shorten >=21pt, maps to, from=2-1, to=2-3]
		\arrow["{\circled{\overline{u_{m\circ h}}}}"', tail, from=4-3, to=4-4]
	\end{tikzcd}\]
	To show that $ \tau \circ \inv{m} = 1 $, take any closed subobject $ k : C\rightarrowtail E $ of $ E $, then $ \tau \left (\inv{m}(C)\right ) = \overline{\image (\inv{m}(C))}$. Now since $\overline{u_{m\circ (\inv{m}(k))}}  : \overline{\image (\inv{m}(C))} \rightarrowtail E$ is as given below:
	\begin{align*}
		\chr{\overline{\image (\inv{m}(C))}} &= j\circ \chr{\image (\inv{m}(C))}\\
		&= j\circ \meet \upair{\chr{m}}{\chr{k}}\\
		&= \meet \circ (j\times j) \circ \upair{\chr{m}}{\chr{k}}\\
		&= \meet \circ \upair{j\circ \chr{m}}{j\circ \chr{k}}\\
		&= \chr{\bar{m}}\intrs \chr{\bar{k}}\\
		&= \chr{\bar{A}} \intrs \chr{\bar{C}}\\
		&= \chr{E}\intrs \chr{C}\\
		&= \chr{C}
	\end{align*}
	and therefore $ \tau \circ \inv{m} =1 $. For $ \inv{m}\circ \tau = 1 $, take a closed subobject $ h : B\rightarrowtail A $, so that $ \inv{m}(\tau (h)) = \inv{m}(\overline{\image (B)}) \isomorph \overline{\inv{m}(\image (B))} = \bar{B} = B$ where the second-to-last equality is obtained via what is called Beck-Chevalley condition, which we hadn't discussed here. Hence $ \inv{m}\circ \tau = 1 $.
\end{proof}
Now we can safely say that $ \Omega_j $ is a sheaf object:
\begin{corollary}\label{C-4Con}
	Suppose $ \cat{E} $ is a topos and $ j : \Omega \longrightarrow \Omega  $ is an LT topology in $ \cat{E} $, then $ \Omega_j $, the equalizer of $ j $ and $ 1_\Omega $, is a $ j $-sheaf in $ \cat{E} $.
\end{corollary}
\begin{proof}
	Take any dense subobject $ m : A\rightarrowtail E $. Then:
	\begin{align*}
		\homset{\cat{E}}{E}{\Omega_j} &\isomorph \ClSub{E}{\cat{E}}&&\text{By Lemma \ref{L-12Con}}\\
		&\isomorph\ClSub{A}{\cat{E}}&&\text{By Lemma \ref{L-13}}\\
		&\isomorph\homset{\cat{E}}{A}{\Omega_j}&&\text{By Lemma \ref{L-12Con}}
	\end{align*}
	Hence proved.
\end{proof}



%We shall not cease from exploration, and, at the end of all our exploring, we will arrive where we started and know the place for the first time.
\begin{lemma}\label{L-14}
	Suppose $ \cat{E} $ is a topos. If $ E $ is a sheaf in $ \cat{E} $, then 
	\begin{align*}
		m: A\rightarrowtail E \text{ is closed in $ E $} \iff \text{ $ A $ is also a sheaf.}
	\end{align*}
\end{lemma}
\begin{proof}
	(L $ \implies  $ R) Let $ E $ be a sheaf in $ \cat{E} $ and $ m :A\rightarrowtail E $ be closed. Take any dense subobject $ k : S\rightarrowtail B $ and let $ f : S\rightarrow A $ be any arrow. Since $ E $ is a sheaf, therefore $ \exists ! b : B\rightarrow E $ such that $ b\circ k  = m\circ f $. Let's now take the following pullback:
	\[\begin{tikzcd}
		{\inv{b}(A)} & {\bar A = A} \\
		B & E
		\arrow["{\bar m = m}", tail, from=1-2, to=2-2]
		\arrow["b"', from=2-1, to=2-2]
		\arrow["{\pi_1}"', tail, from=1-1, to=2-1]
		\arrow["{\pi_2}", from=1-1, to=1-2]
		\arrow["\lrcorner"{anchor=center, pos=0.125}, draw=none, from=1-1, to=2-2]
	\end{tikzcd}\]
and since $ b\circ k = m\circ f $, therefore $ \exists ! u : S\rightarrow \inv{b}(A) $ such that $ \pi_1 \circ u = k $ and $ \pi_2 \circ u = f $. Now $ S\subset \inv{b}(A) \implies \bar{S} \subset \overline{\inv{b}(A)} = \inv{b}(\bar{A}) = \inv{b}(A) \implies B \subset \inv{b}(A)$ where the last implication is drawn from the fact that $ B = \bar{S}$ ($ k $ is dense). Therefore $ \exists g : B \longrightarrow \inv{b}(A) $ such that $ \pi_1\circ g = 1 $. Hence $ (\pi_1 \circ g)\circ k = k \implies \pi_1\circ u = k = (\pi_1 \circ g)\circ k\implies u = g\circ k$ and so $ \pi_2 \circ g : B\rightarrow A $ is the required arrow.\\
(R $ \implies  $ L) Suppose $ A $ is a sheaf. We need to show that $ m : A\rightarrowtail E $ is closed, i.e. $ \bar{A} = A $, where $ E $ itself is a sheaf. Take the trivially dense subobject $ d : A\rightarrowtail \bar{A} $. Since $ A $ and $ E $ are sheaves, then we have that $ \exists ! u_d : \bar{A} \rightarrow A $ such that $ u_d\circ d = 1 $ and $ \exists ! u_m : \bar{A} \rightarrow E $ such that $ u_m \circ d= m $. Now since $ (m\circ u_d)\circ d = m\circ 1 = m $ and since $ u_m $ is unique such that $ u_m\circ d = m $, therefore $ u_m = m\circ u_d $. Since the closure of $ m $ is such that $ \bar m \circ d = m $, therefore $ \bar{m} = u_m $ as $ u_m $ is unique with the property that $ u_m \circ d = m $. But $ \bar{m} $ is a monic and also $ (\bar{m} \circ d) \circ u_d = (m)\circ u_d = \bar{m}$, we get $ d\circ u_d = 1 $, therefore $ \bar{A} \isomorph A $.
\end{proof}
\begin{lemma}\label{L-15}
	Suppose $ \cat{E} $ is a topos and $ j : \Omega \longrightarrow \Omega $ is an LT topology in it. Then the $ \true_j : \bm{1} \longrightarrow \Omega_j $ is the subobject classifier of $ \sh{j}{\cat{E}} $ which is given as composition of $ \true $ with the epic part of the mono-epi factorization of $ j $ as follows:
	\[\begin{tikzcd}
		{\bm 1} \\
		{\Omega_j} & \Omega
		\arrow["\true", tail, from=1-1, to=2-2]
		\arrow["m"', tail, from=2-1, to=2-2]
		\arrow["{\true_j:= r\circ \true}"', tail, from=1-1, to=2-1]
	\end{tikzcd}.\]
\end{lemma}
\begin{proof}
	By Lemma \ref{C-4Con}, we have that $ \Omega_j $ is a $ j $-sheaf object in $ \cat{E} $. Now if $ \true_j $ ought to be the subobject classifier of $ \sh{j}{\cat{E}} $, then for each subobject $ m : A\rightarrowtail E $ of sheaves (that is $ A $ and $ E $ are sheaves), we must have a unique arrow $ \chr{m} : E \rightarrow \Omega_j  $ such that the following is a pullback:
	\[\begin{tikzcd}
		A & \bm1 \\
		E & {\Omega_j}
		\arrow["m"', tail, from=1-1, to=2-1]
		\arrow["{\true_j}", tail, from=1-2, to=2-2]
		\arrow["{\chr{m}}"', dashed, from=2-1, to=2-2]
		\arrow[from=1-1, to=1-2]
		\arrow["\lrcorner"{anchor=center, pos=0.125}, draw=none, from=1-1, to=2-2]
	\end{tikzcd}.\] 
	But by Lemma \ref{L-12Con}, $ A $ must equivalently be closed! Therefore we would be done if we could show that any subobject of sheaves $ m : A\rightarrowtail E $ is always closed. This just follows from Lemma \ref{L-14}.
\end{proof}
We finally can prove that $ \sh{j}{\cat E} $ is a topos:
\begin{theorem}
	Suppose $ \cat{E} $ is a topos and let $ j : \Omega \longrightarrow \Omega $ be any Lawvere-Tierney topology in $ \cat{E} $. Then the full subcategory $ \sh{j}{\cat{E}} $ of sheaf objects in $ \cat{E} $ is a topos.
\end{theorem}
\begin{proof}
		$ \sh{j}{\cat{E}} $ has finite limits and exponentials by Lemma \ref{L-11}. Subobject classifier of $ \sh{j}{\cat{E}} $ is $ \true_j $ as showed in Lemma \ref{L-15}. 
\end{proof}
\subsection{The Sheafification Functor $ a : \cat{E} \longrightarrow \sh{j}{\cat{E}} $}
We will now construct a left adjoint to the inclusion functor $ i : \sh{j}{\cat{E}} \hookrightarrow \cat{E} $ where $ \cat{E} $ is a topos and $ j : \Omega \longrightarrow \Omega  $ is a Lawvere-Tierney topology in $ \cat{E} $. We would achieve our task in two steps. First, we would take any object $ E $ of $ \cat{E} $ to a separated object $ E^{\prime} $ and then we would take a separated object to a sheaf object $ \overline{E^{\prime}} $. In essence, we would need to construct two functors, $ L_1 : \cat{E} \longrightarrow \sep{j}{\cat{E}}$ and $ L_2 : \sep{j}{\cat{E}} \longrightarrow \sh{j}{\cat{E}} $, where $ L_1 $ and $ L_2 $ both must be left adjoints of the corresponding inclusions.\\
We begin with constructing the separated object $ E^{\prime} $:
\subsubsection{From an object $ E $ to a separated object $ E^{\prime} $}
To construct such a separated object $ E^{\prime} $, we first note the following definition and the lemmas:
\begin{definition}
	(\textbf{Graph of an arrow})\emph{ Suppose} $ \cat{E} $\emph{ is a topos and} $ f : A\longrightarrow B $ \emph{is an arrow in it. The graph of $ f $ is defined to be the following subobject:}
	\[\begin{tikzcd}
		A && {A\times B}
		\arrow["{\upair{1}{f}}", tail, from=1-1, to=1-3]
	\end{tikzcd}.\]
	\emph{We also write this subobject as }$ G(f) := \upair{1}{f} $.
\end{definition}
Now, we show that any subobject of a separated object is also separated:
\begin{lemma}\label{L-16}
	Suppose $ \cat{E} $ is a topos and $ m : B\rightarrowtail C $ is a subobject. If $ C $ is separated, then so is $ B $.
\end{lemma}
\begin{proof}
	Take any dense subobject $ k : S\rightarrowtail E $. We then have the following
	\[\begin{tikzcd}
		{\homset{\cat{E}}{E}{C}} && {\homset{\cat{E}}{S}{C}} \\
		{\homset{\cat{E}}{E}{B}} && {\homset{\cat{E}}{S}{B}}
		\arrow["{\homset{\cat{E}}{E}{m}}", tail, from=2-1, to=1-1]
		\arrow["{\homset{\cat{E}}{S}{m}}"', tail, from=2-3, to=1-3]
		\arrow["{\homset{\cat{E}}{k}{C}}", from=1-1, to=1-3]
		\arrow["{\homset{\cat{E}}{k}{B}}"', from=2-1, to=2-3]
	\end{tikzcd}\]
	where both the left and right vertical arrows are injective. Now, the bottom arrow $\homset{\cat{E}}{k}{B}  $ is injective because if for two $ x,y\in \homset{\cat{E}}{E}{B} $ we have $ x\circ k = y\circ k $, then since $ \homset{\cat{E}}{k}{C} $ is injective because $ C $ is separated, therefore we will have \begin{align*}
		m\circ x\circ k &= m\circ y \circ k\\
		\implies m\circ x &= m\circ y &&\text{ because $ -\circ k $ of top arrow is injective}\\
		\implies x&= y
		\end{align*}
	Hence proved.
\end{proof}
We next show the equivalent conditions for graph of any arrow to be a closed subobject:
\begin{lemma}
	Suppose $ \cat{E} $ is a topos. Let $ C $ be any object in $ \cat{E} $ and $ j : \Omega \longrightarrow \Omega $ is an LT topology in $ \cat{E} $. Then, the following are equivalent:
	\begin{enumerate}
		\item {$ C $ is separated.}
		\item {The diagonal $ \Delta_C \in \Sub{C\times C}{\cat{E}}$ is a closed subobject.}
		\item {The following commutes 
	\[\begin{tikzcd}
		C & {\Omega^C} \\
		& {\Omega^C}
		\arrow["{\{\cdot\}_C}", from=1-1, to=1-2]
		\arrow["{j^C}", from=1-2, to=2-2]
		\arrow["{\{\cdot\}_C}"', from=1-1, to=2-2]
	\end{tikzcd}.\]	
	}
\item {For any $ f : A\longrightarrow C $, the graph $ f $, $ G(f) $, is a closed subobject of $ A\times C $.}
	\end{enumerate} 
\end{lemma}
\begin{proof}
	($ 1  \implies 2$) If $ C $ is separated, then because $ \Delta_C \subset \overline{\Delta_C}$, and so we have the usual dense subobject $ k : C\to \overline{C} $. Therefore, we have $ \homset{\cat{E}}{k}{C} : \homset{\cat{E}}{\overline{C}}{C} \rightarrowtail \homset{\cat{E}}{C}{C} $. Take $ \pi_1\circ \bar{\Delta_C} $ and $ \pi_2 \circ \bar{\Delta_C}\in \homset{\cat{E}}{\bar{C}}{C} $ where $ \pi_1,\pi_2 : C\times C\rightrightarrows C$ are the projections. But then $ \pi_1\circ \bar{\Delta_C} \circ k =  \pi_2\circ \bar{\Delta_C} \circ k \implies \pi_1\circ \bar{\Delta_C} = \pi_2\circ \bar{\Delta_C} $ as $ -\circ k $ is injective as $ C $ is separated. Therefore $ \bar{\Delta_C} $ forms a cone over $\pi_1,\pi_2:  C\times C \rightrightarrows C $, therefore there exists unique $ l : \bar{C} \longrightarrow C $ with $ \Delta_C \circ l = \bar{\Delta_C} $ which means that $ \bar{\Delta_C} \subset \Delta_C $ so that $ \bar{\Delta_C} = \Delta_C $.\\
	($ 2\implies 3 $) This is trivial because if $\Delta_C  : C\times C \rightarrowtail C$ is a closed subobject, then $ j\circ \chr{\Delta_C} = \chr{\Delta_C} $. This commuting diagram gives rise to another commuting diagram obtained by it's $ P $-transpose, which proves the result:
	\[\begin{tikzcd}
		{C\times C} && \Omega && C && {PC \isomorph \Omega^C} \\
		&& \Omega &&&& {PC \isomorph \Omega^C}
		\arrow["{\chr{\Delta_C}}", from=1-1, to=1-3]
		\arrow[""{name=0, anchor=center, inner sep=0}, "j", from=1-3, to=2-3]
		\arrow["{\chr{\Delta_C}}"', from=1-1, to=2-3]
		\arrow[""{name=1, anchor=center, inner sep=0}, "{\Ptr{\chr{\Delta_C}}}"', from=1-5, to=2-7]
		\arrow["{\Ptr{\chr{\Delta_C}}}", from=1-5, to=1-7]
		\arrow["{j^C}", from=1-7, to=2-7]
		\arrow["{P-\text{Transpose}}", shorten <=30pt, shorten >=30pt, from=0, to=1]
	\end{tikzcd}.\]	
    ($ 2\implies 4 $) For $ f : A\to C $, the graph $ G(f) = \upair{1}{f} : A \rightarrowtail A\times C $ is obtained by the pullback of $ \Delta_C $ along $ f\times 1 $. The naturality of the closure operator proves the rest.\\
    ($ 4 \implies 1 $) Suppose for $ f : A\to C $ the graph $ G(f) = \upair{1}{f} : A \rightarrowtail A\times C $ is closed. Take a dense subobject $ m : S\rightarrowtail B $ and let $ b_1,b_2 : B\rightrightarrows C $ be such that $ b_1\circ m = b_2 \circ m $. We wish to prove that $ b_1 = b_2 $. Next, let's look at the graph of $ b_1 $ and $ b_1\circ m $:
    \[\begin{tikzcd}
    	S && B && C \\
    	{S\times C} && {B\times C} && {C\times C}
    	\arrow["{\Delta_C}", tail, from=1-5, to=2-5]
    	\arrow["{b_1\times 1}"', from=2-3, to=2-5]
    	\arrow["{b_1}", from=1-3, to=1-5]
    	\arrow["{\upair{1}{b_1}}", tail, from=1-3, to=2-3]
    	\arrow["\lrcorner"{anchor=center, pos=0.125}, draw=none, from=1-3, to=2-5]
    	\arrow["{m\times 1}"', tail, from=2-1, to=2-3]
    	\arrow["{\upair{1}{b_1\circ m}}"', tail, from=1-1, to=2-1]
    	\arrow["m", tail, from=1-1, to=1-3]
    \end{tikzcd}\]
	where $ \Delta_C $ is closed, $ \upair{1}{b_1} $ is then closed and then $ \upair{1}{b_1\circ m} $ is also closed. Right and the whole square are pullbacks, therefore left one is, and $ m\times 1$ is dense, which means $ \overline{\upair{1}{b_1\circ m}} = \upair{1}{b_1\circ m}= \upair{1}{b_1} $. Hence $ b_1\circ m = b_2\circ  m\implies \upair{1}{b_1} = \upair{1}{b_2}\implies b_1 = b_2$. 
\end{proof}
We now construct the separated object $ E^{\prime} $ for each object $ E $ in a topos $ \cat{E} $, in the following lemma:
\begin{lemma}\label{L-18}
	Suppose $ \cat{E} $ is a topos and $ j : \Omega \longrightarrow \Omega $ is an LT topology in it. For any object $ E $ in $ \cat{E} $, there is an epimorphism 
	\begin{align*}
		\theta_E : E \longrightarrow E^{\prime}
	\end{align*}
where $ E^{\prime} $ is a separated object in $ \cat{E} $.
\end{lemma}
\begin{proof}
	Consider the arrow 
	\[\begin{tikzcd}
		E & {\Omega^E} & {\Omega_j^E}
		\arrow["{\{\cdot\}_E}", from=1-1, to=1-2]
		\arrow["{r^E}", from=1-2, to=1-3]
	\end{tikzcd}\]
 where as usual, the $ \Omega_j $ is the mono-epic factor of $ j $ or equivalently the equalizer of $ 1 $ and $ j $. Now denote $ E^{\prime} $ as the mono-epic factor of $ r^{E}\circ \{\cdot\}_E$, as shown below:
 \[\begin{tikzcd}
 	& {E^\prime} \\
 	E && {\Omega_j^E}
 	\arrow["{r^E \circ \{\cdot\}_E}"', from=2-1, to=2-3]
 	\arrow[tail, from=1-2, to=2-3]
 	\arrow["{\theta_E}", two heads, from=2-1, to=1-2]
 \end{tikzcd}.\]
Now, by Lemma \ref{L-16}, $ E^{\prime} $ is a separated object because $ \Omega_j^{E} $ is.
\end{proof}
\subsubsection{The left adjoint of $ i : \sep{j}{\cat{E}}\hookrightarrow \cat{E} $}
Remember our aim is to construct first a left adjoint of inclusion $ i : \sep{j}{\cat{E}} \hookrightarrow \cat{E} $. To this end, we have found a way to form a separated object for any object of $ \cat{E} $. We now wish to show that this construction of separated object is indeed a left adjoint to inclusion. For which we have to check it's universality. For that, we have the following lemmas
\begin{lemma}\label{L-19}
	Suppose $ \cat{E} $ is a topos. For any object $ E $ of $ \cat{E} $, there exists an epimorphism $ \theta_E : E \longrightarrow E^{\prime} $ such that the kernel pair of $ \theta_E $ is the closure $ \overline{\Delta_E} $ of the subobject $ \Delta_E : E \rightarrowtail E\times E $.
\end{lemma}
\begin{proof}
	Section 5.3, p.p. 229, Lemma 5, \cite{MacMoer}.
\end{proof}
An immediate corollary of the above lemma proves the universality of the construction in Lemma \ref{L-18}:
\begin{corollary}\label{C-5}
	Suppose $ \cat{E} $ is a topos and $ j : \Omega\longrightarrow \Omega $ is an LT topology in it. For each object $ E $ of $ \cat{E} $, the corresponding epimorphism to a separated object $ E^{\prime} $
	\begin{align*}
		\theta_E : E \longrightarrow E^{\prime}
	\end{align*} 
is universal amongst all arrows from $ E $ to a separated object.
\end{corollary}

\begin{proof}
	If there is an arrow $ f : E\to S $ where $ S $ is separated, then by Theorem \ref{T-12} and Lemma \ref{L-19}, the epic $ \theta_E : E \longrightarrow E^{\prime}$ is given as the following coequalizer 
	\[\begin{tikzcd}
		&&&& S \\
		{\bar{E}} && E && {E^\prime}
		\arrow["{\pi_1\circ \bar{\Delta}_E}", shift left=2, from=2-1, to=2-3]
		\arrow["{\pi_2 \circ \bar{\Delta}_E}"', shift right=2, from=2-1, to=2-3]
		\arrow["{\theta_E}"', two heads, from=2-3, to=2-5]
		\arrow["f", from=2-3, to=1-5]
	\end{tikzcd}.\]
	Now, because $ f\circ \pi_1\circ \Delta_E = f = f\circ \pi_2 \circ \Delta_E $, therefore $ f\circ \pi_1 \circ \bar{\Delta}_E \circ k=  f\circ \pi_2 \circ \bar{\Delta}_E\circ k  $, where $ k : E \rightarrowtail \bar{E}$, so, because $ -\circ k : \homset{\cat{E}}{\bar{E}}{S} \longrightarrow \homset{\cat{E}}{E}{S}  $ is injective because $ S $ is separated, therefore, $ f\circ \pi_1 \circ \bar{\Delta}_E =  f\circ \pi_2 \circ \bar{\Delta}_E $ and hence $ \exists ! : l : S\to E^{\prime} $.
\end{proof}
Finally, we can now conclude that indeed, the construction of separated object $ E^{\prime} $ is the left adjoint of inclusion:
\begin{corollary}\label{C-6}
	Suppose $ \cat{E} $ is a topos and $ j : \Omega \longrightarrow \Omega $ is an LT topology in it. Then there is an adjunction as given below:
	\[\begin{tikzcd}
		{\cat{E}} && {\sep{j}{\cat{E}}}
		\arrow[""{name=0, anchor=center, inner sep=0}, "{L_1}", curve={height=-24pt}, from=1-1, to=1-3]
		\arrow[""{name=1, anchor=center, inner sep=0}, "i", curve={height=-24pt}, hook', from=1-3, to=1-1]
		\arrow["\dashv"{anchor=center, rotate=-90}, draw=none, from=0, to=1]
	\end{tikzcd}\]
where $ L_1 $ is given by:
\begin{align*}
	L_1 : \cat{E} &\longrightarrow \sep{j}{\cat{E}}\\
			E &\longmapsto E^{\prime}\\
			(f : E \to F) &\longmapsto (f^{\prime} : E^{\prime} \to F^{\prime}) 
\end{align*}
where $ f^{\prime} $ is given by the following universal arrow because of $ \theta_F \circ f $ forming a cocone over the top coequalizer diagram as in the following:
\[\begin{tikzcd}
	{\bar{E}} & E & {E^\prime} \\
	{\bar{F}} & F & {F^\prime}
	\arrow["{\pi_1\circ \bar{\Delta}_E}", shift left=2, from=1-1, to=1-2]
	\arrow["{\pi_2\circ \bar{\Delta}_E}"', shift right=2, from=1-1, to=1-2]
	\arrow["{\theta_E}", two heads, from=1-2, to=1-3]
	\arrow["{p_1\circ \bar{\Delta}_F}", shift left=2, from=2-1, to=2-2]
	\arrow["{p_2\circ \bar{\Delta}_F}"', shift right=2, from=2-1, to=2-2]
	\arrow["{\theta_F}", two heads,from=2-2, to=2-3]
	\arrow["f", from=1-2, to=2-2]
	\arrow["{f^\prime}", dashed, from=1-3, to=2-3]
\end{tikzcd}.\]
\end{corollary}
\begin{proof}
	To show that $ L_1 $ as above is indeed the left adjoint, take any object $ E $ and a separated object $ G $ and then take any arrow $  f : E \longrightarrow iG $ in $ \cat{E} $, and then just observe that the following commutes:
	\[\begin{tikzcd}
		E \\
		\\
		{iL_1(E) = i(E^\prime)} && {i(G)}
		\arrow["f", from=1-1, to=3-3]
		\arrow["{\eta_E}"', from=1-1, to=3-1]
		\arrow["{\text{By Coroll. \ref{C-5}}}"', dashed, from=3-1, to=3-3]
	\end{tikzcd}\]
Hence proved that $ L_1 $ is left adjoint of inclusion.
\end{proof}
\subsubsection{The left adjoint of $ i : \sh{j}{\cat{E}} \hookrightarrow \sep{j}{\cat{E}}$}
And finally, we have the left adjoint of the inclusion $ i : \sh{j}{\cat{E}} \hookrightarrow \sep{j}{\cat{E}} $:
\begin{lemma}\label{L-20}
	Suppose $ \cat{E} $ is a topos and $ j : \Omega \longrightarrow \Omega $ is an LT topology. Then there is an adjunction as given below:
	\[\begin{tikzcd}
		{\sep{j}{\cat{E}}} && {\sh{j}{\cat{E}}}
		\arrow[""{name=0, anchor=center, inner sep=0}, "i", curve={height=-24pt}, hook', from=1-3, to=1-1]
		\arrow[""{name=1, anchor=center, inner sep=0}, "{L_2}", curve={height=-24pt}, from=1-1, to=1-3]
		\arrow["\dashv"{anchor=center, rotate=-90}, draw=none, from=1, to=0]
	\end{tikzcd}\]
	where $ L_2 $ is given by:
	\begin{align*}
		L_2 : \sep{j}{\cat{E}} &\longrightarrow \sh{j}{\cat{E}}\\
		E &\longmapsto \bar{E}\\
		(f : E \to F) &\longmapsto (\bar{f} : \bar{E} \to \bar{F})
	\end{align*}
	where the closure of a separated subobject is a sheaf because $ \overline{r^{E}\circ \{\cdot\}_E} : \bar{E} \rightarrowtail \Omega_j^{E} $ is a closed subobject of the sheaf $ \Omega_j^{E} $ and so by Lemma \ref{L-14}, $ \bar{E} $ is a sheaf.
\end{lemma}
\begin{proof}
	To show that $ L_2 $ is indeed the left adjoint of inclusion, take any $ f : E \longrightarrow i(F) $ in $ \sep{j}{\cat{E}} $ where $ E $ is separated and $ F $ is a sheaf. We then have the following commuting diagram, which establishes the result:
	\[\begin{tikzcd}
		E \\
		\\
		{iL_2(E)=i(\bar{E})=\bar{E}} &&& {i(F)} && {}
		\arrow["f", from=1-1, to=3-4]
		\arrow["{\text{(dense)}}"', tail, from=1-1, to=3-1]
		\arrow["{\text{(sheaf)}}"{description}, draw=none, from=3-6, to=3-4]
		\arrow["{\text{by sheaf condition of }F}"', dashed, from=3-1, to=3-4]
	\end{tikzcd}\]
Hence proved.
\end{proof}
Finally, we have the sheafification functor:
\begin{theorem}\label{T-14}
	Suppose $ \cat{E} $ is a topos and $ j : \Omega\longrightarrow \Omega $ is an LT topology in it. Then the full-subcategory $ \sh{j}{\cat{E}} $ is reflective. That is, there is a left adjoint of inclusion:
	\[\begin{tikzcd}
		{\cat{E}} && {\sh{j}{\cat{E}}}
		\arrow[""{name=0, anchor=center, inner sep=0}, "a", curve={height=-24pt}, from=1-1, to=1-3]
		\arrow[""{name=1, anchor=center, inner sep=0}, "i", curve={height=-24pt}, hook', from=1-3, to=1-1]
		\arrow["\dashv"{anchor=center, rotate=-90}, draw=none, from=0, to=1]
	\end{tikzcd}\]
where $ a = L_2\circ L_1 $ as in Lemma \ref{L-20} and Corollary \ref{C-6}.
\end{theorem}
\begin{proof}
	Take any $ f : E \longrightarrow i(F) $ in $ \cat{E} $ where $ E $ is any object in $ \cat{E} $ and $ F $ is any sheaf. We then have:
	\[\begin{tikzcd}
		&& {} \\
		&& {E^\prime} & E \\
		\\
		{} && {a(E) =\bar{E^\prime}} &&&& {i(F)}
		\arrow["f", from=2-4, to=4-7]
		\arrow["{\theta_E}"', two heads, from=2-4, to=2-3]
		\arrow["{(\text{dense})}"', tail, from=2-3, to=4-3]
		\arrow["{\text{(sep.)}}"{description}, draw=none, from=1-3, to=2-3]
		\arrow["{\text{(sheaf)}}"{description}, draw=none, from=4-1, to=4-3]
		\arrow["{\text{by sheaf cond. of }F}"{description}, dashed, from=4-3, to=4-7]
	\end{tikzcd}.\]
	Hence proved.
\end{proof}
The following shows, like all sheafification adjunction studied previously, that the left adjoint $ a $ is left-exact:
\begin{proposition}
	Suppose $ \cat{E} $ is a topos and $ j : \Omega\longrightarrow \Omega $ is an LT topology in it. Then $ a $ in the adjunction $ i \vdash a $ of Theorem \ref{T-14} is left exact.
\end{proposition}
\begin{proof}
	Section 3.3, p.p. 232, \cite{MacMoer}.
\end{proof}

\newpage 
\section{Geometric Morphisms}
One of the important aspect that we are witnessing continuously in the above sections is the repetitive rise of left exactness of the left adjoint in the sheafification adjunction. This is a general phenomenon for maps between two topoi, the study of which leads to a natural notion of \emph{points of a topos} and generalization of tensor product. We hence define a geometric morphism between two topoi as an adjoint pair where the left adjoint is left exact:
\begin{definition}\label{D-44}
	(\textbf{Geometric Morphism}) \emph{Suppose} $ \cat{E} $ \emph{and} $ \cat{F} $ \emph{are two topoi. An adjunction }
	\[\begin{tikzcd}
		{\cat{E}} && {\cat{F}}
		\arrow[""{name=0, anchor=center, inner sep=0}, "{f_*}", curve={height=-24pt}, from=1-3, to=1-1]
		\arrow[""{name=1, anchor=center, inner sep=0}, "{f^*}", curve={height=-24pt}, from=1-1, to=1-3]
		\arrow["\dashv"{anchor=center, rotate=-90}, draw=none, from=1, to=0]
	\end{tikzcd}\]
	\emph{where the left adjoint $ f^* $ is also left exact (preserves finite limits) is then called a geometric morphism and is denoted as}
	\[\begin{tikzcd}
		{\cat{F}} && {\cat{E}}
		\arrow["f", from=1-1, to=1-3]
	\end{tikzcd}.\]
	\emph{The left adjoint $ f^* $ is called the inverse-image part and the right adjoint $ f_* $ is called the direct-image part of the geometric morphism $ f $.}
\end{definition}
\begin{example}
	A trivial example is that of direct and inverse image of sheaves. Take two topological spaces $ X $ and $ Y $ and a continuous map $ f : X\to Y $. Then there is the following adjunction
	\[\begin{tikzcd}
		{\Sh{Y}} && {\Sh{X}}
		\arrow[""{name=0, anchor=center, inner sep=0}, "{f_* }", curve={height=-24pt}, from=1-3, to=1-1]
		\arrow[""{name=1, anchor=center, inner sep=0}, "{f^*}", curve={height=-24pt}, from=1-1, to=1-3]
		\arrow["\dashv"{anchor=center, rotate=-90}, draw=none, from=1, to=0]
	\end{tikzcd}\]
	where 
	\begin{align*}
		f_* : \Sh{X}&\longrightarrow \Sh{Y}\\
		F &\longmapsto F(\inv{f}(-))
	\end{align*}
and 
\begin{align*}
	f^* : \Sh{Y} &\longrightarrow \Sh{X}\\
			F &\longrightarrow \Gamma_{(f^*(\Lambda_F))}.
\end{align*}
Note that the $ f^* $ in sub-script is the pullback functor. The functors $ \Gamma_{(-)} $ and $ \Lambda_{-} $ in the adjunction $ \Lambda_{(-)} \vdash\Gamma_{(-)}$ are as follows:
\begin{align*}
	\Gamma_{(-)} : \Bund{X} &\longrightarrow \Sh{X}\\
					(p : Y \to X) &\longmapsto (F: \opcat{O(X)} \to \cat{Sets})\\
								  &\;\;\;\;\;\;\;\;\;\;\;\;\;\;\;\;\;\;\;\;\;\;U \mapsto \left \{s : U \to Y \;\vert\; p\circ s = \iota : U \hookrightarrow X\right \}.
\end{align*}
and
\begin{align*}
	\Lambda_{(-)} : \Sh{X} &\longrightarrow \Bund{X}\\
					F & \longmapsto (p : \Lambda_F \to X) 
\end{align*}
where $ \Lambda_F := \left \{ \germ{s}{x} \;\vert\; \forall s\in FU,\forall U \in \nbdsys{x},\forall x\in X\right \}$. Hence $ \Gamma $ is just the sheaf of cross-sections functor and the $ \Lambda $ is the trivial \'etale bundle of germs functor.
\end{example}
We have the following characterization of geometric morphisms between two sheaf topoi $ \Sh{X} $ and $ \Sh{Y} $:
\begin{lemma}
	Consider two topological spaces $ X $ and $ Y $ where $ Y $ is Hausdorff and their corresponding sheaf topoi $ \Sh{X} $ and $ \Sh{Y} $. Then there is a bijection between geometric morphisms between $ \Sh{X} $ and $ \Sh{Y} $ and continuous maps between $ X $ and $ Y $.
\end{lemma}
\begin{proof}
	Take any continuous map $ f : X\longrightarrow Y $. The above example shows that we get a geometric morphism $ f : \Sh{X} \longrightarrow \Sh{Y} $. For a geometric morphism $ f : \Sh{X} \longrightarrow \Sh{Y} $, where $ Y $ is Hausdorff, we wish to find a continuous mapping $ \hat{f}: X\to Y $. We first note that because $ f^* $ is left exact, therefore all subobjects of $ \bm{1} $ of $ \Sh{Y} $ are preserved, that is, a subobject $ F\rightarrowtail \bm{1} $ in $ \Sh{Y} $ is mapped to a subobject $ f^*F \rightarrowtail \bm{1} $ in $ \Sh{X} $. But a subobject of $ \bm{1} $ in $ \Sh{Y} $ is an open set of $ Y $ (Proposition \ref{P-7}). Therefore we have the map $ f^* : O(Y) \to  O(X)$. Now define a function $ \hat{f} : X\longrightarrow Y $ which takes $ x\in X $ to that $ y \in Y$ for which $ x\in f^*(V) \;\forall V\in \nbdsys{y} $. Let's first show that this function $ \hat{f} $ is well defined. If $ y_1\neq y_2 \in Y$ are such that $ \exists x\in X$ with $ \hat{f}(x) = y_1 $ and $ \hat{f}(x) = y_2 $, then $ x\in f^*(V) \forall V\in \nbdsys{y_1} $ and similarly for $ y_2 $. Since $ Y $ is Hausdorff, therefore $ \exists $ open $ V_1 \ni y_1 $ and $ V_2 \ni y_2 $ such that $ V_1\intrs V_2 = \phi$, which means that $x\in f^*(V_1) \intrs f^*(V_2) = f^*(V_1\intrs V_2) = \phi $ which is a contradiction, where the first equality holds because $ f^* $ preserves limits and $ -\intrs - $ is the pullback of subobjects. Similarly, if $ x\in X $, then $ \exists y\in Y$ such that $ \hat{f}(x) = y $ because if it's not $ \forall y\in Y$, $ \exists V_y\in \nbdsys{y} $ such that $ x\notin f^*(V) $. But then if we collect all such open $ V_y \ni y \;\forall y\in Y$ as in $ \bunion_{y\in Y} V_y $, then clearly $ \bunion_{y\in Y}V_y = Y $ and $ f^*(Y) = X \ni x$, which is a contradiction. This proves that $ f^* $ is well defined. We now wish to show that this function $ \hat{f} $ is continuous. For this, take any open $ V\subseteq Y $. We have $ \inv{\hat{f}}(V) =\{x\in X\;\vert\; \hat{f}(x) \in V \} = \{x\in X\;\vert\; x\in f^*(V)\} = f^*(V) $ which is clearly open. 
\end{proof}
\begin{example}
	There are other examples of geometric morphisms, like the change of base adjunction, sheafification adjunction and global sections adjunction among others. More succinctly, the following adjunctions are geometric morphisms:
	\[\begin{tikzcd}
		{\cat{E}/A} && {\cat{E}/B} && {\cat{E}} && {\sh{j}{\cat{E}}} && {\cat{Sets}} && {\cat{E}}
		\arrow[""{name=0, anchor=center, inner sep=0}, "{k^* := \inv{k}}", curve={height=-24pt}, from=1-1, to=1-3]
		\arrow[""{name=1, anchor=center, inner sep=0}, "{k_* := \prod_k}", curve={height=-24pt}, from=1-3, to=1-1]
		\arrow[""{name=2, anchor=center, inner sep=0}, "{i^*:= a}", curve={height=-24pt}, from=1-5, to=1-7]
		\arrow[""{name=3, anchor=center, inner sep=0}, "{i_*:=i}", curve={height=-24pt}, hook', from=1-7, to=1-5]
		\arrow[""{name=4, anchor=center, inner sep=0}, "{\Delta(-):= \coprod_{s\in (-)}\bm{1}_{\cat{E}}}", curve={height=-24pt}, from=1-9, to=1-11]
		\arrow[""{name=5, anchor=center, inner sep=0}, "{\Gamma(-) := \homset{\cat{E}}{\bm{1}}{-}}", curve={height=-24pt}, from=1-11, to=1-9]
		\arrow["\dashv"{anchor=center, rotate=-90}, draw=none, from=0, to=1]
		\arrow["\dashv"{anchor=center, rotate=-90}, draw=none, from=2, to=3]
		\arrow["\dashv"{anchor=center, rotate=-90}, draw=none, from=4, to=5]
	\end{tikzcd}.\]
\end{example}
It is interesting to note that the collection of all topoi and geometric morphisms is a $ 2 $-category:
\begin{definition}
	(\textbf{$ 2 $-Category of Topoi \& Geometric Morphisms}) \emph{Consider the category} $ \cat{Topoi} $ \emph{whose objects are topoi and arrows are geometric morphisms between them. Then, }
	\begin{itemize}
		\item {$ \homset{\cat{Topoi}}{\cat{E}}{\cat{F}} $ \textbf{is a category} \emph{where objects are geometric morphisms} $ f : \cat{E} \longrightarrow\cat{F} $\emph{ and an arrow between two geometric morphisms $ f $ and $ g $ is given by a natural transformation} $ \eta^* : f^* \Rightarrow g^* $\emph{ or equivalently by a natural transformation} $ \eta_* : g_* \Rightarrow f_* $.}
		\item {\textbf{The $ 2 $-category} $ \cat{Topoi} $ \emph{is formed by noticing that any geometric morphism }$ g : \cat{G} \longrightarrow \cat{F} $ \emph{induces a functor} $ \homset{\cat{Topoi}}{g}{\cat{E}} $ \emph{for any topoi} $ \cat{E} $ \emph{as follows:}
		\begin{align*}
			\homset{\cat{Topoi}}{g}{\cat{E}} : \homset{\cat{Topoi}}{\cat{F}}{\cat{E}} &\longrightarrow \homset{\cat{Topoi}}{\cat{G}}{\cat{E}}\\
			(f : \cat{F} \to \cat{E}) &\longmapsto (f\circ g : \cat{G} \to \cat{E})\\
			(\eta : h\to f) &\longmapsto (\eta \circ g : h \circ g  \to f\circ g)
		\end{align*}	
	\emph{	where $ (\eta\circ g)^* : (h\circ g)^* \to (f\circ g)^* $ is same as $ g^* \circ \eta^* : g^* \circ h^* \Rightarrow g^* \circ f^*$.}
	}
	\end{itemize}
\end{definition}
Therefore the 2-category $ \cat{Topoi} $ is where objects are topoi $ \cat{E},\cat{F}, \dots $, $ 1 $-cells are geometric morphisms $ f : \cat{F} \to \cat{E} $, $ \dots $ and $ 2 $-cells are natural transformations between $ 1 $-cells $ \eta : f\rightarrow g $.
\subsection{Tensor Products}
We now study the generalization of tensor products of modules to that of arbitrary contravariant and covariant functors. First, let us note the famous $ \tens $-hom adjunction of modules:
\begin{align*}
	\homset{R}{X_S\tens {_SZ_R}}{Y_R} \isomorph \homset{S}{X_S}{\underline{\text{Hom}_R}(_SZ_R,Y_R)}
\end{align*}
where $ X_S, {_SZ_R}, Y_R $ is a left $ S $-module, left $ S $ right $ R $-module and a right $ R $-module respectively. Moreover, $ \underline{\text{Hom}_R}(_SZ_R,Y_R) $ is the right $ S $-module of $ R $-linear maps $ _SZ_R \to Y_R $. It is important to note here that the above adjunction is not a geometric morphism. But tensor products would subsequently help us in making new geometric morphisms.\\
The tensor product above is, in essence, between two functors because the right $ S $-module $ X_S $ can be represented as the contravariant functor $ F: \opcat{1}_{\cat{AbGrp}} \to \cat{AbGrp} $ where $ \cat{1} $ is the one object $ \cat{AbGrp} $ enriched category and $ F $ as the $ \cat{AbGrp} $ enriched functor and similarly $ _SZ_R $ would be a bifunctor given by the $ \cat{AbGrp}  $ enriched functor $ G : \opcat{1}_{\cat{AbGrp}} \times \cat{1}_{\cat{AbGrp}} \longrightarrow \cat{AbGrp} $.\\
Now, suppose that we have a small category $ \cat{C} $ and a co-complete category $ \cat{E} $ with a functor $ A: \cat{C} \longrightarrow\cat{E} $. What is then the meaning of tensor product of a presheaf $ P : \opcat{C} \longrightarrow \cat{Sets} $ and $ A $?
\[\begin{tikzcd}
	{\opcat{C}} & {\cat{Sets}} & \bigotimes & {\cat{C}} & {\cat{E}}
	\arrow["{??}", curve={height=-6pt}, draw=none, from=1-2, to=1-4]
	\arrow["P", from=1-1, to=1-2]
	\arrow["A", from=1-4, to=1-5]
\end{tikzcd}\]
To obtain such a general notion of tensor product, we will need to understand the \emph{category of elements} (Definition \ref{D-4}) construction in the very first result, the Theorem \ref{T-1}. We would essentially define the left adjoint $ L $ as the tensor product functor $ - \tens_{\cat{C}} A $ with $ A $. But elaborating this would make it clear on how one should approach tensor products.\\
Recall that all colimits can be constructed from coproducts and coequalizers. In particular, take $  H : \cat{J}\longrightarrow\cat{E} $ to be a diagram in $ \cat{E} $ of index $ J $. Now consider the coproducts $ \coprod_{(u : i\to j) \in \arr{\cat{J}}} H(\dom{u}) $ and $ \coprod_{i\in \obj{\cat{J}}} H(i)$. Consider the following two parallel arrows: 
\[\begin{tikzcd}
	{\coprod_{(u : i\to j)\in \arr{\cat{J}}} H(\dom{u})} && {\coprod_{i\in \obj{\cat{J}}} H(i)}
	\arrow["\theta", shift left=3, from=1-1, to=1-3]
	\arrow["\tau"', shift right=3, from=1-1, to=1-3]
\end{tikzcd}.\]
The $ \theta $ is formed because $ \coprod_{i\in \obj{\cat{J}}} H(i) $ forms a cocone over $ \{H(\dom{u})\}_{u\in \arr{\cat{J}}} $ by injections $ \kappa_i $ and $ \tau $ is formed because $ \coprod_{i\in \obj{\cat{J}}} H(i) $ forms a cocone over $ \{H(\dom{u})\}_{u : i\to j\in \arr{\cat{J}}} $ by $ \kappa_j\circ H(u) $. The colimit of the diagram $ H $ is then given as the coequalizer of $ \theta $ \& $\tau $:
\[\begin{tikzcd}
	{\coprod_{(u : i\to j)\in \arr{\cat{J}}} H(\dom{u})} && {\coprod_{i\in \obj{\cat{J}}} H(i)} && {\colim H}
	\arrow["\theta", shift left=3, from=1-1, to=1-3]
	\arrow["\tau"', shift right=3, from=1-1, to=1-3]
	\arrow["\phi", from=1-3, to=1-5]
\end{tikzcd}.\]
Now, replacing $ H $ by $ A\circ \pi_P $, we get the following (where $ \pi_P : \int_\cat{C} P \to \cat{C} $ is the projection):
\[\begin{tikzcd}
	{\coprod_{\left (u : (C,p)\to (C^\prime,p^\prime)\right )\in \arr{\int_{\cat{C}}P}}A(C)} && {\coprod_{(C,p)\in \obj{\int_{\cat{C}}P}} A(C)} && {\colim (A\circ \pi_P) =: L(P)}
	\arrow["\theta", shift left=3, from=1-1, to=1-3]
	\arrow["\tau"', shift right=3, from=1-1, to=1-3]
	\arrow["\phi", from=1-3, to=1-5]
\end{tikzcd}.\]
We simply define $L(P) := \colim (A\circ \pi_P) $ as the tensor product of $ P $ with $ A $, $ P\tens_{\cat{C}} A $. Then, by the Theorem \ref{T-1}, we have the following adjunction:
\[\begin{tikzcd}
	{\psheaf{C}} && {\cat{E}}
	\arrow[""{name=0, anchor=center, inner sep=0}, "{L := -\tens_{\cat{C}} A}", curve={height=-30pt}, from=1-1, to=1-3]
	\arrow[""{name=1, anchor=center, inner sep=0}, "{R:= \homset{\cat{E}}{A(\cdot)}{-}}", curve={height=-30pt}, from=1-3, to=1-1]
	\arrow["\dashv"{anchor=center, rotate=-90}, draw=none, from=0, to=1]
\end{tikzcd}.\]
This adjunction in terms of hom-sets is:
\[\begin{tikzcd}
	{\homset{\cat{E}}{P\tens_{\cat{C}} A}{E} } && {\homset{\psheaf{C}}{P}{\homset{\cat{E}}{A(-)}{E}}}
	\arrow["\isomorph"{description}, no head, from=1-1, to=1-3]
\end{tikzcd}.\]
\subsection{Points of a Topos}
Consider a topological space $ X $. Let $ x \in X $ be a point of it. One can alternatively write $ x \in X$ as an arrow $ x :\bm{1}\to X $ in the $ \cat{Top} $. But as we saw below Defn. \ref{D-44}, the fact that we then have a geometric morphism $ x : \cat{Sets} \longrightarrow \Sh{X} $ as follows:
\[\begin{tikzcd}
	{\Sh{X}} && {\Sh{\bm 1}\isomorph \cat{Sets}}
	\arrow[""{name=0, anchor=center, inner sep=0}, "{x^*}", curve={height=-24pt}, from=1-1, to=1-3]
	\arrow[""{name=1, anchor=center, inner sep=0}, "{x_*}", curve={height=-24pt}, from=1-3, to=1-1]
	\arrow["\dashv"{anchor=center, rotate=-90}, draw=none, from=0, to=1]
\end{tikzcd}.\]
Hence we are representing the point $ x $ of the \emph{underlying space} of the sheaf topos $ \Sh{X} $ as a geometric morphism $ x : \cat{Sets} \longrightarrow\Sh{X} $. This becomes the motivation for the following definition:
\begin{definition}
	(\textbf{Points of a Topos}) \emph{Let} $ \cat{E} $ \emph{be a topos. A point $ f $ of topos} $ \cat{E} $ \emph{is defined to be a geometric morphism} $ f : \cat{Sets} \longrightarrow \cat{E} $.
\end{definition}
However, the more interesting observations lies in trying to characterize the points of a Grothendieck topos.
\subsubsection{Points of a Presheaf Topos}
We wish to characterize the points of a Grothendieck topos. In order to do so, we begin by studying the points of a presheaf topos $ \psheaf{C} $. Take a point of $ \psheaf{C} $
\begin{align*}
	f : \cat{Sets} \longrightarrow \psheaf{C}.
\end{align*}
We know that each object in $ \psheaf{C} $ is a colimit of representables by Proposition \ref{P-2}. But $ f^* $ preserves those colimits as it is a left adjoint. Therefore $ f^* $ can be studied by studying $ f^*\circ \yembed{-} : \cat{C} \longrightarrow \cat{Sets}$ only. Our aim is now to show that for each point $ f $, the functor $-\tens_{\cat{C}}( f^* \circ \yembed{-}) : \psheaf{C} \longrightarrow \cat{Sets} $ is isomorphic to $ f^* $. To further our discussion, we have to first simplify the definition of tensor product to set valued functors as it will make analysis easier:
\begin{definition}
	(\textbf{Tensor Product of $ \cat{Set} $ Functors}) \emph{Let} $ \cat{C} $ \emph{be a small category and} $ R : \opcat{C}\longrightarrow \cat{Sets} $ \emph{and }$ A : \cat{C} \longrightarrow\cat{Sets} $.\emph{ Then the tensor product} $ R\tens_{\cat{C}} A $ \emph{is alternatively given as:}
	\begin{align*}
		R\tens_{\cat{C}} A:= \coprod_{C\in \obj{\cat{C}}} RC\times AC / \sim
	\end{align*}
\emph{where} $ (r,a)_C \sim (r^{\prime},c^{\prime})_{C^{\prime}} $ \emph{if and only if} $ \exists (r_0,c_0)_{C_0}, (r_1,c_1)_{C_1} ,\dots, (r_n,c_n)_{C_n} \in \coprod_{C\in \obj{\cat{C}}} RC\times AC$ \emph{such that }
\begin{enumerate}
	\item {$ (r_0,c_0)_{C_0} = (r,a)_{C} $ \emph{and} $ (r_n,c_n)_{C_n} = (r^{\prime},a^{\prime})_{C^{\prime}}  $.}
	\item {$ \forall 1\le k \le n $, $ \exists u_k : C_k \longrightarrow C_{k-1} $ \emph{in} $ \cat{C} $ \emph{such that}
\begin{align*}
	Ru_k(r_{k-1}) &= r_k\\
	Au_k(a_k)  &= a_{k-1}
\end{align*}
\emph{OR, equivalently, }$ \exists u_k : C_{k-1} \longrightarrow C_k $ in $ \cat{C} $ \emph{such that }
\begin{align*}
	Ru_k(r_k) &= r_{k-1}\\
	Au_k(a_{k-1}) &= a_k.
\end{align*}	
}
\end{enumerate}
\end{definition}
With the above definition, we can see now the following:
\begin{proposition}
	Suppose $ \cat{C} $ is a small category and $ f : \cat{Sets} \longrightarrow \psheaf{C}$ is a point of the $ \psheaf{C} $. Then, there exists a unique functor $ A= f^{*} \circ \yembed{-} : \cat{C} \longrightarrow\cat{Sets}$ such that 
	\begin{align*}
		f^* \isomorph -\tens_{\cat{C}} A.
	\end{align*}
\end{proposition}
\begin{proof}
	Take any $ R : \opcat{C} \longrightarrow \cat{Sets} $. We then have $ f^*(R) $ and $ R\tens_{\cat{C}} A $ both in $ \cat{Sets} $. But for these two sets in $ \cat{Sets} $, we have a canonical map:
	\begin{align*}
	e_R : R\tens_{\cat{C}} A&\longrightarrow f^*(R)\\
	(r\tens a)_{C} &\longmapsto f^*(\eta^r)(a)
	\end{align*}
because for any $ r\in RC $, $ \exists ! \;\eta^r : \yembed{C} \Rightarrow R $ by Yoneda Lemma where $ f^*(\eta^r) : f^*(\yembed{C}) =: AC \longrightarrow f^*(R) $ takes $ a\in AC $ to $ f^*(\eta^r)(a) $. The well definiteness of $ e_R $ can be checked readily, that is, the fact that $ e_R((Rg(r) \tens a^{\prime})_{C^{\prime}} ) = e_R((r\tens Ag(a^{\prime}))_C ) $ can be seen via unraveling of definitions and Yoneda lemma. Now, we wish to show that $ e_R $ is an isomorphism. To this extent we just need to show that $ e_{\yembed{C}} $ is an isomorphism because every presheaf is a colimit of representables and $ -\tens_{\cat{C}}A $ is itself a colimit construction. In order to show this, we have
\begin{align*}
	e_{\yembed{C}} : \yembed{C} \tens_{\cat{C}} A \longrightarrow f^*(\yembed{arg1})
\end{align*}
But we have that $ A= f^*\circ \yembed{-} : \cat{C} \longrightarrow\cat{Sets}$ and because of the fact that $ \yembed{C} \tens_{\cat{C}} F \isomorph FC$ for any $ F : \cat{C}  \to \cat{Sets}$\footnote{This happens by Adjoint isomorphism and Yoneda Lemma:
\begin{align*}
	\homset{\cat{Sets}}{\yembed{C}\tens_{\cat{C}}A }{S} &\isomorph \homset{\psheaf{C}}{\yembed{C}}{\homset{\cat{Sets}}{A(-)}{S}}\\
	&= \Nat{\homset{\cat{C}}{-}{C}}{\homset{\cat{Sets}}{A(-)}{S}}\\
	&\isomorph \homset{\cat{Sets}}{AC}{S}.
\end{align*}
By generalized elements, we have the isomorphism $ \yembed{C}\tens_{\cat{C}} A \isomorph AC $.
}, therefore $ \yembed{C} \tens_{\cat{C}} A \isomorph f^*(\yembed{C}) $, hence $ e_{\yembed{C}} $ is an isomorphism.
\end{proof}
Now, by the virtue of tensor product, we already have an adjunction:
\[\begin{tikzcd}
	{\psheaf{C}} && {\cat{Sets}}
	\arrow[""{name=0, anchor=center, inner sep=0}, "{-\tens_{\cat{C}} A}", curve={height=-24pt}, from=1-1, to=1-3]
	\arrow[""{name=1, anchor=center, inner sep=0}, "{\homset{\cat{Sets}}{A(\cdot)}{-}}", curve={height=-24pt}, from=1-3, to=1-1]
	\arrow["\dashv"{anchor=center, rotate=-90}, draw=none, from=0, to=1]
\end{tikzcd}.\]
So if we forcefully assume that $ -\tens_{\cat{C}}  A$ is left-exact, a condition we then define as \emph{flatness }of $ A $, then, we can safely say that any such flat functor $ A: \cat{C} \longrightarrow \cat{Sets} $ gives a point of the presheaf topos $ \psheaf{C} $ because we then have the geometric morphism as above. This leads to following proposition, which we have just proved:
\begin{proposition}
	Suppose $ \cat{C} $ is a small category and $ A : \cat{C} \longrightarrow\cat{Sets} $ is a \textbf{flat} functor, then $ \exists $ a unique point of $ \psheaf{C} $ 
	\begin{align*}
		f : \cat{Sets} \longrightarrow \psheaf{C}
	\end{align*} 
where $ f^* := -\tens_{\cat{C}} A $ and $ f_* := \homset{\cat{Sets}}{A(\cdot)}{-} $.
\end{proposition}
We therefore have the following equivalence:
\begin{align*}
	\boxed{\text{Flat}(\cat{C},\cat{Sets}) \equiv \homset{\cat{Topoi}}{\cat{Sets}}{\psheaf{C}}}
\end{align*}
where $ \text{Flat}(\cat{C},\cat{Sets}) $ is the category of flat functors $ A : \cat{C}\longrightarrow\cat{Sets} $ and natural transformations.
\newpage
\section{Categorical Semantics}
What we have seen so far is the fact that a topos is an another framework/universe in which one can do \emph{sets-like mathematics}. But one of the important facets of the category of $ \cat{Sets} $ is that one can interpret logical theories in it, for example, the interpretation of theory of abelian groups in sets leads to an abelian group in the usual set-theoretic sense. But we already saw above that one can interpret abelian groups in a category with \emph{enough structure}, which was an internal group object. What we saw there was an example of categorical semantics; interpreting syntactic languages in a category (with enough structure). In this section we would develop this line of thinking more formally. We follow the \emph{Elephant} Section D1.2 \cite{Elephant}  for the discussion below. 
\subsection{$ \sign $-Structures}
\begin{definition}\label{D-39}
	(\textbf{$ \sign$-Structure in a Category}) \emph{Suppose $ \sign $ is a first order signature of a language and $ \cat{C} $ is a category with finite products. A $ \sign $-structure on $ \cat{C} $ is a map $ M $ from the signature $ \sign $ to $ \cat{C} $ which takes:}
		\begin{itemize}
			\item {\emph{\textbf{Finite list of Sorts}} $ A_1,\dots, A_n \in \ssort$ \emph{to an object 
				\begin{align*}
					MA_1 \times \dots \times MA_n
				\end{align*}
				 in $ \cat{C} $. The empty list is mapped to terminal object $ \bm{1} $ in $ \cat{C} $.}}
			\item {\emph{\textbf{Function symbol}} $ f: A_1\dots A_n \longrightarrow B \in \sfun $ \emph{to an arrow 
			\begin{align*}
				 Mf : MA_1\times \dots \times MA_n \longrightarrow MB
			\end{align*}	
		in $ \cat{C} $.}
		}
	\item {\emph{\textbf{Relation Symbol}} $ R\rightarrowtail A_1\dots A_n \in \srel$ \emph{to a monomorphism
\begin{align*}
	MR \rightarrowtail MA_1\times MA_n
\end{align*}	
in $ \cat{C} $.
 }}
		\end{itemize}
	
\end{definition}
The collection of all $ \sign $-structures over a category themselves form a category:
\begin{definition}\label{D-40}
	(\textbf{Category of $ \sign $-Structures}) \emph{Suppose $ \sign $ is a signature and $ \cat{C} $ is a category with finite products. We can then form a category denoted}
	\begin{align*}
		\sstruc{C}
	\end{align*}
\emph{whose:}
\begin{itemize}
	\item {\emph{\textbf{Objects} are $ \sign $-structures $ M,N,\dots $.}}
	\item {\emph{\textbf{Arrows} are $ \sign $-structure homomorphisms between two $ \sign $-structures $ M $ \& N, which are denoted by:}
\begin{align*}
	h : M\longrightarrow N
\end{align*}	
\emph{and defined as a collection of arrows in $ \cat{C} $}
\begin{align*}
	\left \{h_A : M_A \longrightarrow N_A \right \}_{A\in \ssort}
\end{align*}
\emph{for which the following two conditions hold:}
\begin{itemize}
	\item {\emph{The following commutes for each} $ f : A_1\dots A_N \longrightarrow B $ \emph{in }$ \sfun $:
\[\begin{tikzcd}
	{MA_1\times \dots\times MA_n} && MB \\
	\\
	{NA_1\times \dots\times NA_n} && NB
	\arrow["Mf", from=1-1, to=1-3]
	\arrow["Nf", from=3-1, to=3-3]
	\arrow["{h_{A_1}\times \dots\times h_{A_n}}"{description}, from=1-1, to=3-1]
	\arrow["{h_B}"{description}, from=1-3, to=3-3]
\end{tikzcd}.\]	
}
	\item {\emph{The following commutes for each} $ R \rightarrowtail A_1\dots A_n $\emph{ in} $ \srel $:
		\[\begin{tikzcd}
			MR && {MA_1\times \dots\times MA_n} \\
			\\
			NR && {NA_1\times \dots\times NA_n}
			\arrow["{h_{A_1}\times \dots\times h_{A_n}}"{description}, from=1-3, to=3-3]
			\arrow[tail, from=1-1, to=1-3]
			\arrow[tail, from=3-1, to=3-3]
			\arrow["{h_R}"{description}, from=1-1, to=3-1]
		\end{tikzcd}.\]	
}
\end{itemize}

}
\end{itemize}
\end{definition}
\subsection{Terms}
The term of some sort of a signature can also be interpreted as an arrow in a category with finite products:
\begin{definition}\label{D-41}
	(\textbf{Term in a $ \sign $-Structure}) \emph{Suppose $ \sign $ is a signature and $ M $ is a $ \sign $-structure over a category} $ \cat{C} $ \emph{with finite products. Let $ \vec{x}.t $ be a term in a context $ \vec{x} = \{x_1,\dots,x_n\} $ where $ x_i : A_i $, $ 1\le i\le n $ and $ t : B $. Then the same term $ \vec{x}. t $ is interpreted in the $ \sign $-structure $ M $ as an arrow denoted}
	\[\begin{tikzcd}
		{MA_1\times\dots\times MA_n} && MB
		\arrow["{[[\vec x.t]]_M}", from=1-1, to=1-3]
	\end{tikzcd}\]
	 \emph{generated by the following conditions:}
	\begin{itemize}
		\item {\emph{If the term $ t : B $ is simply a variable of sort $ B $, then $ t $ must be some $ x_i :A_i $ from the context $ \vec{x} $, and therefore the corresponding arrow simply becomes the following projection:}
	\begin{align*}
		 [[\vec{x}.t]]_M := \pi_i : MA_1\times \dots\times  MA_n \longrightarrow MA_i.
	\end{align*}	
	}
\item {\emph{If the term $ t : B $ is actually the term $ f(t_1,\dots,t_m) : B $ for some} $ f\in \sfun $ \emph{and $ \vec{x}.t_i : C_i $ are other terms, then the arrow $ [[\vec{x}.t]] $ would be the composite}\footnote{Note that the terms $ t_i,\;1\le i\le m $ are also in context $ \vec{x} $, so the arrows $ [[\vec{x}.t_i]] : MA_1\times \dots \times MA_n \longrightarrow MC_i$ are the term arrows in $ \cat{C} $ for them.}:
\[\begin{tikzcd}
	{MA_1\times \dots\times MA_n} &&& {MC_1\times \dots MC_m} && MB
	\arrow["{\uprs{[[\vec x.t_1]]_M,\dots, [[\vec x.t_m]]_M}}", from=1-1, to=1-4]
	\arrow["Mf", from=1-4, to=1-6]
\end{tikzcd}\]
}
	\end{itemize}
\end{definition}
\begin{example}
	A term $ \vec{x}.t : B$ in a signature $ \sign $ can be constructed by $ f(g(h(a(x_1),b(c(x_2))))) $ where $ \vec{x} = \{x_1,x_2\} $ and $ x_1,x_2 $ are variables of sorts $ A_1,A_2 \in \ssort $ respectively and $ f,g,h,a,b,c \in \sfun$ with \emph{target} of $ f $ being $ B $. Then the corresponding interpretation $ [[\vec{x}.t]] : MA_1\times MA_2 \longrightarrow MB$ would be given by the following composition:
	 \[\begin{tikzcd}
	 	{MA_1\times MA_2} && {M\cod{a}\times M\cod{b}} && MB
	 	\arrow["{\uprs{Ma,Mb\circ Mc}}", from=1-1, to=1-3]
	 	\arrow["{Mf\circ Mg\circ Mh}", from=1-3, to=1-5]
	 \end{tikzcd}.\]
\end{example}
We can also interpret the usual substitution of a term $ \vec{x}.t $ by list of terms $ \vec{s} $, denoted $ t[\vec{s}/\vec{x}] $, in a category by the help of implicit composition in that category.
\begin{proposition}\label{P-26}
	Suppose $ M $ is a $ \sign $-structure over a category $ \cat{C} $ with finite products. Let $ \vec{y}.t : C $ be a term of sort $ C $ in context $ \vec{y},\;y_i : B_i $. Suppose $ \vec{s} $ is a list of terms and this list has same length and type as $ \vec{y} $ and let $ \vec{x},\; x_i : A_i $ be a common context for each term $ s_i \in \vec{s} $. Then, the term $ \vec{x}.t[\vec{s}/\vec{y}] $ is interpreted as the following arrow given by composition in $ \cat{C} $: 
	\[\begin{tikzcd}
		{MA_1\times \dots\times MA_n} &&& {MB_1\times \dots\times MB_m} && MC
		\arrow["{\uprs{[[\vec x.s_1]]_M,\dots,[[\vec x.s_m]]_M }}", from=1-1, to=1-4]
		\arrow["{[[\vec y.t]]_M}", from=1-4, to=1-6].
	\end{tikzcd}\]
\end{proposition}
\begin{proof}
		Each term is given by a chain of application of function symbols over all or some of the variables present in context. Therefore, the substitution in term $ \vec{y}.t $ by terms $ \vec{s} $ is again a term in which there are more applications of function symbols on some or all the arguments of the context $ \vec{x} $. Now, using Definition \ref{D-41}, specifically the second point, we get the desired result by unrolling the whole chain of application of functions inductively.
\end{proof}
We now see that the homomorphism of $ \sign $-structures preserves the interpretation of terms upto \emph{naturality}:
\begin{proposition}\label{P-27}
	Suppose $ M $ \& $ N $ are two $ \sign $-structures and $ h : M\longrightarrow N $ is a $ \sign $-structure homomorphism. If $ \vec{x}.t $ is term in $ \sign $ where $ t : B $ and $ x_i : A_i $, then, the following commutes:
	\[\begin{tikzcd}
		{MA_1\times \dots\times MA_n} && MB \\
		\\
		{NA_1\times\dots\times NA_n} && NB
		\arrow["{[[\vec x.t]]_M}", from=1-1, to=1-3]
		\arrow["{[[\vec x.t]]_N}"', from=3-1, to=3-3]
		\arrow["{h_{A_1}\times\dots\times h_{A_n}}"{description}, from=1-1, to=3-1]
		\arrow["{h_B}", from=1-3, to=3-3]
	\end{tikzcd}\]
\end{proposition}
\begin{proof}
	Denote the term $ \vec{x}.t $ as a chain of application of function symbols as in the example above. Since the arrow $ [[\vec{x}.t]]_M $ would be the composition of all arrows involved, similarly for $ [[\vec{x}.t]]_N $, and since each individual arrow in the composition for $ M $ would have corresponding arrows from domain object and target object to that of $ N $ for which the natural square would commute by Definition \ref{D-40}, therefore the whole big rectangle will commute. This big rectangle is clearly the one required.
\end{proof}
\subsection{Formulae}
Definition \ref{D-41} tells us how to interpret terms of a signature. The next step would thus be to interpret a formula of some signature in a category. As we know, formulas of some signature themselves are categorized by several restrictions. These restrictions are atomic, horn, regular, coherent, first order, geometric \& infinitary first order formulas. Each of which would thus be interpreted in a category which would have enough structure suitable for their residence in it.
\begin{definition}\label{D-46}
	(\textbf{Formula in a $ \sign $-Structure}) Suppose $ \sign $ is a signature and $ M $ is a $ \sign $-structure in a category $ \cat{C} $ which has finite limits. Let $ \vec{x}.\phi $ be a formula in context $ \vec{x},\;x_i : A_i,\;1\le i \le n $. The formula $ \vec{x}.\phi $ would be interpreted in $ M $ as the subobject
	\[\begin{tikzcd}
		{[[\vec x.\phi]]} && {MA_1\times\dots\times MA_n}
		\arrow[tail, from=1-1, to=1-3]
	\end{tikzcd}\]
in $ \cat{C} $ and this subobject $ [[\vec{x}.\phi]] $ is generated recursively by the following :
\begin{enumerate}
	\item {\textbf{RELATIONS} : If $ \phi $ is simply $ R(t_1,\dots,t_m) $ where $ R\in \srel $ and $ t_j : B_j,\;1\le j\le m $, then the subobject $ [[\vec{x}.\phi]] $ is the following pullback:
	\[\begin{tikzcd}
		{[[\vec x.\phi]]} && MR \\
		\\
		{MA_1\times \dots\times MA_n} && {MB_1\times \dots\times MB_m}
		\arrow[tail, from=1-3, to=3-3]
		\arrow["{\uprs{[[\vec x.t_1]]_M,\dots,[[\vec x.t_m]]_M}}"', from=3-1, to=3-3]
		\arrow[tail, from=1-1, to=3-1]
		\arrow[from=1-1, to=1-3]
		\arrow["\lrcorner"{anchor=center, pos=0.125}, draw=none, from=1-1, to=3-3]
	\end{tikzcd}.\]
}
\item {\textbf{EQUALITY} : If $ \phi $ is simply  $ \vec{x}.s=\vec{x}.t $, an equality of terms of some sort $ B $, then the subobject $ [[\vec{x}.\phi]] $ is the following equalizer:
\[\begin{tikzcd}
	{[[\vec x.\phi]]} && {MA_1\times\dots\times MA_n} && MB
	\arrow["{[[\vec x.s]]_M}", shift left=2, from=1-3, to=1-5]
	\arrow["{[[\vec x.t]]_M}"', shift right=2, from=1-3, to=1-5]
	\arrow[tail, from=1-1, to=1-3]
\end{tikzcd}.\]
}
\item {\textbf{TRUTH} : If $ \phi $ is $ \top $, i.e. truth, then the subobject $ [[\vec{x}.\phi]] $ is simply the top element of the lattice $ \Sub{MA_1\times \dots \times MA_n}{\cat{C}} $:
\[\begin{tikzcd}
	{[[\vec x.\phi]] = \top} && {MA_1\times \dots\times MA_n}
	\arrow[tail, from=1-1, to=1-3]
\end{tikzcd}.\]
}
\item {\textbf{BINARY MEETS} : If $ \phi $ is $ \psi \meet \chi $ for other formulas $ \psi $ and $ \chi $, then the subobject $ [[\vec{x}.\phi]] $ is the following pullback:
\[\begin{tikzcd}
	{[[\vec x.\phi]]} & {[[\vec x.\chi]]} \\
	{[[\vec x.\psi]]} & {MA_1\times\dots\times MA_n}
	\arrow[tail, from=2-1, to=2-2]
	\arrow[tail, from=1-2, to=2-2]
	\arrow[tail, from=1-1, to=2-1]
	\arrow[tail, from=1-1, to=1-2]
	\arrow["\lrcorner"{anchor=center, pos=0.125}, draw=none, from=1-1, to=2-2]
\end{tikzcd}.\]
}
\item {\textbf{FALSITY} : If $ \phi $ is $ \bot $, then the subobject $ [[\vec{x}.\phi]] $ is the bottom element of the lattice $ \Sub{MA_1\times \dots\times MA_n}{\cat{C}} $:
\[\begin{tikzcd}
	{[[\vec x.\phi]] = \bot} && {MA_1\times \dots\times MA_n}
	\arrow[tail, from=1-1, to=1-3]
\end{tikzcd}.\]
}
\item {\textbf{BINARY JOINS} : If $ \phi $ is $ \psi \join \chi $ for other formulas $ \psi $ and $ \chi $ and $ \cat{C} $ is a coherent category\footnote{A coherent category is a regular category in which each subobject poset have finite joins and change of base functor for any arrow preserves these finite joins. Also remember the arrow factorization property in a regular category.}, then the subobject $ [[\vec{x}.\phi]] $ is the join of the two terms as subobjects:
	\[\begin{tikzcd}
		{[[\vec x.\psi]]\amalg[[\vec x.\chi]]} && {[[\vec x.\chi]]} \\
		& {[[\vec x.\phi]]} \\
		{[[\vec x.\psi]]} && {MA_1\times\dots\times MA_n}
		\arrow[tail, from=3-1, to=3-3]
		\arrow[tail, from=1-3, to=3-3]
		\arrow["{i_1}", from=3-1, to=1-1]
		\arrow["{i_2}"', from=1-3, to=1-1]
		\arrow["{!}"{description}, curve={height=18pt}, from=1-1, to=3-3]
		\arrow["e"{description}, two heads, from=1-1, to=2-2]
		\arrow["m"{description}, tail, from=2-2, to=3-3]
	\end{tikzcd}\]
}
\item {\textbf{IMPLICATION} : If $ \phi $ is $ \psi \Rightarrow \chi $ for other formulas $ \psi $ and $ \chi $ and $ \cat{C} $ is a Heyting category\footnote{A category $ \cat{C} $ is Heyting if it is coherent and for any arrow $ f : X\to Y $, the change of base functor
\begin{align*}
	f^{*} : \Sub{Y}{\cat{C}} \longrightarrow \Sub{X}{\cat{C}}
\end{align*}
has a right adjoint $ \forall_f : \Sub{X}{\cat{C}} \longrightarrow \Sub{Y}{\cat{C}}$. It follows that in a Heyting category, as expected, each subobject lattice is a Heyting algebra.
}, then the subobject $ [[\vec{x}.\phi]] $ is the implication $ [[\vec{x}.\psi]] \Rightarrow [[\vec{x}.\chi]] $ in the Heyting algebra $ \Sub{MA_1\times\dots\times MA_n}{\cat{C}} $:
\[\begin{tikzcd}
	{[[\vec x.\phi]] = \forall_{m_{[[\vec x.\psi]]}}([[\vec x.\psi]] \intrs [[\vec x.\chi]]) } && {MA_1\times\dots\times MA_n}
	\arrow[tail, from=1-1, to=1-3]
\end{tikzcd}\]
where $ \forall_{m_{[[\vec{x}.\psi]]}} : \Sub{[[\vec{x}.\psi]]}{\cat{C}} \longrightarrow \Sub{MA_1\times \dots\times MA_n}{\cat{C}} $.
}
\item {\textbf{NEGATION} : If $ \phi $ is $ \neg \psi $ for some formula $ \psi $ and $ \cat{C} $ is a Heyting category, then the subobject $ [[\vec{x}.\phi]] $ is the negation $ \neg[[\vec{x}.\psi]] $ in the Heyting algebra $ \Sub{MA_1\times \dots\times MA_n}{\cat{C}} $:
\[\begin{tikzcd}
	{[[\vec x.\phi]] = \left([[\vec x.\psi]] \Rightarrow \bot\right)} && {MA_1\times \dots\times MA_n}
	\arrow[from=1-1, to=1-3]
\end{tikzcd}.\]
}
\item {\textbf{EXISTENTIAL QUANTIFICATION} : If $ \phi $ is $ (\exists y)\psi $ where $ y $ is a variable of sort $ B $ and $ \cat{C} $ is a regular category, then the subobject $ [[\vec{x}.\phi]] $ is the following image\footnote{$\vec{x},y$ denotes that we have extended our initial context $ \vec{x} $ to now also include $ y $.}:
\[\begin{tikzcd}
	{[[(\vec x,y).\psi]]} \\
	& {[[\vec x.\phi]]} \\
	{MA_1\times \dots\times MA_n\times MB} && {MA_1\times \dots\times MA_n}
	\arrow[tail, from=1-1, to=3-1]
	\arrow["\pi"', from=3-1, to=3-3]
	\arrow["e", two heads, from=1-1, to=2-2]
	\arrow["m", tail, from=2-2, to=3-3]
\end{tikzcd}\]
where $ \pi $ is the unique arrow due to projection onto first $ n $ terms.}
\item {\textbf{UNIVERSAL QUANTIFICATION} : If $ \phi $ is $ (\forall y)\psi $ where $ y $ is a variable of sort $ B $ and $ \cat{C} $ is a Heyting category, then the subobject $ [[\vec{x}.\phi]] $ is the following (see footnote 27):
\[\begin{tikzcd}
	{[[\vec x.\phi]] = \forall_\pi\left([[(\vec x,y).\phi]]\right)} && {MA_1\times\dots\times MA_n}
	\arrow[tail, from=1-1, to=1-3]
\end{tikzcd}\]
}
\item {\textbf{INFINITARY JOINS} : If $ \phi $ is $ \bjoin_{i\in I}\psi_i $ where $ \psi_i $ are other formulas and $ \cat{C} $ is a geometric category\footnote{A geometric category is just an infinitary coherent category, meaning that each subobject poset have infinitary joins and any change of base functor preserves them.}, then the subobject $ [[\vec{x}.\phi]] $ is the join $ \bunion_{i\in I}[[\vec{x}.\psi_i]] $ in the join-semilattice $ \Sub{MA_1\times\dots\times MA_n}{\cat{C}} $:
\[\begin{tikzcd}
	{[[\vec x.\phi]] = \bunion_{i\in I} [[\vec x.\psi_i]]} && {MA_1\times\dots\times MA_n}
	\arrow[tail, from=1-1, to=1-3]
\end{tikzcd}.\]
} 
\item {\textbf{INFINITARY MEETS} : If $ \phi $ is $ \bmeet_{i\in I}\psi_i $ where $ \psi_i $ are other formulas and $ \cat{C} $ has arbitrary meets of subobjects, then the subobject $ [[\vec{x}.\phi]] $ is given by the meet $ \bintrs_{i\in I}[[\vec{x}.\psi_i]] $ in the meet-semilattice $ \Sub{MA_1\times\dots MA_n}{\cat{C}} $:
\[\begin{tikzcd}
	{[[\vec x.\phi]] = \bintrs_{i\in I} [[\vec x.\psi_i]]} && {MA_1\times\dots\times MA_n}
	\arrow[tail, from=1-1, to=1-3]
\end{tikzcd}.\]
}
\end{enumerate}

\end{definition}
\begin{remark}
	(\textbf{Infinitary first order formulas can just be interpreted in a geometric category}) Suppose $ \cat{C} $ is a geometric category. Then $ \cat{C} $ is also a Heyting category. Therefore each subobject lattice has infinitary meets, hence the heading. However, the change of base functor may not preserve the infinitary meets, i.e., change of base functor in this geometric category may not preserve non-geometric formulas.
\end{remark}
\begin{remark}
	Suppose $ M $ is a $ \sign $-structure on a category $ \cat{C} $ and $ \cat{C} $ has enough structure to interpret a particular formula $ \vec{x}.\phi $. We then call $ \phi $ to be interpretable in category $ \cat{C} $.
\end{remark}
The substitution property extends from terms to formulas:
\begin{proposition}
	Suppose $ \sign $ is a signature and $ \vec{y}.\phi $ is a formula in $ \sign $ in context $ \vec{y} $ which can be interpreted in a category $ \cat{C} $\footnote{That is, $ \cat{C} $ has enough structure to interpret $ \vec{y}.\phi $ in itself, as instructed in Definition \ref{D-46}.} and $ M $ is a $\sign $-structure on $ \cat{C} $. Let $ \vec{s} $ be a list of terms with same length and type as that of context $ \vec{y} $, where each term $ s_i \in \vec{s} $ is in common context $ \vec{x} $. Then, the formula $ [[\vec{x}.\phi[\vec{s}/\vec{y}]]] $ is given as the following pullback (Note $ x_i : A_i,\;1\le i\le n $, $ y_i : B_i,\;1\le i\le m $):
	\[\begin{tikzcd}
		{[[\vec x.\phi[\vec s/\vec y]]]_M} && {[[\vec y.\phi]]_M} \\
		\\
		{MA_1\times\dots\times MA_n} && {MB_1\times\dots\times MB_m}
		\arrow[tail, from=1-1, to=3-1]
		\arrow[tail, from=1-3, to=3-3]
		\arrow["\lrcorner"{anchor=center, pos=0.125}, draw=none, from=1-1, to=3-3]
		\arrow["{\uprs{[[\vec x.s_1]]_M,\dots,[[\vec x.s_m]]_M}}"', from=3-1, to=3-3]
		\arrow[from=1-1, to=1-3]
	\end{tikzcd}.\]
\end{proposition}
\begin{proof}
	Any formula, as Definition \ref{D-46} instructs, is fundamentally generated from relations and equality. The fact that the formula $ \vec{y}.\phi $ is substituted by $ \vec{s} $ to give a new formula $ \vec{x}.\phi[\vec{s}/\vec{y}] $ just makes the previous formula stated in new terms. We thus have two formulas, $ [[\vec{y}.\phi]] $ and $ [[\vec{x}.\phi[\vec{s}/\vec{y}]]] $. If $ \vec{y}.\phi $ is simply a relation or an equality, then by Proposition \ref{P-26} one can see that the later subobject $ [[\vec{x}.\phi[\vec{s}/\vec{y}]]] $ is just the relevant pullback. Since the result hold for atomic formulas, therefore it will hold for all generated from other constructions.
\end{proof}
Unfortunately, not all formulas in context are natural with respect to $ \sign $-structure homomorphisms:
\begin{proposition}
	Let $ \cat{C} $ be atleast a cartesian\footnote{One which has all finite limits.} category and let $ \vec{x}.\phi $ be a geometric formula in context over $ \sign $ which is interpretable in $ \cat{C} $. Suppose $ h : M\longrightarrow N $ is a homomorphism of $ \sign $-structures. Then there is a commutative square (Note $ x_i : A_i,\;1\le i\le n $):
	 \[\begin{tikzcd}
	 	{[[\vec x.\phi]]_M} && {MA_1\times\dots\times MA_n} \\
	 	\\
	 	{[[\vec x.\phi]]_N} && {NA_1\times\dots\times NA_n}
	 	\arrow[tail, from=1-1, to=1-3]
	 	\arrow[tail, from=3-1, to=3-3]
	 	\arrow[from=1-1, to=3-1]
	 	\arrow["{h_{A_1}\times\dots\times h_{A_n}}", from=1-3, to=3-3]
	 \end{tikzcd}.\]
\end{proposition}
\begin{proof}
	Again, we wish to see whether there is a commutative square for relations and equality. For equality, this can be seen easily; suppose $ \phi $ is $ \vec{x}.s = \vec{x}.t $ where $ s $ and $ t $ are terms of sort $ B $. Then, we have the following diagram (The two squares on the right commute due to Proposition \ref{P-27}):
	\[\begin{tikzcd}
		&& {[[\vec x.\phi]]_N} && {NA_1\times\dots\times NA_n} & NB \\
		&& P && {MA_1\times\dots\times MA_n} & MB \\
		{[[\vec x.\phi]]_M}
		\arrow[shift left=2, from=2-5, to=2-6]
		\arrow[shift right=2, from=2-5, to=2-6]
		\arrow[tail, from=3-1, to=2-5]
		\arrow[shift right=2, from=1-5, to=1-6]
		\arrow[shift left=2, from=1-5, to=1-6]
		\arrow[""{name=0, anchor=center, inner sep=0}, tail, from=1-3, to=1-5]
		\arrow["{h_{A_1}\times\dots\times h_{A_n}}", from=2-5, to=1-5]
		\arrow["{h_B}"', from=2-6, to=1-6]
		\arrow[tail, from=2-3, to=2-5]
		\arrow[from=2-3, to=1-3]
		\arrow[dashed, from=3-1, to=1-3]
		\arrow["\lrcorner"{anchor=center, pos=0.125, rotate=90}, draw=none, from=2-3, to=0]
	\end{tikzcd}\]
where the dotted arrow is obtained from universal property of equalizer $ [[\vec{x}.\phi]]_N $. Therefore we have a cone over the pullback $ P $ and so there is a unique arrow $ [[\vec{x}.\phi]]_M \to P $ through which the subobject $ [[\vec{x}.\phi]]_M $ factors through. Clearly, this arrow $ [[\vec{x}.\phi]]\to P $ has to be a monic. Hence we have a commutative square as required. Note that we only needed the universal property of the equalizer for the equality, hence the same will hold in case of relations. Since a geometric formula is generated by those basic formulas interpretation of whom preserves finite limits, therefore we have the required commutative square for all geometric formulas interpretable in $ \cat{C} $.
\end{proof}
The above motivates the following definitions:
\begin{definition}
	\emph{Suppose $ \sign $ is a signature and $ h : M\longrightarrow N $ is a homomorphism of $ \sign$-structures over} $ \cat{C} $\emph{. Then,}
	\begin{enumerate}
		\item {(\textbf{Elementary Morphism}) \emph{The morphism $ h : M \longrightarrow N $ is called an elementary morphism if for each \textbf{first-order formula} in context $ \vec{x}.\phi $ over $ \sign $ where $ x_i : A_i,\;1\le i\le n $, there is a \textbf{commutative square} as shown:}
			\[\begin{tikzcd}
				{[[\vec x.\phi]]_M} && {MA_1 \times\dots \times MA_n} \\
				\\
				{[[\vec x.\phi]]_N} && {NA_1\times\dots\times NA_n}
				\arrow[tail, from=1-1, to=1-3]
				\arrow[tail, from=3-1, to=3-3]
				\arrow["{h_{A_1}\times\dots\times h_{A_n}}", from=1-3, to=3-3]
				\arrow[from=1-1, to=3-1]
			\end{tikzcd}.\]
}
\item {(\textbf{Elementary Embedding}) \emph{The morphism $ h : M \longrightarrow N $ is called an elementary morphism if for each \textbf{first-order formula} in context $ \vec{x}.\phi $ over $ \sign $ where $ x_i : A_i,\;1\le i\le n $, there is a \textbf{pullback} as shown:}
\[\begin{tikzcd}
	{[[\vec x.\phi]]_M} && {MA_1 \times\dots \times MA_n} \\
	\\
	{[[\vec x.\phi]]_N} && {NA_1\times\dots\times NA_n}
	\arrow[tail, from=1-1, to=1-3]
	\arrow[tail, from=3-1, to=3-3]
	\arrow["{h_{A_1}\times\dots\times h_{A_n}}", from=1-3, to=3-3]
	\arrow[from=1-1, to=3-1]
	\arrow["\lrcorner"{anchor=center, pos=0.125}, draw=none, from=1-1, to=3-3]
\end{tikzcd}.\]
}
\item {(\textbf{Embedding}) \emph{The morphism $ h : M \longrightarrow N $ is called an elementary morphism if for each \textbf{atomic formula} in context $ \vec{x}.\phi $ over $ \sign $ where $ x_i : A_i,\;1\le i\le n $, there is a \textbf{pullback} as shown:}
\[\begin{tikzcd}
	{[[\vec x.\phi]]_M} && {MA_1 \times\dots \times MA_n} \\
	\\
	{[[\vec x.\phi]]_N} && {NA_1\times\dots\times NA_n}
	\arrow[tail, from=1-1, to=1-3]
	\arrow[tail, from=3-1, to=3-3]
	\arrow["{h_{A_1}\times\dots\times h_{A_n}}", from=1-3, to=3-3]
	\arrow[from=1-1, to=3-1]
	\arrow["\lrcorner"{anchor=center, pos=0.125}, draw=none, from=1-1, to=3-3]
\end{tikzcd}.\]
}
	\end{enumerate}
\end{definition}
\newpage
\subsection{Theories and Models}
We now come to models of a theory over a signature, defining both:
\begin{definition}
\emph{	Suppose $ \sign $ is a signature and $ M $ be a $ \sign $-structure over a category }$ \cat{C} $\emph{. Then,}
	\begin{enumerate}
		\item {(\textbf{Satisfiability of a Sequent}) \emph{Let $\sigma = \phi \vdash_{\vec{x}} \psi$ be a sequent over $ \sign $ interpretable}\footnote{A sequent is \emph{interpretable} if each formula in the sequent is interpretable in $ \cat{C} $.}\emph{ in} $ \cat{C} $\emph{. Then the sequent $ \sigma $ is defined to be satisfiable in the structure $ M $ if}
		\begin{align*}
			[[\vec{x}.\phi]]_M \le [[\vec{x}.\psi]]_M 
		\end{align*}
\emph{	in} $ \Sub{MA_1\times \dots\times MA_n}{\cat{C}} $\emph{ where $ \le $ is the order induced by the subobject lattice. The satisfiability of the sequent $ \sigma $ in structure $ M $ is denoted as:}
	\begin{align*}
		M \vDash\sigma.
	\end{align*}
	}
\item {(\textbf{Model of a Theory}) \emph{Let $ \theory $ be a theory over $ \sign $ interpretable in} $ \cat{C} $\footnote{A theory is \emph{interpretable} in $ \cat{C} $ if all the sequents of the theory are interpretable.}.\emph{ Then the structure $ M $ over} $ \cat{C} $\emph{ is said to be a model of the theory $ \theory $ if all the axioms/sequents of the theory $ \theory $ is satisfiable in $ M $. We denote a model $ M $ of a theory $ \theory $ by:}
\begin{align*}
	M\models \theory.
\end{align*}
}
\item {(\textbf{Category of Models of a Theory}) \emph{Let $ \theory $ be a theory over a signature $ \sign $. The full-subcategory of} $ \sstruc{C} $ \emph{where structures(objects) are all the models of the theory $ \theory $ and structure morphisms between them is called the category of models of theory $ \theory $ and is denoted as:}
\begin{align*}
	\tmod{C}.
\end{align*}
}
	\end{enumerate}

\end{definition}

\begin{remark}
	The subcategory of all the models of a theory $ \theory $ over a signature $ \sign $ on category $ \cat{C} $ between whom the structure morphisms are elementary morphisms is denoted as:
	\begin{align*}
		\tmod{C}_e.
	\end{align*}
\end{remark}
Before going further, we need to know particularities about functors between two categories with appropriate structures:
\begin{enumerate}
	\item {If categories $ \cat{C} $ and $ \cat{D} $ are cartesian, then a \textbf{cartesian functor} $ F : \cat{C} \longrightarrow \cat{D}$ is defined to be a functor which preserves finite limits.}
	\item {If categories $ \cat{C} $ and $ \cat{D} $ are regular, then a \textbf{regular functor} $ F : \cat{C} \longrightarrow \cat{D}$ is defined to be a functor which preserves finite limits and regular epimorphisms.}
	\item {If categories $ \cat{C} $ and $ \cat{D} $ are coherent, then a \textbf{coherent functor} $ F : \cat{C} \longrightarrow \cat{D}$ is defined to be a functor which preserves finite limits, regular epimorphisms and finite joins of subobjects.}
	\item {If categories $ \cat{C} $ and $ \cat{D} $ are Heyting, then a \textbf{Heyting functor} $ F : \cat{C} \longrightarrow \cat{D}$ is defined to be a functor which preserves finite limits, regular epimorphisms and right adjoints $ \forall_f : \Sub{X}{\cat{C}} \to \Sub{Y}{\cat{C}}$ for any arrow $ f: X\to Y$ in $ \cat{C} $.}
	\item {If categories $ \cat{C} $ and $ \cat{D} $ are geometric, then a \textbf{geometric functor }$ F : \cat{C} \longrightarrow \cat{D}$ is defined to be a functor which preserves finite limits, regular epimorphisms and infinitary joins of subobjects.}
\end{enumerate} 
The above defined functors preserves satisfiability of a sequent:
\begin{proposition}\label{P-30}
	Let $ T : \cat{C}\longrightarrow \cat{D} $ be a cartesian (regular, coherent, Heyting, geometric) functor. Let $ M $ be a $ \sign $-structure over $ \cat{C}$ and let $ \sigma $ be a sequent over $ \sign $ interpretable in $ \cat{C} $. If $ M \vDash \sigma $, then, $ \sign-\text{\textbf{Str}}(T) : \sstruc{C} \longrightarrow \sstruc{D}$, which takes each interpretation through $ T $ to get an interpretation over $ \cat{D} $, gives a structure over $ \cat{D} $ for which $ \sigma $ is still satisfiable, that is,
	\begin{align*}
		\sign-\text{\textbf{Struc}}(T) (M) \vDash \sigma.
	\end{align*}
\end{proposition}
\begin{proof}
	Take $ \cat{C} $ to be cartesian (respectively, all else). Suppose $ \sigma = \phi \vdash_{\vec{x}} \psi $ is a sequent satisfiable in $ M $ and interpretable in $ \cat{C} $, where $ x_i : A_i,\;1\le i\le n $. Hence, we have $ [[\vec{x}.\phi]]_M \le [[\vec{x}.\psi]]_M $ in the $ \Sub{MA_1\times \dots\times MA_n}{\cat{C}} $. Since for two subobjects $ S,T $ in any $ \Sub{A}{\cat{C}} $, we say $ S_1\le S_2 $ if and only if $ S_1\intrs S_2 = S_1 $ where $ \intrs $ is the meet of two subobjects given by pullback of them (which exists since $ \cat{C} $ has finite limits (atleast)). Since $ T $ preserves limits as it is cartesian (atleast), therefore if $ S_1\intrs S_2 = S_1 $, then $T(S_1) = T(S_1\intrs S_2) \isomorph T(S_1) \intrs T(S_2) $. Hence, if $ [[\vec{x}.\phi]]_M\le [[\vec{x}.\psi]]_M $, then $ T([[\vec{x}.\phi]]_M) \le T([[\vec{x}.\psi]]_M) $, which just means that if $ M\vDash \sigma $, then $ \sign-\text{\textbf{Struc}}(T)(M) \vDash \sigma $. 
\end{proof}
The converse of the above proposition additionally requires $ T $ to be conservative:
\begin{proposition}\label{P-31}
	Let $ T : \cat{C}\longrightarrow \cat{D} $ be a cartesian (regular, coherent, Heyting, geometric) functor which is additionally conservative\footnote{A functor $ F : \cat{C}\longrightarrow \cat{D} $ is conservative if for any $ f : C\to C^{\prime} $ in $ \cat{C} $, we have that $ Ff : FC \to FC^{\prime} $, then $ f :C\to C^{\prime} $ was an isomorphism.}. Let $ M $ be a $ \sign $-structure over $ \cat{C}$ and let $ \sigma $ be a sequent over $ \sign $ interpretable in $ \cat{C} $. If $ \sign-\text{\textbf{Struc}}(T) (M) \vDash \sigma$, then, $ M\vDash \sigma$.
\end{proposition}
\begin{proof}
	If $ \sign-\text{\textbf{Struc}}(T) (M) \vDash \sigma $, then $ T([[\vec{x}.\phi]]_M) \le T([[\vec{x}.\psi]]_M) $ which means that $ T([[\vec{x}.\phi]]_M \intrs [[\vec{x}.\psi]]_M) \isomorph T([[\vec{x}.\phi]]_M) $. But $[[\vec{x}.\phi\meet \psi]]_M := [[\vec{x}.\phi]]_M \intrs [[\vec{x}.\psi]]_M $ (Definition \ref{D-46}). Therefore $ T([[\vec{x}.\phi\meet \psi]]_M) \isomorph T([[\vec{x}.\phi]]_M) $. By conservativity of $ T $, we have that $ [[\vec{x}.\phi]]_M \isomorph [[\vec{x}.\phi\meet \psi]]_M $, which is what we wanted.
\end{proof}
\begin{example}
	(\textbf{The Category of Models of the Theory of Abelian Groups over $ \cat{Sets} $ is $ \cat{AbGrp} $})\\
	We wish to show the following where $ \theory $ is the theory of abelian groups over it's canonical signature with one sort $ G $, three function symbols: $ f : G,G\to G $, $ 1 : [] \to G $, $ i : G\to G $; and no relation symbols. We wish to show:
	\begin{align*}
		\tmod{Sets} \isomorph \cat{AbGrp}
	\end{align*}
	First, the theory of abelian groups consists of the following four atomic sequents/axioms:
	\begin{enumerate}
		\item {$ \top \vdash_{\{x,y,z\}} m(m(x,y),z) = m(x,m(y,z)) $}
		\item {$ \top \vdash_{\{x,y\}} m(x,y) = m(y,z) $}
		\item {$ \top \vdash_{\{x\}} m(x,1()) = x $}
		\item {$ \top \vdash_{\{x\}} m(x,i(x)) = 1() $}
	\end{enumerate}
	To show first that each model defines a unique group, take any model $ M $ of $ \theory $ over $ \cat{Sets} $. Since $ M $ is a model, therefore for each axiom $ \top \vdash_{\vec{x}} \phi $ of $ \theory $, the following is true:
	\begin{align*}
		[[\vec{x}.\top]]_M \le [[\vec{x}.\phi]]_M.
	\end{align*}
 	But this means that 
 	\begin{align*}
 		[[\vec{x}.(\top \meet \phi)]]_M \isomorph [[\vec{x}.\top]]_M.
 	\end{align*}
 	Moreover, we know that 
 	\begin{align*}
 		[[\vec{x}.\top]]_M = MA_1\times \dots MA_n
 	\end{align*}
 where $ x_i : A_i $. This means that 
 \begin{align*}
 	[[\vec{x}.\top]]_M \intrs [[\vec{x}.\phi]]_M = [[\vec{x}.(\top \meet \phi)]]_M = [[\vec{x}.\top]]_M = MA_1\times \dots\times MA_n
 \end{align*}
and so 
\begin{align*}
	[[\vec{x}.\phi]]_M = MA_1\times \dots \times MA_n.
\end{align*}
Therefore, for associativity of $ Mm  : MG\times MG \longrightarrow MG$:
\begin{align*}
	[[\{x,y,z\}.m(m(x,y),z) = m(x,m(y,z))]]_M = MG\times MG\times MG
\end{align*}
which means $ Mm $ is associative for all $ x,y,z $ in $ MG $. Hence $ Mm $ is associative. Similarly, $ Mm $ is commutative, each $ x $ in $ G $ has inverse and $ 1()\in MG $ is the identity. Hence $ (MG,Mm)$ is an abelian group determined by the model $ M $. For each $ \sign $-structure homomorphism $ h : M\to N $, the only component gives rise to a group homomorphism $ h_G : MG \to NG $ in $ \cat{Sets} $. This follows because of the natural square of the three function symbols. Similarly, each group homomorphism $ h : G\to H $ determines a unique $ \sign $-structure homomorphism because the conditions of group homomorphism are the ones which make the three natural squares commute.
%   \begin{align*}
%   	x < y \;&\equiv \; \exists a (x-y = a \cdot a) \meet \neg(a=0)\\
%   	x \text{ is prime} \;&\equiv \; (x > 1) \bmeet \left (\forall y \left ((y < x) \meet  \exists z(x = y\cdot z) \implies (y=1)\right )\right )
%   \end{align*}
\end{example}
Proposition \ref{P-31} has an interesting corollary:
\begin{proposition}
	Let $ \theory $ be a geometric theory over a signature $ \sign $. Then,
		\begin{enumerate}
			\item {For any small category $ \cat{C} $, a $ \sign $-structure $ M $ in $ \cat{Sets}^{\cat{C}} $ is a $ \theory $-model if and only if the $ \sign$-structure $ \ev{M}{C} $ on $ \cat{Sets} $ for each $ C \in \obj{\cat{C}} $ is a $ \theory $-model. The functor $ \ev{-}{C} : \cat{Sets}^{\cat{C}} \longrightarrow \cat{Sets}$ takes a set-functor to the set obtained by evaluating it at $ C $. 
		}
	\item {For any topological space $ X $ and for any element $ x\in X $, the inverse image $ x^{*} : \Sh{X} \longrightarrow \cat{Sets}$ of the geometric morphism:
\[\begin{tikzcd}
	{x :\cat{Sets}} & {} & {\Sh{X}}
	\arrow[""{name=0, anchor=center, inner sep=0}, "{x^{*}}"{description}, curve={height=-24pt}, from=1-3, to=1-1]
	\arrow[""{name=1, anchor=center, inner sep=0}, "{x_*}"{description}, curve={height=-24pt}, from=1-1, to=1-3]
	\arrow[from=1-2, to=1-3]
	\arrow[shorten >=7pt, no head, from=1-1, to=1-2]
	\arrow["\dashv"{anchor=center, rotate=90}, draw=none, from=0, to=1]
\end{tikzcd}\]	
is such that for a $ \sign $-structure $ M $ in the category $ \Sh{X} $ is a $ \theory $-model if and only if the $ \sign $-structure $ x^{*}(M) $ in $ \cat{Sets}$ is a $ \theory $-model. $ x^{*}(M) $ takes the interpreation of $ M $ in $ \Sh{X} $ to an interpretation in $ \cat{Sets} $ via composition with $ x^{*} $. 
}
		\end{enumerate}
\end{proposition}
\begin{proof}
$ \bm{1}. $	(L $ \implies $ R) Let $ M $ be a $ \theory $-model over the category $ \cat{Sets}^{\cat{C}} $. Now, for a fixed object $ C \in \obj{\cat{C}} $, we have by composition a $ \sign $-structure over the category $ \cat{Sets} $, $ \ev{M}{C} $, which takes each sort, function symbols \& relation symbols first to it's interpretation in $ \cat{Sets}^{\cat{C}} $ via $ M $ and then to $ \cat{Sets} $ via evaluating that interpretation at $ C \in \obj{C}$. Note that $ \ev{-}{C} : \cat{Sets}^{\cat{C}} \longrightarrow \cat{Sets}$ is a geometric functor because limits of set functors is determined point-wise, an epimorphic natural transform is by definition one whose each component is an epimorphism and join of two subobjects in $ \cat{Sets}^{\cat{C}} $ is join of their corresponding components (colimits are computed point-wise). By Proposition \ref{P-30}, since each sequent of $ \theory $ is satisfied by $ M $ as it is a $ \theory $-model, therefore $ \ev{M}{C} $ is also a $ \sign $-structure over $ \cat{Sets} $ where each sequent of $ \theory $ is satisfiable. Hence $ \ev{M}{C} $ is also a $ \theory $-model, for each $ C\in \obj{\cat{C}} $.\\
	(R $ \implies $ L) It is quite simple to see that $ \ev{-}{C} : \cat{Sets}^{\cat{C}}\longrightarrow \cat{Sets}$ is conservative, because if for some natural transform $ \eta : F \Rightarrow G $ in $ \cat{Sets}^{\cat{C}} $, the function $ \ev{\eta}{C} := \eta_C :FC \longrightarrow GC $ is an isomorphism for each $ C\in\obj{\cat{C}} $, then $ \eta $ is also a natural isomorphism as each natural transform is determined by it's components. Therefore by Proposition \ref{P-31}, we have that $ M $ is also a $ \theory $-model.\\\\
	$ \bm{2}. $ For $ x : \bm{1}\longrightarrow X $, we have that the induced geometric morphisms $ x : \Sh{\bm{1}} \isomorph \cat{Sets}  \longrightarrow \Sh{X}$ is such that the inverse image (left-adjoint) $ x^{*} : \Sh{X}\longrightarrow \cat{Sets}$ is a stalk functor, that is, it takes each sheaf $ F $ over $ X $ to stalk at $ x $, $ F_x $. Now because $ x^{*} $ is left adjoint so it preserves small colimits (and so coequalizers, hence regular epics), $ x^{*} $ is inverse image of a geomteric morphism so it preserves finite limits. Therefore $ x^{*} $ is a geometric functor. Now $ x^{*} : \Sh{X}\longrightarrow\cat{Sets}$ is also conservative because if for some $ \eta : F\Rightarrow G $ it s true that $ x^{*}(\eta) : F_x \rightarrow G_x $ is an isomorphism for each $ x\in X $, then $ \eta $ is a natural isomorphism. We can then follow the same argument as in $ \bm{1} $ to conclude the result.
\end{proof}
\newpage
\section{Topoi and Logic}
We now study some of the interconnections between topoi and logic, studying Cohen's proof of independence of continuum hypothesis from ZFC axioms and a brief introduction to synthetic differential geometry, in between.
\subsection{Natural Numbers Object in a Topos : $ \nno{E} $ }
The axioms of set theory demand existence of an infinite set, the set of natural numbers $ \mathbb{N} $. In a topos, this axiom is interpreted as the existence of the following object:
\begin{definition}\label{D-49}
	(\textbf{Natural Numbers Object}) \emph{Suppose }$ \cat{E} $\emph{ is a topos. An object} $ \nno{E} $\emph{ in }$ \cat{E} $\emph{ is defined to be a natural numbers object if it has two arrows}
\[\begin{tikzcd}
	{\bm 1} && {\nno{E}} && {\nno{E}}
	\arrow["0", tail, from=1-1, to=1-3]
	\arrow["s", from=1-3, to=1-5]
\end{tikzcd}\]
\emph{such that for any other object $ X $ with arrows} $ \bm{1} \overset{x}{\longrightarrow} X \overset{f}{\longrightarrow} X $,\emph{ there exists a unique arrow} $ h : \nno{E} \longrightarrow X$ \emph{such that the following commutes:}
\[\begin{tikzcd}
	{\bm 1} && {\nno{E}} && {\nno{E}} \\
	\\
	{\bm 1} && X && X
	\arrow["0", tail, from=1-1, to=1-3]
	\arrow["s", from=1-3, to=1-5]
	\arrow["x"', tail, from=3-1, to=3-3]
	\arrow["f"', from=3-3, to=3-5]
	\arrow[Rightarrow, no head, from=1-1, to=3-1]
	\arrow["h"', dashed, from=1-3, to=3-3]
	\arrow["h", dashed, from=1-5, to=3-5]
\end{tikzcd}.\]
\end{definition}
\begin{remark}
	(\textbf{Natural Numbers Objects are unique upto isomorphism}) Take any other object $ N $ in $ \cat{E} $ for which satisfies the condition of an NNO as in Definition \ref{D-49} with the following defining arrows: $ \bm{1} \overset{n}{\longrightarrow} N \overset{f}{\longrightarrow} N$. Hence there exists the unique arrows $ a : \nno{E}\longrightarrow N $ and $ b : N \longrightarrow \nno{E} $ which are universal, for which $ a\circ b : N \longrightarrow N$ is such that $ f\circ a\circ b = a\circ b\circ f $ and $ a\circ b\circ x = x $. Now we have another $ \bm{1} \overset{x}{\longrightarrow} N \overset{a\circ b}{\longrightarrow} N$, therefore $ \exists $ unique $ h : N\to N $ with $ h\circ x =x  $ and $ a\circ b\circ h=h\circ f $. But $ a\circ b\circ x = x $ too and $ a\circ b $ is also unique, therefore $ a\circ b = h $. Moreover, because $ h\circ f = a\circ b\circ h $, therefore by uniqueness, $ f = h\implies f= a \circ b $. Hence, $ h = 1_N =a\circ b $. Similarly, $ b\circ a = 1_{\nno{E}} $.
\end{remark}
We first see that each geometric morphism between two topoi in which one has an NNO, implies that the other one has it too:
\begin{lemma}\label{L-12}
	Let $ \cat{E}, \cat{F} $ be a topoi where $ \cat{E} $ has a natural numbers object $ \nno{E} $. Let there be a following geometric morphism between $ \cat{F} $ and $ \cat{E} $:
	\[\begin{tikzcd}
		& {} \\
		{g : \cat{F}} & {} & {\cat{E}} \\
		& {}
		\arrow[no head, from=2-1, to=2-2]
		\arrow[from=2-2, to=2-3]
		\arrow["{g^*}"{description}, curve={height=-24pt}, from=2-3, to=2-1]
		\arrow["\dashv"{anchor=center, rotate=90}, draw=none, from=3-2, to=1-2]
		\arrow["{g_*}"{description}, curve={height=-24pt}, from=2-1, to=2-3]
	\end{tikzcd}.\]
	Then, $ g^{*}(\nno{E}) $ is an NNO for $ \cat{F} $.
\end{lemma}
\begin{proof}
	It can be seen, that for each object $ X $ of $ \cat{F} $ for which there are arrows $ \bm{1}_{\cat{F}} \overset{x}{\longrightarrow} X \overset{f}{\longrightarrow} X $, there exists unique arrow $ g^{*}(\nno{E})\longrightarrow X $ which is the transpose of the unique arrow $ \nno{E}\longrightarrow g_* (X) $, and hence it makes the corresponding square in $ \cat{F} $ commute. This also depends on the fact $ g^{*} $ preserves finite limits as it is inverse image of a geometric morphism and so preserves terminals.
\end{proof}
This lemma has very important corollaries, first of which shows that \textbf{each presheaf category has an NNO}:
\begin{corollary}\label{C-4}
	For a small category $ \cat{C} $, the presheaf category $ \psheaf{C} $ has a natural numbers object.
\end{corollary}
\begin{proof}
	We have the global sections adjunction (a geometric morphism):
	\[\begin{tikzcd}
		& {} \\
		{\psheaf{C}} & {} & {\cat{Sets}} \\
		& {}
		\arrow[no head, from=2-1, to=2-2]
		\arrow[from=2-2, to=2-3]
		\arrow["\Delta"{description}, curve={height=-24pt}, from=2-3, to=2-1]
		\arrow["\dashv"{anchor=center, rotate=90}, draw=none, from=3-2, to=1-2]
		\arrow["\Gamma"{description}, curve={height=-24pt}, from=2-1, to=2-3]
	\end{tikzcd}.\]
Now use Lemma \ref{L-12}. Therefore the NNO for $ \psheaf{C} $ is the constant to $ \mathbb{N}$ presheaf, $ \Delta (\mathbb{N}) : \opcat{C}\longrightarrow \cat{Sets}$.
\end{proof}
We now see that \textbf{each Grothendieck topos has an NNO}:
\begin{corollary}
	Let $ (\cat{C},J) $ be a site. The sheaf topos $ \Sh{\cat{C},J} $ has a natural numbers object, given by:
	\begin{align*}
		\mathbb{N}_{\Sh{\cat{C},J}} = \coprod_{n\in \mathbb{N}} \bm{1}_{\Sh{\cat{C},J}} 
	\end{align*}
\end{corollary}
\begin{proof}
	The site $ (\cat{C},J) $ determines a Lawvere-Tierney topology on $ \psheaf{C} $, $ j $ [REF]. We also know that there is a geometric morphism (sheafification):
\[\begin{tikzcd}
	& {} \\
	{\sh{j}{\cat{E}}} & {} & {\cat{E}} \\
	& {}
	\arrow[no head, from=2-1, to=2-2]
	\arrow[from=2-2, to=2-3]
	\arrow["a"{description}, curve={height=-24pt}, from=2-3, to=2-1]
	\arrow["\dashv"{anchor=center, rotate=90}, draw=none, from=3-2, to=1-2]
	\arrow["i"{description}, curve={height=-24pt}, hook', from=2-1, to=2-3]
\end{tikzcd}.\]
Now for any $ \cat{E} $ being the presheaf category $ \psheaf{C} $, we have by Lemma \ref{L-12} and Corollary \ref{C-4} that $ \Sh{\cat{C},J} $ has an NNO, given by:
\begin{align*}
	a(\Delta(\mathbb{N})) : \opcat{C} \longrightarrow \cat{Sets}.
\end{align*}
Now since $ \mathbb{N} \isomorph \coprod_{n\in \mathbb{N}}\bm{1}$ and $ a $ \& $ \Delta $ are left adjoints of geometric morphisms so they preserve the terminals and the small coproducts to give the desiderata. 
\end{proof}
\subsection{The $ \neg\neg $ Lawvere-Tierney Topology in a Topos}
There is an LT topology in a topos which gives us as it's sheaf topos, a Boolean topos. To get to that result, we would need to understand how the operations in the two Heyting lattices $ \Sub{F}{\sh{j}{\cat{E}}} $ and $ \Sub{F}{\cat{E}}$ interacts.
\subsubsection{$ \Sub{F}{\sh{j}{\cat{E}}} $ and $ \Sub{F}{\cat{E}}$}
The structures of the Heyting algebra structures $ \Sub{F}{\sh{j}{\cat{E}}} $ and $ \Sub{F}{\cat{E}} $ are comparable (where $ F $ is $ j $-sheaf):
\begin{proposition}\label{P-33}
	Let $ \cat{E} $ be a topos and let $ j : \Omega\longrightarrow \Omega $ be a Lawvere-Tierney topology on it. Then, for any closed subobjects $ S,T $ of a $ j $-sheaf $ F $ in $ \cat{E} $, the following identities hold in $ \Sub{F}{\sh{j}{\cat{E}}} $:
	\begin{enumerate}
		\item {$ 1_j = 1 $}
		\item {$ S\meet_j T = S\meet T $}
		\item {$ 0_j = \bar{0}$}
		\item {$ S\join_j T = \overline{S\join T} $}
		\item {$ S\Rightarrow_j T = S\Rightarrow T$}
		\item {$ \neg_j S = \overline{\neg S}$}
	\end{enumerate}
where $ (-)_{j} $ denotes corresponding operation in the Heyting algebra $ \Sub{F}{\sh{j}{\cat{E}}} $.
\end{proposition}
\begin{proof}
	1. Because $ 1 : F\to F $ is closed, hence $ 1 $ is also the top element of $ \Sub{F}{\sh{j}{\cat{E}}} $.\\
	2. The meet of two closed subobjects $ S,T $ in $ \Sub{F}{\sh{j}{\cat{E}}} $ would be closed by LTT.3. Hence $ S\meet T = \bar{S} \meet \bar{T} = \overline{S\meet T} = S\meet_j T$.\\
	3. $ 0 $ is the bottom element of $ \Sub{F}{{\cat{E}}} $, the closure of $ 0 $ would hence be the smallest closed subobject of $ F $.\\
	4. The join of two closed subobjects, $ S\join_j T $, would be the smallest closed subobject containing $ S $ and $ T $, which is $ \overline{S\join T} $.\\
	5. If we could show that $ S\Rightarrow T $ is closed, then we could argue that: Since $ S\Rightarrow_j T  $ is the unique subobject in $ \Sub{F}{\sh{j}{\cat{E}}} $ with the property for any closed subobject $ R $ that $ R\le S\Rightarrow_j T $ if and only if $ R \meet_j S \le T $, therefore we need to show that the same is true for $ S\Rightarrow T $:
	\begin{align*}
		R\meet_j S\le T &\iff R\meet S \le T\\
		&\iff R\le S\Rightarrow T.
	\end{align*}
Therefore we would be done if we could show that $ S\Rightarrow T $ is also closed. To this extent, for any subobject $ W $ of $ F $ in $ \Sub{F}{\cat{E}} $, we have:
\begin{align*}
	W\le S\Rightarrow T &\iff W\meet S \Rightarrow T
	\iff W\meet \bar{S} \le \bar{T}
	\iff \overline{W\meet \bar{S}} \le \bar{\bar{T}}
	\iff \bar{W} \meet \bar{\bar{S}}\le \bar{T}
	\iff \bar{W} \meet \bar{S} \le \bar{T}\\
	&\iff \bar{W}\meet S\le T
		\iff \bar{W} \le S\Rightarrow T
	\iff \bar{\bar{W}} \le \overline{S\Rightarrow T}
	\iff \bar{W} \le \overline{S\Rightarrow T}
	\iff W \le \overline{S\Rightarrow T}
\end{align*}
where last line follows from the fact that $ W \le \bar{W} $.\\
6. Negation in a Heyting algebra is simply $ \neg A := (A \Rightarrow 0) $. Hence from the above results (particularly 3 \& 5), $ \neg_j S := (S\Rightarrow_j 0_j ) = (S\Rightarrow 0) =: \neg S$.
\end{proof}
\subsubsection{To a Boolean Topos from any Topos}
A \textbf{Boolean topos} is a topos whose internal/external subobject lattice is an internal/external Boolean lattice. That is, $ PX $ is an internal Boolean lattice or, equivalently, $ \Sub{X}{\cat{E}} $ is an external Boolean lattice for each object $ X $ of $ \cat{E} $. We will now see that, for the negation $ \neg : \Omega \longrightarrow \Omega $ of the internal Heyting lattice $ \Omega $, the arrow $ \neg\neg := \neg\circ \neg  : \Omega \longrightarrow \Omega$ is a Lawvere-Tierney topology in $ \cat{E} $ and it's sheaf topos is in-fact a Boolean topos:
\begin{theorem}
	Suppose $ \cat{E} $ is a topos. Consider the internal Heyting algebra $ \Omega $, whose negation operator is $ \neg  : \Omega\longrightarrow \Omega$. Then, the arrow
	\begin{align*}
		\neg\neg : \Omega \longrightarrow \Omega
	\end{align*}
is a Lawvere-Tierney topology in $ \cat{E} $. Moreover, the following then holds:
\begin{align*}
	\sh{\neg\neg}{\cat{E}} \text{ is a \textbf{Boolean Topos}.} 
\end{align*}
\end{theorem}
\begin{proof}
	It's a basic result that for $ x,y\in H $ for any Heyting algebra $ H $ that the following holds:
	\begin{align*}
		x\le \neg\neg x\;,\;\;\neg\neg x = \neg\neg\neg\neg x\;,\;\; \neg\neg \left (x\meet y\right ) = \neg\neg x \meet \neg \neg y.
	\end{align*}
	So the corresponding requirements for the closure operator in $ \Sub{X}{\cat{E}} $ is already satisfied. But we wish to show that this is also natural. Since for any $ f : X\to Y $ in $ \cat{E} $, we have by Proposition \ref{P-22} that $ f^{*} : \Sub{Y}{\cat{E}} \longrightarrow\Sub{X}{\cat{E}}$ is a Heyting algebra homomorphism, therefore the negation is preserved by $ f^{*} $. Hence the $ \neg\neg : \Sub{X}{\cat{E}}\longrightarrow \Sub{X}{\cat{E}}$ is a natural operator. Hence, $ \neg\neg $ is an LT topology.  \\\\
	The latter result can be seen to hold because for any subsheaf $ S $ of sheaf $ X$, $ S $ has to be $ \neg\neg $-closed in $ \cat{E} $, i.e. $ \neg\neg S = S $ in $ \cat{E} $. But we wish to show the result in $ \sh{\neg\neg}{\cat{E}} $. Now note that we wish to show that $ \neg_{\neg\neg}\neg_{\neg\neg} S= S $ (i.e. in $ \sh{j}{\cat{E}} $). But by Proposition \ref{P-33}, $ \neg_{\neg\neg} S = \neg \left (\bar{S}^{\neg\neg}\right ) = \neg\neg\neg S$, so $ \neg_{\neg\neg} \neg_{\neg\neg} S = (\neg)^{6}S $. Hence, the statement $ \neg\neg S= S $ in $ \cat{E} $ is equivalent to $ (\neg)^{6}S = S $ in $ \sh{\neg\neg}{\cat{E}} $. The result then follows.
\end{proof}
The next result would help us in establishing the independence of axiom of choice with continuum hypothesis:
\begin{proposition}
	Suppose $ \cat{E} $ is a topos which is generated by the subobjects of $ \bm{1} $ and each subobject lattice $ \Sub{A}{\cat{E}} $ is a complete Boolean algebra. Then, $ \cat{E} $ satisfies axiom of choice.
\end{proposition}
\begin{proof}
	Take an epimorphism $ p : X \longrightarrow I $. Since $ \Sub{I}{\cat{E}} $ is a complete Boolean algebra, therefore the collection of all subobjects $ n : N \rightarrowtail I $ of $ I $ which has a section $ s : N \longrightarrow X $ with $ p\circ s = n $, has a maximal subobject $ m : M \rightarrowtail I $. We wish to show that there is a section $ I \longrightarrow X $ of $ p $. To do this, we have to argue that the maximal element $ M $ of all sections of $ p $ must be the $ I $ itself. Hence let us assume that this maximal subobject of sections $ M $ is not $ I $. Now, since $ \Sub{I}{\cat{E}} $ is a Boolean algebra so it has $ \neg M $. Now since $ \Sub{\bm{1}}{\cat{E}} $ generates $ \cat{E} $, therefore there is a monic from a subobject $ V \rightarrowtail \bm{1} $ with $ V \rightarrow \neg M $. Now construct the pullback 
	\[\begin{tikzcd}
		{X^{\prime}} && X \\
		V & {\neg M} & I
		\arrow["p", two heads, from=1-3, to=2-3]
		\arrow["{p^{\prime}}"', two heads, from=1-1, to=2-1]
		\arrow["t", tail, from=2-1, to=2-2]
		\arrow[tail, from=2-2, to=2-3]
		\arrow[tail, from=1-1, to=1-3]
		\arrow["\lrcorner"{anchor=center, pos=0.125}, draw=none, from=1-1, to=2-3]
	\end{tikzcd}.\]
	Again, there is a subobject $ W \rightarrowtail \bm{1} $ which has a monic $ r : W \rightarrowtail X^{\prime} $. Clearly, $ p^{\prime} \circ r : W \rightarrowtail V$ is a monic. Therefore we also have $ t\circ p^{\prime} \circ r : W\rightarrowtail \neg M $. Hence $ W \meet \neg M = \bot$ and so $ W\join M = W\amalg M $. Since we already have $ W \rightarrowtail X $ and $ M \rightarrow X $ which are sections of $ p $. But $ M $ was the greatest subobject of $ I $ which has a section of $ p $, therefore our assumption $ M\neq I $ is wrong, and hence $ M = I $ and hence we have a section $ I \rightarrowtail X $.
\end{proof}
The $ \neg\neg $ sheaves over a poset forms a topos which follows axiom of choice:
\begin{proposition}\label{P-35}
	Let $ \cat{P} $ be a poset regarded as a category. Then the topos $ \Sh{\cat{P},\neg\neg} $ satisfies axiom of choice.
\end{proposition}
\begin{proof}
	We have to just show that $ \Sh{\cat{P},\neg\neg} $ is generated by $ \Sub{\bm{1}}{\Sh{\cat{P},\neg\neg}} $ because $ \Sh{\cat{P},\neg\neg} $ has complete subobject lattices as it is a Grothendieck topos. Now because each presheaf is a colimit of representables (Proposition \ref{P-2}) and so by the sheafification geometric morphism $ i \vdash  a$, any sheaf $ F $ is such that $ i(F) \isomorph  \colim D$ where $ D : I \rightarrow \psheaf{P}$ is a diagram of representables. Now, $ F \isomorph a\circ i (F) = a (\colim D) \isomorph \colim a(D)  $ where last isomorphism comes from the fact that $ a $ is the left adjoint. Hence, $ \Sh{\cat{P},\neg\neg} $ is generated by the sheafification of representables. But since $ \yembed{p} \rightarrowtail \bm{1}_{\psheaf{P}}$, and because $ a $ is left-exact (geometric morphism), therefore $ a(\yembed{p})  \rightarrowtail \bm{1}_{\Sh{\cat{P},\neg\neg}}$.
\end{proof}
\subsubsection{Dense Topology \& $ \neg\neg $}
Let's first revisit the dense topology:
\begin{definition}
	(\textbf{Dense Topology on a Poset}) \emph{Let} $ \cat{P} $ \emph{be a poset regarded as a category. We first define a subset} $ D_p \subseteq  \{q\in \obj{\cat{P}}\;\vert\; q\le p\}$ \emph{to be \textbf{dense below $ p $} if for any} $ r\le p $, $ \exists \;q\in D_p $ \emph{such that} $ q\le r $\emph{. We then define a Grothendieck topology} $ J $ \emph{of} $ \cat{P} $ \emph{given as} (\emph{for any }$ p\in \obj{\cat{P}} $)\emph{:}
	\begin{align*}
		J(p) :=  \left \{ D_p\;\vert\; D_p \text{ is dense below $ p $} \right \}.
	\end{align*}
\end{definition}
Extending the above definition to any category is obvious:
\begin{definition}
	(\textbf{Dense Topology on a Category}) \emph{Let} $ \cat{C} $ \emph{be a small category. The dense topology on} $ \cat{C} $ \emph{is defined as follows: for any object} $ C $ \emph{of }$ \cat{C} $,
	\begin{align*}
		JC := \left \{ S \;\vert\; \text{for any }f : D\to C\;,\;\;\exists \;g : E\to D \text{ such that }f\circ g\in S \right \}.
	\end{align*}
\end{definition}
We will now see that dense topology on the $ \psheaf{C} $ is exactly the $ \neg\neg $ topology on it:
\begin{proposition}\label{P-36}
	Suppose $ \cat{C} $ is a small category. Then, the $ \neg\neg $ Lawvere-Tierney topology on $ \psheaf{C} $ is equivalent to the dense topology on $ \cat{C} $.
\end{proposition}
\begin{proof}
	First note that in $ \psheaf{C} $, for a subobject $ A \rightarrowtail E$, it's negation $ (\neg A) \rightarrowtail E $ is given as follows:
	\begin{align*}
		(\neg A ) (C) := \{x\in EC \;\vert\; \forall\; f : \dom{f} \to C\;,\;\;Ef(x) \not \in A(\dom{f})\}.
	\end{align*}
Therefore $ \neg\neg A $ would be:
\begin{align*}
	(\neg\neg A)(C) := \left \{ x\in EC \;\vert\; \forall f : \dom{f} \to  C\;,\;\;\exists \; g : \dom{g} \to \dom{f} \text{ such that } E(f\circ g) (x) \in A(\dom{g}) \right \}.
\end{align*}
Secondly, for any given site $ (\cat{C},J) $, the closure operation of LT topology on $ \psheaf{C} $ induced by Grothendieck topology $ J $ is given by: for $ A\rightarrowtail E $ in $ \psheaf{C} $,
\begin{align*}
	x\in \bar{A}(C) \iff \left \{f : \dom{f} \to C\;\vert\; \bar{A}(f)(x) \in E(\dom{f})\right \} \in JC.
\end{align*} 
In particular, if we let $ (\cat{C},J) $ to be a dense topology, then the above condition would be:
\begin{align*}
	x\in \bar{A}(C) \iff \forall \;f :\dom{f} \to C\;,\;\exists\; g : \dom{g} \to \dom{f} \text{ such that } E(f\circ g)(x) \in A(\dom{g})
\end{align*}
and this is same as that of $ \neg\neg A $.
\end{proof}
\subsection{Axiom of Choice in a Topos}
Axiom of choice says that for a collection of non-empty sets $ \{X_i\}_{i\in I} $, the set $ \prod_{i\in I}X_i $ is also non-empty. This condition can also be stated equivalently as: A surjective function $ p : X \longrightarrow I $ has a section $s : I\longrightarrow X $ so that $p\circ s = 1_I $. We hence define the following:
\begin{definition}
	(\textbf{Axiom of Choice}) \emph{Suppose} $ \cat{E} $ \emph{is a topos. Then} $ \cat{E} $ \emph{is said to follow axiom of choice if for each epimorphism }
	\[\begin{tikzcd}
		X && Y
		\arrow["p", two heads, from=1-1, to=1-3]
	\end{tikzcd}\]
	\emph{has a section} 
	\[\begin{tikzcd}
		Y && X
		\arrow["s", from=1-1, to=1-3]
	\end{tikzcd}.\]
	\emph{That is, for every epis $ p $, there is an arrow $ s $ as above such that $ p\circ s =  1_Y$.}
\end{definition}
There is a weaker property for AC, called the internal axiom of choice.
\begin{definition}
	(\textbf{Internal Axiom of Choice}) \emph{Suppose }$ \cat{E} $ \emph{is a topos. Consider the functor for any object $ E $ of }$ \cat{E} $:
	\begin{align*}
		(-)^{E} : \cat{E} &\longrightarrow\cat{E}\\
		X&\longmapsto X^{E}\\
		(f : X\to Y) &\longmapsto (f^{E} : X^{E}\to Y^{E}).
	\end{align*}
	\emph{The topos} $ \cat{E} $ \emph{is said to follow the internal axiom of choice if the above functor $ (-)^{E} $ for any object $ E $, preserves epimorphisms.}
\end{definition}
\begin{remark}
	(\textbf{AC $ \implies $ IAC}) If $ p : X\to Y $ is any epimorphism in $ \cat{E} $ which has a section $ s : Y\to X $, then $ p^{E} : X^{E} \to Y^{E} $ and $ s^{E} : Y^{E} \to X^{E} $ are such that $ p^{E} \circ s^{E} = (p\circ s)^{E} = (1_Y)^{E} = 1_{Y^{E}}$, because $ (-)^{E} $ is a functor, and therefore $ s^{E} $ is a section of $ p^{E} $. Hence axiom of choice implies internal axiom of choice.
\end{remark}
An interesting property of $ \cat{Sets} $ is that the terminal object $ \bm{1} $ generates the whole category. Here, a collection of objects $ \mathcal{G} $ of a category $ \cat{C} $ is said to \textbf{generate} $ \cat{C} $ if and only if for any two non-equal parallel pair of arrows $ f\neq g : A\rightrightarrows B $ there exists an object $ G\in \mathcal{G} $ and an arrow $ u : G\to A $ such that $ f\circ u \neq g\circ u $. A topos $ \cat{E} $ is said to be \textbf{well-pointed} if the terminal $ \bm{1} $ generates the $ \cat{E} $. A topos $ \cat{E} $ is additionally said to be \textbf{non-degenerate} if $ \bm{0}\not\isomorph \bm{1} $.\\
Clearly $ \cat{Sets} $ is a non-degenerate, well-pointed topos. Moreover:
\begin{lemma}
	 Suppose $ \cat{E} $ is a non-degenerate topos. Then $ \cat{E} $ is also a well-pointed topos if and only if the functor 
	\begin{align*}
		\homset{\cat{E}}{\bm{1}}{-} : \cat{E} \longrightarrow \cat{Sets}
	\end{align*}
	is faithful.
\end{lemma}
\begin{proof}
	(L $ \implies $ R) Suppose $ \cat{E} $ is well-pointed and non-degenerate. If we have $ \homset{\cat{E}}{1}{A}\isomorph \homset{\cat{E}}{1}{B} $, then since $ \bm{1} $ generates $ \cat{E} $, therefore $ \homset{\cat{E}}{X}{A}\isomorph \homset{\cat{E}}{X}{B} $ because if $ f\neq g : X\rightrightarrows A $, then $ \exists \; u : \bm{1}\rightarrow X $ (because $ \bm{0} \not\isomorph \bm{1}$) such that $ f\circ u \neq g\circ u : \bm{1}\longrightarrow A \in \homset{\cat{E}}{\bm{1}}{A}$. Since $ \homset{\cat{E}}{X}{A}\isomorph\homset{\cat{E}}{X}{B} $ for any object $ X $ of $ \cat{E} $, hence by generalized elements, $ A\isomorph B $, proving that the functor $ \homset{\cat{E}}{\bm{1}}{-} $ is injective over hom-sets.\\
	(R$ \implies $ L) If $ \homset{\cat{E}}{\bm{1}}{-} $ is faithful, then if we take any two non-equal parallel arrows $ f\neq g : A \rightrightarrows B$, because $ \bm{0}\not\isomorph \bm{1} $, then we can conclude that $ \homset{\cat{E}}{\bm{1}}{f} \neq \homset{\cat{E}}{\bm{1}}{g} : \homset{\cat{E}}{\bm{1}}{A}\rightrightarrows \homset{\cat{E}}{\bm{1}}{B} $, which means that for any $ u : \bm{1}\to A $, $ \homset{\cat{E}}{\bm{1}}{f}(u) \neq \homset{\cat{E}}{\bm{1}}{g}(u) \implies f\circ u \neq g\circ u$. Hence $ \bm{1} $ generates $ \cat{E} $, so $ \cat{E} $ is well-pointed.
\end{proof}
\subsection{Independence of Continuum Hypothesis : The Cohen Topos}
We now prove that there is a Boolean topos (a model of set theory) in which the continuum hypothesis doesn't hold. We show the entire construction in the theorem below:
\begin{theorem}
	(\textbf{Independence of Continuum Hypothesis}) There is a Boolean topos satisfying the axiom of choice in which continuum hypothesis doesn't hold.
\end{theorem}
\begin{proof}
	\textbf{Act 1:} \emph{Requirement in} $ \cat{Sets} $ \emph{to follow CH}\\
	We first understand what we need in $ \cat{Sets} $ in order for it to follow CH. The continuum hypothesis says that there is no set whose cardinality is between $ \N $ and $ P\N = \mathbb{R}$. If this doesn't hold, then we must have a set $ X $ such that $ \N \rightarrowtail X \rightarrowtail P\N $ where the subobjects are strict, that is, there are no epimorphisms $ \N \to X $ and $ X \to P\N $. We will construct a Boolean topos (a model of set theory) where we would indeed have an object $ X $ with the above mentioned monics. In order to construct this Boolean topos, let us begin with the usual $ \cat{Sets} $ where we take a set $ B $ with cardinality strictly greater than that of $ P\N $. We would use\footnote{Note that the idea here is to use $ B $ to force another set to be in between them in some other model of set theory} $ B $ to \emph{force} some other unique set to be in between the nno and it's power object in a so constructed Boolean topos. We would then conclude that this so constructed topos will not follow CH.\\\\
	\textbf{Act 2:} \emph{Construction of the Cohen Poset} $ \cat{P} $\\
	To make this Boolean topos, let's first analyze our requirement of $ g : B \rightarrowtail P\N $. We can equivalently state it by the power adjunction:
	\begin{align*}
	\Ptr{g} : B\times \N &\longrightarrow \Omega \isomorph \bm{2}\\
	(b,n) &\longmapsto \begin{cases}
		0 &\text{ if }n\in g(b)\\
		1&\text{ if }n\notin g(b)
	\end{cases}
	\end{align*}
	where $ \Omega $ is the subobject classifier of $ \cat{Sets} $ . If $ g $ ought to be a monic, then we must have that
	\begin{align*}
		b\neq b^{\prime} \implies g(b)\neq g(b^{\prime}) \iff \exists\;n\text{ such that } \Ptr{g}(b,n) \neq \Ptr{g}(b^{\prime} ,n).
	\end{align*}
	or the contrapositive:
	\begin{align*}
		\Ptr{g}(b,n) = \Ptr{g}(b^{\prime},n) \forall \;n \implies b=b^{\prime}
	\end{align*} 
	These conditions are particularly important as we would try to reach this condition for two $ b\neq b^{\prime} $. \\
	Now, define a \textbf{condition} as a tuple $ (F_p, p) $ where $ F_p \subseteq B\times \N $ is finite and $ p : F_p \longrightarrow \bm{2} $. What we wish to do is to get closer and closer to the whole $ B\times \N $ gradually, this means that as $ F_p $ gets bigger and bigger, we wish to get the corresponding $ p $ closer and closer to $ \Ptr{g} $. In order to argue this more concretely, we construct the following poset, called the Cohen poset:
	\begin{align*}
		P := \left \{ (F_p,p) \;\vert\; F_p \subseteq B\times\N \text{ is finite} \;\& \; p : F_p \to \bm{2}\right \}
	\end{align*}
	and the order $ \le  $ in $ P $ is given by:
	\begin{align*}
		p\le q \iff F_q\subseteq F_p \;\& \; \rest{p}{F_q} = q.
	\end{align*}
	For the following we regard the Cohen poset $ P $ as a category $ \cat{P} $.\\\\
	\textbf{Act 3:} \emph{Transferring $ B\times \N $ from} $ \cat{Sets} $ \emph{to} $ \psheaf{P} $\\
	We now use the adjunction\footnote{As alluded to earlier, this is actually a geometric morphism.} given in Definition \ref{D-19}, as follows:
	\[\begin{tikzcd}
		& {} \\
		{\psheaf{P}} & {} & {\cat{Sets}} \\
		& {}
		\arrow[no head, from=2-1, to=2-2]
		\arrow[from=2-2, to=2-3]
		\arrow["\Delta"{description}, curve={height=-24pt}, from=2-3, to=2-1]
		\arrow["\dashv"{anchor=center, rotate=90}, draw=none, from=3-2, to=1-2]
		\arrow["\Gamma"{description}, curve={height=-24pt}, from=2-1, to=2-3]
	\end{tikzcd}.\]
	The left adjoint $ \Delta $ takes $ B\times \N $ to $ \Delta (B\times \N) $ but since $ \Delta $ is left-exact (footnote 35), hence $ \Delta (B\times \N) \isomorph \Delta B \times \Delta \N $. Now, as we pointed to earlier, we wish to get a $ p $ as close to $ \Ptr{g} $ as possible, so we consider the following subobject of $ \Delta (B\times \N) $:
	\begin{align*}
		A \rightarrowtail \Delta (B\times \N) 
	\end{align*}
which takes a condition $ p $ of $ \cat{P} $ to:
\begin{align*}
	A : \opcat{P} &\longrightarrow \cat{Sets}\\
	p &\longmapsto \left \{ (b,n) \in B\times \N \;\vert\; p(b,n) = 0\right \}.
\end{align*}
We now observe the following:\\\\
\textbf{Act 4: }\emph{$ A $ is a closed subobject of $ \Delta (B\times \N) $ with respect to $ \neg\neg $ topology on} $ \psheaf{P} $\\
The proof of the above statement is as follows: We just wish to show $ \neg\neg A \subseteq  A $ in the lattice $ \Sub{\Delta(B\times \N)}{\cat{P}} $ as the other side is trivial. As the proof of Proposition \ref{P-36} showed, we have $ (b,n) \in \neg\neg A(p)$ iff $ \forall q\le p $, $ \exists r\le q $ such that $ \Delta(B\times\N)(r \le p)((b,n))\in A(r)$ or $ (b,n) \in A(r) $ or $ r(b,n) = 0 $.  Assume $ (b,n) \notin A(p) $. Hence $ p(b,n) \neq 0 $. But then we are not sure whether $ p(b,n)= 1 $ or $ p(b,n) $ is undefined. For the former, if $ p(b,n) = 1 $, then clearly for any $ r\le p $, $ r(b,n) = 1 $ because $ \rest{r}{F_p} = p $ and so $ (b,n) \notin \neg\neg A(p) $. For the latter, if $ p(b,n)  $ is undefined, then for some $ q\le p $ we would have $ q(b,n) = 1 $, and then for any $ r\le q $, $ r(b,n) =1 $ to conclude that $ (b,n) \notin A(p) $. Both cases suggest $ (b,n)\notin \neg\neg A(p) $. Hence proved the fact that if $ (b,n)\notin A(p) \implies (b,n) \notin \neg\neg A(p)$ and it's contrapositive gives the required result.\\\\
%\textbf{Act 5:} \emph{Transferring $ \Delta (B\times \N) $ from} $\psheaf{P}$ \emph{to} $ \Sh{\cat{P},\neg\neg} $ \emph{via sheafification}
\textbf{Act 5:} \emph{Yielding an arrow $ \Delta B \longrightarrow \Omega_{\neg\neg}^{\Delta \N} $}\\
Since $ A $ is a closed subobject of $ \Delta (B\times \N) $, therefore the characteristic arrow $ \Delta(B\times \N) \longrightarrow \Omega $ factors via the LT Topology $ \neg\neg  : \Omega \longrightarrow \Omega$ and $ \neg\neg : \Omega \longrightarrow \Omega $ itself factors via the $ \Omega_{\neg\neg} $ which is the subobject classifier of $ \Sh{\cat{P},\neg\neg} $, as shown below:
\[\begin{tikzcd}
	{\Delta(B\times \N)} & \Omega \\
	\Omega & {\Omega_{\neg\neg}}
	\arrow["{\chr{A}}", from=1-1, to=1-2]
	\arrow["{\chr{A}}"', from=1-1, to=2-1]
	\arrow["\neg\neg"{description}, from=2-1, to=1-2]
	\arrow["m"', tail, from=2-2, to=1-2]
	\arrow["r"', two heads, from=2-1, to=2-2]
\end{tikzcd}.\]
Therefore, we have an arrow
\begin{align*}
	f := r\circ \chr{A} : \Delta B\times\Delta \N \longrightarrow \Omega_{\neg\neg}
\end{align*}
and hence it's $ P $-transpose would be:
\begin{align*}
	\Ptr{f} : \Delta B \longrightarrow \Omega_{\neg\neg}^{\Delta \N}.
\end{align*}
\textbf{Act 6:} $ \Ptr{f} : \Delta B\longrightarrow \Omega_{\neg\neg}^{\Delta \N} $ \emph{is a monomorphism in} $ \psheaf{P} $\\
To show the above, we just have to show that each component $ \Ptr{f}_p : \Delta B(p) = B \longrightarrow \Omega_{\neg\neg}^{\Delta \N}(p)$ is a monomorphism. To this extent, take any two non-equal elements $ b\neq b^{\prime}$ from $ B $. We wish to show that $ \Ptr{f}_p(b) \neq  \Ptr{f}_p(b^{\prime})$. First, we note that $ \Omega_{\neg\neg}^{\Delta \N} $, by Proposition \ref{P-3}, is given as follows:
\begin{align*}
	\Omega_{\neg\neg}^{\Delta \N} : \opcat{P} &\longrightarrow \cat{Sets}\\
	p &\longmapsto \Nat{\homset{\cat{P}}{-}{p} \times \Delta \N}{\Omega_{\neg\neg}}.
\end{align*}
Therefore, $ \Ptr{f}_p(b) $ is a natural transformation as:
\begin{align*}
	 \Ptr{f}_p(b) : \homset{\cat{P}}{-}{p} \times \Delta \N \Rightarrow \Omega_{\neg\neg}.
\end{align*}
To fulfill our aim, we must show that $ \Ptr{f}_p(b) \neq \Ptr{f}_p(b^{\prime}) $. Again, as both are natural transforms, so we would be done if we would show that for each $ q\le p $, $  (\Ptr{f}_p(b))_q \neq (\Ptr{f}_p(b^{\prime}))_q $. Again, 
\begin{align*}
	 (\Ptr{f}_p(b))_q : (\homset{\cat{P}}{-}{p}\times \Delta \N)(q) \isomorph \homset{\cat{P}}{q}{p}\times \Delta \N(q) \isomorph \{\star\}\times \N \isomorph \N &\longrightarrow \Omega_{\neg\neg}(q) \\
	 n&\longmapsto \{r\in \cat{P}\;\vert\; r\le q \;\&\; r(b,n) = 0\}.
\end{align*}
Now consider $  (\Ptr{f}_p(b))_q(n) $ as above. If $ r\in  (\Ptr{f}_p(b))_q (n) $, then $ r(b,n) = 0 $ and so all $ t\le r$ is in $ (\Ptr{f}_p(b))_q (n)  $. Since all conditions $ F_r $ are finite, hence for large enough $ n $, neither $ (b,n) $ nor $ (b^{\prime},n) $ would be defined for $ r : F_r \to \bm{2} $. One can now easily construct some $ t\le r $ with $ t(b,n) = 0 $ and $ t(b,n) = 1 $ and so $ t\in  (\Ptr{f}_p(b))_q(n) $ but $ t\notin  (\Ptr{f}_p(b))_q(n) $. Hence we are done.\\\\
\textbf{Act 7:} \emph{Transferring monic} $ \Ptr{f} : \Delta B\longrightarrow \Omega_{\neg\neg}^{\Delta \N} $ \emph{from} $ \psheaf{P} $ \emph{to a monic in} $ \Sh{\cat{P},\neg\neg} $ \emph{via sheafification}\\
Since sheafification is inverse image of a geometric morphism, therefore it would trivially preserve the monic $ \Ptr{f} $. In particular, we would have the following monic:
\begin{align*}
	a(\Ptr{f}) : a(\Delta B) \rightarrowtail a(\Omega_{\neg\neg}^{\Delta\N})
\end{align*}
and because $ a(X^{Y}) \isomorph a(X) ^{a(Y)} $, therefore $ a(\Omega_{\neg\neg}^{\Delta\N}) \isomorph (a(\Omega_{\neg\neg}))^{a(\Delta \N)} \isomorph \Omega_{\neg\neg}^{a(\Delta\N)}$, we can rewrite above as (denoting $ \widehat{(-)} := a(\Delta(-))$):
\begin{align*}
	\widehat{\Ptr{f}} : \widehat{B} \rightarrowtail \Omega_{\neg\neg}^{\widehat{\N}}.
\end{align*}
\textbf{Act 7:} \emph{Conclusion}\\
We have finally proved that $ \widehat{B} $ is a subobject of $ \Omega_{\neg\neg}^{\widehat{\N}} \isomorph P(\widehat{\N})$ where $ P(\widehat{\N}) $ is the power object of the nno $ \widehat{\N} $ in $ \Sh{\cat{P},\neg\neg} $. One can also show that (but we won't here for space concerns! Refer to Section 6.3, pp 284 of \cite{MacMoer} instead) the cardinal inequality of our choice of $ B $ in $ \cat{Sets} $ as $ \N < P\N < B $ is preserved in $ \Sh{\cat{P},\neg\neg} $ as
\begin{align*}
	\widehat{\N} < \widehat{P\N} < \widehat{B}
\end{align*}   
and this, combined with $ \widehat{B} < P(\widehat{\N}) $, gives us that in the Boolean Grothendieck topos (which is a model of set theory!) $ \Sh{\cat{P},\neg\neg} $, we would have the following cardinal inequality:
\begin{align*}
\boxed{	\widehat{\N} < \widehat{P\N} < P(\widehat{\N}) \;\;\text{ in } \Sh{\cat{P},\neg\neg}.} 
\end{align*}
Therefore, $ \Sh{\cat{P},\neg\neg} $ is a Boolean Grothendieck topos in which continuum hypothesis fails as above but axiom of choice holds by Proposition \ref{P-35}.
\end{proof}
\begin{remark}
	(\textbf{The Cohen Topos}) The Boolean Grothendieck Topos of $ \neg\neg $-sheaves over the Cohen poset $ \cat{P} $, $ \Sh{\cat{P},\neg\neg} $, is usually called the Cohen topos.
\end{remark}

%
%\newpage
%\subsection{Independence of Axiom of Choice}
%We will now study a way to construct a Boolean Grothendieck topos in which internal axiom of choice would fail.
\newpage
\subsection{Integers in a Topos}
The concept of Dedekind cuts is usually used to generate irrationals from rationals. One can essentially extend the idea on to a sheaf topos $ \Sh{X} $ where $ X $ is some topological space.\\
Let's first construct the integers from naturals:
\subsubsection{From $ \mathbb{N}_\cat{E} $ to $ \mathbb{Z}_\cat{E} $}
In the usual category $ \cat{Sets} $, we have the $ \mathbb{N} $. One can construct all integers by collecting all pairs of naturals whose difference between them is same, that is, $ \mathbb{Z} $ can be constructed as the following quotient set:
\begin{align*}
	\mathbb{Z} :=  \{(n,m)\;\vert\; n,m\in \mathbb{N}\}/\sim 
\end{align*} 
where $ \sim$ is the following equivalence relation,
\begin{align*}
	(n,m) \sim (n^{\prime},m^{\prime}) \iff n+m^{\prime} = m+n^{\prime}.
\end{align*}
The relation $ \sim $ essentially collects all the pairs $ (n,m) $ whose difference $ (n-m) $ are same. More categorically, we can represent above as an universal construction in $ \cat{Sets} $ as follows:
\begin{enumerate}
	\item {Construct the kernel pair of $ + :\N\times\N \longrightarrow \N $ in $ \cat{Sets} $:
		\[\begin{tikzcd}
			E & \N\times\N \\
			\N\times\N & \N
			\arrow["{+}"', from=2-1, to=2-2]
			\arrow["{+}", from=1-2, to=2-2]
			\arrow["a"', from=1-1, to=2-1]
			\arrow["b", from=1-1, to=1-2]
			\arrow["\lrcorner"{anchor=center, pos=0.125}, draw=none, from=1-1, to=2-2]
		\end{tikzcd}.\]
}
\item {Then construct the following coequalizer where $ \pi_1,\pi_2 : \N\times\N \rightrightarrows \N $ are the product projections:
\[\begin{tikzcd}
	E && \N\times\N && \Z
	\arrow["{\uprs{\pi_1\circ a,\pi_2\circ b}}", shift left=2, from=1-1, to=1-3]
	\arrow["{\uprs{\pi_2\circ a, \pi_1\circ b}}"', shift right=2, from=1-1, to=1-3]
	\arrow[two heads, from=1-3, to=1-5]
\end{tikzcd}.\]
}
\end{enumerate}
The fact that this universal description is equivalent to the previous set-theoretic one can be seen via the observation that $ E $ is the collection of 4-tuples $ (n,m^{\prime},n^{\prime},m) $ with $ n+m^{\prime} = m+ n^{\prime} $ with $ a $ and $ b $ being the respective projections, and the arrows $ \pi_1\circ a $ takes $ (n,m^{\prime},n^{\prime},m) $ to $ n $, $ \pi_2\circ b $ takes it to $ m $, $ \pi_2 \circ a  $ takes it to $ m^{\prime} $ and $ \pi_1\circ b $ takes it to $ n^{\prime} $.\\
With this, we are motivated to state the following:
\begin{definition}
	(\textbf{Integer Object in a Topos}) \emph{Let} $ \cat{E} $ \emph{be a topos and} $ \nno{E} $ \emph{being the natural numbers object in} $ \cat{E} $\emph{. Then, we define the integer object} $ \Z_\cat{E} $ \emph{in} $ \cat{E} $\emph{ as the following coequalizer:}
	\begin{enumerate}
		\item {\emph{Let $ E $ be the following pullback:}
				\[\begin{tikzcd}
				E & {\N_\cat{E}\times \N_\cat{E}} \\
				{\N_\cat{E}\times\N_\cat{E} } & \N_\cat{E}
				\arrow["{+}"', from=2-1, to=2-2]
				\arrow["{+}", from=1-2, to=2-2]
				\arrow["a"', from=1-1, to=2-1]
				\arrow["b", from=1-1, to=1-2]
				\arrow["\lrcorner"{anchor=center, pos=0.125}, draw=none, from=1-1, to=2-2]
			\end{tikzcd}\]	
}
\item {\emph{And then define the} $ \Z_\cat{E} $ \emph{is defined as:}
\[\begin{tikzcd}
	E && \N_\cat{E}\times\N_\cat{E} && \Z_\cat{E}
	\arrow["{\uprs{\pi_1\circ a,\pi_2\circ b}}", shift left=2, from=1-1, to=1-3]
	\arrow["{\uprs{\pi_2\circ a, \pi_1\circ b}}"', shift right=2, from=1-1, to=1-3]
	\arrow[two heads, from=1-3, to=1-5]
\end{tikzcd}.\]
}
	\end{enumerate}
%Let us now demonstrate a construction in any well-pointed topos $ \cat{E} $ with an NNO $ \N_{\cat{E}} $. Let us focus on the slice topos $ \cat{E}/\nno{E}$. Consider the object $ n : \bm{1} \to \nno{E} $ in $ \cat{E}/\nno{E} $. Since we have the arrow $ s : \nno{E} \to \nno{E} $, therefore let us now construct the exponential object $ s^n : \nno{E}^{\bm{1}} \isomorph \nno{E} \to \nno{E} $ in $ \cat{E}/\nno{E} $. Now, consider an arrow $ (n,m) : \bm{1} \longrightarrow \nno{E} \times \nno{E} $. Consider now the operator 
%\begin{align*}
%	\homset{\cat{E}}{\bm{1}}{\nno{E}\times\nno{E}} & \longrightarrow \homset{\cat{E}}{\bm{1}}{\nno{E}}\\
%	(n,m)&\longmapsto s^n \circ s^m \circ 0
%\end{align*}
%We wish to show that this operator is natural in $ \bm{1} $. To see if that's the case or not, let's take any arrow $ f : X\to Y $ of $ \cat{E} $. Since the above operator in the form as stated doesn't makes sense because $ \homset{\cat{E}}{Y}{\nno{E}\times \nno{E}} \longrightarrow \homset{\cat{E}}{Y}{\nno{E}} $ which takes $ (a,b) \longmapsto s^a\circ s^b\circ 0$ because latter is not a possible arrow ($ s^a $ and $ s^b $ are parallel arrows $ \nno{E}^Y \rightrightarrows \nno{E} $), therefore we need to modify the definition of our operator. So let us redefine our operator as 
%\begin{align*}
%	\homset{\cat{E}}{Y}{\nno{E}\times \nno{E}} &\longrightarrow \homset{\cat{E}}{Y}{\nno{E}}\\
%	(a,b) : Y \to \nno{E}\times \nno{E} &\longmapsto 
%\end{align*}

\end{definition}


























\newpage
\appendix
\section{Notes on : A theory of Enriched Sheaves - F. Borceux, C. Quinteiro}
The following are some of the notes made while attempting to understand the following work : \href{http://www.numdam.org/item/CTGDC_1996__37_2_145_0.pdf}{A theory of enriched sheaves, F. Borceux, C. Quinteiro}.
\subsection{Some non-trivial Introductory Definitions}
\begin{enumerate}
	\item {\textbf{Locally Presentable Category} : A locally small category $ \cat{C} $ is a locally presentable category precisely if $ \exists $ a small set $ S \hookrightarrow \obj{\cat{C}} $ such that each object in $ S $ is a \textbf{small object} and every other object in $ \cat{C} $ is a colimit of some diagram $ D: I \to S $ of these objects from $ S $. $ S $ is then called the \emph{generating set} of the category $ \cat{C} $.
		\begin{remark}
			\textbf{Small Object} : An object $ A  $ in $ \cat{C} $ is said to be small if $ \exists  $ a regular cardinal $ \kappa $ such that $ A $ commutes with $ \kappa $-filtered colimit of any $ \kappa $-filtered diagram $ X : J \to \cat{C} $. Commutativity here means that:
			\begin{align*}
				\colim \left (\homset{\cat{C}}{A}{X(-)} \right )\isomorph \homset{\cat{C}}{A}{\colim \left (X(-)\right )}
			\end{align*}
			More succinctly, the object $ A $ is said to be a \textbf{$ \kappa $-compact object} if it follows the above condition.
		\end{remark}
		\begin{remark}
			\textbf{Locally Finitely Presentable Category} : A locally presentable category in which the generating set $ S $ of small objects are $ \aleph_0 $-small. That is, each object in $ S $ is a $ \aleph_0 $-compact object. Remember $ \aleph_0 = \vert\mathbb{N}\vert $.
		\end{remark}
	}
	\item {\textbf{Symmetric Monoidal Closed Category} : (\emph{tersely}) A category $ \cat{V} $ having the following:
		\begin{itemize}
			\item {Tensor product bifunctor : $\cat{V} \tens \cat{V} \to \cat{V} $.}
			\item {Tensor unit object : $ I $.}
			\item {Associative natural isomorphism : $ a_{x,y,z} : (x\tens y)\tens z \longrightarrow x\tens (y\tens z) $.}
			\item {Left unit natural isomorphism : $ l_x : I \tens x \longrightarrow x $.}
			\item {Right unit natural isomorphism : $ r_x : x\tens I \longrightarrow x $.}
			\item {Symmetry natural isomorphism : $ s_{x,y} : x\tens y \longrightarrow y\tens x $.}
			\item {Internal Hom functor : $ (-)^{y} : \cat{V} \to \cat{V}$, the right adjoint of $( - \tens y) : \cat{V} \to \cat{V}$. }
		\end{itemize}
		combined with the corresponding canonical associativity commutative diagrams.
	}
	\item {\textbf{Locally Finitely Presentable Symmetric Monoidal Closed Category/LFPSMC} : A category $ \cat{V} $ which satisfies both property 1 and 2 above. This also is combined with the assumption that the tensor unit object $ I $ is $ \aleph_0 $-compact and for two $ \aleph_0 $-compact objects $ A,B $ from the generating set $ S $, we must have that $ A\tens B  $ is also $ \aleph_0 $-compact in $ S $. }
	\item {\textbf{$ \cat{V} $-Enriched Category} : (\emph{tersely}) A category $ \cat{C} $ is said to be $ \cat{V} $-Enriched where $ \cat{V} $ is a monoidal category if we have the following:
		\begin{itemize}
			\item {A set of objects $ \obj{\cat{C}} $ of category $ \cat{C} $.}
			\item {For any pair of objects $ (x,y)\in \cat{C}\times\cat{C} $, an object of $ \cat{V} $ denoted as 
				\begin{align*}
					\cat{C}(x,y)\text {, the \textbf{hom object} of $ x $ and  $ y $ in $ \cat{V} $.}
				\end{align*}	
			}
			\item {For any two arrow objects $ \cat{C}(x,y) $ and $ \cat{C}(y,z) $, an arrow in $ \cat{V} $ given by:
				\begin{align*}
					\circ_{x,y,z}:	\cat{C}(y,z) \tens \cat{C}(x,y) \longrightarrow \cat{C}(x,z) 
				\end{align*}
				called the \textbf{composition arrow} in $ \cat{V} $.
			}
			\item {For any object $ x $ of $ \cat{C} $, the \textbf{unital arrow }in $ \cat{V} $ given by:
				\begin{align*}
					\eta_x : I \longrightarrow \cat{C}(x,x)
				\end{align*}
			}
		\end{itemize}
		combined with the canonical unital and associative composition commutative diagrams.
	}
	\item {\textbf{$ \cat{V} $-Enriched Functor} : Suppose $ \cat{C} $ and $ \cat{D} $ are two $ \cat{V} $-enriched categories. A $ \cat{V} $-enriched functor $ F: \cat{C}\longrightarrow \cat{D} $ is defined as:
		\begin{itemize}
			\item {A function $ F_0 : \obj{\cat{C}} \longrightarrow \obj{\cat{D}} $.}
			\item {A collection of arrows $ \{f_{x,y}\} $ in $ \cat{V} $ indexed by $(x,y)\in  \obj{\cat{C}}\times\obj{\cat{C}} $ given as:
				\begin{align*}
					f_{x,y} : \cat{C}(x,y) \longrightarrow \cat{D}(F_0x,F_0y)
				\end{align*}
				between the hom-objects, present as objects in $ \cat{V} $.
			}
			\item {Following the functor's distribution over composition and respect of unit, both described as the following two commutative diagrams respectively:
				\[\begin{tikzcd}
					{\cat{C}(y,z)\tens\cat{C}(x,y)} & {\cat{C}(x,z)} \\
					{\cat{D}(F_0y,F_0z)\tens\cat{D}(F_0x,F_0,y)} & {\cat{D}(F_0x,F_0z)}
					\arrow["{f_{y,z}\tens f_{x,y}}"', from=1-1, to=2-1]
					\arrow["{\circ_{x,y,z}}", from=1-1, to=1-2]
					\arrow["{\circ_{F_0x,F_0y,F_0z}}"', from=2-1, to=2-2]
					\arrow["{f_{x,z}}", from=1-2, to=2-2]
				\end{tikzcd}\]
				\[\begin{tikzcd}
					I & {\cat{C}(x,x)} \\
					& {\cat{D}(F_0x,F_0x)}
					\arrow["{\eta_x}", from=1-1, to=1-2]
					\arrow["{f_{x,x}}", from=1-2, to=2-2]
					\arrow["{\eta_{F_0x}}"', from=1-1, to=2-2]
				\end{tikzcd}\]
			}
		\end{itemize}
	}
	
	
	\item {\textbf{Finite $ \cat{V} $-Enriched Category} : A $ \cat{V} $-enriched category $ \cat{C} $ where $ \cat{V} $ is LFPSMC, in which each hom object $ \cat{C}(x,y) $ in $ \cat{V} $ is in the generating set $ S $ of $ \cat{V} $, so it's finitely presentable/$ \aleph_0 $-compact.
	}
	\item {\textbf{Finite Indexing Type} : A functor $ F: \cat{C} \longrightarrow \cat{V} $ where for any object $ C $ of $ \cat{C} $, $ FC $ is in the generating set $ S $ of $ \cat{V} $ or equivalently, $ FC $ is finitely presentable/$\aleph_0  $-compact.}
	
\end{enumerate}

\subsubsection{Ends of $ \cat{V} $-Functors and $ \cat{V} $-Enriched Functor Categories.}
\begin{enumerate}
	\item {\textbf{Covariant Action of a category on a functor} : Suppose $ \cat{C} $ is a $\cat{V} $-enriched category and $ F : \opcat{\cat{C}} \times \cat{C} \longrightarrow \cat{V}$ is a $ \cat{V} $-enriched functor by assuming $ \cat{V} $ is $ \cat{V} $-enriched. A covariant action of $ \cat{C} $ on $ F $ is a collection of arrows in $ \cat{V} $ given as:
		\begin{align*}
			\lambda_{c,d,e} : F(c,d) \tens \cat{C}(d,e) \longrightarrow F(c,e)
		\end{align*}	
		for any objects $ c,d,e $ in $ \cat{C} $.
	}
	\item {\textbf{Contravariant Action of a category on a functor} : Suppose $ \cat{C} $ is a $ \cat{V} $-enriched functor and $ F : \opcat{C} \times \cat{C}\longrightarrow \cat{V} $ is a $ \cat{V} $-enriched functor. A contravariant action of $ \cat{C} $ on $ F $ is a collection of arrows in $ \cat{V} $ given as:
		\begin{align*}
			\rho_{c,d,e} : F(d,e)\tens \cat{C}(c,d) \longrightarrow F(c,e)
		\end{align*}
		for any objects $ c,d,e $ of $ \cat{C} $.
	}
	\item {\textbf{$ \cat{V} $-Extranatural Transformation} : Suppose $ F: \opcat{\cat{C}} \times \cat{C} \longrightarrow \cat{V} $ is a $ \cat{V} $-enriched functor and $ v $ is an object in $ \cat{V} $. An extranatural transformation 
		\begin{align*}
			\theta : v\xn F
		\end{align*}
		from $ v $ to $ F $ is defined as a family of arrows:
		\begin{align*}
			\left \{\theta_c : v \to F(c,c)\right \}_{c\in \obj{\cat{C}}}
		\end{align*}
		such that for any pair of objects $ (c,d) $ in $ \obj{\cat{C}}\times \obj{\cat{C}} $, we must have that:
		\[\begin{tikzcd}
			& {v\tens \cat{C}(c,d)} \\
			{F(d,d)\tens \cat{C}(c,d)} && {F(c,c) \tens \cat{C}(c,d)} \\
			& {F(c,d)}
			\arrow["{\theta_C \tens 1}", from=1-2, to=2-3]
			\arrow["{\lambda_{c,c,d}}", from=2-3, to=3-2]
			\arrow["{\theta_D\tens 1}"', from=1-2, to=2-1]
			\arrow["{\rho_{c,d,d}}"', from=2-1, to=3-2]
		\end{tikzcd}\]
		commutes
	}
	\item {\textbf{$ \cat{V} $-Enriched End of an Enriched bifunctor} : Suppose $ F : \opcat{C} \times \cat{C} \longrightarrow \cat{V} $ is a $ \cat{V} $-enriched bifunctor. Then a $ \cat{V} $-enriched end of $ F $ is an object of $ \cat{V} $ denoted $ \int_{c :\cat{C}} F(c,c) $, which is equipped with a $ \cat{V} $-enriched extranatural transformation :
		\begin{align*}
			\pi : \int_{c:\cat{C}} F(c,c) \xn F
		\end{align*}
		such that for any other extranatural transformation $ \theta: v \xn F $, any component $ \theta_c $ of it is obtained via pre-composition of $ \pi_c $ with an unique arrow 
		\begin{align*}
			f : v\longrightarrow \int_{c: \cat{C}}F(c,c)
		\end{align*}
		in $ \cat{V} $. That is, for any extranatural transformation $ \theta $, $ \exists $ unique $ f $ as above such that 
		\begin{align*}
			\theta_c = \pi_c \circ f \;\forall \; c\in \obj{\cat{C}}
		\end{align*}
		
	}
	\item {\textbf{$ \cat{V} $-Enriched Natural Transformation} : Suppose $ F,G : \cat{C}\longrightarrow \cat{D}$ are two $ \cat{V} $-enriched functors between $ \cat{V} $-enriched categories. A $ \cat{V}$-enriched natural transformation $ \alpha : F \Longrightarrow G $ is a family of arrows in $ \cat{V} $
		\begin{align*}
			\left \{ \alpha_c : I \longrightarrow \cat{D}(Fc,Gc) \right \}_{c\in \obj{\cat{C}}}
		\end{align*}
		where $ I $ is the tensor unit of $ \cat{V} $, such that for any pair $ (c,d) \in \obj{\cat{C}} \times \obj{\cat{C}}$, the following diagram commutes:
		\[\begin{tikzcd}
			{\cat{C}(c,d)} && {\cat{C}(c,d) \tens I} && {\cat{D}(Gc,Gd)\tens \cat{D}(Fc,Gc)} \\
			\\
			{I\tens \cat{C}(c,d)} && {\cat{D}(Fd,Gd)\tens \cat{D}(Fc,Fd)} && {\cat{D}(Fc,Gd)}
			\arrow["{l_{\cat{C}(c,d)}}", from=1-1, to=1-3]
			\arrow["{g_{c,d}\tens\alpha_c}", from=1-3, to=1-5]
			\arrow["{\circ_{Fc,Gc,Gd}}", from=1-5, to=3-5]
			\arrow["{r_{\cat{C}(c,d)}}"', from=1-1, to=3-1]
			\arrow["{\alpha_d\tens f_{c,d} }"', from=3-1, to=3-3]
			\arrow["{\circ_{Fc,Fd,Gd}}"', from=3-3, to=3-5]
		\end{tikzcd}.\]
	}
	\item {\textbf{$ \cat{V} $-Enriched Functor Category $ \func{{C}}{{D}} $} : Suppose $ \cat{C} $ and $ \cat{D} $ are two $ \cat{V} $-enriched categories. We can then form a $ \cat{V} $-enriched category of $ \cat{V} $-functors $ F: \cat{C}\to \cat{D} $ and $ \cat{V} $-natural transformations $ \alpha: F\Rightarrow G $, denoted $ \func{{C}}{{D}} $, in which :
		\begin{itemize}
			\item {\textbf{Objects} are the $ \cat{V} $-functors $ F : \cat{C}\to \cat{D} $.}
			\item {\textbf{Hom Objects} in $ \cat{V} $, denoted $ \func{C}{D}(F,G) $ between two functors $ F $ \& $ G $, is given by the $ \cat{V} $-enriched end of the bifunctor:
				\begin{align*}
					\cat{D}(F(-),G(-)) : \opcat{C} \times \cat{C} \longrightarrow \cat{V}.
				\end{align*}
				That is, the hom-object is given as below:
				\begin{align*}
					\func{C}{D} (F,G) := \int_{c: \cat{C}} \cat{D}(Fc,Gc)
				\end{align*}
				where the corresponding $ \cat{V} $-enriched extranatural transformation is then denoted as:
				\begin{align*}
					\pi_c : \func{C}{D} (F,G) \longrightarrow \cat{D}(Fc,Gc).
				\end{align*}
				
			}
			\item {\textbf{Composition} of two hom-objects in $ \cat{V} $ is then given as:
				\begin{align*}
					\circ_{F,G,H} : \func{C}{D}(G,H)\tens\func{C}{D}(F,G) \longrightarrow \func{C}{D}(F,H)
				\end{align*}
				which is the universal arrow into the end $ \func{C}{D}(F,G) $ because the following 
				\[\begin{tikzcd}
					{ \func{C}{D}(G,H)\tens\func{C}{D}(F,G) } && {\cat{D}(Gc,Hc)\tens \cat{D}(Fc,Gc)} && {\cat{D}(Fc,Hc)}
					\arrow["{\pi_c\tens \pi_d}", from=1-1, to=1-3]
					\arrow["{\circ_{Fc,Gc,Hc}}", from=1-3, to=1-5]
				\end{tikzcd}\]
				forms a $ \cat{V} $-enriched extranatural transformation $ \func{C}{D}(G,H)\tens \func{C}{D}(F,G) \xn \cat{D}(F(-),G(-)) $.
			} 
			
		\end{itemize}
		
	}
\end{enumerate}
\subsubsection{An important result}
\begin{proposition}
	In a monoidal closed category $ \cat{V} $, we have that for any three objects $ x,y,z $ in $ \cat{V} $, the following "internal" isomorphism:
	\begin{align*}
		(z)^{x\tens y} \isomorph (z^{y})^{x}.
	\end{align*}
\end{proposition}
\begin{proof}
	Just notice that by the exponential adjunction, we have the natural isomorphism $ \homset{\cat{V}}{x\tens y}{z} \isomorph \homset{\cat{V}}{x}{z^{y}} $ for any objects $ x,y,z $. But then for any other object $ w $ of $ \cat{V} $, we must have that:
	\begin{align*}
		\homset{\cat{V}}{w}{z^{x\tens y}} \isomorph \homset{\cat{V}}{w\tens x\tens y}{z} \isomorph \homset{\cat{V}}{w\tens x}{z^{y}} \isomorph \homset{\cat{V}}{w}{(z^{y})^{x}}.
	\end{align*}
	Now, since all above isomorphisms are natural, therefore by fully faithfulness of Yoneda embedding, we have an isomorphism $ z^{x\tens y}  \isomorph (z^{y})^{x}$. Alternatively, one can deduce an isomorphism by the presence of universal elements of two functors $ \homset{\cat{V}}{-}{z^{x\tens y}} $ and $ \homset{\cat{V}}{-}{(z^{y})^{x}} $.
\end{proof}
\begin{corollary}\label{C-2}
	In a symmetrical monoidal closed category $ \cat{V} $, we will have the following natural isomorphism, thanks to the symmetry of tensor product now:
	\begin{align*}
		(z^{y})^{x} \isomorph (z^{x})^{y}.
	\end{align*}
\end{corollary}
\begin{remark}
	We will now use the following notation for exponentials, because they will now be made to behave similar to internal hom-objects:
	\begin{align*}
		[y,x] : =x^{y}.
	\end{align*}
In this notation, the Corollary \ref{C-2} would be denoted as:
\begin{align*}
	[x,[y,z]] \isomorph [y,[x,z]]
\end{align*}
\end{remark}
\begin{definition}
	(\textbf{Power of an object \& Cotensored Category}) Let $ \cat{V} $ be a monoidal closed category. Suppose $ \cat{C} $ is $ \cat{V} $-enriched and so is $ \cat{V} $\footnote{A monoidal category $ \cat{V} $ being enriched over itself means that the hom-object between two objects in $ \cat{V} $ is defined as some other object in $ \cat{V} $ satisfying those properties.}. Then the power of an object $ c\in \cat{C} $ by an object $ v\in \cat{V} $ is defined to be the object $ \power{c}{v} \in \cat{C}$ with the following natural isomorphism:
	\begin{align*}
		\cat{C}(d,\power{c}{v}) \isomorph \cat{V}(v,\cat{C}(d,c)).
	\end{align*}
Note that the object power operation $ \pitchfork $ is the following functor:
\begin{align*}
	\power{-}{-} : \opcat{V}\times \cat{C} \longrightarrow \cat{C}.
\end{align*}
If all such power objects $ \power{c}{v} $ for all $ c\in \obj{\cat{C}} $ and $ v\in \obj{\cat{V}} $ exists, then we call $ \cat{C} $ to be \textbf{powered} or \textbf{cotensored category over $ \cat{V} $}.
\end{definition}

%\subsection{Enriched Grothendieck Topology}
%\begin{definition}
%	(\textbf{Localization of Enriched Functor Category}) Suppose $ \cat{C} $ is a small category and $ \cat{V} $ is a LFPSMC category. A localization of the $ \cat{V} $-enriched functor category $ \cat{V}^{\opcat{C}} $ is a full and reflective subcategory $ L \subseteq \cat{V}^{\opcat{C}} $ whose reflection (the left-adjoint) preserves finite limits.
%\end{definition}
%\begin{remark}
%	This is the generalization of the usual case of the full-subcategory of sheaves over a site, which forms a full-reflective subcategory of the presheaf category. Remember that the sheafification functor in this example preserves finite limits.
%\end{remark}
%\begin{definition}
%	(\textbf{Enriched Grothendieck Topology}) Suppose $ \cat{C} $ is a $ \cat{V} $-enriched small category where $ \cat{V} $ is LFPSMC. A $ \cat{V} $-Enriched Grothendieck Topology is then an assignment $ T(c) $ for each object $ c $ of $ \cat{C} $ of subobjects\footnote{A subobject of $ \cat{C}(-,c) :\opcat{C} \longrightarrow \cat{V}$ would be an ordinary natural transformation $ \eta: S \Rightarrow \cat{C}(-,c) $ with components $ \eta_d : Sd \rightarrowtail \cat{C}(d,c) $, where $ \eta_d $ is a monic arrow in the LFPSMC category $ \cat{V} $, for all objects $ d $ of $ \cat{C} $.} of $ \cat{C}(-,c) : \opcat{C} \to \cat{V}$ in the ordinary functor category $ \cat{V}^{\opcat{C}}$, which satisfies the following axioms:
%	\begin{itemize}
%		\item [\textbf{EGT.1}]{\textbf{Maximal Subobject} The subobject $ \cat{C}(-,c) $ is present in $ T(c) $.}
%	\end{itemize}
%\end{definition}

\newpage
\begin{thebibliography}{9}
	\bibitem{MacMoer} 
	Maclane, S., Moerdijk, I (1994). Sheaves in Geometry and Logic, A First Introduction to Topos Theory. Springer, New York, NY, 978-1-4612-0927-0, 0172-5939.
	\bibitem{Elephant}
	Johnstone, P. (2002), Sketches of an Elephant – A Topos Theory Compendium, Volume 2, Oxford University Press. ISBN:9780198515982.
\end{thebibliography}

\end{document}

